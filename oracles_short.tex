\documentclass[12pt]{article}
\usepackage{personal}
\usepackage{realoracles}


\newtheorem{theorem}{Theorem}[section]
\newtheorem{lemma}{Lemma}[section]
\newtheorem{corollary}{Corollary}[section]
\newtheorem{proposition}{Proposition}[section]



\title{Real Numbers As Rational Betweenness Relations}

\jtauthor
\date{\today}



%\sloppy%\openup-.1\jot
\begin{document}\maketitle
\begin{abstract}
Irrational numbers cannot be known in a precise way in the same fashion that rational numbers can be. But one can be precise about the uncertainty. The assertion is that if one knows whether or not a real number is in any given rational interval, then one can claim to know the real number. This insight leads to a two tiered definition of real number. The top tier is the idealized rational betweenness relation which details what properties a relation on rational numbers must satisfy to qualify as a real number. The other tier is the practical approach to real numbers. These are non-unique procedures that give definite pathways to bringing forth the rational betweenness relations. Through these procedures, it will be established that the rational betweenness relations satisfy the axioms of the real numbers. 
\end{abstract}

Real numbers are a fundamental part of mathematics. There are many different definitions and they each have their own inadequacies. This paper will present a new definition based on intervals. Intervals have been used before, but the definitions here should be more flexible and hopefully give a clearer structure in dealing with the tension of a real number as an infinite entity that practically is often only glimpsed at with finite means. 

Decimals are the first presentation of real numbers that most people are introduced to and, for non-mathematicians, often the only one. They seem nice at first, giving a good intuitive sense as to where the real number is roughly located on a number line. But the issue is that computing out a decimal to $n$ places of known accuracy may not be possible in finite time. A primary example is that of adding two numbers whose decimals happen to add up to 9 for many places. Until a non-9 is found or a proof is made that they will always be 9, there is uncertainty in all of those decimal places. Multiplication compounds this difficulty. A nice example is computing $1/9 * 1/9$ purely in terms of decimals; Norm Wildberger has a video entitled ``Fractions and repeating decimals'' which has a section, about 40 minutes in, on this computation. The multiplication has arbitrarily large carries forming a pattern which is hard to deduce from the decimal version alone. In producing the decimals, even as the uncertainty in the $n$-th decimal place may continue for awhile, the further computation is narrowing on an interval that contains the real number. This naturally leads one to thinking of a sequence of nested intervals being generated. This is a valid way of defining a real number as Bachmann essentially did in 1892 and as reviewed in the survey of real number definitions \cite{ittay-2015}. This is one generalization of the decimal approach; it introduces an extreme abundance of different representations for a given real number. 

Another approach using sequences is that of Cauchy sequences. This is the idea of a sequence of rational numbers  with some notion of the terms eventually being arbitrarily close; they are taken to converge to the real number which is itself considered to be the sequence and its equivalents. One of the issues is the immense non-uniqueness of Cauchy sequences. In particular, any Cauchy sequence can have any finite sequence prepended without changing the tail. This highlights that the sequence of numbers is useless unless one also has a shrinking distance  for which the following elements of the sequence are within that distance of the given element of the sequence. That is, the useful version of a Cauchy sequence is to see it as a sequence of intervals given by specifying the center of the interval and its radius. The sequence of intervals has the property that eventually the radii get arbitrarily close to 0 and all of the intervals overlap. Generalizing this to a set of intervals instead of a sequence is what this paper calls fonsis and is an approach to defining real numbers given by constructivists \cite{bridger}.

A third approach is that of Dedekind cuts. One version has a real number be the set of all rational numbers less than itself. This works surprising well formally. But it is a little unclear what is being computed. In particular, to get a sense of what the number actually is, there has to be some computation of an upper bound on that set. Pairing an element of this set with an upper bound then provides an interval which ought to contain the real number. By pairing all the lower bounds with all the upper bounds leads to the idea of the set of all intervals that contain the real number. This is what this paper calls rational betweenness relations and is what is asserted to be the singular entity of what a real number is. Its equivalence to Dedekind cuts is detailed in \cite{taylor24dedekind}. A related construct, which includes many more objects than just intervals, is that of minimal Cauchy filters as detailed in \cite{ittay-2015}.

In all of the approaches above, the object taken to be the real number is something of infinite scope presented as if it is completely known. That is, it is an object which exists in some fully realized sense. This is not possible for many real numbers of interest. The main approach of this paper is to use an auxiliary object which is, at its heart, a computational structure that allows one to refine the real number presentation as much as one wants, but there is not a notion of having a finished state. The procedure itself, when satisfying the required properties, is taken to be sufficient information to say that one knows the real number. 

As a guide in contemplating different notions, it is useful to consider how equality and arithmetic are handled, particularly on how an irrational minus something that may be itself is computationally known to be 0 or not. 

The first step towards this idea is to think of a real number $x$ as providing a betweenness relation on rational numbers by $a$ and $b$ being $x$-related if $x$ is, inclusively, between $a$ and $b$. This almost works. But there are many situations in which one cannot definitively conclude that a given real number is between two particular rational numbers. The resolution of this is to have an underlying object managing the information. It allows for progressive refinements to be made. 

These procedures have the property that given any interval of rational numbers, the procedure can answer whether there is a small interval containing the real number that intersects the interval in question. That small interval can then be used for further computations. The term oracle will be used for such procedures to indicate that it produces a kind of self-fulfilling prophecy. The setup allows for the computation of intervals, but does not require the computations be done in order to define the real number. The main downside is that multiple procedures can represent the same real number, but the ambiguity is nonconstructively removed at the higher level presentation of the $x$-relation viewpoint.  

This paper will first establish some basic notations for working with rational intervals before defining both the rational betweenness relations and the oracles. A brief section on some examples will be given which will be light on details. See \cite{taylor23main} for a more in-depth discussion of examples, uses, and comparisons to other definitions of real numbers. Equality, inequality, completeness, and arithmetic are all explored to fully establish the real number properties of these objects. 

One aspect of this paper is to highlight when something is constructively known or not. For this to be relevant, it is necessary that the given computational objects, the oracles, can be computed in a predictable amount of finite time. This will not always be the case, but that is not the concern here. The main concern is to distinguish whether the claimed knowledge of the statements could be constructive if the oracle was constructive. The viewpoint of constructivity that is adopted here is whether or not one can say ``this knowledge will be known after this amount of finite work.'' 

\section{Interval Notations}

This section is in common to the other papers on this subject by this author, such as \cite{taylor24dedekind}.

The set of all rationals $q$ such that $q$ is between $a$ and $b$, including the possibility that $q=a$ or $q=b$, is a \textbf{rational interval} denoted by $a\rint b$. If $a=b$, then this is a \textbf{rational singleton} denoted by $a\rint a$. A singleton is a set of exactly one rational number, namely, $a$. To indicate $a \leq b$, the notation for the interval can be $a \lte b$ as well as $b \gte a$. If $a < b$ and the interval is being presented, then the notation $a \lt b$ may be used as well as $b \gt a$. To indicate that $a \neq b$ but without knowing the order, the symbol $a \betneq b$ may be used; these are the \textbf{neighborly intervals}. The symbol $\mathbb{I}$ will represent the set of all rational intervals. 

The notations above will also be used to indicate betweenness. For example, if $a \leq b \leq c$ or $c \leq b \leq a$, then $a\rint b\rint c$ will be used to denote that. By definition, $c\rint b\rint a$ and $a\rint b\rint c$ represent the same betweenness assertion. This can be extended to any number of betweenness relations, such as $a\rint b\rint c\rint d$ implying either $a \leq b \leq c \leq d$ or $d \leq c \leq b \leq a$. There are also some trivial ways to extend given betweenness chains. For example, if $a\rint b\rint c$, then $a\rint a\rint b\rint c$ holds as well. Another example is that if $a\rint b\rint c$ and $b\rint c\rint d$, then $a\rint b\rint c\rint d$ holds true. This all follows from standard inequality rules for rational numbers. 

If $b$ and $c$ are between $a$ and $d$, but it is not clear whether $b$ is between $a$ and $c$ or between $c$ and $d$, then the notation $a\rint \{b,c\}\rint d$ can be used. This can also be extended to have, for example, $a\rint b\rint \{c,d\}$ which would be shorthand for saying that both $a\rint b\rint c$ and $a\rint b\rint d$ hold true. In addition, the notation \sout{$a\rint b\rint c$} will be used to indicate that $b$ is not between $a$ and $c$. One could also indicate this by $b\rint \{a,c\}$ which could be said in words that $a$ and $c$ are on the same side of $b$. 

Rational numbers satisfy the fact that, given three distinct rational numbers, $a, b, c$, exactly one of the following holds true:  $a\rint b\rint c$, $a\rint c\rint b$, or $b\rint a\rint c$. That is, exactly one of them is between the other two. It can be written in notation as $a\rint b\rint c$ holds true if and only if both \sout{$a\rint c\rint b$} and \sout{$b\rint a\rint c$} hold true. This follows from the pairwise ordering of each of them as provided by the Trichotomy property for rational numbers along with the Transitive property. 

In this paper, often a potential relabeling will be invoked. This is to indicate that there are certain assumptions that are needed to be made, but they are notational assumptions and, in fact, some arrangement of the labels of that kind must hold. For example, if $\{a,d\}\rint b\rint c$ holds true, then either $a\rint d\rint b\rint c$ or $d\rint a\rint b\rint c$ holds true. If these are generic elements, then relabeling could be used to have $a\rint d\rint b\rint c$  be  true for definiteness, avoiding breaking the argument into separate, but identical, cases. If $a$ and $d$ were distinguished in some other way, such as being produced by different processes, then relabeling would not be appropriate to use. 

If $a\rint b\rint c\rint d$, then the union of $a\rint c$ with $b\rint d$ is the interval $a\rint d$. If $b \neq c$, then the union of $a\rint b$ and $c\rint d$ as a set is not an interval. One can still consider the \textbf{intervalized union} $a\rint d$ as the shortest interval that contains both intervals. 

An \textbf{$a$-rooted} interval is an interval who has an endpoint that is $a$, that is, they are of the form $a\rint b$ for some $b$. An  \textbf{$a$-neighborly} interval is an interval that strictly contains $a$. The set of all $a$-rooted intervals will be denoted $\mathbb{I}_a$ while the set of all $a$-neighborly intervals will be denoted by $\mathbb{I}_{(a)}$

For rational $\delta > 0$, the notation $\halo{a}$ represents the \textbf{$\delta$-halo of $a$} which is defined as $a -\delta \rint  a+ \delta$. An interval is $\halo{a}$ \textbf{compatible }if it is strictly contained in $\halo{a}$ and is an $a$-neighborly interval. That is, the interval $c \lt d$ is $\halo{a}$ compatible if $a- \delta < c < a < d < a+ \delta$. The halo notation can be extended to any interval. The notation $\halo{a\rint b}$ will refer to $\halo{a} \cup a\rint b \cup b_\delta$. If $a \leq b$, this is the same as the interval $a-\delta\rint b+\delta$. 

The notation $b\rint a|_\delta$ will indicate that the interval goes from $b$ to the farthest endpoint of $\halo{a}$ from $b$. If, for example, $b < a$, then $b\rint a|_\delta$ is the same as $b\rint a+\delta$ while ${}_\delta|b\rint a$ would be the interval $b-\delta\rint a$. Also, ${}_\delta|a \rint  b$ is the same as $b\rint a|_\delta$. This notation makes the most sense for $b$ outside of $\halo{a}$. The intent of this notation is to avoid having to specify the inequality relation of $a$ and $b$ while still being able to expand outside of $a\rint b$ in the specified direction. The potential notation of ${}_\delta | a\rint b|_\delta$ would have the same meaning as $[a\rint b]_\delta$; this paper will generally use the halo notation. 

Given $m$ in $a \rint  b$, a \textbf{subwidth} $\delta$ shall mean a positive rational number such that $\halo{m}$ is strictly contained in $a\rint b$. For a subwidth to exist, it must be case that $m$ is strictly contained in $a\rint b$. A notation that can express this is $a \betneq \halo{m} \betneq b$ which is asserting that if $p \in \halo{m}$, then $p \in a\rint b$, $p \neq a$ and $p \neq b$. 

The term subinterval of $a\rint b$ will include $a\rint b$ as a subinterval but it does not include the empty set. 

The length of $a\rint b$ will be denoted by $|a\rint b|$ and is equal to $|b-a|$. If $a \betneq b \rint c$, then the distance between $a$ and the interval $b \rint c$ will be the $|a-b|$. If $a \rint b \betneq c \rint d$, then the distance between the intervals $a \rint b$  and $c \rint d$ is $|b-c|$. That is, the distance uses the closest endpoints for its computation. 

Throughout the paper, unless noted otherwise, the letters $a$ through $w$ will represent rational numbers, $x, y, z, \alpha, \beta$ will represent real numbers, and $\delta, \varepsilon$ will represent positive rational numbers. Primes on symbols will be assumed to be of the same type as the unprimed version. 

\section{Rational Betweenness Relations}

The definition of a \textbf{rational betweenness relation}, with relational symbol $\xrel$, is that it is a symmetric relation on rational numbers which satisfies the properties listed below. If $a \xrel b$, then the interval $a\rint b$ is said to be an \textbf{$x$-interval} and $a$ and $b$ are \textbf{$x$-related}.
\begin{enumerate}
    \item Existence. There exists $a$ and $b$ such that $a\xrel b$.
    \item Separation. If $a \xrel b$ and $c$ is strictly between $a$ and $b$, then exactly one of the following holds:  1) $a \xrel c$ and \sout{$c \xrel b$}, 2) $c \xrel b$ and \sout{$a \xrel c$}, or 3) $c \xrel c$. 
    \item Consistency. If $c \rint  a \rint  b \rint  d$ and $a \xrel b$, then $c \xrel d$. 
    \item Singular. If $c \xrel c$ and $d \xrel d$, then $c=d$. 
    \item Closed. If $c$ is a rational number such that $c$ is included in every $x$-interval $a\rint b$, then  $c \xrel c$. 
\end{enumerate}

The set of all such relations when coupled with appropriate ordered field operations satisfies the axioms of the complete field of real numbers. This definition was shown to work in \cite{taylor24dedekind} by demonstrating that they are equivalent to Dedekind cuts. The focus of this paper is to establish these as a model of the real numbers using the practical approach of the oracle procedures defined in the next section. 

For a given relation $\xrel$, it can be convenient to use the term Yes interval for intervals $a\rint b$ that satisfy $a \xrel b$ and the term No interval if \sout{$ a \xrel b$}. These terms will be reused in other sections to refer to intervals using oracle procedures though it will eventually be established that the Yes intervals of an oracle representing $x$ coincide with the Yes intervals of $\xrel$.

It can be useful to recast the definitions above in the Yes/No interval language. Also of use is the term a \textbf{root of the relation} which is a rational number $c$ such that $c \xrel c$.

\begin{enumerate}
    \item Existence. There exists a Yes interval. Without this, one could have the null relation as a rational betweenness relation. Existence gives a starting point for further computations using Separation. 
    \item Separation. Any rational number $c$ strictly contained in a Yes interval will divide that Yes interval into two new intervals one of which will be Yes and the other will be No unless $c$ is a root of the relation in which case the two intervals and the singleton $c\rint c$ are Yes intervals. Separation is inspired by the Intermediate Value Theorem and is the computational property allowing for a narrowing down of intervals. 
    \item Consistency. If an interval contains a Yes interval, then it is a Yes interval. This also implies that an interval contained in a No interval is a No interval. Consistency allows Separation to rule out two disjoint separated Yes intervals. 
    \item Singular. There is at most one root of the relation. This prevents the possibility of an interval of roots. The Separation property does not rule this out. Separation with consistency would rule out a finite number of roots, but not an interval of roots.  
    \item Closed. If a rational number $c$ is in every Yes interval, then $c$ is a root of the relation. This establishes a unique representative relation for rational numbers. Without this, one could have, for example, the relation consisting of neighborly intervals of 0 along with 0-rooted intervals of the form $0 \lt b$. Another variant would instead have the rooted intervals just be $0 \gt b$. Both would represent 0 and be distinct. The Closed property, along with Consistency, rules out this multiplicity. 
\end{enumerate}

These properties ensure that two different rational betweenness relations cannot represent the same real number. 

To get a little familiarity with these properties, it can be instructive to see how, given an interval $c\rint d$, the properties give a pathway for determining whether $c \xrel d$ or \sout{$c \xrel d$}. It starts with existence yielding an $x$-Yes interval $a\rint b$. If $a\rint b$ does not strictly contain $c\rint d$, one can assume by relabeling that $a\rint \{b, c\} \rint d$. Then $a\rint d|_1$ is an interval that strictly contains $c\rint d$ and contains $a\rint b$. Thus, by Consistency $a\rint d|_1$ is an $x$-Yes interval that strictly contains $c\rint d$. Relabel $a\rint d|_1$ as $a\rint b$ such that $a\rint c\rint d\rint b$. Given that, Separation can be used, using $c$ and $d$. If either $c \xrel c$ or $d \xrel d$, then $c \xrel d$ by Consistency. Assuming  not, Separation applied to $c$ leads to either \sout{$c \xrel b$} or $c \xrel b$. In the first instance, Consistency implies \sout{$c \xrel d$}. In the second, use $d$ in Separation of $c \xrel b$. Then either $c \xrel d$ or \sout{$c \xrel d$}. Thus, the properties of Existence, Separation, and Consistency lead to a conclusion being able to be drawn about every pairing of rational numbers. 


Arithmetic is defined using interval arithmetic, as will be discussed later, but there is a closure step that may be needed. For example, if $x$ is irrational, then $x - x = 0$ is still a statement that is desired, but the arithmetic of intervals will never produce any Yes intervals with an endpoint of $0$, let alone $0\rint 0$. When all the intervals of $x$ are subtracted from one another, every resulting interval will contain 0, but not all intervals that contain 0 are included. Thus, the arithmetic does not directly produce rational betweenness relations from rational betweenness relations. It does seem reasonable, however, to have a mechanism that would take from this process that the relation under discussion is the one that includes all intervals that contain $0$. Since the mechanism to do so is also a convenient mechanism for how real numbers are actually used, it seems reasonable to focus on it. That mechanism consists of the oracles and their prophecies. 

\section{Oracles}

An \textbf{oracle} is a procedure $R$, satisfying the properties listed below, which should be able to handle any input of the form of a rational interval along with a positive rational number. The output, which is not necessarily exclusively defined by the input, should be an ordered pair where the first entry is one of three values:  $-1, 0 , 1$ while the second entry is either a rational interval or the empty set. The intention is that $-1$ will indicate that the real number is not in the input interval while both $0$ and $1$ indicate it could be while $1$ additionally indicates that the returned interval is small enough in relation to the given positive rational number and given interval. 

%To make it into a function, one could view it as $R \mathbin{\col} \mathbb{I} \times \mathbb{Q}^+ \to \mathcal{P}( \{-1, 0, 1\} \times \mathbb{I})$ where $\mathcal{P}$ is the power set operator. This would suggest, however, computing the full output of $R$ for a given input which is not something generally needed nor desired. 

The idea for the returned intervals is that they are considered to contain the real number. Any interval $c\rint d$ that appears as the second entry of a pair for an input will be said to be in the range of $R$, denoted as $c\rint d \in R$. The whimsical language of calling the nonempty interval $c\rint d \in R$ a \textbf{prophecy} of the oracle may also be used.

Writing $R(a\rint b, \delta)=(m ,A)$ implies that one of the possible outputs for that input is $(m, A)$. The expression $R(a\rint b, \delta) = c\rint d$ will be shorthand to imply that one of the outputs for that pair of inputs is $(1, c\rint d$).  The expression $R(a \rint b, \delta) = m$ implies that there is an output pair for that input such that $m$ is the first entry of the pair. The expression $R(a\rint b, \delta) \neq m$ means that none of the outputs for that input pair has $m$ as its first entry.  

The notation $\mathbb{I}_R$ represents the set of all intervals that contain a prophecy, that is, there is a subinterval in the range of $R$. 

To be an oracle, the procedure must satisfy six properties.  The properties are
\begin{enumerate}
    \item Range. If $R(a\rint b, \delta) = (n, c\rint d)$, then one of the following must hold
    \begin{itemize}
        \item $n=1$. $c\rint d$ is a subinterval of $\halo{a\rint b}$ and intersects $a\rint b$. 
        \item $n=0$. $c\rint d$ is not contained in $\halo{a\rint b}$ but does intersect $a \rint b$.
        \item $n=-1$. $c\rint d$ is disjoint from $a\rint b$.
    \end{itemize} 
    If $R(a \rint b, \delta) = (n, \emptyset)$, then $n = -1$. 
    \item Existence. There exists $a \rint b$ and $\delta$ such that $R(a \rint b, \delta)$ has an output whose second entry is a rational interval. That is, there exists $c\rint d \in R$. 
    \item Separation. 
    If $a\rint b \in R$, then for a given $m$ contained in $a\rint b$ and given a $\delta$, there exists an $\halo{m}$ compatible interval $e\rint f$ such that exactly one of the following holds true:   ${}_\delta |a\rint e \in \mathbb{I}_R$, $e\rint f \in \mathbb{I}_R$,  or $f\rint b|_\delta \in \mathbb{I}_R$.
   \item Disjointness. 
   If $c\rint d \in R$ and $a\rint b$ is disjoint from $c\rint d$, then there exists a $\delta$ such that $R(a\rint b, \delta) = -1$. %If $R(a\rint b, \delta) = -1$, then there exists $c \rint d \in R$ such that $a\rint b$ and $c\rint d$ are disjoint. 
    \item Consistency. 
    If $a\rint b  \in \mathbb{I}_R$, then for all $\delta$, $R(a\rint b, \delta) \neq -1$  and $R(a\rint b, \delta) = 1$.
    \item Closed. 
    If $\halo{a} \in \mathbb{I}_R$ for all $\delta $, then, for all $\delta$ and $b$, $R(a\rint b, \delta) \neq -1$  and $R(a \rint b, \delta)=1$. Such an $a$ is called a \textbf{root} of the oracle. 
\end{enumerate}

Oracles may also take additional inputs, such as a starting interval for a process that is different than the given interval. This allows for more flexibility and is particularly helpful for moving from the ambiguous $n=0$ situation to the more definite $n=1$ or $n=-1$ responses. The Separation property implies that all oracles can be used in some fashion in an iterative fashion, such as in the Bisection Algorithm established later, but many oracles naturally come along with an iterative process. 

If multiple oracles representing distinct real numbers are being discussed, such as $x$ and $y$, then $R_x$ and $R_y$ will represent their respective oracles. If multiple oracles that represent the same real number are used, then a prime will be used such as $R'_x$ representing a different oracle than $R_x$ but both are representing $x$. A later section will explore establishing when two oracles represent the same real number or not. 

The expression $R(a\rint b, \delta) \neq -1$ implies that neither the empty set nor intervals disjoint from $a\rint b$ are part of any output of $R$ for that input. The expression $R(a\rint b, \delta) = -1$ is asserting that there is an output of $R$ for that input that either has the empty set or an interval disjoint from $a \rint b$, but there may be other outputs for that input which could have overlapping intervals with $a \rint b$.

The range of $R$ can be thought of as intervals that are known to include the real number. The set $\mathbb{I}_R$ contains all the intervals that contain an interval in the range of $R$. Consistency ensures that intersecting intervals are always assigned which is possible since all intervals in $\mathbb{I}_R$ contain an interval that can be returned for any $\delta$.

 For example, if approximating the square root of 2 and the interval is $1.3\rint 1.4$ with a $\delta$ of $0.1$, then the procedure might generate $(1, 1.39\rint 1.42$), but another computation with the procedure might generate $(-1, 1.41\rint 1.42)$. Both $R(1.3\rint 1.4, 0.1) = 1.39\rint 1.42$ and $R(1.3\rint 1.4, 0.1) = -1$ would be valid statements given these computations. Only the latter would be definitive in excluding $1.3\rint 1.4$ from containing the square root. The interval $1.3\rint 1.4$ would not be in $\mathbb{I}_R$ while $1.39\rint 1.43$ would be as it contains $1.39\rint 1.42$ which was a returned output of $R$.

A rational interval $a\rint b$ is a \textbf{Yes interval} of the oracle $R$ if $R(a\rint b, \delta) \neq -1$ for all $\delta >0$. By Conistency, this includes all intervals that are in  $\mathbb{I}_R$, but, by Closure, it also includes the \textbf{$a$-rooted} intervals where $a$ is a root of the oracle.   A rational interval $a\rint b$ is a \textbf{No interval} if $R(a\rint b, \delta) = -1$ for some $\delta$. It can be argued nonconstructively that every No interval is disjoint from at least one prophecy.  

In a nonconstructive sense, each interval $a\rint b$ is either a Yes interval or a No interval. Given any interval, one can take a prophecy and test it against the interval. If the interval is disjoint from the prophecy, then the interval is a No interval. If the interval contains the prophecy, then it is a Yes interval. But if the prophecy only overlaps with the interval, part of it in, part of it out, then there is no conclusion. This latter indeterminate state can only happen for all prophecies if the given interval has a root for one of its endpoints. If that is the case, however, then the Closure property would indicate that the interval is  Yes interval. The difficulty is that there is no finite testing that leads to this conclusion; it must be proven by other means. The above reasoning does lead to a  nonconstructive argument that if an interval intersects every prophecy, it is a Yes interval; see Proposition \ref{os-inter}. As, nonconstructively, every interval either intersects all prophecies or is disjoint from at leat one, it is the case that all intervals nonconstructively are either a Yes or a No interval. 

If an interval $a\rint b$ is a Yes interval for $R$ and $R$ is to represent the real number $x$, then this can be expressed with the notation $a \xora b$. For No intervals, the notation is \sout{$a \xora b$}. The notation $a\rint c|_\delta$ extends to the Yes intervals as $a \xora c|_\delta$. These notations reflect that the Yes / No interval designation has created an $x$-betweenness relation on the rational numbers as will be established later.

It may be helpful to expand a little on what the properties mean. The basic idea is that a Yes interval ought to contain the real number; since this is defining the real number, this becomes more of a guiding idea rather than a deduction. The concept of the oracle is that a possible Yes interval is given along with a little error tolerance. The oracle ought to respond with a prophecy which helps move the process along in ascertaining what the real number is and will determine if the fuzzy version of the given interval may contain the real number or not. If it happens to not be able to respond with a prophecy, then the given interval is a No interval. 

Here is a bit of explanation for the properties: 
\begin{enumerate}
    \item Range. The idea of the oracle is that its first entry indicates success (1), indeterminate (0), or failure (-1). Thanks to Separation, given a prophecy, one can make a smaller prophecy. Thus, if a 0 is returned, there is an iterative procedure for continuing on. The Bisection Algorithm gives an explicit amount of computations to definitively reach a given size for a prophecy. When the size of the prophecy is less than $\delta$, then either a 1 is guaranteed to be a response or a -1 is. 
    
    Given an interval for a second entry, Separation allows for further refinement. If an empty set is returned, then that does not yield in any further direction, but the Existence property promises that there is always something to be used. 
    
    \item Existence. Without this, the oracle could always return the empty set. Such a procedure would not represent any $x$. On a practical level, having a prophecy returned is the starting point of where to start narrowing in on the real number. It can be quite a large interval which makes this doable if some very rough knowledge of the number is known. 
    
    \item Separation. This property is inspired by the Intermediate Value Theorem. The idea is that a Yes interval should be able to be continually narrowed down by selecting a rational number $m$ inside of it and then testing which of the two created intervals is a Yes interval and the other one would then be a No interval. Because of the possibility of not being able to decide the issue at $m$, there is a $\delta$-halo in which to examine it. 
    
    It can also be the case that while $a\rint b$ is in the range of $R$, none of its subintervals are. For example, in representing the real number $2$, the range of $R$ could be $1\rint 2$ along with intervals of the form $2 \lt b$. Intervals of the form $\halo{a\lte 2}$ would then contain, for example, the interval $2\rint (\delta/2)$ which is in the range of $R$. Picking $m=1.5$ in $1\rint 2$ and $\delta = 0.2$, an interval of $1.6\rint 2.1$ would contain an element of the range, say $2\rint 2.05$. If it had to be strictly in the interval $1\rint 2$, this procedure would not satisfy the Separation property. 

    Having a bit of fuzziness outside of $a\rint b$ also allows $m$ to be one of the endpoints, say, $a$. Then $\halo{a} = \halo{m}$ and the subinterval of ${}_\delta | a \rint  e$ would be disjoint from $a\rint b$ which would automatically imply by Disjointness that it is a No interval. But that is okay. The interval $e\rint f$, containing $a$, could still work if it contains a prophecy as could $f\rint b|_\delta$.
 
    \item Disjointness. Disjointness ensures that a single real number is being discussed. Without an assertion of negativity, one could have multiple disjoint regions.  As an example, imagine a procedure which returns small intervals around 2 and small intervals around 5. Done correctly, this will be able to satisfy the other properties, including the Separation property because that property only applies to prophecies. Consistency does not help as that is about intervals containing prophecies, but does not demand that there is a prophecy containing other prophecies.  Since rational betweenness relations do not make this distinction between prophecies and Yes intervals, it does not need this postulated as it can be deduced via Consistency and the exclusionary aspect of Separation. 
    
    \item Consistency. Consistency asserts that the oracle never contradicts itself. Since a prophecy is supposed to represent a real number being in the interval, if another interval contains it, then that containing interval would also contain the real number and ought to be a Yes interval. 
    
    \item Closed. This ensures that if there is a narrowing in to a single rational number, then the intervals with that rational number as endpoint are Yes intervals. It does not require that the interval in question be a prophecy nor contain one. The assumption that $\halo{a}$ is in $\mathbb{I}_R$ for all $\delta >0$ implies that $R$ can always return a subinterval of $\halo{a}$ for $R(a\rint b, \delta)$. Being Closed implies that $R$ does this or returns another suitable interval. Disjointness forces $a$ to be in those prophecies in this case. 

\end{enumerate}

The term procedure is being used instead of function to suggest a more constructive approach. While it is perfectly fine for an $R$ to be given as an explicit function, the more typical case is that $R$ is computed out as needed and may not return the same result for the same inputs. For example, if a procedure is based on Newton's method restricted to a region of unique convergence but no specific starting point is given, then different choices of starting points could lead to different returned intervals. The design here is to allow for that variation. 




\subsection{Examples}

The first set of examples is that of rational numbers. Given a rational number $q$, one oracle, the \textbf{Singular Oracle at $q$}, is to have the procedure $R(a\rint b, \delta) = q\rint q$ if $a\rint q\rint b$ and is the empty set otherwise. This is the nice version of a rational number. The \textbf{Reflexive Oracle at $q$} is the procedure $R(a\rint b, \delta) = a\rint b$ if $q \in a\rint b$ and the empty set otherwise. A less nice version, but one more in line with what is produced from arithmetic operations on irrational numbers is the \textbf{Fuzzy Oracle at $q$} whose procedure is that $R(a\rint b, \delta)$ returns $\halo{q}$ if $\halo{q}$ intersects $a\rint b$ and returns the empty set otherwise. A multi-valued version would be that given $a\rint b$ and $\delta$, consider the set of intervals of size $\delta$ or less that contain $q$ and select one. If it intersects $a\rint b$, then the selected interval is returned and, if not, the empty set is returned. All these versions have that the Yes intervals are the intervals that contain $q$ while the No intervals are those that do not contain $q$. These oracles are considered equivalent, as discussed in a later section. Oracles equivalent to them will be known as an Oracle of $q$.  

The next set of examples are the $n$-th roots. Let $x$ represent the positive real number such that $x^n$ ought to be the positive rational $q$. The comprehensive procedure for these is that $R(a\lt b, \delta) = a\rint b$ whenever $b > 0$ and $\max(a, 0)^n\rint q\rint b^n$; it returns the empty set otherwise.

A common situation for real number estimates is that the real number is computed as a sequence of intervals. For example, the $n$-th root can be computed using Newton's method. Let $a_0 >0$ be some positive rational. Given $a_m$, define $b_m = q/a_m^{n-1}$. It can be shown that $a_m^n \rint  q \rint  b_m^n$ will hold true. The iteration is defined as $a_{m+1} = a_m + (b_m - a_m)/n$. This is what results from applying Newton's method to $x^n - q$. The oracle procedure $R$ would then be to compute $m, a_m, b_m$ such that $|a_m - b_m| < \delta$.  Then $R(a\rint b, \delta) = a_m\rint b_m$ if $a_m\rint b_m$ intersects $a\rint b$ and is the the empty set otherwise. Due to the length, $a_m\rint b_m$ will be in $\halo{a\rint b}$ if it intersects $a\rint b$. 

This particular way of using Newton's method leads to a family of procedures. Given a different $a_0$, the intervals computed will differ and will lead to some computed intervals overlapping the endpoints of some intervals that other starting points would not overlap with. The Yes / No determination, however, would not  change. If the root is the rational $p$, that is, $p^n = q$,  then intervals of the form $p\rint b$ will never be returned by this procedure unless $a_0 = p = b_0$. Nevertheless, the procedure will never return the empty set for an interval that contains $p$. All Yes intervals in this case will contain $p$. 

More general uses of Newton's method can be discussed, but it is more efficient to discuss a general construct based on interval families. This is discussed in Section \ref{os-fonsis} about fonsis. 

Cauchy sequences can be viewed as pairs of rationals $a_n, \varepsilon_n$ where $a_n$ is the $n$-th element of the sequence and $\varepsilon_n$ is a bound for all future elements to be within $\varepsilon_n$ of $a_n$. This leads to the sequence of intervals $a_n-\varepsilon_n\rint a_n+\varepsilon_n = \halo[ \varepsilon_n]{a_n}$. The Cauchy criterion ensures that the $\varepsilon_n$ exists and that they can be taken to approach 0. An oracle procedure would then be to select, possibly arbitrarily, some $n$ such that $\varepsilon_n < \delta$, and then  $R(b\rint c, \delta) = \halo[ \varepsilon_n]{a_n}$ if $\halo[\varepsilon_n]{a_n}$ intersects $b\rint c$; it returns the empty set otherwise. If $R(b\rint c, \delta) \neq \emptyset$, this implies that the entire tail of the sequence of intervals intersects $b\rint c$ and thus the sequence converges to a real number that is in $b\rint c$.  

For a set $E$ of rationals bounded above, the least upper bound can be defined as $R(a \lte b, \delta)$ to be $a\rint b$ when $a$ is bounded above by an element of $E$ and $b$ is an upper bound of $E$; it would be the empty set otherwise. For practical reasons, it may not be easy to determine this. Therefore, one could extend it such that if $\halo{a}$ or $\halo{b}$ contains both elements of $E$ and upper bounds, then $\halo{a}$, respectively $\halo{b}$, is returned. 

The last example of this section are the real numbers produced by the Intermediate Value Theorem. Let $f(x)$ be a function which is continuous and monotonic on $a\rint b$ with $f(a)*f(b) < 0$. For rational intervals $c\rint d$ contained in $a\rint b$, define the oracle $R$ by $R(c\rint d, \delta) = c\rint d$ if $f(c)*f(d) \leq 0$ and the empty set otherwise. For intervals that are not contained in $a\rint b$, take the intersection with $a\rint b$ and return its result. Monotonicity is required for the Disjointness property. Continuity is required to show that the oracle produced here is a zero of $f$. Without continuity, the most that can be said is that the oracle is the location of a sign change for the function. The initial $a\rint b$ yields Existence.

The above is assuming that a sign can be determined for $f(q)$ for any rational $q$ in $a\rint b$. It is possible that if $f(q)$ is close to zero, something which is desired, then computing $f(q)$ leads to fuzziness which prevents a sign determination. In that case, the continuity of the function allows, given $\varepsilon > 0$, the existence of a $\delta$ such that $f(\halo{q})$ is contained in the interval $\halo[\varepsilon]{0}$. Thus, one could specify an $\varepsilon$ tolerance and if $f$ is in that region, then $\halo{q}$ is returned by the procedure. Practically speaking, this is when a process will terminate, having hit the resolution limits of the computational system. 

Without monotonicity, uniqueness potentially fails. Some other mechanism or condition needs to be added to make it into an oracle. One can still do the usual process of dividing up the interval and testing the signs to create a sequence of nested intervals. That sequence could be used to define an oracle particularly if, after finitely many steps, it wanders into a subinterval with a unique solution contained in it. The oracle that is found would generally be path dependent on the choice of division points. 

\subsection{Small Yes Intervals}

To approximate a real number well, one needs a very small interval in which it is known to be in. Given an oracle, the existence and separation properties are the ingredients necessary to refine the intervals. This, and its implications, is what this section is about. 


\begin{proposition}[Bisection Algorithm]
    Given a rational $\varepsilon >0$ and an oracle $R$, there exists a prophecy whose length is less than $\varepsilon$.
\end{proposition}

\begin{proof}
    By Existence, there exists a prophecy $a_0\rint b_0$, that is, an interval in the range of $R$. If $|a_0\rint b_0| < \varepsilon$, then that interval is sufficient. Otherwise, the desired interval is found via iteration. The iterative step for determining $a_{i+1}$ and $b_{i+1}$ given $a_i\rint b_i$ begins with letting $m_i$ be the average of $a_i$ and $b_i$. Apply Separation to $a_i\rint b_i$ with $m_i$ the dividing point and choose a subwidth $\delta < \min(|a_i\rint b_i|/6, \varepsilon /2)$. By Separation and definition of $\mathbb{I}_R$, there exists an interval $e_i\rint f_i$ in $\halo{m_i}$ and a prophecy $a_{i+1}\rint b_{i+1}$ such that one of the following holds:  
    \begin{enumerate}
        \item $a_{i+1}\rint b_{i+1}$ is in $e_i\rint f_i$. Since the length of $\halo{m_i}$ is less than $\varepsilon$, this interval works as the desired interval.  
        \item  $a_{i+1}\rint b_{i+1}$ is in ${}_\delta|a_i \rint e_i$. The length will be no more than $\frac{|a_i\rint b_i|}{2}  + \frac{|a_i\rint b_i|}{6} = 2|a_i\rint b_i|/3$.
        \item  $a_{i+1}\rint b_{i+1}$ is in $f_i\rint b_i|_\delta$. The length will be no more than $2|a_i\rint b_i|/3$.
    \end{enumerate}
    The length of $a_n\rint b_n$ will be at most $(2/3)^n |a_0\rint b_0|$. To find the $n$ in which this procedure has definitively achieved the goal of $|a_n\rint b_n|<\varepsilon$,  take, in reduced form, $p/q = \varepsilon/|a_0\rint b_0|$ and let $m$ be the number of digits in $q$ expressed in base 10. Then $n \geq  6m$ leads to  $(2/3)^n \leq (2/3)^{6m} < 10^{-m} < 1/q \leq p/q$. Thus, $n \geq 6m$ would have $|a_n\rint b_n| \leq (2/3)^n |a_0\rint b_0| < \varepsilon$ as desired. 
\end{proof}

Since every prophecy is a Yes interval, this immediately implies: 
\begin{corollary}
    Given an oracle $R$, there are arbitrarily small Yes intervals. 
\end{corollary}

The following statement is independent of the Bisection Algorithm, but it is needed in the proof of the following proposition. 

\begin{proposition}\label{os-prointer}
    Given an oracle $R$, any pair of prophecies $a\rint b$ and $c\rint d$ will intersect. 
\end{proposition}

Given a procedure that is potentially an oracle, this can be a quick item to check to give some confidence as to whether it is an oracle or not. 

\begin{proof}
    If they were disjoint, then the Disjointness property would say, for example, that $R(c\rint d, \delta)= \emptyset$ for some $\delta$. But $c\rint d \in \mathbb{I}_R$ as it contains itself. Thus, Consistency says that $R(c\rint d, \delta) \neq \emptyset$. This is a contradiction and the two intervals must intersect. 
\end{proof}

\begin{proposition}\label{os-yescat}
    Let an oracle $R$ be given. If $a\rint b$ is a Yes interval, then either $a\rint b$ strictly contains a prophecy of $R$ or an endpoint of $a\rint b$ is a root of $R$. 
\end{proposition}

\begin{proof}
    By definition of $a \xora b$, $R(a\rint b, \delta) \neq \emptyset$ for all $\delta > 0$. This implies that, by Disjointness, all prophecies must intersect $a\rint b$. If any of them are strictly contained in $a\rint b$, then that satisfies what was to be shown. Thus, for the rest of this proof, assume all prophecies are not strictly contained in $a\rint b$.

    Let $L = |a\rint b|$. By the Bisection Algorithm, there exists $c\rint d \in R$ such that $|c\rint d| < L/2$. Due to its length, $c\rint d$ cannot contain both $a$ and $b$. As $c\rint d$ is a prophecy, it must intersect $a\rint b$. For it not to be contained in it, it must overlap one of the endpoints. By relabeling, assume $a \in c\rint d$.
    
    Let $\delta < L/2$ be given. The task is to show that $\halo{a} \in \mathbb{I}_R$. Let $e\rint f \in R$ be given such that $|e\rint f| < \delta$. By the same logic as before, the prophecy $e\rint f$ must contain one of the endpoints of $a\rint b$. Since it must intersect $c\rint d$ and the length of both intervals combined is less than $L$, the endpoint contained in $e\rint f$ must be $a$ as well. Since the length of $e\rint f$ is less than $\delta$, this implies that $e\rint f \subset \halo{a}$.

    Thus, as this holds for arbitrarily small $\delta$ and prophecies $e\rint f$, $a$ is a root of $R$ as was to be shown in this case. 
\end{proof}

\begin{proposition}\label{os-singular}
    If an oracle $R$ has a root $q$, then that root is unique and an interval is a Yes interval exactly when $q$ is in the interval.
\end{proposition}

\begin{proof}
    There are three possible cases for how an interval $a\rint b$ can be related to $q$: 

    \begin{enumerate}
        \item $q$ is an endpoint of $a\rint b$. Since $q$ is a root, any interval of the form $q\rint r$ is a Yes interval by the Closed property.
        \item $q$ is strictly contained in $a\rint b$. If $q \in a\rint b$ but is not an endpoint, then let $L$ be the distance from $q$ to the closest endpoint of $a\rint b$. Any interval $\halo{q}$ with $\delta < L $ will then be contained in $a\rint b$ and hence $a\rint b \in \mathbb{I}_R$ since $\halo{q}$ is. 
        \item $q$ is outside of $a\rint b$.  Let $L$ be the distance from $q$ to the closest endpoint. Then let $c\rint d$ be a prophecy contained in $\halo{q}$ with $\delta < L$. This exists as $q$ is a root. Due to the length, $c\rint d$ is disjoint from $a\rint b$. Thus, Disjointness yields a $\delta'$ such that $R(a\rint b, \delta')= \emptyset$. Hence, $a\rint b$ is a No interval. 
    \end{enumerate}
    
    As for the uniqueness of the root, let $b$ be any rational not equal to $q$ and $\delta = \frac{|b-q|}{2}$. Then $\halo{b}$ does not contain $q$ and hence is a No interval. This means it does not contain any prophecy and $b$ is not a root of the oracle. 
\end{proof}


\begin{proposition}\label{os-rootsmallpro}
    If $q$ is contained in arbitrarily small prophecies of an oracle, then $q$ is the root of the oracle. 
\end{proposition}

\begin{proof}
    Given $\delta$, let $a\rint b$ be a prophecy whose length is less than $\delta$ and contains $q$; this exists by assumption. Since $q$ is contained in $a\rint b$, $a\rint b$ is contained in $\halo{q}$. Thus, $\halo{q} \in \mathbb{I}_R$. As this was an arbitrary $\delta$, the Closed property applies to conclude  $q$ is the root of the oracle. 
\end{proof}

\begin{corollary}\label{os-root}
    If $q$ is contained in every prophecy of an oracle, then $q$ is the root of that oracle. 
\end{corollary}

\begin{proof}
    By the Bisection Algorithm, there exists arbitrarily small prophecies. As $q$ is contained in them, the proposition applies. 
\end{proof}



\begin{corollary}
    Given an oracle, $q$ is a root of the oracle if and only if $q$ is an element of every Yes interval. 
\end{corollary}

\begin{proposition}
    Given an oracle, $q$ is a root of the oracle if and only if $q$ is in arbitrarily small Yes intervals. 
\end{proposition}

\begin{proof}
   The one direction immediately follows from the previous statements. For the other direction, assume that $q$ is in arbitrarily small Yes intervals. Let $\delta$ be given and let $a\rint b$ be a Yes interval whose length is less than $\delta$ and contains $q$. Then $a\rint b$ is strictly contained in $\halo{q}$. From above, either $a\rint b \in \mathbb{I}_R$ in which case $\halo{q}$ is as well, or one of the endpoints is a root. In the latter case, by relabeling, let $a$ be the endpoint that is the root. The distance from $a$ to $q$ is less than $\delta$; call that $L$. Let $\delta' = \delta - L > 0$. As $a$ is a root, $\halo[\delta']{a} \in \mathbb{I}_R$ and it is contained in $\halo{q}$ as $\delta' + L = \delta$. Thus, $\halo{q}$ is in $\mathbb{I}_R$ in all cases and for all $\delta$.
\end{proof}


\subsection{Intersections}

The Yes intervals are united by supposedly containing the real number being represented. Thus, intersecting two such intervals should lead to a further narrowing towards the real number. This is indeed what happens. This section deals with various statements involving intersections and oracles. 


\begin{proposition}\label{os-yesinter}
    Given an oracle $R$, all Yes intervals intersect. 
\end{proposition}

\begin{proof}
    If $R$ is rooted at $q$, then $q$ is common to all Yes intervals. Hence, they intersect. 

    If $R$ is not rooted, then all Yes intervals contain at least one prophecy. Those prophecies intersect implying the intervals that contain them will intersect as well. 
\end{proof}


\begin{proposition}\label{os-inter}
    If $a\rint b$ is an interval that intersects every prophecy of an oracle, then, nonconstructively, $a\rint b$ is a Yes interval. 
\end{proposition}

\begin{proof}
    Let $\delta$ be less than half the length of $a\rint b$. By the Bisection Algorithm, there exists a prophecy $c\rint d$ whose length is less than $\delta$. By assumption, $c\rint d$ intersects $a\rint b$. This implies that either $c\rint d$ is contained in $a\rint b$, contains $a$ but not $b$, or contains $b$ but not $a$. If it is the first, then $a\rint b \in \mathbb{I}_R$ and is therefore a Yes interval. If not, by relabeling, take $a$ to be the endpoint that is contained in $c\rint d$. All prophecies whose lengths are less than $\delta$ will contain either $a$ or be contained in $a\rint b$; it cannot contain the other endpoint due to it needing to intersect $c\rint d$ and the two combined lengths are less than the length of $a\rint b$. If there is no prophecy that is a subinterval of $a\rint b$, then, nonconstructively, $a$ is contained in arbitrarily small prophecies. This in turn implies $a$ is a root of the oracle and, thus, $a\rint b$ is a Yes interval. 
\end{proof}

A useful idea is that every pair of disjoint intervals has at least one No interval in it. 

\begin{proposition}
    Given an oracle and two disjoint intervals $a\rint b$ and $c\rint d$, at least one of them is constructively known to be a No interval of that oracle. 
\end{proposition}

\begin{proof}
    As the intervals are disjoint, by potentially relabeling, it can be assumed $a\rint b\rint c\rint d$ with $b < c$. Then define $L = c-b > 0$. By the Bisection Algorithm, which is a finite constructive procedure, there is a Yes interval $e\rint f$ whose length is less than $L$. The interval $e\rint f$ must be disjoint from at least one of the intervals because of the length. The interval it is disjoint from is then a No interval by Disjointness. 
\end{proof}



   
\begin{proposition}
    If $ a\rint b$ and $c\rint d$ are Yes intervals for the same oracle, then their intersection is a Yes interval. 
\end{proposition}

\begin{proof}
    If there are Yes intervals that do not contain a prophecy, then by Proposition \ref{os-yescat}, it is a rooted oracle. If the oracle is rooted with root $q$, then the Yes intervals all contain $q$. Thus, they intersect and the intersection is an interval that contains $q$. Thus, it is a Yes interval as well. 

    The other situation is that all Yes intervals contain a prophecy. This will be assumed for what follows below. Note that there is no assumption that the oracle is not rooted; it can be the case that all Yes intervals contain prophecies and  the oracle is rooted. 

    Up to relabeling, the various cases are:  
    \begin{enumerate}
        \item $a\rint c\rint d\rint b$. In this case, the intersection is $c\rint d$ which is a Yes interval. 
        \item $a\rint b \betneq c\rint d$. This is the case of two disjoint Yes intervals. That was established to not be possible. 


        \item $a\rint (b=c)\rint d$. In this case, $b=c$ is the intersection.  If $a=b$ or $c=d$, then the intersection is a Yes interval as it is one of the given (singleton) intervals. Assume, therefore, that $a \neq b$ and $c \neq d$. 
        
        By assumption, $a\rint b$ and $c\rint d$ both contain prophecies. The prophecies must intersect and their intersection can only be the rational $b=c$. Since all prophecies must intersect both of these prophecies, they must all contain $b=c$. Any rational that is  contained in arbitrarily small prophecies is a root of the oracle by Proposition \ref{os-rootsmallpro}.
        
        \item $a\rint c \betneq b\rint d$. By assumption there is a prophecy $a'\rint b'$ in $a\rint b$ and a prophecy $c'\rint d'$ in $c\rint d$. The goal is to establish that there is a prophecy in $c\rint b$. This would then imply $c\rint b$ is a Yes interval. 
        
        If $a'\rint b'$ is contained in $c\rint b$ or $c'\rint d'$ is contained in $c\rint b$, then $c\rint b$ is a Yes interval. In the case that they are not, this implies, by potentially relabeling, that $a'$ and $d'$ are outside of $c\rint b$. Furthermore, as  $a'\rint b'$ and $c'\rint d'$ must intersect from being prophecies, it can be inferred that $c\rint c'\rint b'\rint b$. Thus, the following holds:   $a\rint a' \betneq c\rint c'\rint b'\rint b \betneq d'\rint d$. If $c'=b'$, then the case before can be adapted to claim that $c'=b'$ is the root of the oracle implying $c\rint b$ is a Yes interval. 

        Assuming $b' \neq c'$, let $\delta$ be a length less than half the distance between $b'$ and $c'$. Note that $\halo{b'}$ and $\halo{c'}$ are disjoint. Thus, at least one of them is a No interval. By potentially relabeling, let $\halo{c'}$ be a No interval. 

        Use Separation on the prophecy $a'\rint b'$ with separation point $c'$ and $\delta' < \delta$ taken to be a subwidth of $a'\rint b'$. Separation gives us a $\halo[\delta']{c}$ compatible interval $e\rint f$ such that $a'\betneq \{e,c\}\rint c'\rint f \betneq b'\rint b$  with one of $a'\rint e$, $e\rint f$, or $f\rint b'$ containing a prophecy. As $a'\rint e$ is disjoint from $c'\rint d'$, it is a No interval. The interval $e\rint c'\rint f$ is contained in $\halo[\delta']{c'}$ which is contained in the No interval $\halo{c'}$. Thus, it is a No interval as well due to Consistency. This implies that $f\rint b'$ must contain a prophecy and, as it is contained in $c\rint b$, $c\rint b$ contains that prophecy as well making the intersection of $a\rint b$ with $c\rint d$ a Yes interval.  
     \end{enumerate}
\end{proof}

\begin{corollary}
    If $a \xora b$, $b \xora c$ and $a\rint b\rint c$, then $b\rint b$ is a Yes interval.
\end{corollary}

\begin{proof}
    $b\rint b$ is the intersection of $a\rint b$ and $b\rint c$.
\end{proof}

A common trick is to look at small enough Yes intervals, or prophecies, such that they are all contained within a known interval. This is an immediate consequence of them all intersecting. 

\begin{proposition}\label{os-yescontain}
    Let $a\rint b$ be a Yes interval of a given oracle. Then any Yes interval of that oracle whose length is less than a given $\delta$ will be contained in $\halo{a\rint b}$.
\end{proposition}

\begin{proof}
    Given any Yes interval $c\rint d$ whose length is less than $\delta$, it is the case that $c\rint d$ intersects $a\rint b$ as they are both Yes intervals of the same oracle. Since the maximum distance an element of $c\rint d$ can be from another element in $c\rint d$ is less than $\delta$, the maximum distance an element can be from $a\rint b$ is less than $\delta$. Thus, $c\rint d$ is contained in $\halo{a\rint b}$.
\end{proof}

\section{Fonsis}\label{os-fonsis}

 A \textbf{Family of Overlapping, Notionally Shrinking Intervals} (fonsi, pronounced faan-zee,) is a set of rational intervals such that any pair of rational intervals in the set intersect and, given a rational $\varepsilon >0$, there exists at least one interval in the fonsi such that its length is less than $\varepsilon$. Sequences of nested intervals whose lengths approach 0 is an example of a fonsi. In constructivist works, such as \cite{bridger}, these objects are often taken as the definition of a real number and are likened to a set of measurements. 

Fonsis cover a range of examples. They can be thought of as the set of prophecies of an oracle without the procedure; they are just given somehow. 

\begin{proposition}
    Given a fonsi, there is an oracle associated with it such that all of the elements of the fonsi are precisely the prophecies of the oracle. 
\end{proposition}

\begin{proof}
    Let $a\rint b$ and rational $\delta>0$ be given. Choose an interval $c\rint d$ from the fonsi such that $|c\rint d| < \delta$. 

    The procedure $R$ will have it that if $c\rint d$ intersects $a\rint b$, then $R(a\rint b, \delta) = c\rint d$; otherwise it is equal to the empty set. Given the that length is less than $\delta$, the intersection with $a\rint b$ forces $c\rint d$ to be contained in $\halo{a\rint b}$.

    Whenever it is required to show that $R(a\rint b, \delta) \neq \emptyset$, it will be necessary to demonstrate that every element of the fonsi whose length is less than $\delta$ intersects $a\rint b$. 
    
    The intervals in the range of $R$ are precisely the elements of the fonsi. That the range is contained in the fonsi is clear from the definition. That every element of the fonsi is in the range follows by considering $R(c\rint d, |c\rint d|+1)$ for a given interval $c\rint d$ of the fonsi. Since $|c\rint d| < |c\rint d|+1$, $c\rint d$ itself matches the procedure constraint and is therefore in the range of $R$. 

    The properties are verified as follows:  
    \begin{enumerate}
        \item Range. Satisfied by definition. 
        \item Existence. The fonsi must be non-empty since given a length, it is required to produce an interval. Let $a\rint b$ be any element of the fonsi and $\delta =1$. Let $c\rint d$ be any element of the fonsi whose length is less than 1. By definition of a fonsi, $c\rint d$ and $a\rint b$ intersect. Thus, $R(a\rint b, 1) 
        \neq \emptyset$.
        \item Separation. 
        Let $a\rint b \in R$, that is, it is an element of the fonsi. Let $m$ and a subwidth $\delta$ be given such that $a\rint m\rint b$. Let $c\rint d$ be an element of the fonsi whose length is less than $\delta$. Since $a\rint b$ and $c\rint d$ are in the fonsi, they intersect. This implies $c\rint d$ is contained in $\halo{a\rint b}$. If $c\rint m\rint d$, then, because of the length being less than $\delta$, $c\rint d$ is wholly contained in $\halo{m}$. Thus, take $\halo{m}$ as the $e\rint f$ with $a\rint e\rint f\rint b$ and $e\rint f \in \mathbb{I}_{R}$ as required. 
        
        If $m$ is not in $c\rint d$, then, possibly via relabeling, $c\rint d$ will be contained within ${}_\delta |a\rint m$ with $c\rint d\rint m$ as a relabeling choice and noting $m \neq d$. Let $e$ be, say, the midpoint between $m$ and $d$ and $f$ be some number that is strictly contained in $\halo{m}$ and $m\rint b$. By the containment of $c\rint d$, $|a_\delta\rint e$ is in $\mathbb{I}_R$.
        
        \item Disjointness. Let $a\rint b \in R$ and $c\rint d$ be disjoint from $a\rint b$. Let $\delta$ be less than the distance from $a\rint b$ to $c\rint d$. Let $e\rint f$ be any element of the fonsi whose length is less than $\delta$. Since $e\rint f$ must intersect $a\rint b$, it cannot intersect $c\rint d$ given the lengths. Thus, $R(c\rint d, \delta) = \emptyset$. 
    
        \item Consistency. Let $a\rint b \in \mathbb{I}_R$ which implies there exists an element $c\rint d$ of the fonsi which is a subinterval of $a\rint b$. Let  $ \delta$ be given. The task is to show $R(a\rint b, \delta) \neq \emptyset$. Let $e\rint f$ be any element of the fonsi whose length is less than $\delta$. Since $e\rint f$ and $c\rint d$ intersect as they are both in the fonsi and as $c\rint d$ is contained in $a\rint b$, it is the case that $e\rint f$ intersects $a\rint b$. Therefore, the procedure returns $R(a\rint b, \delta) = e\rint f$ in this instance. As $e\rint f$ was arbitrary given the length, $R(a\rint b, \delta) \neq \emptyset$. 
        
        \item Closed. Assume that $a$ is given such that $\halo{a} \in \mathbb{I}_R$ for all $\delta$. Given $a\rint b$ and $\delta$, the task is to show $R(a\rint b, \delta) \neq \emptyset$. Let $c\rint d$ be an element of the fonsi whose length is less than $\delta$. The question is whether $c\rint d$ intersects $a\rint b$ or not. Suppose $a\rint b$ does not intersect $c\rint d$, then let $\delta'$  be less than the length from $c\rint d$ to $a\rint b$. This implies that $\halo[\delta']{a}$ is disjoint from $c\rint d$. Since $\halo[\delta']{a} \in \mathbb{I}_R$, there exists an interval $e\rint f$ in the fonsi which is contained in $\halo[\delta']{a}$. But being an element of the fonsi, it must intersect $c\rint d$. As this is not possible if $\halo[\delta']{a}$ is disjoint from $c\rint d$, it must be the case that $\halo[\delta']{a}$ and $c\rint d$ intersect. As this is a contradiction of the choice of $\delta'$ whose existence is based on $c\rint d$ being disjoint from $a\rint b$, it must be the case that $c\rint d$ intersects $a\rint b$. Thus, $R(a\rint b, \delta) \neq \emptyset$. As a side note, since this is true for all $b$ and elements $c\rint d$ of the fonsi, it must be the case that $a$ is contained in every element of the fonsi. 
    \end{enumerate}

\end{proof}

The above shows that a fonsi gives rise to an oracle whose prophecies is the fonsi. Given any oracle, its prophecies form a fonsi. It is also the case that the set of all Yes intervals of an oracle is a maximal fonsi. 


\begin{corollary}
    The prophecies of an oracle form a fonsi.
\end{corollary}

\begin{proof}
    The bisection algorithm establishes the notionally shrinking portion of the fonsi. That the intersection of the prophecies is nonempty was established in Proposition \ref{os-prointer}.
\end{proof}


\begin{corollary}
    The Yes intervals form a fonsi. 
\end{corollary}

\begin{proof}
    The containment of the prophecies handles the notionally shrinking. The intersection of the Yes intervals is the claim of Proposition \ref{os-yesinter}.
\end{proof}


\begin{corollary}
    Given an oracle such that every interval can be established to be either a Yes interval or a No interval, the complete set of all Yes intervals form a maximal fonsi.
\end{corollary}

\begin{proof}
A maximal fonsi is one which is not contained in any other fonsi. That is equivalent to the fonsi having the property that every interval not in the fonsi is disjoint from at least one element of the fonsi. If an interval is not a Yes interval, then by assumption this is a No interval. For an interval to be a No interval, it must be disjoint from a prophecy. Thus, no interval can be added to the fonsi. 
\end{proof}



\section{Oracles and Rational Betweenness Relations}

Each oracle is associated with a single rational betweenness relation, but a rational betweenness relation is associated with multiple oracles. Two oracles represent the same relation if all the elements of their respective ranges intersect. This section will establish these claims. 

\subsection{Oracles Give Rise to Rational Betweenness Relations}

The complete set of Yes intervals for an oracle is the rational betweenness relation associated with that oracle. The notation $a \xora b$ is used to indicate that $a\rint b$ is a Yes interval for the oracle $R_x$ while the notation $a \xrel b$ is used for the Yes intervals of the rational betweenness relation $x$. Complete means in this context that the complement of the set of Yes intervals is the set of No intervals; there is no interval that is considered unknown. Practically speaking, this cannot necessarily be achieved for the set of oracle Yes intervals. The assumption essentially requires being able to ascertain for every interval whether or not it intersects every prophecy of the oracle or not. 

In this, properties from the oracle may be prepended with the term Oracle for clarity while those of the rational betweenness relations may have RBR prepended. 

\begin{lemma}[Consistency]\label{os-con}
    Let an oracle $R$ be given representing $x$. Assume $c\rint a\rint b\rint d$. If $a \xora b$, then $c \xora d$. If \sout{$c \xora d$}, then \sout{$a \xora b$}. 
\end{lemma}

\begin{proof}
    If $a \xora b$, then this means that either $a\rint b \in \mathbb{I}_R$ or, potentially by relabeling, $R$ is rooted at $a$. In the first instance, there is a prophecy $e\rint f$ contained in $a\rint b$. This implies $c\rint d$ contains $e\rint f$ since $c\rint d$ contains every interval contained in $a\rint b$. Thus, $c\rint d \in \mathbb{I}_R$ and hence $c \xora d$. 
    
    If $R$ is rooted at $a$, there are two cases. In the first case, $a$ is strictly contained in $c\rint d$. Then there is a positive distance $L$ from $a$ to the closest endpoint of $c\rint d$. Let $\delta < L$. Then $\halo{a}$ is contained in $c\rint d$ and hence $c\rint d \in \mathbb{I}_R$. The other case is that $a$ is an endpoint of $c\rint d$, say, $ a= c$. Then $R(a\rint d, \delta) \neq \emptyset$ for all $\delta$ by the Closed property. Thus, $c \xora d$ in both cases.  

    If \sout{$c \xora d$}, then, by the definition of a No interval, there exists a prophecy $e\rint f$ which is disjoint from $c\rint d$. Since it is disjoint from $c\rint d$, it is also disjoint from any of its subintervals, such as $a\rint b$. Thus, \sout{$a \xora b$}.    
\end{proof}

The Separation property as postulated does not apply to all Yes intervals; this makes being an oracle easier to verify, but harder to use. Separation, however, can be extended to all Yes intervals though potentially non-constructively for rooted oracles that are not established as such.

\begin{lemma}[Yes Interval Separation]
    If $a\rint  b$ is a Yes interval for an oracle $R$, $m$ is a rational number strictly between $a$ and $b$, and a subwidth $\delta > 0$ is given, then there exists an $\halo{m}$ compatible interval $e\rint f$ such that one of the following holds true: 
    \begin{enumerate}
        \item $a \xora e$, \sout{$e\xora f$}, \sout{$f\xora b$};
        \item $e \xora f$, \sout{$a\xora e$}, \sout{$f\xora b$};
        \item $f \xora b$, \sout{$a\xora e$}, \sout{$e\xora f$}.
    \end{enumerate}
\end{lemma}

\begin{proof}
    Yes intervals are either in $\mathbb{I}_R$ or they are rooted. 
    
    If it is a rooted interval, then relabeling allows for the root to be $a$ which implies $\halo[\delta']{a} \in \mathbb{I}_R$ for all rational $\delta' >0$. With $a\rint e\rint b$, it is immediately the case that $a\rint e$ is an $a$-Rooted interval and hence is a Yes interval by the Closed property. Any $\halo{m}$ compatible interval $e\rint f$ will then have the property that $a \xora e$ while \sout{$e\xora f$} and \sout{$f\xora b$} because $e\rint f$ and $f\rint b$ are disjoint from the prophecies in $\halo{a}$ for $\delta < |a\rint e|$.
 
    The other case is that of $a\rint b \in \mathbb{I}_R$. This consists of two cases. Let $c\rint d$ be a prophecy contained in $a\rint b$. 
    
    If $m$ is not in $c\rint d$, then, by possibly relabeling, $a\rint m\rint c\rint d\rint b$ with $m$ strictly contained in $a\rint c$.  Let $e\rint f$ be an $\halo{m}$ compatible interval such that $a\rint e\rint m\rint f\rint c\rint d\rint b$ with $f \neq c$. Then $a\rint e$ and $e\rint f$ are disjoint from $c\rint d$ while $f\rint b$ contains $c\rint d$. Thus, $f \xora b$, \sout{$a\xora e$} and \sout{$e\xora f$}.

    If $m$ is in $c\rint d$, then apply the Separation property using  $\delta' < \delta$. If the produced $e'\rint f'$ is in $\mathbb{I}_R$, then, as it is in $\halo[\delta']{m}$ which is strictly contained in $\halo{m}$, there exists an $\halo{m}$ compatible interval $e\rint f$ that strictly contains $e'\rint f'$. Such an interval $e\rint f$ would then lead to both $a\rint e$ and $f\rint b$ being disjoint from $e'\rint f'$. Thus, $e \xora f$, \sout{$a\xora e$}, \sout{$f\xora b$}.

    If the produced $e'\rint f'$ is not in $\mathbb{I}_R$, then, by relabeling, it can be assumed that ${}_{\delta'} |c\rint e' \in \mathbb{I}_R$. Then $e$ can be chosen to be strictly contained in $e'\rint m$ so that \sout{$e\xora f$} and \sout{$f\xora b$} by Disjointness. If $a\rint \halo[\delta']{c}\rint e$, then $a \xora e$. If not, then a non-constructive approach is needed.

    If there is any Yes interval contained in $a\rint e$, then that is sufficient to conclude $a\rint e$ is a Yes interval. So assume there is not; this is a nonconstructive step. This means that while every Yes interval must intersect both $a\rint b$ and $a\rint e'$, none of them can be contained in $a\rint e$. Intervals whose lengths are less than the distance between $e'$ and $b$ imply that $a$ is the common point of intersection for all of these intervals. As established in Corollary \ref{os-root}, this implies $a$ is a root of the oracle. Thus, $a \xora e$ is a Yes interval. 

\end{proof}

\begin{lemma}[RBR Separation]\label{os-intsep}
  If $a \xora b$, then exactly one of the following holds:  
  \begin{enumerate}
        \item $a\xora m$, \sout{$m \xora b$};
        \item $m \xora m$; 
        \item $b \xora m$, \sout{$m \xora a$}.
    \end{enumerate}
\end{lemma}

\begin{proof}
Given any $\delta$, if $e\rint f$, as in the previous proof, is not the Yes interval, then Consistency and Disjointness leads to either the first or third outcome, depending on which one contains the Yes interval. 

If $e\rint f$ is the Yes interval for each $\delta$, then $\halo{m} \in \mathbb{I}_R$ for all $\delta$. This is a nonconstructive step. The Closed property yields that $m$ is the root of the oracle and thus, $m \xora m$.
\end{proof}


\begin{theorem}
    Given an oracle $R_x$ and assuming that for every interval $a\rint b$, either $a \xora b$ or \sout{$a \xora b$} can be established, then $\xora$ is a rational betweenness relation. 
\end{theorem}




\begin{proof}
    By assumption, $\xora$ is a relation on all pairs of rational numbers. It is symmetrical as the outputs of $R(a\rint b, \delta)$ are the same as the outputs of $R(b\rint a, \delta)$; this holds for all intervals $a\rint b$ and positive rationals $\delta$.

    The properties are established as follows:  
    \begin{enumerate}
        \item RBR Existence. By Oracle Existence, there exists $a\rint b$  and a $\delta$ such that $R(a\rint b, \delta) \neq \emptyset$. This implies there exists a prophecy $c\rint d$ such that $R(a\rint b, \delta) = c\rint d$. The interval $c\rint d \in \mathbb{I}_R$ which, by Oracle Consistency, implies $R(c\rint d, \delta) \neq \emptyset$ for all $\delta$.  
        \item RBR Separation. This is Lemma \ref{os-intsep}.
        \item RBR Consistency. This is Lemma \ref{os-con}.
        \item RBR Singular. Since $c \xora c$ is equivalent to saying that $c$ is a root of the oracle, Proposition \ref{os-singular}  states that $c$ is unique. Thus, if $d \xora d$, then $d$ must be $c$. 
        \item RBR Closed. Assume $c$ is such that $a \xora b$ implies $c \in a\rint b$. This is equivalent to saying that $c$ is in every Yes interval. By Corollary \ref{os-root}, $c$ is a root of the oracle. By the Oracle Closed property, $R(c\rint c, \delta) \neq \emptyset$ for all $\delta$. Thus, $c \xora c$.
    \end{enumerate}
\end{proof}

\subsection{Equivalent Oracles}

Having established that a given oracle does have an associated rational betweenness relation, at least nonconstructively, this allows an equivalence between oracles based on having the same rational betweenness relation. Specifically, two oracles are \textbf{equivalent}, denoted $R_x \equiv R_y$, exactly when $\xora$ and $\xora[y]$ are equal as completed relations. Since the (completed) rational betweenness relation is unique for a given oracle, a relation between oracles based on equality of the rational betweenness relation will automatically be reflexive, symmetric, and transitive. 

 \begin{proposition}
Let $R_x$ and $R_y$ be two oracles. All of the prophecies of $R_x$ will intersect all of the prophecies of $R_y$ if and only if $R_x \equiv R_y$.
\end{proposition}

\begin{proof}\label{os-equal}
    If $R_x \equiv R_y$, then they have the same Yes intervals, which will include both sets of prophecies. Since all Yes intervals of a given oracle intersect each other, the two sets of prophecies intersect as well. 
    
    The other direction is to assume that all of the prophecies of $R_x$ intersect all of the prophecies of $R_y$. The proof will proceed to establish that the Yes intervals of $R_x$ are Yes intervals of $R_y$ and that the No intervals of $R_x$ are No intervals of $R_y$. Switching $x$ and $y$ then gives both directions.  

    Assume $a \xora b$. By the non-constructive Proposition \ref{os-yescat}, either there exists a prophecy $c\rint d$ strictly contained in $a\rint b$ or one of the endpoints, say $a$, is a root of the oracle. In the first case, consider a prophecy $e\rint f$ of $R_y$ whose length is less than the distance from $c\rint d$ to $a\rint b$ (this is the smallest distance between the endpoints). By assumption, $e\rint f$ intersects $c\rint d$ and by choice of length, $e\rint f$ must be strictly contained in $a\rint b$. Thus, $a\rint b$ is a Yes interval of $R_y$.

    In the case that $a$ is the root of $R_x$, this implies that every prophecy of $R_x$ contains $a$. Assume that there is a prophecy $c\rint d$ of $R_y$ which does not contain $a$. Then by letting $\delta$ be less than the distance from $c\rint d$ to $a$, there exists, by $a$ being a root, a prophecy $e\rint f$ of $R_x$ in $\halo{a}$. It is disjoint from $c\rint d$ due to the choice of $\delta$. But it should intersect $c\rint d$ by the assumption. Thus, all of the prophecies of $R_y$ contain $a$. That means it is also rooted at $a$  and the Yes intervals of both $R_x$ and $R_y$ are exactly the intervals that contain $a$. 
    
    As for the No intervals, let $a\rint b$ be a No interval of $R_x$. By definition of No, there is a prophecy $c\rint d$ of $R_x$ such that $a\rint b$ and $c\rint d$ are disjoint. Let $L$ be the distance between $a\rint b$ and $c\rint d$. Let $e\rint f$ be a prophecy of $R_y$ whose length is less than $L/2$. This must intersect $c\rint d$ by assumption. Thus, all elements of $e\rint f$ are strictly contained in $\halo[L]{c\rint d}$. This means that $e\rint f$ is disjoint from $a\rint b$ and, hence, $a\rint b$ is a No interval of $R_y$ as was to be shown. 
\end{proof} 

Because of having to test whether infinitely many intervals intersect another set of infinitely many intervals, it may be impossible to establish that two oracles are equivalent. Instead, one can say that they are \textbf{$a\rint b$ compatible} for a given interval $a\rint b$ if $a\rint b$ is a Yes interval for both oracles. All oracles share Yes intervals as can be seen by intervalizing the union of two of their separate Yes intervals. For example, if $a\rint b$ is a Yes interval of $x$, $c\rint d$ is a Yes interval of $y$, and the betweenness relation of $a\rint \{b,c\}\rint d$ holds, then the two oracles are $a\rint d$ compatible.  


 Given a rational number $q$, the Singular Oracle at $q$, the Fuzzy Oracle at $q$, and the Reflexive Oracle at $q$ are all equivalent as oracles. This is easy to see by considering their prophecies. 
 
 The prophecies of the Fuzzy Oracle at $q$ are all of the form $q\rint q$. The prophecies of the Fuzzy Oracle at $q$ are of the form $\halo{q}$. The prophecies of the Reflexive Oracle at $q$ are of the form $a\rint b$ where $q \in a\rint b$. In all three oracles, every prophecy contains $q$. Thus, all the prophecies intersect and these are equivalent oracles. 

The associated rational betweenness relation is $a \xora[q] b$ if and only if $q \in a\rint b$. The above would constitute a proof of these claims if the procedures in question were, in fact, established to be oracles, which, while not difficult, was not done here. See \cite{taylor23main} for a more full treatment of these oracles. 

The Oracle of $q$ will refer to any oracle which is equivalent to the Singular Oracle at $q$ which will be the canonical oracle for $q$. 

A particular example to explore is $(\sqrt{2})^2$. While arithmetic will be defined later, what follows is a taste of it in this particular case. Let $R$ be the oracle of this number. It should be equal to the Oracle of 2. The arithmetic procedure leads to the following. Given $a\rint b$ and $\delta$, consider prophecies $c\lte d$ and $e\lte f$ of the $\sqrt{2}$ that satisfy $|ce\rint df| < \delta$ and $0\lte \{c,e\}$. Essentially, these will satisfy  $ce\rint 2\rint df$. Then $R(a\rint b, \delta) = ce\rint df$ if $ce\rint df$ intersects $a\rint b$ and is the empty set otherwise. After showing that this is an oracle, it is, by definition, trivial to observe that all of the prophecies contain 2. Thus, this is the Oracle of 2. 

One can also ponder comparing $(\sqrt{2})^2$ to $(\sqrt{1.9\ldots93})^2$ where the number of 9s is, say, $10^{23}$. And let us say that this is in the context of solving $f(x) = 0$ for some function $f$ where one can use Newton's method to find the root, but not be able to explicitly produce it. It would be impossible to distinguish these computationally. This is an issue for all versions of real numbers though an interval approach has the advantage that it is explicitly giving the range of compatible numbers. Indeed, one can at least say that they are $1.9999999999\rint 2.000000001$ compatible. The previous paragraph's argument fails to apply in that it is not clear that all the prophecies of this uncertain $x$ when multiplied together will contain $2$ or simply contain a number very close to $2$. 

There is also a slightly different approach that can be used to establish equivalence of oracles. It is essentially that they agree on the No intervals. 

\begin{proposition}
    Let $R_x$ and $R_y$ be oracles such that whenever there is a prophecy of $R_x$ that is disjoint from an interval $a\rint b$, then there exists a prophecy of $R_y$ that is also disjoint from $a\rint b$. Then $R_x$ and $R_y$ are equivalent as oracles. 
\end{proposition}

\begin{proof}
    It is necessary to show that all the prophecies intersect. Let $a\rint b$ be any prophecy of $R_x$ and $c\rint d$ be any prophecy of $R_y$. Either $a\rint b$ and $c\rint d$ intersect or they are disjoint. The unwelcome case is that of being disjoint. If they are disjoint, then, by the hypothesis, there is a prophecy of $R_y$ disjoint from $c\rint d$. But all prophecies of the oracle $R_y$ must intersect each other. Thus,  $a\rint b$ and $c\rint d$ cannot be disjoint. As these were arbitrary prophecies of their respective oracles, all of the prophecies must intersect. 
\end{proof}



\subsection{Rational Betweenness Relations Give Rise to the Reflexive Oracle}

The above has established that given an oracle, there is a rational betweenness relation associated with it. The other direction needs to be established as well. While there is not a unique oracle given a rational betweenness relation, there is a canonical one. 

Given a rational betweenness relation $\xrel$, the \textbf{Reflexive Oracle of $x$} is defined by the procedure $R(a\rint b, \delta)$ is $a\rint b$ if $a \xrel b$ and is the empty set otherwise. This procedure is single-valued and the result is independent of $\delta$. As all the prophecies of $R$ are the Yes intervals of the relation, it should be clear that the oracle does generate this rational betweenness relation. 

It does have to be shown that the procedure defined here is an oracle. One quick fact needs to be established first before proving the procedure is an oracle. 

\begin{proposition}
    Given a rational betweenness relation $\xrel$, it is the case that if $a \xrel b$ and $c \xrel d$, then $a\rint b$ and $c\rint d$ intersect. 
\end{proposition}

While this is basically a postulate for oracles, it is a deduction for rational betweenness relations. 

\begin{proof}
    The only case, after relabeling, of the two intervals being disjoint is $a\rint b\rint m\rint c\rint d$. It will be shown that $b = m = c$. This will then imply that the intervals intersect. 
    
    By RBR Consistency, $a \xrel d$. By RBR Separation using $m$, exactly one of the following holds true:  1) $a \xrel m$ with \sout{$m \xrel d$}, 2) $d \xrel m$ with \sout{$m \xrel a$}, or 3) $m \xrel m$. By RBR Consistency, since $c \xrel d$, it is the case that $m \xrel d$. This rules out case 1. Also since $a \xrel b$, it is also the case by RBR Consistency that $a \xrel m$. This rules out case 2. Thus, case 3 must hold which is $m \xrel m$. 
    
    Next use RBR Separation on $a \xrel m$ with $b$ the Separation point. Since $a \xrel b$ by assumption and $b \xrel m$ by RBR Consistency using $m \xrel m$, it must be the case that $b \xrel b$. But the RBR Singular property would then assert that $m = b$. The same argument applies for $c$. Therefore, $b = m = c$ in the one possible case where the intervals could have been disjoint. 
\end{proof}

\begin{proposition}
    Given the rational betweenness relation $\xrel$, the associated Reflexive procedure is an oracle. 
\end{proposition}

\begin{proof}

    The first step is to describe the set $\mathbb{I}_R$. If $a\rint b \in \mathbb{I}_R$, then there must be an $x$-related interval $c\rint d$ contained in $a\rint b$. But by RBR Consistency, this implies $a \xrel b$. Thus, $a\rint b \in \mathbb{I}_R$ if and only if $a \xrel b$ if and only if $R(a\rint b, \delta) = a\rint b$ for all $\delta$ if and only if $R(a\rint b, \delta) \neq \emptyset$ for all $\delta$. This implies the set $\mathbb{I}_R$ is the set of prophecies itself and these are the Yes intervals of the relation $\xrel$. 

    \begin{enumerate}
        \item Oracle Range. Given $a\rint b$ and $\delta$, $R(a\rint b, \delta)$ is either $\emptyset$ or $a\rint b$ which certainly intersects $a\rint b$ and is a subinterval  of $\halo{a\rint b}$.
        \item Oracle Existence. RBR existence yields an interval $a\rint b$ such that $a \xrel b$. Thus, $R(a\rint b, 1) = a\rint b$. As this is the only output for this input, $R(a\rint b, 1) \neq \emptyset$. 
        \item Oracle Separation. Let $a\rint b \in R$, i.e., $a \xrel b$, $m$ contained in $a\rint b$, and $\delta >0$ be given. If $m$ is an endpoint, then choose any $\halo{m}$ compatible interval $e\rint f$ such that one of its endpoints is contained in $a\rint b$. By potentially relabeling, assume $m =a$ and ${}_\delta |a\rint e\betneq a \betneq f\rint b\rint b|_\delta$. Apply RBR Separation to determine whether $a \xrel f$, $f \xrel f$, or $f \xrel b$. If it is the first, then $e\rint f$ is a Yes interval. If is the third, then $f\rint b|_\delta$ can be taken as the Yes interval. If $f \xrel f$, then choose $f'$ to be the average of $a$ and $f$ leading to $f'\rint b|_\delta$ being the Yes interval with $e\rint f'$ becoming the $\halo{m}$ compatible interval for the property. 

        Assume $m$ is not an endpoint. Then $m$ is strictly contained in $a\rint b$ and RBR Separation applies to give three outcomes. If $m \xrel m$, then choose any $\halo{m}$ compatible interval $e\rint f$  contained in $a\rint b$ and that is the required Yes interval. If not, then either $a \xrel m$ or $m \xrel b$. By relabeling, assume $a \xrel m$ with \sout{$m \xrel b$}. Pick a $\halo{m}$ compatible interval $e\rint f$ such that $a\rint e\rint m\rint f\rint b$. As $a \xrel m$, RBR Separation applies using $e$ as the separation point. If $a \xrel e$, then ${}_\delta | a\rint e$ is the required Yes interval. Otherwise, $e\rint f$ can be taken as the Yes interval based on the other two possible outcomes of the Separation property. 

        \item Oracle Disjointness. Assume $a\rint b \in \mathbb{I}_R$, i.e, $a \xrel b$, and that $c\rint d$ is disjoint from it. Since all Yes intervals of the relation intersect as shown above, $c\rint d$ is a No interval. By definition of the procedure, $R(c\rint d, \delta) = \emptyset$ for all $\delta$. 
        \item Oracle Consistency. If $a\rint b \in \mathbb{I}_R$, then by definition of the procedure, $R(a\rint b, \delta) = a\rint b$ for all $\delta >0$ and it is never equal to the empty set as was to be shown. 
        \item Oracle Closed. Assume $\halo{c} \in \mathbb{I}_R$ for all $\delta$ which implies $\halo{c}$ is an $x$-related interval of the relation. Let $a \xrel b$. If $c$ is an endpoint, then $c \in a\rint b$. Otherwise, let $L$ be the distance from $c$ to the closest endpoint. Consider $\halo{c}$ for $\delta < L$. It cannot contain either of the endpoints. Thus, it is either contained in $a\rint b$ or disjoint from it. But since $\halo{c}$ is an $x$-related interval, it must intersect the $x$-related interval $a\rint b$. Therefore, it is contained in it. The conclusion is that $c$ is in every $x$-related interval implying by RBR Closed that $c \xrel c$. This in turn implies by RBR Consistency that every interval that contains $c$ is a an RBR Yes interval and hence a Yes interval for the oracle. In particular, $R(c\rint b, \delta) \neq \emptyset$ for all $b$ and $\delta$. 
    \end{enumerate}
\end{proof}


It has now been established that every rational betweenness relation arises from at least one oracle and every oracle gives rise to exactly one rational betweenness relation. 




\section{Establishing the Real Number Field}

In this section, the task is to establish that the rational betweenness relations with the appropriate inequality relations and field operations are a model for the complete real number field. The approach is to define these relations and operations on oracles and establish that they respect the equivalence relation on the oracles. While in a theoretical sense it seems as if this filters the operations from the RBR to the oracles and then back to the RBRs, in practice, it is oracles that are generally at hand to use. 

As a note, equality of RBR is as relations and equality of oracles is via the equivalence relation as already established. In this section, the oracle $R_x$ will often be implicitly linked to the associated relation $x$, with notation switching as the need arises. Thus, $x = y$ or $R_x \equiv R_y$ may be used to convey equality. Both $a \xora b$ and $a \xrel b$ may be used interchangeably and will alter depending on whether oracles or relations are being emphasized. 

\subsection{Inequality}

Two oracles $R_x$ and $R_y$ are inequivalent if there are disjoint intervals $a\rint b$ and $c\rint d$ such that $a \xora b$ and $c \xora[y] d$. Disointness then implies \sout{$a \xora[y] b$} and \sout{$c \xora d$}. It is clear that the oracles are not equivalent as they do not have the same Yes/No intervals. These disjoint intervals then allow for a comparison of oracles. 

On the interval level, $a\rint b < c\rint d$ means that for every $p \in a\rint b$ and every $q \in c\rint d$, it is the case that $p < q$. Necessarily, the intervals must be disjoint to have this be possible and, for any two disjoint intervals, they will be related by one inequality. 

We define the inequality relations as: 
\begin{enumerate}
    \item $R_x < R_y$ and $x < y$, if there exists $a \xora b$ and $c \xora[y] d$ such that $a\rint b<c\rint d$.
    \item $R_x > R_y$ and $x > y$, if there exists $a \xora b$ and $c \xora[y] d$ such that $a\rint b > c\rint d$.
\end{enumerate}

If the intervals $a\rint b$ and $c\rint d$ were not disjoint, then oracle equivalence is possible. If it can be shown that all Yes intervals of $x$ are not greater than any of the Yes intervals of $y$, then that is noted as $x \leq y$. This would translate to the interval level as given any $x$-Yes interval $a \lte b$ and any $y$-Yes interval $c \lte d$ that it must be the case that $a \leq d$. 

The oracle inequality operation satisfies transitivity, relying on the transitivity of inequality for rational intervals which in turn relies on the transitivity of rational numbers. 

\begin{proposition}[Transitivity]
    If $x <y$ and $y<z$, then $x < z$.
\end{proposition}

\begin{proof}
    Let $a \xora b$, $c \xora[y] d$, $e \xora[y] f$, and $g \xora[z] h$ such that $a\rint b < c\rint d$ and $e\rint f < g\rint h$. The goal is to show that $a\rint b < g\rint h$. Let $r$ be in the intersection of $c\rint d$ and $e\rint f$. This exists as $c\rint d$ and $e\rint f$ are Yes intervals of the same oracle. Let $p \in a\rint b$ and $q \in g\rint h$. As $r \in c\rint d$, $p < r$; as $r \in e\rint f$, $r < q$. By the transitivity of the inequality of rational numbers, $p < q$. As this holds for all $p \in a\rint b$ and $q \in g\rint h$, it is the case that $a\rint b <g\rint h$.
\end{proof}

It does need to be shown that there can be no contradiction. 

\begin{proposition}
    If $x < y$, then it is not true that $x > y$ and it is also not true that $x = y$.
\end{proposition}

\begin{proof}
    The issue is that the comparison is based on two particular Yes intervals. It needs to be shown that two other Yes intervals would not contradict this statement. 

    Let $a\lte b < c\lte d$ be given that exemplifies $x<y$. Then $a\rint b$ is disjoint from $c\rint d$. This implies that $a\rint b$ is a No interval for $y$ while $c\rint d$ is a No interval for $x$. Thus, the two oracles are not equivalent as their rational betweenness relations differ.

    The other task is to show there are no Yes intervals that yield the opposite inequality. Let $p\lte q$ be a Yes interval for $x$ and $r \lte s$ be a Yes interval for $y$. If $p\rint q$ and $r\rint s$ overlap, then there is no contradictory information. 

    Assume, therefore, that they are disjoint. There are exactly two configurations for disjoint intervals:  $p \lte q < r \lte s$ or $ r \lte s < p \lte q$. Since $p\rint q$ must intersect $a\rint b$ due to both being Yes intervals of the same oracle, let $m$ be in their intersection. Similarly, let $n$ be in the intersection of $r\rint s$ and $c\rint d$. Since $a\rint b < c\rint d$, it is the case that $m < n$. This is only compatible with $p \lte q < r \lte s$ which was to be shown. 

\end{proof}



The classical story is that the real numbers satisfy the Trichotomy property:  Given $x$ and $y$, exactly one of the following holds:  $x<y$, $x>y$, or $x=y$. This holds non-constructively for the oracles. If there exists two disjoint Yes intervals, one for $x$ and one for $y$, then one of the inequality holds as was just explored. The other case is that every Yes interval of $x$ intersects every Yes interval of $y$. Proposition \ref{os-equal} establishes that $x=y$ in that situation. 
 

This was non-constructive as, generically, it requires potentially checking infinitely many intervals. 

\begin{corollary}
    Let $x$ and $y$ be two oracles. Then, nonconstructively, exactly one of the following holds true:  $x<y$, $x>y$, or $x=y$.
\end{corollary}

The constructivists use a property called $\varepsilon$-Trichotomy. This allows a definite determination with a finite, predictable amount of work. 

\begin{proposition}[$\varepsilon$-Trichotomy]
    Given oracles $x$ and $y$ and a positive rational $\varepsilon$, exactly one of the following holds:  $x<y$, $x>y$, or there exists an interval $a\rint b$ of length no more than $\varepsilon$ such that $a\rint b$ is a Yes interval for both $x$ and $y$.
\end{proposition}

\begin{proof}
    By the Bisection algorithm, there exists an $x$-Yes interval $c\rint d$ and a $y$-Yes interval $e\rint f$ such that both intervals have length less than $\varepsilon/2$. If $c\rint d$ and $e\rint f$ are disjoint, then the oracles are unequal with the inequality being that of the intervals. If $c\rint d$ and $e\rint f$ overlap, then their union is a Yes interval for both $x$ and $y$. That interval has length less than $\varepsilon$.
\end{proof}

It is also the case that given $x < y$, there exists a length such that all Yes intervals of $x$ and $y$ of that length are disjoint. 

\begin{proposition}
    If $ x< y$, there exists $\delta$ such that if $a \xora b$, $c \xora[y] d$, $|a\rint b| < \delta$ and $|c\rint d| < \delta$, then $a\rint b < c\rint d$
\end{proposition}

\begin{proof}
    Assume $e \xora f$ and $g \xora[y] h$ are such that $e\lte f < g \lte h$. Let $L = g-f$. Then $\delta = L/2$ will satisfy the requirements of the statement. 

    Let $a \xora b$ with $|a\rint b| < \delta$ and $c \xora d$ with $|c\rint d| < \delta$. By Proposition \ref{os-yescontain}, $a\rint b$ is contained in $\halo{e\rint f}$ and $c\rint d$ is contained in $\halo{g\rint h}$. By choice of $\delta$, $\halo{e\rint f}$ and $\halo{g\rint h}$ are disjoint and thus obey the same inequality as $e\rint f$ and $g\rint h$. Their subintervals are therefore disjoint and related in the same fashion. Thus, $a\rint b < c\rint d$. 
\end{proof}

In this section, the focus has been on the Yes intervals. It is also the case that if two oracles have Yes intervals that are disjoint, then there are prophecies of each that are disjoint from each other.

\begin{corollary}
    Given oracles $R_x$ and $R_y$ such that $a \xora b$ and $c \xora[y] d$ with $a\rint b$ disjoint from $c\rint d$, then there exists prophecies of $R_x$ and $R_y$ that are disjoint. 
\end{corollary}

\begin{proof}
    By the disjointness of the intervals, one of them is less than the other. By potentially relabeling, take $a\rint b < c\rint d$ implying $x < y$. The Bisection Algorithm gives arbitrarily small prophecies and the previous proposition states that for small enough Yes intervals, they satisfy the same inequality and, therefore, are disjoint. 
\end{proof}

Thus, all the computations about inequalities can be done solely with the prophecies which may be a more concrete set of intervals to inspect. For example, this immediately establishes that if $q < r$, then the Oracle of $q$ is less than the Oracle of $r$ using the fact that the Singular Oracle version of these oracles have the sole prophecies of $q\rint q$ and $r\rint r$, respectively. 

It is also important to establish that equivalent oracles have the same relation with non-equivalent oracles. 

\begin{proposition}
    Let $R_x \equiv R'_{x}$, $R_y \equiv R'_{y}$, and $R_x < R_y$. Then $R'_{x} < R'_{y}$.
\end{proposition}

\begin{proof}
    Let $a \xora b$ and $c \xora[y] d$ such that $a\rint b < c\rint d$; this exists by assumption of the inequality of $R_x$ and $R_y$. It can also be assumed these are prophecies by the above corollary. Let $\delta$ be less than half the distance between $a\rint b$ and $c\rint d$. By the Bisection Algorithm, let $a'\rint b'$ be a prophecy of $R'_x$ whose length is less than $\delta$ and let $c'\rint d'$ be a prophecy of $R'_y$ whose length is less than $\delta$. Since all prophecies of equivalent oracles intersect, it is the case that $a'\rint b'$ intersects $a\rint b$ and $c'\rint d'$ intersects $c\rint d$. Thus, by the length of the intervals, $a'\rint b' \subset \halo{a\rint b}$ and $c'\rint d' \subset \halo{c\rint d}$. By choice of $\delta$, the halos are disjoint and satisfy $\halo{a\rint b} < \halo{c\rint d}$. Thus, $a'\rint b' < c'\rint d'$ and 
    $R'_{x} < R'_{y}$.
\end{proof}

With the oracle inequalities explored, the rational betweenness relations can be ordered in the following way. The relation $\xrel$ is less than $\xrel[y]$ if an oracle representative $R_x$ of $\xrel$ is less than an oracle representative $R_y$ of $\xrel[y]$. Due to the equivalence proposition, different representatives will lead to the same conclusion. By considering the Reflexive oracle of the relation, it is clear that this inequality is the same one as obtained by saying that $\xrel < \xrel[y]$ if and only if there exists $a \xrel b$ and $c \xrel[y] d$ such that $a\rint b < c\rint d$. 

Letting $x$, $y$, and $z$ represent relations, the statement $x\rint y\rint z$ will mean that either $x \leq y \leq z$ or $z \leq y \leq x$. That is, $y$ is between $x$ and $z$. This notation can be extended as was done with rational numbers. 

\begin{proposition}
    If $a \xrel b$, then $\xrel[a] \rint  \xrel \rint  \xrel[b]$.
\end{proposition}

That is, if $a\rint b$ is a $x$-Yes interval, then $x$ is between $a$ and $b$.

\begin{proof}
    By the Bisection Algorithm, given a length $\delta$, there exists $c \xrel d$ such that $|c\rint d|<\delta$. Choosing $\delta$ to be less than the length of $a\rint b$, at least one of the endpoints is excluded; by relabeling, this can be chosen so that $b$ is excluded. This establishes that $\{a, x\} \rint  b$. If $c\rint d$ excludes $a$ as well, then $a \rint  c\rint d\rint b$ and then either $a\rint a < c\rint d < b\rint b$ or $b\rint b < c\rint d < a\rint a$. This establishes that as relations, $a \rint  x \rint  b$. 

    The other possibility is that there is no $\delta$ such that there is an $x$-interval excluding $a$. This implies that $\halo{a} \in \mathbb{I}_R$ for all $\delta$. This implies that $a$ is the root of the oracle of $x$ and thus $x = a$ as relations which implies $a\rint x\rint b$. 
\end{proof}

Notationally, if $a \lte b$ is an $x$-Yes interval, then $a \leq x \leq b$ is understood as being the true statement $\xrel[a] \leq \xrel \leq \xrel[b]$. 


\subsection{Completeness}

Completeness is a defining feature of real numbers. It can come in a variety of guises as wonderfully detailed by James Propp in \cite{propp}. While any of the equivalent versions could be used, this paper will go with the choice Propp suggests as a good foundation:  the Cut property. It is a simplified and symmetrized version of the least upper bound property. 

For this section, the letters $x$ and $y$ will represent relations. That is, $x$ is used instead of $\xrel$. For rational numbers, the symbols such as $q$ will also represent the relation version $\xrel[q]$. Thus, rational $q \in A$ means the relation $\xrel[q]$ is an element of the set of relations $A$. 

\begin{theorem}[The Cut Property] 
Let $A$ and $B$ be two disjoint, nonempty sets of rational betweenness relations such that $A \cup B$ is the whole set of such relations. In addition, for all $x \in A$ and for all $y \in B$, it is the case that $x < y$. Then there exists a rational betweenness relation $\kappa$ such that whenever $x < \kappa < y$, it will be the case that $ x \in A$ and $y \in B$.
\end{theorem}

Note that if $a \leq b$ and $a-\delta \in B$, then $\halo{a\rint b}$ is entirely contained in $B$. Similarly, if $b + \delta \in A$, then $\halo{a\rint b}$ is entirely contained in $A$. If neither of these hold, then $a-\delta \in A$, $b+\delta \in B$, and $\halo{a\rint b}$ has elements of both $A$ and $B$ in it. It is also useful to note that if $a \in A$, then $a-\delta \in A$ and if $b \in B$, then $b+\delta \in B$ for all lengths $\delta$.

\begin{proof}
This property can be established for the rational betweenness relations by using a single-valued procedure which returns intervals that intersect both $A$ and $B$. In particular, given $a \lte b$ and a length $\delta$, if $a-\delta \in A$ and $b+\delta \in B$, then $R(a\rint b, \delta) = \halo{a\rint b}$. Otherwise, the empty set is returned. The expansion beyond $a\rint b$ facilitates dealing with rational cut points as detailed by the Closed property. 

To establish this is an oracle, the properties follow by:  
\begin{enumerate}
    \item Range. By definition, a prophecy will be the interval $\halo{a\rint b}$ which contains $a\rint b$. 
    
    \item Existence. Let $a \in A$ and $b \in B$. These exist by assumption. Thus, $R(a\rint b, \delta)= \halo{a\rint b}$  and it is not the empty set.
    
    \item Separation. Let a prophecy $a \lte b$ be given. This implies $a \in A$, $b \in B$, and, due to the disjointness of $A$ and $B$, $a \neq b$. Let $m$ be given in $a\rint b$ and a length $\delta$ be given. 
    
    If $m$ is strictly contained in $a\rint b$, take $\delta'= \min(|m-a|, |m-b|, \delta)/3$ which implies $\halo[2\delta']{m}$ is contained in both $a\rint b$ and $\halo{m}$.  Let $c = m-\delta'$ and $d=m+\delta'$. If $c \in A$ and $d \in B$, then $R(c\rint d, \delta') = \halo[\delta']{c\rint d} = \halo[2 \delta']{m}$. As this is a prophecy, $e = m - 2 \delta'$ and $f = m + 2 \delta'$ satisfy the required role for the Separation property. If $ c\in B$, then $R(a\rint m-2\delta', \delta') = a-\delta'\rint m-\delta'$ is a prophecy implying ${}_\delta| a \rint  c$ contains a prophecy and $c\rint d$ works in the role of $e\rint f$ in the Separation property. If $d \in A$, then $R(m+2\delta'\rint b, \delta') = m+\delta'\rint b+\delta'$ is a prophecy implying $d\rint b|_{\delta}$ contains a prophecy and $c\rint d$ works in the role of $e\rint f$ in the Separation property. 

    The other case is if $m$ is an endpoint; let $n$ be the other endpoint. Let $L = |m-n|$ and $s = (n-m)/L$. Then take $c = m + s2\delta'/3$ where $\delta' = \min(L, \delta)/2$. This was chosen so that $c$ is strictly contained in $m\rint n$. If $c$ is in the same cut set as $m$ then so is $c -s \delta'/3 = m +s \delta'/3$ as it is closer to $m$ than $c$ is. Also $n + s \delta'/3$ will be in the same cut set as $n$ since it is on the other side of $n$ from $m$. Thus, $R(c\rint n, \delta'/3) = c-s\delta'/3 \rint  n+s \delta'/3$ is a prophecy implying that $e=m-s\delta'/3$ and $f=c-s\delta'/3 = m + s \delta'/3$ will satisfy the Separation property requirement in which $f\rint n|_{\delta'}$ contains  the prophecy $c-s\delta'/3 \rint  n+s\delta'/3$. If $c$ is in the same set as $n$, then so is $c + s \delta'/3$ being closer to $n$. Also $m-s \delta'/3$ is in the same set as $m$ since it is on the other side of $m$ from $n$.  Thus, $R(m\rint c, \delta'/3) =  m-s\delta'/3\rint c+s\delta'/3$. Then $e = m-s \delta'/3$ and $f = c + s\delta'/3 = m + s\delta'$ satisfies the Separation property requirement as $e\rint f \subset \halo{m}$ and it contains a prophecy, namely, itself. 

    \item Disjointness. If $a\lte b$ is a prophecy, then $a \in A$ and $b \in B$. If $c\rint d$ is disjoint from $a\rint b$, then let $\delta$ be less than the distance from $a\rint b$ to $c\rint d$. If $a\rint b < c\rint d$, then $\halo{c\rint d}$ is entirely contained in $B$; if $a\rint b > c\rint d$, then $\halo{c\rint d}$ is entirely contained in $A$. In either case, the procedure returns the empty set. 
    
    \item Consistency. If $a \lte b \in \mathbb{I}_R$, then let $c\lte d$ be a prophecy in $a\rint b$. This means $c \in A$ and $d \in B$. Thus, $a-\delta < a \leq c < d \leq b < b + \delta$ implying $a-\delta \in A$ and $b+\delta \in B$. Thus, the procedure returns $\halo{a\rint b}$ and not the empty set. 
    
    \item Closed. Assume $\halo{a} \in \mathbb{I}_R$ for all $\delta$. Let $b$ and $\delta$ be given. The task is to show that $R(a\rint b, \delta) \neq \emptyset$. As $\halo{a} \in \mathbb{I}_R$, $a-\delta \in A$ and $a+\delta \in B$ as was shown in the Consistency property. If $b \leq a$, then $b-\delta \leq a- \delta$ and so $b-\delta \in A$. Thus, the procedure returns $\halo{a\rint b}$. If $b \geq a$, then $b+\delta \geq  a+\delta$ and so $b+\delta \in B$. Thus, the procedure returns $\halo{a\rint b}$. In all cases, the empty set is not returned and $a$ is a root of the oracle. 
\end{enumerate}

Having established that $\kappa$ is an oracle, it now has to be established that it satisfies being a cut point. Let $x < \kappa < y$. That statement implies the existence of the prophecies $a \xrel b$, $c \xrel[\kappa] d$, and $ e \xrel[y] f$ such that $a \lte b < c \lte d < e \lte f$. Being a prophecy of $\kappa$, $c \lte d$ has the property that $c \in A$ and $d \in B$. Since $c > b$, $b \in A$ and since $x \leq b$, $x \in A$. Similarly, $d < e$ implies $e \in B$ and $y \geq e$ implies $y \in B$.

\end{proof}

Having shown the Cut property, the rational betweenness relations satisfies all the equivalent completeness properties. 

The above assumes that membership in the sets $A$ and $B$ can always be determined. It is possible to handle a situation in which the boundary between $A$ and $B$ is fuzzier. The oracles are particularly helpful here as they come with a bit of fuzziness built in. The assumption on the sets is that given a fuzziness tolerance, any number can be either determined to be in one of the sets or is in an interval whose length is less than the tolerance whose endpoints are in the opposing sets from each other. 

In particular, the governing assumption on the fuzzy cut sets is that given $x$ and $\varepsilon >0$, then either $x \in A$, $x \in B$, or there exists $\delta< \varepsilon$ such that $x-\delta \in A$ and $x + \delta \in B$. The latter will be referred to as $\halo{x}$ straddling the cut. In that instance, the procedure $R(a \lte b, \delta)$ for the cut oracle is defined in the following way. By the nature of $A$ and $B$, either 1) $a - \delta \in A$, 2) $a - \delta \in B$, or 3) there exists $\delta' < \delta$ such that $\halo[\delta']{a-\delta}$ straddles the cut. In case 2 or 3, the procedure returns the empty set. Similarly, for $b$, the cases are 1) $b + \delta \in B$, 2) $b+\delta \in A$, or 3) there exists $\delta' < \delta$ such that $\halo[\delta']{b+\delta}$ straddles the cut. Again, if either case 2 or 3 holds, then the procedure returns the empty set. Otherwise, case 1 holds for both and the procedure returns $a-\delta\rint b+\delta$. The verification that this procedure is an oracle should be similar to the above. One key property of the prophecies is that it is still the case that if $a \lte b$ is a prophecy, then $a \in A$ and $b \in B$.


\subsection{Arithmetic}

For arithmetic, the operators of addition and multiplication need to be defined. Then it needs to be shown that the usual properties hold including the existence of additive and multiplicative identities and inverses, as appropriate. 

The idea of arithmetic with oracles is to apply the operations to the intervals. The ideal approach would be that the Yes intervals of $x+y$ would be the intervals that result from adding Yes intervals of $x$ to those of $y$. This almost works, but it fails with, say, $\sqrt{2} - \sqrt{2}$. This ought to be 0. All the Yes intervals of $\sqrt{2}$ added to those of $-\sqrt{2}$ do lead to intervals that contain 0. But there is no interval of the form $0\rint b$ that is a result of subtracting two Yes intervals of $\sqrt{2}$. This is where having the oracles becomes useful. The range of $R_{x+y}$ will consist of the result of combining the range of $R_x$ and $R_y$ with the arithmetic operator, but with that scenario, the omission of $0\rint b$ intervals is not an issue as they can be deduced in the step going from the range of $R$ to the Yes intervals. The Yes step is theoretically doable, but may not be always actionable. For example, if solving $f(x)=0$ with Newton's method to something that looks like $\sqrt{2}$ but cannot be proven as such, then $x - \sqrt{2}$ would be seen as compatible with 0, but could not be proven to be so via any finite computation. 

Interval arithmetic is largely that of doing the operation on the endpoints. For addition, $a \lte b \oplus c \lte d = (a+c) \lte (b+d)$. The length of the new interval is $|a\rint b \oplus c\rint d| = |(b+d) - (a + c)| = |b-a+ c-d| = |a\rint b|+|c\rint b|$.

For multiplication, $a\rint b \otimes c\rint d = \min(ac, ad, bc, bd)\lte \max(ac, ad, bc, bd)$. For $0 \lte a \lte b$ and $0 \lte c \lte d$, the interval multiplication becomes $ac \lte bd$. For multiplication, the length is a bit more complicated. Let $M$ be an absolute bound for $a\rint b$ and $c\rint d$; an absolute bound on an interval is a number such that for any $p$ in the interval, $|p| \leq M$. Then $|a\rint b \otimes c\rint d| \leq M(|a\rint b| + |c\rint d|)$. This follows by considering cases. The first case is that of all four endpoints being involved. By relabeling in this case,  $ac \lte bd$ can be taken to be the multiplication interval leading to $|bd - ac| = |bd - cb + cb - ac| = |b(d-c) + c (b-a)| \leq M (|a\rint b| + |c\rint d|)$. The second case is that of three of the endpoints being involved which implies one of them is repeated. This may happen, for example, if an interval contains 0. By relabeling, $ac \lte ad$ can be taken to be the multiplication interval leading to $|ad - ac| = |a(d-c)| < M (|c\rint d|) \leq M (|c\rint d| + |a\rint b|)$. The third case is that of only two endpoints being used. By relabeling, $ac\rint ac$ can be taken to be the interval with a length of 0 which is less than or equal to any non-negative quantity.  That last case is the scenario of multiplying two singletons. 

 Operating on subintervals of $a\rint b$ and $c\rint d$ yields a subinterval of the resulting interval operation on $a\rint b$ and $c\rint d$. This is because if $p \in a\rint b$ and $q \in c\rint d$, then $p \oplus q \in a\rint b \oplus c\rint d$ and $p \otimes q \in a\rint b \otimes c\rint d$. 

Most of the arithmetic properties hold for interval arithmetic, but the distributive property does not nor are there any additive and multiplicative inverses. The additive identity is $0\rint 0$ while the multiplicative identity is $1\rint 1$.

The negation operator is $\ominus a\rint b  = -a\rint -b$; this is not the additive inverse as $a\rint b \oplus (\ominus(a \rint b )) = (a-b)\rint (b-a)$ which does contain 0, but it is not $0\rint 0$ unless $a = b$. 

The reciprocity operator is $1 \oslash a\rint b = 1/a \rint  1/b$ though this only applies to intervals excluding 0. If 0 was included, with $a < 0$, the resulting set would be $-\infty\rint 1/a \cup 1/b \rint  \infty $ which is not an interval as used here. This is not the multiplicative inverse, but when multiplied with the original interval, it does contain 1. To see this, first consider the case of $0 \lt a \lt b$ which implies $0 \lt 1/b \lt 1/a$ Then $a\rint b \otimes (1 \oslash a\rint b) = a/b \lte b/a$; since $a \leq b$, $a /b \leq b/b =1 $ and $b/a \leq a/a = 1$. Thus, the interval contains 1. In the case of $a \lt b \lt 0$, it is the case that $|b| < |a|$. As $a/b = |a|/|b|$ and $b/a = |b|/|a|$, the interval $a\rint b \otimes (1 \oslash a\rint b) = |b| \lte |a| \otimes (1 \oslash |b|\rint |a|)$. Applying the first case, that interval includes 1. Neither case yields the singleton $1\rint 1$ unless $a=b$. 

The distributive property is replaced with $ a\rint b \otimes ( c\rint d \oplus e\rint f) \subset (a\rint b \otimes c\rint d) \oplus (a\rint b \otimes e\rint f)$. To see this, note that the left-side has boundaries chosen from $\{a(c+e), a(d+f), b(c+e), b(d+f)\}$ while the second has a boundary of the form $(\min(ac, ad, bc, bd) + \min(ae, af, be, bf) ) \lte (\max(ac, ad, bc, bd) + \max(ae, af, be, bf) )$. To demonstrate that they are indeed not equal for some examples, consider $2\rint 3 \otimes ( 4\rint 7 \oplus -6\rint -3) = 2\rint 3 \otimes -2\rint 4 = -6\rint 12$ compared to $(2\rint 3 \otimes 4\rint 7) \oplus (2\rint 3 \otimes -6\rint -3) = 8\rint 21 \oplus -18\rint -6 = -10\rint 15$. 

For more on interval analysis, see, for example, \cite{moore} or, in this context, \cite{taylor23main}. In \cite{moore}, there is also a discussion of the difference between $a\rint b \otimes a\rint b$ and a potential understanding of $(a\rint b)^2 = \{c^2 | c \in a\rint b\}$. The former will include all products which for intervals containing 0 will include negatives. The latter will only include nonnegative results. Oracle arithmetic will only use the former, but it ultimately does not matter as the intervals of main concern are arbitrarily small intervals which will either be away from 0 or one is squaring the oracle of 0 which comes out to 0 in any event.  

Having discussed interval arithmetic, oracle arithmetic can now be defined based on that. The essential extra ability is that of being able to use narrower intervals. Much of the discussion will be the same for both arithmetic and multiplication. To facilitate that, the symbol $\odot$ will be used to represent either $\oplus$ or $\otimes$ operating on intervals and $\cdot$ will then be used for the corresponding operator operating on individual numbers.

Let oracles $R_x$ and $R_y$ be given. The oracle $R_{x \cdot y}$ can be defined by the following. Given $a\rint b$ and $\delta$, take an interval $c\rint d = (e\rint f) \odot (g\rint h)$ such that $e\rint f$ is a prophecy of $R_x$, $g\rint h$ is a prophecy of $R_y$, and $|c\rint d| < \delta$. Then $R_{x \cdot y}(a\rint b, \delta) = c\rint d$ if $c\rint d$ intersects $a\rint b$ and is the empty set otherwise. 

The existence of an interval of the form $c\rint d$ follows from the computational bounds on the arithmetic operators as well as the Bisection Algorithm applied to $R_x$ and $R_y$. For addition, that choice is ensured by choosing the lengths of each to be less than $\delta/2$. For multiplication, let $M$ be a bound on given intervals of $R_x$ and $R_y$. Then choose intervals of $R_x$ and $R_y$ such that their lengths are less than $\delta/(2M)$. 

A variant of this is to enlarge this to use the Yes intervals of $x$ and $y$; the choice to use the prophecies is to keep the arithmetic constructive. It will be the case that using equivalent oracles leads to an equivalent oracle for the result; this will be established below. 

Rather than prove that the above procedure defines an oracle, the path here is to recognize that when the prophecies of two oracles are combined either under addition or multiplication, the newly formed set is a fonsi. The oracle and relation then follows from the fonsi. 

\begin{lemma}
    Let $a\rint b, a'\rint b' \in R_{x}$, $c\rint d, c'\rint d' \in R_y$, $I =  a\rint b \odot c\rint d$, $I' =  a'\rint b' \odot c'\rint d'$. Then $I$ and $I'$ intersect. Furthermore, if $|I'| < \delta$, then $I' \subset \halo{I}$.
\end{lemma}

\begin{proof}   
    As prophecies of the same oracle intersect, let $p$ be in the intersection of $a\rint b$ and $a'\rint b'$ and let $q$ be in the intersection of $c\rint d$ and $c'\rint d'$. Then $p \cdot q = r$ is contained in both $I$ and $I'$. 
    
    If the length of $|I'| < \delta$, the distance of any element in $I'$ is less than $\delta$ from $r$. As $r$ is in $I$, any element of $I'$ is within $\delta$ of an element of $I$. This means that $I' \subset \halo{I}$.
\end{proof}


\begin{proposition}
    Given oracles $R_x$, $R_y$, the set $\mathcal{A} = \{a\rint b \oplus c\rint d| a\rint b \in R_x, c\rint d \in R_y\}$ is a fonsi. 
\end{proposition}

\begin{proof}

    Let $\delta > 0$ be given. 

    Choose $\delta'$ and $\delta''$ such that $\delta' + \delta'' < \delta$, such as $\frac{\delta}{3}$ for both of them. 

    Choose intervals $a\rint b \in R_x$ and $c\rint d \in R_y$  such that  $|a\rint b| < \delta'$ and $|c\rint d| < \delta''$. This is allowed by the Bisection Algorithm. Let $e\rint f = a\rint b \oplus c\rint d$.
     
    Earlier, the interval length of a sum was computed:   $|a\rint b \oplus c\rint d| \leq |b-a| + |c-d| < \delta' + \delta'' < \delta$. Therefore, there are arbitrarily small summed prophecies in $\mathcal{A}$. By the lemma, any pair of summed prophecies in $\mathcal{A}$ intersect. Thus, it is a fonsi. 
    
\end{proof}


The associated oracle with this fonsi will be denoted as $R_{x + y}$; it has the same rule as described previously. The associated rational betweenness relation will simply be written as $x+y$. 

\begin{proposition}
    Given oracles $R_x$, $R_y$, the set $\mathcal{P} = \{a\rint b \otimes c\rint d| a\rint b \in R_x, c\rint d \in R_y\}$ is a fonsi. 
\end{proposition}

\begin{proof}
    That the intervals intersect was established in the lemma. 

    Let $\delta > 0$ be given. Let intervals $e'\rint f' \in R_x$ and $e''\rint f'' \in R_y$ be chosen as allowed by the existence property. Let $M = \max(|e'|, |f'|, |e''|, |f''|) +1$. 

    Choose $\delta'$ and $\delta''$ such that $M(\delta' + \delta'') < \delta$, such as $\frac{\delta}{3M}$ for both of them. 

    Choose intervals $a\rint b \in R_x$ and $c\rint d \in R_y$  such that  $|a\rint b| < \min(\delta', 1)$ and $|c\rint d| < \min(\delta'', 1)$. This is allowed by the Bisection Algorithm. As all prophecies of length no greater than $1$ are within the $1$-halo of a given prophecy, it is the case that $|a|, |b|, |c|, |d| < M$. 
     
    Earlier, the interval length of a product interval was bounded:   $|a\rint b \otimes c\rint d| \leq M (|b-a| + |c-d|) < M (\delta' + \delta'') < \delta$. Therefore, there are arbitrarily small product prophecies in $\mathcal{P}$. By the lemma, any pair of product prophecies in $\mathcal{P}$ intersect. Thus, it is a fonsi. 
    
\end{proof}

The associated oracle with this fonsi will be denoted as $R_{xy}$; it has the same rule as described previously. The associated rational betweenness relation will simply be written as $xy$. In the case that the lack of a symbol is awkward, the notation $x * y$ may be used as well. 

One also has to establish that different oracles for a real number lead to an equivalent oracle under these operations. 

\begin{proposition}
    Let $R_x \equiv R'_x$ and $R_y \equiv R'_y$, then $R_{x \cdot y} \equiv R'_{x \cdot y}$.
\end{proposition}

\begin{proof}
The task is to show that the prophecies of $R_{x \cdot y}$ intersect the prophecies of $R'_{x \cdot y}$. By the equivalence, all the prophecies of $R_x$ intersect all the prophecies of $R'_x$ and all the prophecies of $R_y$ intersect the prophecies of $R'_y$. Let $a\rint b \in R_{x \cdot y}$ which means there exists $c\rint d \in R_x$ and $e\rint f \in R_y$ such that $a\rint b = c\rint d \odot e\rint f$. Similarly, let $a'\rint b'\in R'_{x \cdot y}$ with $c'\rint d' \in R'_x$, $e'\rint f' \in R'_y$ such that $a'\rint b' = c'\rint d' \odot e'\rint f'$. Then by assumption of the equivalences, there exists $p$ that is in both $c\rint d$ and $c'\rint d'$ along with a $q$ in both $e\rint f$ and $e'\rint f'$. Then $p \cdot q$ will be in both $a\rint b$ and $a'\rint b'$. Thus, they intersect. This shows that all prophecies of $R_{x \cdot y}$ intersect all the prophecies of $R_{x' \cdot y'}$ which implies they are equivalent. 
\end{proof}

The equivalence implies that they represent the same completed rational betweenness relation.

Having established that these are oracles, it is necessary to check the field properties. The key fact to use is that if two oracles have prophecies that are always intersecting, let alone being equal, then they are equivalent oracles and their corresponding rational betweenness relations are identical. 

\begin{enumerate}
    \item Commutativity. To show $x\cdot y = y \cdot x$, the interval fact that $a\rint b \odot c\rint d = c\rint d \odot a\rint b$ establishes that the two have the same set of prophecies and hence are equivalent. The interval equality follows from the commutativity of $\cdot$  on the rationals. 
    \item Associativity. To show $(x\cdot y) \cdot z = x \cdot (y \cdot z)$, the interval fact to use is that $(a\rint b \odot c\rint d) \odot e\rint f = a\rint b \odot (c\rint d \odot e\rint f)$. This establishes that the two have the same set  of prophecies and hence are equivalent. The interval equality follows from the associativity of $\cdot$ on the rationals. 
    \item Identities. Let $R_0$ be the Singular Oracle of 0 and $R_1$ be the Singular Oracle of 1. The task is to show $x+0=x$ and $1x  = x$.  Since $a\rint b \oplus 0\rint 0 = a\rint b$, the prophecies of $R_{x+0}$ and $R_x$ are identical. As $a\rint b \otimes 1\rint 1 = a\rint b$, the prophecies of $R_{x*1}$ and $R_x$ are identical. This establishes the additive identity as 0 and the multiplicative identity as 1. 
    \item Distributive Property. As previously mentioned, $( a\rint b \otimes ( c\rint d \oplus e\rint f) \subset (a\rint b \otimes c\rint d) \oplus (a\rint b \otimes e\rint f)$. Thus, the prophecies of $R_{x(y+z)}$ are contained in the prophecies of $R_{xy + yz}$. Since that means they intersect, these are equivalent oracles and the Distributive Property holds. 
    \item Additive Inverse.  The additive inverse for $x$ can be given by the procedure $R_{-x}(a\rint b, \delta) = \ominus (c\rint d)$ where $R_x (\ominus (a\rint b), \delta) = c\rint d$  and is the empty set if $R_x$ is the empty set for those inputs. It can be shown that it is an oracle basically because the procedure returns the empty set under the same conditions as the original procedure.

    Alternatively, given the fonsi that is the set of prophecies of $R_x$, define the additive inverse fonsi as $\{-a\rint -b| a\rint b \in R_x\}$. This is a fonsi as $|-a\rint -b| = |a\rint b|$ and  $q \in a\rint b$ if and only if $-q \in -a\rint -b$ which implies that intersection of two intervals is non-empty if and only if the intersection of the two negated intervals is. 

    By either method, there is an oracle $R_{-x}$ whose prophecies are the negated versions of the prophecies of $R_x$. The corresponding rational betweenness relation will be denoted by $-x$.
    
    To verify that this is the additive inverse, compute $x + (-x)$ by looking at $a\rint b \oplus c\rint d$ where $a \lte b \in R_x$ and $c \lte d \in R_{-x}$. For $c\rint d \in R_{-x}$, it must be the case that $-c\rint -d \in R_x$. The intervals $-c \rint  -d$ and $a\rint b$ must intersect as they are both prophecies of $R_x$. Let $p$ be a point of that intersection which implies $a\rint p\rint b$ and $c\rint -p\rint d$. Then $a\rint b \oplus c\rint d$, which is $(a+c) \lte (b+d)$, has $(a+c) \rint  ( p + (-p) =0 ) \rint  (b+d)$ in its interval. Since $a\rint b$ and $c\rint d$ were random prophecies of their oracles, it must be true that all intervals in the range of the sum procedure must contain 0. This leads to the prophecies of $R_{x-x}$ all containing 0 which means 0 is the root of the oracle $R_{x-x}$ which in turn means $R{x-x} \equiv R_0$.  Thus, $-x$ is the additive inverse of $x$. 
    
    \item Multiplicative Inverse. For the multiplicative inverse of $x$, it is slightly more complicated. This only applies if $x \neq 0$. Being not 0, there exists at least one prophecy $u\rint v$ of $x$ that does not contain 0. Once one has that prophecy, then if $L$ is the distance to 0 from $u\rint v$, which would be $\min(|u|, |v|)$, any prophecy whose size is less than $L$ also excludes 0. 
    
    Define a fonsi to be $\mathcal{F} = \{1/a \rint  1/b | a\rint b \in R_x \wedge |a\rint b| < L \}$. To show this is a fonsi, let $a\lte b$ and $c \lte d$ be two intervals in $\mathcal{F}$. It is required to show that they intersect. Since $1/a\rint 1/b \in R_x$ and $1/c\rint 1/d \in R_x$, those reciprocated intervals intersect. Let $\{1/a, 1/c\} \rint  p \rint  \{1/b, 1/d\}$. Then $\{a, c\} \rint  1/p \rint  \{b, d\}$ as all of these are the same sign. Thus, $1/p$ is in both $a\rint b$ and $c\rint d$. 

    The other aspect of a fonsi is to have arbitrarily small intervals in it. Let $\delta > 0$ be given. Let $M = L/2$. By the Bisection Algorithm, let $a\rint b \in R_x$ be such that $|a\rint b| < \min(L/2, \delta/M^2)$. This implies $a\rint b \subset \halo[L/2]{u\rint v}$ implying that $|a|, |b| >  L/2 = M$. By the length of $a\rint b$, it does not contain 0, and thus its reciprocal is an interval in $\mathcal{F}$ and $ab > 0$. Then $|1/a\rint  1/b| = |b-a|/ab$. As $|a|, |b| > M$, it is the case that $1/(ab) < 1/M^2$. Thus, $|1/a\rint 1/b| = |a\rint b|/ab < M^2|a\rint b| < M^2 \delta/M^2 = \delta$ demonstrating that there are arbitrarily small interval in $\mathcal{F}$ and completing the proof that $\mathcal{F}$ is a fonsi. 
    
    The fonsi generates an oracle $R_{1/x}$ which generates the betweenness relation denoted as $1/x$. 

    To verify that this is the multiplicative inverse, compute $R_{x*1/x}$. As with the additive inverse, given $a\rint b \in R_x$ and $c\rint d \in R_{1/x}$, there exists $p$ that is common to both $a\rint b$ and $1 \oslash (c\rint d)$. Thus, the multiplication of these intervals leads to $p *1/p = 1$ being in the product interval and, as this is true for all pairings of prophecies of $R_{x}$ with $R_{1/x}$, it is the case that $R_{x*1/x} \equiv R_1$. Thus, $1/x$ is the multiplicative inverse of $x$; this exists for all $x \neq 0$. 

\end{enumerate}

Having established that the field properties hold for oracles, these properties transfer to the rational betweenness relations which establishes that they are a field. 


\subsection{The Real Numbers}

The rational betweenness relations are ordered, complete, and a field. It needs to be shown that it is an ordered field, that is, that the arithmetic operations respects the ordering. To do this, it is sufficient to show that 1) for all $x, y, z$ in the field, if $x<y$, then $x + z < y +z$; and 2) for all $x, y$ in the field, if $x >0$ and $y>0$, then $xy > 0 $.

By definition of the inequality, there exists intervals $ a \xrel[x] b$ and $c \xrel[y] d$ such that $a \lte b < c \lte d$. Thus, $c-b > 0$. By Bisection, there exists an interval $e \xrel[z] f$, $e \lte f$, such that $0 \leq f-e < c-b$. Then $a+e \xrel[x+z] b+f$ and $c+e \xrel[y+z] d+f$. Since $a+e \leq  b+f$ and $c+e \leq d+f$, it suffices to show that $b+f < c+e$. This is equivalent to showing $f-e < c-b$, but that is true by choice of $e\rint f$. Thus, $x+z < y+z$. As there was nothing special about $x$, $y$, or $z$, this holds for all rational betweenness relations. 

The multiplication is a bit shorter. Let $x >0 $, $y >0$, $a \xrel b$ such that $0<a \lte b$, $c \xrel[y] d$ such that $0 < c \lte d$. Then $ac \xrel[xy] bd$ and $0\rint 0 < ac \lte bd$. Hence, the rational betweenness relation $xy$ is greater than 0. 

The rationals are also dense in the rational betweenness relations. Let $x < y$ be two given relations. Let $a \lte b < c \lte d$ where $a \xrel b$ and $c \xrel[y] d$; this is allowed by the meaning of the inequality. Let $m$ be the average of $b$ and $c$. Then the betweenness version of $m$, say $\xrel[m]$, satisfies $x < \xrel[m] < y$ as evidenced by $a\rint b < m\rint m < c\rint d$.

Taken as a whole, all of the above translates into the following theorem.

\begin{theorem}
    The set of rational betweenness relations equipped with the arithmetic and inequality operators as defined above satisfy all the axioms of the real numbers. 
\end{theorem}


\section{Concluding Thoughts}

The idea was quite simple:  a real number is best represented by the intervals that contain it. To do this, there are three distinct structures. 

The initial structure is that of the rational betweenness relations. While Dedekind cuts emphasize, in some sense, being away from the real number, the relations suggest enclosing the real number. The idea is that if one knows which rational intervals contain the real number, then it is reasonable to say that one knows the real number. Unfortunately, this is not achievable for all real numbers by finite processes. 

What is achievable is the structure of oracles. They are computationally accessible. They are constructive in nature. Indeed, they are best thought of as a prescription as to what to compute and do rather than some given completed object. 

But given some attempt to represent a particular real number, many oracles are possible. This suggests trying to find a structure that oracles are approximating. This is the betweenness relations. These are pure. They represent the intervals that contain the real number. In order to compute out all the Yes intervals for a given oracle, it may be necessary to check infinitely many intervals. This can mean that to state the relations is to delve into nonconstructivist territories. 

Both versions have their trade offs. In terms of practical uses, the oracles are very implementable in, say, a computer program. The betweenness relations are very useful in theoretical uses. 

The third structure that emerges is that of fonsis. While algorithms generally fit the oracles better, the fonsi notion naturally arises in arithmetic. The fonsis are the prophecies of the oracles and they are a partial, but functional, version of the rational betweenness relations. 

A famous motivation for Dedekind was to prove that $\sqrt{2} \sqrt{3} = \sqrt{6}$. How might that go with the relations? The Yes intervals for $\sqrt{p}$ are those intervals $\{0, a\} \lte b$ that satisfy $\max(0,a)^2 \rint  p \rint  b^2$. Let $0\rint 0 < a \lte b$ be a $\sqrt{2}$-Yes interval and $0\rint 0 < c\lte d$ be a $\sqrt{3}$-Yes interval. Multiplication of them leads to $ac \lte bd$. Squaring the multiplication leads to $a^2 c^2 \lte b^2 d^2$. Since $a^2 \lte 2 \lte b^2$ and $c^2 \lte 3 \lte d^2$, multiplication leads to $a^2 c^2 \lte 6 \lte b^2 d^2$. Thus, $\sqrt{2} \sqrt{3} = \sqrt{6}$. This can obviously be generalized to any $n$-th roots. 

In general, using intervals is how arithmetic can be explored. The computation of $e + \pi$ can be explored with intervals. One would take Yes intervals of $e$ and Yes intervals of $\pi$ and then add them together. The intervals can be shrunk as small as one likes if one has the computational power to do it. There is nothing infinite in the computation up to a certain desired level of precision. Using fractions allows for accurate precision. 

There are other works relevant to this idea being worked on by the author. The paper \cite{taylor24dedekind} explores the betweenness relations and how they relate to Dedekind cuts. The paper \cite{taylor23metric} extends this idea to give a new completion of metric spaces. Basically, inclusive balls replace the inclusive rational intervals. The Separation property gets replaced by the property that given two points in a Yes ball, there exists a Yes ball that excludes at least one of them. There is also work, \cite{taylor23maudlin}, to extend this idea to the novel topological spaces of linear structures from \cite{maudlin}. 

Functions are also a topic of interest to extend oracles to. With the idea of a real number being based on rational intervals, it suggests that functions ought to respect that. Exploring the implications of that is the business of \cite{taylor23funora}. A comprehensive tome, \cite{taylor23main}, covers these various topics including comparing the other definitions of real numbers to this one and explaining how it suggests techniques such as using mediants to compute continued fractions. 

\medskip

\normalem %restoring normal emphasis in bibliography 

\printbibliography

\end{document}
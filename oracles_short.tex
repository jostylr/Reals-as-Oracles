\documentclass[12pt]{article}
\usepackage{personal}
\usepackage{realoracles}


\newtheorem{theorem}{Theorem}[section]
\newtheorem{lemma}{Lemma}[section]
\newtheorem{corollary}{Corollary}[section]
\newtheorem{proposition}{Proposition}[section]



\title{Defining Real Numbers By Inclusion in Rational Intervals}

\jtauthor
\date{\today}



%\sloppy%\openup-.1\jot
\begin{document}\maketitle
\begin{abstract}
Irrational numbers cannot be known in a precise way in the same fashion that rational numbers can be. But one can be precise about the uncertainty. The assertion is that if one knows whether or not a real number is in any given rational number, then one can claim to know the real number. This paper will define real numbers as rules that give definite, useful answers to whether a real number is in a given rational interval. It is proven that the collection of these rules satisfies the axioms of the real numbers. 
\end{abstract}

Real numbers are a fundamental part of mathematics. There are many different definitions, but they each have their own inadequacies. Some, such as decimals, Cauchy sequences, and series representation, emphasize computational aspects often at the costs of added complexity and non-uniqueness. Other approaches, such as Dedekind cuts and axiomatic definitions, offer almost no computational guidance. The definition in this paper is an attempt to offer a path towards computation while not requiring it for stating what the number is. As a guide in contemplating different notions, it is useful to consider how equality and arithmetic are handled, particularly how an irrational minus itself is computationally known to be 0. 

The first step towards this idea is to think of a real number $x$ as providing a betweenness relation on rational numbers by $a$ and $b$ being $x$-related if $x$ is, inclusively, between $a$ and $b$. This almost works. But there are many situations in which one cannot definitively conclude that for a given real number. The resolution of this is to have an underlying rule managing the information and then to have a layer on top of this. 

The rules under discussion have the property that they can provide precise, certain answers about a real number. The setup allows for the computation of intervals, but does not require computation in order to define the real number. The main downside is that multiple rules can represent the same real number, but the ambiguity is nonconstructively removed at the higher level presentation of the $x$-relation viewpoint.  

This paper will first establish some basic notations for working with rational intervals before defining the properties that these rules must satisfy to represent a real number. A brief section on some examples will be given which will be light on details. See \cite{taylor23main} for a more in-depth discussion of examples, uses, and comparisons to other definitions of real numbers. Equality is discussed before several sections are explored in fully establishing the real number properties. 

\section{Interval Notations}

This section is in common to the other papers on this subject by this author, such as \cite{taylor24dedekind}.

The set of all rationals $q$ such that $q$ is between $a$ and $b$, including the possibility that $q=a$ or $q=b$, is a \textbf{rational interval} denoted by $a:b$. If $a=b$, then this is a \textbf{rational singleton} denoted by $a:a$. A singleton is a set of exactly one rational number, namely, $a$. To indicate $a \leq b$, the notation $a \lte b$ can be used. If $a < b$ and the interval is being presented, then the compacted notation $a \lt b$ may be used. 

The notation will also be used to indicate betweenness. If $a \leq b \leq c$ or $c \leq b \leq a$, then $a:b:c$ will be used to denote that. By definition, $c:b:a$ and $a:b:c$ represent the same betweenness assertion. This can be extended to any number of betweenness relations, such as $a:b:c:d$ implying either $a \leq b \leq c \leq d$ or $d \leq c \leq b \leq a$. There are also some trivial ways to extend given betweenness chains. For example, if $a:b:c$, then $a:a:b:c$ holds as well. Another example is that if $a:b:c$ and $b:c:d$, then $a:b:c:d$ holds true. This all follows from standard inequality rules for rational numbers. 

If $b$ and $c$ are between $a$ and $d$, but it is not clear whether $b$ is between $a$ and $c$ or between $c$ and $d$, then the notation $a:\{b,c\}:d$ can be used. This can also be extended to have, for example, $a:b:\{c,d\}$ which would be shorthand for saying that both $a:b:c$ and $a:b:d$ hold true. In addition, the notation \sout{$a:b:c$} will be used to indicate that $b$ is not between $a$ and $c$. One could also indicate this by $b:\{a,c\}$ which leads to observing that $a$ and $c$ could be said to be on the same side of $b$. 

Rational numbers satisfy the fact that, given three distinct rational numbers, $a, b, c$, exactly one of the following holds true: $a:b:c$, $a:c:b$, or $b:a:c$. That is, one of them is between the other two. It can be written in notation as $a:b:c$ holds true if and only if both \sout{$a:c:b$} and \sout{$b:a:c$} hold true. This follows from the pairwise ordering of each of them as provided by the Trichotomy property for rational numbers along with the transitive property. 

In this paper, often a potential relabeling will be invoked. This is to indicate that there are certain assumptions that are needed to be made, but they are notational assumptions and, in fact, some arrangement of the labels of that kind must hold. For example, if $\{a,d\}:b:c$ holds true, then either $a:d:b:c$ or $d:a:b:c$ holds true. If these are generic elements, then relabeling could be used to have $a:d:b:c$  be  true for definiteness, avoiding breaking the argument into separate, identical cases. If $a$ and $d$ were distinguished in some other way, such as being produced by different processes, then relabeling would not be appropriate to use. 

If $a:b:c:d$, then the union of $a:c$ with $b:d$ is the interval $a:d$. If $b \neq c$, then the union of $a:b$ and $c:d$ as a set is not an interval. One can still consider the intervalized union of $a:d$ as the shortest interval that contains both intervals. 

An \textbf{$a$-rooted} interval is an interval who has an endpoint that is $a$, that is, they are of the form $a:b$ for some $b$. An  \textbf{$a$-neighborly} interval is an interval that strictly contains $a$. The set of all $a$-rooted intervals will be denoted $\mathbb{I}_a$ while the set of all $a$-neighborly intervals will be denoted by $\mathbb{I}_{(a)}$

The notation $a_\delta$ represents the $\delta$-halo of $a$ which means $a -\delta : a+ \delta$. An interval is $a_\delta$ compatible if it is strictly contained in $a_\delta$ and is an $a$-neighborly interval. That is, the interval $c\lte d$ is $a_\delta$ compatible if $a- \delta < c < a < d < a+ \delta$. The halo can also extend to other intervals. The notation $(a:b)_\delta$ will refer to $a_\delta \cup a:b \cup b_\delta$. If $a \leq b$, this is the same as the interval $a-\delta:b+\delta$. 

The notation $b : |a_\delta$ will indicate the interval that goes from $b$ to the closest endpoint of $a_\delta$ to $b$ while the notation $b:a_{\delta}|$ will indicate that the interval goes from $b$ to the farthest endpoint of $a_\delta$ from $b$. If, for example, $b < a-\delta < a$, then $b:|a_\delta$ is the same as $b:a-\delta$ while $b:a_{\delta}|$ is the same as $b:a+\delta$. Also, $|a_\delta : b$ is the same as $b:a_\delta |$. This notation makes the most sense for $b$ outside of $a_\delta$. 

Given $m$ in $a:b$, a \textbf{subwidth} $\delta$ shall mean a positive rational number such that $m_\delta$ is strictly contained in $a:b$.

The term subinterval of $a:b$ will include $a:b$ as a subinterval but it does not include the empty set as a subinterval. 

The set $\mathbb{I}$ will represent the set of all rational intervals. 

The length of $a:b$ will be denoted by $|a:b|$ and is equal to $|b-a|$.

Throughout the paper, unless noted otherwise, the letters $a$ through $w$ will represent rational numbers, $x, y, z, \alpha, \beta$ will represent real numbers, and $\delta, \varepsilon$ will represent positive rational numbers. Primes on symbols will be assumed to be of the same type. 

\section{Definition of a Real Number}

An \textbf{oracle} is defined to be a rule $R$ which should be able to handle any input of the form of a rational interval along with a positive rational number. The output, which is not necessarily exclusively defined by the input, should be either a rational interval or the empty set. To make it into a function, one could view it as $R \mathbin{\col} \mathbb{I} \times \mathbb{Q}^+ \to \mathcal{P}(\mathbb{I} \cup \emptyset)$. This would suggest, however, computing the full output of $R$ for a given input which is not something needed or desired. Notationally, $R(a:b, \delta) = c:d$ means that one of the outputs for that pair of inputs is $c:d$; $c:d$ can then be said to be in the range of $R$, denoted as $c:d \in R$. The notation $\mathbb{I}_R$ will be all of the intervals that have a subinterval in the range of $R$.

To be an oracle, the rule must satisfy six properties.  The properties are:
\begin{enumerate}
    \item Range. $R(a:b, \delta)$ should either be $\emptyset$ or a subinterval of $(a:b)_\delta$. 
    \item Existence. There exists $a:b$ and $\delta$ such that $R(a:b, \delta) \neq \emptyset$.
    \item Separation. 
    If $a:b \in R$, then for a given $m$ in $a:b$ and given a subwidth $\delta$, there exists an $m_\delta$ compatible interval $e:f$ such that one of the following holds true:  $|a_\delta:e \in \mathbb{I}_R$, $e:f \in \mathbb{I}_R$,  or $f:b_{\delta}| \in \mathbb{I}_R$.
   \item Disjointness. If $a:b \in R$ and $c:d$ is disjoint from $a:b$, then there exists a $\delta$ such that $R(c:d, \delta) = \emptyset$.
    \item Consistency. If $a:b  \in \mathbb{I}_R$, then $R(a:b, \delta) \neq \emptyset$ for all $\delta$.
    \item Closed. If $a_\delta \in \mathbb{I}_R$ for all $\delta $, then $R(a:b, \delta) \neq \emptyset$ for all $\delta$ and $b$. Such an $a$ is called a root of the oracle. 
\end{enumerate}

If multiple real numbers are being discussed, such as $x$ and $y$, then $R_x$ and $R_y$ will represent their respective rules. 

The rule $R$ may be thought of as a multi-valued function. The expression $R(a:b, \delta) \neq \emptyset$ implies that the empty set is not in the range of $R$ for that input. The expression $R(a:b, \delta) = \emptyset$ is implying that the range for that input into $R$ contains the empty set; it need not be exclusively the empty set. 

The range of $R$ can be thought of as intervals that are known to include the real number based directly on its definition. Then Consistency ensures that $\mathbb{I}_R$ is included in the the set of intervals known to contain the rule since containment is transitive. The Closed property covers an edge case dealing with describing rational numbers.

A rational interval $a:b$ is a \textbf{Yes interval} of the rule $R$ if $R(a:b, \delta) \neq \emptyset$ for all $\delta >0$. This includes all intervals that are in  $\mathbb{I}_R$, but it also includes the \textbf{$a$-rooted} intervals where $a$ is a root of the oracle.   A rational interval $a:b$ is a \textbf{No interval} if $R(a:b, \delta) = \emptyset$ for some $\delta > 0$. In a nonconstructive sense, each interval $a:b$ is either a Yes interval or a No interval. It is non-constructive since it requires a potentially infinite number of $\delta$s to check. 

If an interval $a:b$ is a Yes interval for $R$ and $R$ is to represent the real number $x$, then this can be expressed with the notation $a \xora b$. For No intervals, the notation is \sout{$a \xora b$}. The notation $a:|c_\delta$ extends to the Yes intervals as $a \xora |c_\delta$. These notations reflect that the Yes / No interval designation has created an $x$-betweenness relation on the rational numbers as will be discussed later.

It may be helpful to expand a little on what the properties mean. The basic idea is that a Yes interval ought to contain the real number; since this is defining the real number, this becomes more of a guiding idea, than a deduction. The concept of the rule is that a possible Yes interval is given along with a little error tolerance. The rule ought to respond with a Yes interval which helps move the process along in ascertaining what the real number is. If it cannot respond with a Yes interval, then the given interval is a No interval. 

Here is a bit of explanation for the properties:
\begin{enumerate}
    \item Range. Returning a subinterval of $a:b$ means $x$ is definitively in $a:b$ which is ensured by Consistency. Returning $\emptyset$ means that $x$ is not in $a:b$. The return of subintervals of $(a:b)_\delta$ that are not contained in $a:b$ are ambiguous on the question of $a:b$ containing $x$ though it is in the returned subinterval. If the subinterval is disjoint from $a:b$ then Disjointness yields that $a:b$ does not contain $x$. The returned interval can always be used for further work.  
    
    \item Existence. Without this, the rule could yield a No interval for all intervals. Such a rule would not represent any $x$. On a practical level, this is the starting point of where to start narrowing in on the real number. It can be quite a large interval which makes this doable if some very rough knowledge of the number is known. 
    
    \item Separation. This property is inspired by the Intermediate Value Theorem. The idea is that a Yes interval should be able to be continually narrowed down by selecting a rational number $c$ inside of it and then testing which of the two created intervals is a Yes interval and the other one would then be a No interval. Because of the possibility of being able to not decide the issue at $c$, there is a $\delta$-halo extended in which to examine it. 
    
    It can also be the case that while $a:b$ is in the range of $R$, subintervals of it are not. For example, the range of $R$ could be $1:2$ along with intervals of the form $2:b$ with $b >2$. Intervals of the form $(a:2)_{\delta}$ would then contain the interval $2:(\delta/2)$ which is in the range of $R$. Picking $m=1.5$ and $\delta = 0.2$, an interval of $1.6:2.1$ would contain an element of the range. If it had to be strictly in the interval $1:2$, this example would not be allowed with the Separation property. 

    The use of the term subwidth allows for that assumption to be made in general. The intention is to divide the interval $a:b$ into finer pieces, not to create a larger interval. 
 
    \item Disjointness. This ensures that a single real number is being discussed. Without an assertion of negativity, one could have multiple disjoint regions.  As an example, imagine a rule which returns small intervals around 2 and small intervals around 5. Done correctly, this will be able to satisfy the other properties, including the Separation property because that property only applies within intervals in the range of $R$. 
    
    \item Consistency. This is stating that the oracle never contradicts itself. Since it is supposed to represent a real number being in the interval, if another interval contains it, then that containing interval also contains the real number and ought to be a Yes interval. This rule ensures this as will be demonstrated later. 
    
    \item Closed. This ensures that if there is a narrowing in to a single rational number, then the intervals with that rational number as endpoint are Yes intervals. It does not require that the interval in question is in the range of $R$ nor contain an interval in the range of $R$. The assumption that $a_\delta$ is in $\mathbb{I}_R$ for all $\delta >0$ implies that $R$ can always return a subinterval of $a_\delta$ for $R(a:b, \delta)$. Closed is simply asserting that it ought to do this. 

\end{enumerate}

The term rule is being used instead of function to suggest a more constructive approach. While it is perfectly fine for an $R$ to be given as an explicit function, the more typical case is that $R$ is computed out as needed and may not return the same result for the same inputs. The term oracle is also used to suggest that it is an answering to a question about something that is not quite there and answering in vague, uncertain ways. It is, in fact, the questioning which is bringing the real number into the human realm.


\section{Examples}

The first set of examples is that of rational numbers. Given a rational number $q$, one oracle rule, the \textbf{Singular Oracle at $q$}, is to have  $R(a:b, \delta) = q:q$ if $q \in a:b$ and is the empty set otherwise. This is the nice version of a rational number. A less nice version, but one more in line with what is produced from arithmetic operations on irrational numbers is the \textbf{Fuzzy Oracle at $q$} whose rule is that $R(a:b, \delta)$ returns $q_\delta$ if $q \in a:b$ and returns the empty set otherwise. Both versions have that the Yes intervals are the intervals that contain $q$ while the No intervals are those that do not contain $q$. These two oracles are considered equal, as established in a later section. Oracles equal to them will be collectively referred to as the Oracle of $q$ or the Rooted Oracle of $q$.  

The next set of examples are the $n$-th roots. Let $x$ represent the positive real number such that $x^n$ ought to be the rational $q$. The comprehensive rule for these is that $R(a\lt b, \delta) = a:b$ whenever $\max(a, 0)^n:q:b^n$ and is the empty set otherwise.

A common situation for real number estimates is that the real number is computed by a sequence of intervals. For example, the $n$-th root can be computed using Newton's method. Let $a_0 >0$ be some positive rational. Given $a_m$, define $b_m = q/a_m^{n-1}$ and then $a_m^n : q : b_m^n$ will hold true. The iteration is defined as $a_{m+1} = a_m + (b_m - a_m)/n$. This is what results from applying Newton's method to $x^n - q$. The oracle rule $R$ would then be to compute $m, a_m, b_m$ such that $|a_m - b_m| < \delta$.  Then $R(a:b, \delta) = a_m:b_m$ if $a_m:b_m$ is a subinterval of $(a:b)_\delta$ and is the $\emptyset$ otherwise. Note that if it is the empty set that is returned, then $a:b$ and $a_m:b_m$ are disjoint due to the length of $a_m:b_m$; $a_m:b_m$ need not be disjoint from $a_\delta$ or $b_\delta$ in this case. 

Newton's method leads to a family of rules. Given a different $a_0$, the intervals computed will differ and will lead to some computed intervals overlapping the endpoints of some intervals that other starting points would not overlap with. The Yes / No determination, however, would not  change. If the root is the rational $p$, that is, $p^n = q$,  then intervals of the form $p:b$ will never be returned by this procedure unless $a_0 = p = b_0$. Nevertheless, the rule will never return the empty set for an interval that contains $p$. All Yes intervals in this case will contain $p$. 

This idea of containment rules generally goes the same way. Define a \textbf{Family of Overlapping, Notionally Shrinking Intervals} (fonsi, pronounced faan-zee,)  to be a set of rational intervals such that any pair of rational intervals in the set intersect and, given a rational $\varepsilon >0$, there exists at least one interval in the fonsi such that its length is less than $\varepsilon$. Sequences of nested intervals whose lengths approach 0 is an example of a fonsi. In constructivist works, such as \cite{bridger}, these objects are often taken as the definition of a real number and are likened to a set of measurements. 

\begin{proposition}
    Given a fonsi, there is an oracle associated with it such that all of the elements of the fonsi are Yes intervals. 
\end{proposition}

\begin{proof}
    Let $a:b$ and rational $\delta>0$ be given. Choose an interval $c:d$ from the fonsi such that $|c:d| < \delta$. 

    The rule, $R$ will have that if $c:d$ is a subinterval of $(a:b)_{\delta}$, then $R(a:b, \delta) = c:d$; otherwise it is equal to the empty set. 

    An alternate rule would be that if $c:d$ is a subinterval of $(a:b)_{\delta}$ which intersects $a:b$, then $R(a:b, \delta) = c:d$; otherwise, it is equal to the empty set. This rule would also work and not alter any of the details below as in all of the cases in which not being the empty set is demonstrated, it happens that $c:d$ does intersect $a:b$.

    To show that $R(a:b, \delta) \neq \emptyset$, it is necessary to demonstrate that any element of the fonsi whose length is less than $\delta$ is contained in $(a:b)_\delta$. 
    
    The intervals in the range of $R$ are precisely the elements of the fonsi. 

    The properties are verified as follows: 
    \begin{enumerate}
        \item Range. Satisfied by definition. 
        \item Existence. The fonsi must be non-empty since given a length, it is required to produce an interval. Let $a:b$ be any element of the fonsi and $\delta =1$. Let $c:d$ be any element of the fonsi whose length is less than $\delta$. Since $c:d$ and $a:b$ must intersect, it must be the case that $c:d$ is a subinterval of $(a:b)_\delta$. Thus, $R(a:b, \delta) = c:d$. 
        \item Separation. 
        Let $a:b \in R$, that is, it is an element of the fonsi. Let $m$ and a subwidth $\delta$ be given such that $a:m:b$. Let $c:d$ be an element of the fonsi whose length is less than $\delta$. Since $a:b$ and $c:d$ are in the fonsi, they overlap. This implies $c:d$ is contained in $(a:b)_\delta$. If $c:m:d$, then, because of the length being less than $\delta$, $c:d$ is wholly contained in $m_\delta$. Thus, take $m_\delta$ as the $e:f$ with $a:e:f:b$ and $e:f \in \mathbb{I}_{R}$ as required. 
        
        If $m$ is not in $c:d$, then, possibly via relabelling, $c:d$ will be contained within $|a_\delta:m$ with $c:d:m$ as a relabelling choice. Let $e$ be, say, the midpoint between $m$ and $d$ and $f$ be some number that is strictly contained in $m_\delta$ and $m:b$. By the containment of $c:d$, $|a_\delta:e$ is in $\mathbb{I}_R$.
        
        \item Disjointness. Let $a:b \in R$ and $c:d$ be disjoint from $a:b$. Let $\delta$ be less than the distance $L$ from $a:b$ to $c:d$. Then no interval contained in $(c:d)_\delta$ can intersect $a:b$. Thus, there is no element of the fonsi contained in $(c:d)_\delta$ which implies $R(c:d, \delta) = \emptyset$. 
    
        \item Consistency. Let $a:b \in \mathbb{I}_R$ which implies there exists an element $c:d$ of the fonsi which is a subinterval of $a:b$. Let  $ \delta$ be given. The task is to show $R(a:b, \delta) \neq \emptyset$. Let $e:f$ be any element of the fonsi whose length is less than $\delta$. Since $e:f$ and $c:d$ intersect as they are both in the fonsi, it must be the case that $e:f$ is contained in $(a:b)_\delta$ and $e:f$ intersects $a:b$. Hence $R(a:b, \delta) \neq \emptyset$.
        
        \item Closed. Assume that $a$ is given such that $a_\delta \in \mathbb{I}_R$ for all $\delta$. Given $a:b$ and $\delta$, the task is to show $R(a:b, \delta) \neq \emptyset$. Let $c:d$ be an element of the fonsi whose length is less than $\delta$. The task is to show that it is contained in $(a:b)_\delta$ and that it intersects $a:b$. Due to the length, if $c:a:d$, then $c:d$ is contained in $(a:b)_\delta$. If not, then the distance from $c:d$ to $a$ is positive. Let $\delta'$ be less than that distance. Then $a_{\delta'}$ is disjoint from $c:d$. But this cannot be as $a_{\delta'}$ contains an element of the fonsi which must intersect $c:d$. Therefore, $a$ must be contained in $c:d$. Since $c:d$ is contained in $(a:b)_\delta$ and $c:d$ was an arbitrary element of the fonsi whose length was less than $\delta$, $R(a:b, \delta) \neq \emptyset$. This has also established that $a$ is in every element of the fonsi. 

    \end{enumerate}

\end{proof}


Cauchy sequences can be viewed as pairs of rationals $a_n, \varepsilon_n$ where $a_n$ is the $n$-th element of the sequence and $\varepsilon_n$ is the bound for all future elements to be within $\varepsilon_n$ of $a_n$. This leads to the sequence of intervals $a_n-\varepsilon_n:a_n+\varepsilon_n$. The Cauchy criterion ensures that the $\varepsilon_n$ exists and that they approach 0. Thus, Cauchy sequences can be viewed essentially as fonsis. The associated oracle for the fonsi is the oracle that the sequence converges to. From Cauchy sequences, one obtains many common constructs as fonsis, such as convergent infinite sums. 

Given a set $E$ of rationals bounded above by a rational $b_0$, let $a_0 \in E$. Define $m_n$ to be the midpoint between $a_n$ and $b_n$. If $m_n$ is an upper bound of $E$, then let $a_{n+1} = a_n$ and $b_{n+1}=m_n$. If $m_n$ is not an upper bound of $E$, then $a_{n+1} = m_n$ and $b_{n+1} = b_n$. With this definition, $a_n:b_n$ is a sequence of nested shrinking intervals. That collection is a fonsi and defines an oracle $x$. This oracle is the least upper bound of $E$. The role of the $a_n$ is to ensure there is no upper bound below $x$ while the role of $b_n$ is to ensure there is no element of $E$ above $x$. 

Another approach to the least upper bound is to define $R(a:b, \delta)$ to be $a:b$ when $a$ is bounded above by an element of $E$ and $b$ is an upper bound. For practical reasons, it may not be easy to determine this. Therefore, one could extend it such that if $a_\delta$ or $b_\delta$ contains both elements of $E$ and upper bounds, then $a_\delta$, respectively $b_\delta$, is returned. 

The last example of this section are the real numbers produced by the Intermediate Value Theorem. Let $f(x)$ be a function which is continuous and monotonic on $a:b$ with $f(a)*f(b) < 0$. For rational intervals $c:d$ contained in $a:b$, define the oracle $R$ by $R(c:d, \delta) = c:d$ if $f(c)*f(d) \leq 0$ and the empty set otherwise. For intervals that are not contained in $a:b$, take the intersection with $a:b$ and return its result. Monotonicity is required for the Disjointness property. Continuity is required to show that the oracle produced here is a zero of $f$. Without continuity, the most that can be said is that the oracle is the location of a sign change for the function. The initial $a:b$ yields the existence property. 

The above is assuming that a sign can be determined for $f(c)$. It is possible that if $f(c)$ is close to zero, something which is desired, then $f(c)$'s fuzziness will prevent a sign determination. In that case, the continuity of the function allows for the existence of a $\delta$ such that $f(c_{\delta})$ is contained in the interval $-\varepsilon:\varepsilon$ for a specified $\varepsilon$. Thus, one could specify an $\varepsilon$ tolerance and if $f$ is in that region, then $c_\delta$ is returned by the rule. Practically speaking, this is when a process will terminate, having hit the resolution limits of the computational system. 

Without monotonicity, this definition does not work. One can still do the usual process of dividing up the interval and testing the signs to create a sequence of nested intervals. The sequence constructed will be a fonsi and therefore there will be an oracle associated with it. The oracle that is found could, however, be path dependent on the choice of division points. 

\section{Yes Intervals as Fonsis}

This section establishes that the set of Yes intervals for an oracle is a maximal fonsi. It also produces some useful results for working with oracles. 


\begin{proposition}[Bisection Algorithm]
    Given a rational $\varepsilon >0$ and an oracle $R$, there exists a Yes interval whose length is less than $\varepsilon$.
\end{proposition}

\begin{proof}
    By Existence, there exists an interval $a_0:b_0$ in the range of $R$. If $|a_0:b_0| = 0$, then that interval is sufficient. Otherwise, the desired interval is found via iteration. The iteration step for determining $a_{i+1}$ and $b_{i+1}$ given $a_i:b_i$ begins with letting $m_i$ be the average of $a_i$ and $b_i$. Apply Separation to $a_i:b_i$ with $m_i$ the dividing point and choose a subwidth $\delta < \min(|a_i:b_i|/6, \varepsilon /2)$. By Separation and Consistency, there exists an interval $e_i:f_i$ in $(m_i)_\delta$ and an interval $a_{i+1}:b_{i+1}$ in the range of $R$ such that one of the following holds: 
    \begin{enumerate}
        \item $a_{i+1}:b_{i+1}$ is in $e_i:f_i$. Since the length of $(m_i)_\delta$ is less than $\varepsilon$, this interval works as the desired interval.  
        \item  $a_{i+1}:b_{i+1}$ is in $|(a_i)_\delta:e_i$. The length will be no more than $2|a_i:b_i|/3$.
        \item  $a_{i+1}:b_{i+1}$ is in $f_i:(b_i)_\delta|$. The length will be no more than $2|a_i:b_i|/3$.
    \end{enumerate}
    The length of $a_n:b_n$ will be at most $(2/3)^n |a_0:b_0|$. To find $n$, set the positive rational $\varepsilon/|a_0:b_0| = \frac{p}{q}$ and let $m$ be the number of digits in $q$ expressed in base 10. Then $n \geq  6m$ leads to  $(2/3)^n \leq (2/3)^{6m} < 10^{-m} < 1/q < p/q$. Thus, $n \geq 6m$ would have $|a_n:b_n| \leq (2/3)^n |a_0:b_0| < \varepsilon$ as desired. 
\end{proof}


\begin{proposition}
    Given an oracle $R$, $a:b$ in the range of $R$, and $c:d$ in the range of $R$, $a:b$ and $c:d$ intersect. 
\end{proposition}

\begin{proof}
    If they were disjoint, then the Disjointness property would say, for example, that $R(c:d, \delta)= \emptyset$ for some $\delta$. But $c:d \in \mathbb{I}_R$ as it contains itself. Thus, Consistency says that $R(c:d, \delta) \neq \emptyset$. This is a contradiction and the two intervals must intersect. 
\end{proof}

\begin{corollary}
    The intervals in the range of $R$ form a fonsi.
\end{corollary}


\begin{proposition}
    Let an oracle $R$ be given. If $a:b$ is a Yes interval, then one of the following holds: $a:b \in \mathbb{I}_R$, $a$ is a root of $R$, or $b$ is a root of $R$.
\end{proposition}

\begin{proof}
    By definition of $a \xora b$, $R(a:b, \delta) \neq \emptyset$ for all $\delta > 0$. This implies that, by Disjointness, all intervals in the range of $R$ must intersect $a:b$. If any of them are contained in $a:b$, then $a:b \in \mathbb{I}_R$ which satisfies what was to be shown. Thus, for the rest of this proof, assume all intervals in the range of $R$ are not contained in $a:b$.

    Let $ \delta < |a:b|$. By the Bisection Algorithm, there exists $c:d \in R$ such that $|c:d| < \delta$. Due to its length, $c:d$ cannot contain both $a$ and $b$. As $c:d$ is in the range of $R$, it must intersect $a:b$. For it not to be contained in it, it must overlap one of the endpoints. By potentially relabelling, let $a \in c:d$.
    
    Let $\delta' < \delta$ be given. The task is to show that $a_{\delta'} \in \mathbb{I}_R$. Let $e:f \in R$ be given such that $|e:f| < \delta'$. Since $e:f$ must intersect $c:d$ and $a:b$ but not be contained in $a:b$, it must be the case that $e:a:f$. Since its length is less than $\delta'$, this implies that $e:f \subset a_{\delta'}$.

    Thus, $a$ is a root of $R$ as was to be shown. 
\end{proof}

\begin{proposition}
    If an oracle $R$ has a root $a$, then that root is unique and an interval is a Yes interval exactly when $a$ is in the interval.
\end{proposition}

\begin{proof}
    Any interval of the form $a:b$ is a Yes interval by the Closed property. If $c:d$ strictly contains $a$, then let $L$ be the distance from $a$ to the closest endpoint. Any interval $a_\delta$ with $\delta < L $ will then be contained in $c:d$ and hence $c:d \in \mathbb{I}_R$ since $a_\delta$ is. 

    The remaining case is that of $a$ being outside of $c:d$. Let $L$ be the distance from $a$ to the closest endpoint. Then let $e:f \in R$ be contained in $a_{\delta}$ with $\delta < L$. This exists as $a$ is a root. Due to the length, $e:f$ is disjoint from $c:d$. Thus, Disjointness yields a $\delta'$ such that $R(a:b, \delta')= \emptyset$. Hence, $a:b$ is a No interval. 

    Let $b$ be any rational not equal to $a$. Then $b_{|b-a|/2}$ does not contain $a$ and hence is a No interval. This means it does not contain an element of the range of $R$ and $b$ is not a root of the oracle. 
\end{proof}

\begin{proposition}
    Given an oracle $R$, all Yes intervals intersect. 
\end{proposition}

\begin{proof}
    If $R$ is rooted at $a$, then $a$ is common to all Yes intervals. Hence, they intersect. 

    If $R$ is not rooted, then Yes intervals contain elements of the range of $R$. These intersect and thus the intervals that contain them will intersect as well. 
\end{proof}


\begin{corollary}
    The Yes intervals form a fonsi. 
\end{corollary}

\begin{proposition}\label{os:root}
    If $a$ is contained in every Yes interval, then $a$ is the root of the oracle. 
\end{proposition}

\begin{proof}
    Given $\delta$, let $c:d$ be a Yes interval whose length is less than $\delta$. Since $a$ is contained in $c:d$, $c:d$ is contained in $a_\delta$. Thus, $a_\delta \in \mathbb{I}_R$ for all $\delta$ which, by the Closed property, means $a$ is a root of the oracle. 
\end{proof}

\begin{proposition}
    If $a:b$ is an interval that intersects every Yes interval of an oracle, then $a:b$ is a Yes interval. 
\end{proposition}

\begin{proof}
    Let $c:d$ be any Yes interval. By assumption, $c:d$ intersects $a:b$. If it is contained in $a:b$, then $a:b$ is a Yes interval. If this does not happen for any interval $c:d$, then by this nonconstructive step, every Yes interval contains either $a$ or $b$. By considering intervals whose lengths are less than half the length of $a:b$ and noting that Yes intervals intersect each other, one endpoint must be contained in all of the Yes intervals. But this implies that $a$ is the root of the oracle and thus, $a:b$ is a Yes interval. 
\end{proof}

\begin{corollary}
    The Yes intervals form a maximal fonsi.
\end{corollary}

\begin{proof}
A maximal fonsi is one in which no new interval can be added without it failing to be a fonsi. Since any new interval must intersect all the elements of the fonsi, the previous proposition establishes that it must already be a Yes interval and hence a member of the fonsi. 
\end{proof}


\section{Rational Betweenness Relations}

In this section, the collection of Yes intervals of an oracle will be established to be a rational betweenness relation. 

The definition of a \textbf{rational betweenness relation}, with relational symbol $\xrel$, is that it is a symmetric relation on rational numbers which satisfies the following properties:
\begin{enumerate}
    \item Existence. There exists $a$ and $b$ such that $a\xrel b$.
    \item Interval Separation. If $a \xrel b$ and $a : c : b$, then exactly one of the following holds: 1) $a \xrel c$ and \sout{$c \xrel b$}, 2) $c \xrel b$ and \sout{$a \xrel c$}, or 3) $c \xrel c$. 
    \item Consistency. If $c : a : b : d$ and $a \xrel b$, then $c \xrel d$. 
    \item Singular. If $c \xrel c$ and $d \xrel d$, then $c=d$. 
    \item Closed. If $c$ is a rational number such that $c$ is included in every $x$-interval $a:b$, then  $c \xrel c$. 
\end{enumerate}

This is another definition of what a real number is. The idea is that $x$ is defined by the relation such that $x$ is between the two rational numbers if they are $x$-related. This definition was shown to work in \cite{taylor24dedekind} by demonstrating that they are equivalent to Dedekind cuts. Here, the claim is that $\xora$ defines an $x$-betweenness relation. Since $x$-betweenness relations can be shown to define a fonsi, it is also the case that all such relations are covered by an oracle. Since the Yes intervals include all the intervals of the fonsi and the $x$-related intervals form a maximal fonsi, the Yes intervals and the $x$-related intervals are the same. 

Given an oracle, since all intervals are either a Yes interval or a No interval and it is symmetric, this does define a relation on the rational numbers. Existence follows from the existence rule. The Singular property follows from Disjointness as well as from the roots being unique. The Closed property has also been established already. This leaves Consistency and Interval Separation. 

\begin{proposition}[Consistency]
    Let an oracle $R$ be given representing $x$. Assume $c:a:b:d$. If $a \xora b$, then $c \xora d$. If \sout{$c \xora d$}, then \sout{$a \xora b$}. 
\end{proposition}

\begin{proof}
    For $a \xora b$, either $a:b \in \mathbb{I}_R$ or, potentially by relabelling, $R$ is rooted at $a$. In the first instance, there is an interval $e:f \in R$ contained in $a:b$. But then $c:d$ contains $e:f$ since it contains every interval contained in $a:b$. Thus, $c:d \in \mathbb{I}_R$ and hence $c \xora d$. 
    
    If $R$ is rooted at $a$, there are two cases. In the first case, $a$ is strictly contained in $c:d$. Then there is a positive distance $L$ from $a$ to the closest endpoint of $c:d$. Let $\delta < L$. Then $a_\delta$ is contained in $c:d$ and hence $c:d \in \mathbb{I}_R$. The other case is that $a$ is an endpoint of $c:d$, say, $ a= c$. Then $R(a:d, \delta) \neq \emptyset$ for all $\delta$ by the Closed property. Thus, $c \xora d$ in both cases.  

    With the nonconstructive assumption that $a:b$ is either a Yes interval or No interval and since being a Yes interval leads to $c:d$ being a Yes interval, if $c:d$ is known to be a No interval, then $a:b$ must be a No interval as well. 

    In terms of $R$, this is implying that $R(c:d, \delta) = \emptyset$ implies the existence of $\delta'$ such that $R(a:b, \delta') = \emptyset$. 
    
    
\end{proof}



The Separation property as postulated does not apply to all Yes intervals; this makes being an oracle easier to verify, but harder to use. Separation, however, can be extended to all Yes intervals.

\begin{proposition}[Separation]
    If $a: b$ is a Yes interval for an oracle $R$, $m$ is a rational number strictly between $a$ and $b$, and a subwidth $\delta > 0$ is given, then there exists an $m_\delta$ compatible interval $e:f$ such that one of the following holds true:
    \begin{enumerate}
        \item $a \xora e$, \sout{$e\xora f$}, \sout{$f\xora b$};
        \item $e \xora f$, \sout{$a\xora e$}, \sout{$f\xora b$};
        \item $f \xora b$, \sout{$a\xora e$}, \sout{$e\xora f$}.
    \end{enumerate}
\end{proposition}

\begin{proof}
    Yes intervals are either in $\mathbb{I}_R$ or they are rooted. 
    
    If it is a rooted interval, then relabelling allows for the root to be $a$ which implies $a_{\delta'} \in \mathbb{I}_R$ for all rational $\delta' >0$. With $a:e:b$, it is immediately the case that $a:e$ is an $a$-Rooted interval and hence is a Yes interval by the Closed property. Any $m_\delta$ compatible interval $e:f$ will then have the property that $a \xora e$ while \sout{$e\xora f$} and \sout{$f\xora b$} because $e:f$ and $f:b$ are disjoint from the Yes intervals $a_\delta$ for $\delta < |a:e|$.
 
    The other case is that of $a:b \in \mathbb{I}_R$. This consists of two cases. Let $c:d$ be an interval contained in $a:b$ which is in the range of $R$. 
    
    If $m$ is not in $c:d$, then, by possibly relabelling, $a:m:c:d:b$ with $m$ strictly contained in $a:c$.  Let $e:f$ be an $m_\delta$ compatible interval such that $a:e:m:f:c:d:b$ with $f \neq c$. Then $a:e$ and $e:f$ are disjoint from $c:d$ while $f:b$ contains $c:d$. Thus, $f \xora b$, \sout{$a\xora e$} and \sout{$e\xora f$}.

    If $m$ is in $c:d$, then apply the Separation property using a subwidth $\delta' < \delta$. If the produced $e':f'$ is in $\mathbb{I}_R$, then, as it is in $m_{\delta'}$, it can be expanded to an interval $e:f$ such that $e:f$ is $m_\delta$ compatible and both $a:e$ and $f:b$ are disjoint from $e':f'$. Thus, $e \xora f$, \sout{$a\xora e$}, \sout{$f\xora b$}.

    If the produced $e':f'$ is not in $\mathbb{I}_R$, then, by relabelling, it can be assumed that $|c_{\delta'}:e' \in \mathbb{I}_R$. Then $e$ can be chosen to be strictly contained in $e':m$ so that \sout{$e\xora f$} and \sout{$f\xora b$} by Disjointness. If $a:c_{\delta'}:e$, then $a \xora e$. If not, then a non-constructive approach is needed.

    If there is any Yes interval contained in $a:e$, then that is sufficient to conclude $a:e$ is a Yes interval. So assume there is not; this is a nonconstructive step. This means that while every Yes interval must intersect both $a:b$ and $a:e'$, none of them can be contained in $a:e$. Intervals whose lengths are less than the distance between $e'$ and $b$ imply that $a$ is the common point of intersection for all of these intervals. As established in \ref{os:root}, this implies $a$ is a root of the oracle. Thus, $a \xora e$ is a Yes interval. 

\end{proof}

\begin{corollary}[Interval Separation]
  If $a \xora b$, then one exactly one of the following holds: 
  \begin{enumerate}
        \item $a\xora m$, \sout{$m \xora b$};
        \item $m \xora m$; 
        \item $b \xora m$, \sout{$m \xora a$}.
    \end{enumerate}
\end{corollary}

\begin{proof}
Given any $\delta$, if $e:f$, as in the previous proof, is not the Yes interval, then onsistency and Disjointness leads to either the first or third outcome, depending on which one contains the Yes interval. 

If $e:f$ is the Yes interval for each $\delta$, then $m_\delta \in \mathbb{I}_R$ for all $\delta$. This is a nonconstructive step. The Closed property yields that $m$ is the root of the oracle and thus, $m \xora m$.
\end{proof}

    
\begin{proposition}
    If $ a\xora b$ and $c \xora d$, then the intersection of $a:b$ and $c:d$ is a Yes interval. 
\end{proposition}

\begin{proof}
    Up to relabelling, the various cases are: 
    \begin{enumerate}
        \item $a:c:d:b$. In this case, the intersection is $c:d$ which is a Yes interval. 
        \item $a:b:c:d$ with $b \neq c$. This is the case of two disjoint Yes intervals. That was established to not be possible. 
        
        \item $a:c:b:d$ with $b \neq c$. By Consistency, $a:d$ is a Yes interval. Let $\delta$ be less than half the distance between $b$ and $c$. 
        
        Use Separation on $a:b$ with $c$ as the separating number and $\delta$ the overlap. Separation gives us a $c_\delta$ compatible interval $e:f$ such that $a:e:c:f:b$. Since $a:e$ is disjoint from the Yes interval $c:d$, this implies that $a:e$ must be a No interval. If $f:b$ is a Yes interval, then $c:b$ is a Yes interval by Consistency which establishes that the intersection is a Yes interval and not a No interval. What remains is $e:f$ being a Yes interval.  

        By a similar argument using $c:d$ and $b$ to separate, there is a $b_\delta$ compatible interval $m:n$ such that $c:m:b:n:d$. The interval $n:d$ must be a No interval by the disjointness with $a:b$. The interval $c:m$ being a Yes interval implies $c:b$ is a Yes interval by Consistency. That leaves $m:n$.
        
        By the selection of $\delta$, it is the case that $b_\delta$ and $c_\delta$ are disjoint. Thus, $m:n$ and $e:f$ are disjoint and both cannot be Yes intervals implying at least one of them is a No interval. If $m:n$ is No, then $c \xora m$ must hold true. If $e:f$ is No, then $f \xora b$ is true. 

        The conclusion is therefore that $b:c$ is a Yes interval. 

        \item $a:(b=c):d$. In this case, $b=c$ is the intersection.  If $a=b$ or $c=d$, then the intersection is a Yes interval as it is one of the given interval. Assume, therefore, that $a \neq b$ and $c \neq d$. Note that, as before, $a:d$ is a Yes interval by Consistency.
        
        %Apply Interval Separation to $a:d$ with a separating number of $b=c$. Then the property states that either  1) $a \xora b$ with \sout{$c \xora d$}, 2) $c \xora c$, or 3) $c \xora d$ with \sout{$a \xora b$}. As both 1) and 3) are not consistent with the given information, 2) must hold. Thus, the intersection is a   NOT DOING THIS AS IT IS NONCONSTRUCTIVE BASED. 
        
        Assuming the endpoints are distinct, apply Separation to $a:d$ with the separating number being $c$. Use a subwidth $\delta$. Then Separation gives an interval $e:f$ such that $a:e:c:f:d$, all distinct numbers. If $a:e$ is a Yes interval, then $c:d$ is a No interval which contradicts the assumption. If $f:d$ is a Yes interval, then $a:b$ is a No interval which contradicts the assumption. Thus, $e:f$ has to be a Yes interval. Note that this logic would apply for all $0 < \delta' < \delta$. Therefore, $c_\delta$ is a Yes interval for all $\delta > 0$. By the Closed property, the intersection $c:c$ is a Yes interval. 
     \end{enumerate}
\end{proof}

\begin{corollary}
    If $a \xora b$, $b \xora c$ and $a:b:c$, then $b:b$ is a Yes interval.
\end{corollary}

\begin{proof}
    $b:b$ is the intersection of $a:b$ and $b:c$.
\end{proof}


\section{Equality}

Most definitions of real numbers have some issue with non-uniqueness of the representative. Some of them are small in scope such as the trailing nines of decimal representations and the two representatives of rationals in continued fraction representations. Others are quite significant such as Cauchy sequences having uncountably many different and misleading representatives. Dedekind cuts largely avoid non-uniqueness though that approach is the least oriented towards computation as partially evidenced by the representatives of rational numbers not having any special characteristics distinct from irrationals. 

For oracles, the non-uniqueness is largely sitting with $R$. Indeed, $R$ may not even be a single-valued function. It is a foundation for the higher level deduction of which intervals are Yes or No intervals. 

\begin{proposition}
    Given an oracle and two disjoint intervals $a:b$ and $c:d$, at least one of them is constructively known to be a No interval of that oracle. 
\end{proposition}

\begin{proof}
    As the intervals are disjoint, by potentially relabelling, it can be assumed $a:b:c:d$ with $b < c$. Then define $L = c-b > 0$. By the Bisection Algorithm, which is a constructive procedure, there is a Yes interval $e:f$ whose length is less than $L$. The interval $e:f$ must be disjoint from at least one of the intervals because of the length. The interval it is disjoint from is then a No interval by Disjointness. 
\end{proof}

This proposition also applies to singletons which establishes that given two rational numbers, at least one of them can be definitively excluded from being the root of the oracle. The role of the proposition is to say that while there is a nonconstructive portion when definitively narrowing in on a root of an oracle, all other rational numbers can be excluded when asked about. 

Two oracles are \textbf{equal} if their associated betweenness relations are equal as relations. Since, nonconstructively, all intervals are either Yes or No for a given oracle, this can be taken as well-defined. The properties of Reflexivity, Symmetry, and Transitivity follow immediately from those properties on equality of sets. 

Because of the nature of having to affirm equality of infinitely many intervals, in practice, it may be hard to establish that two oracles are equal. They can be said to be $a_\delta$ compatible for a given $a$ and $\delta$ if $a_\delta$ is a Yes interval for both. 

\begin{proposition}
    Given a rational number $q$, the Singular Oracle at $q$ is equal to the Fuzzy Oracle at $q$. 
\end{proposition}

\begin{proof}
    The Fuzzy Oracle at $q$ is defined as $R(a:b, \delta) = q_\delta \neq \emptyset$ exactly when $q \in a:b$ and is the empty set otherwise. Thus, an interval $a:b$ is a Yes interval of the Fuzzy Oracle exactly when $q \in a:b$. 
    
    The Singular Oracle at $q$ has $R(a:b, \delta) = q:q \neq \emptyset $ exactly when $q \in a:b$ and is the empty set otherwise. Thus, an interval $a:b$ is a Yes interval of the Singular Oracle exactly when $q \in a:b$.

    In both cases, $a:b$ is a Yes interval exactly when $q \in a:b$. By the definition of equality, these are equal as oracles. 
\end{proof}

The Oracle of $q$ will refer to any oracle which is equal to the Singular Oracle at $q$ which will be the canonical example. 

A particular example to explore is $(\sqrt{2})^2$. Let $R$ be the oracle of this number. It should be equal to the Oracle of 2. One rule can be the following: $R(a:b, \delta) = c^2:d^2$ if there exists $c:d$ such that $c^2:2:d^2$ and $c^2:d^2$ is a subinterval of $(a:b)_\delta$. The rule returns the empty set in all other cases. If $a:b$ excludes 2,  then when $\delta$ is less than the distance between $a:b$ and 2, there will exist no interval $c:d$ that satisfies the requirements and thus the rule provides the empty set. It is clear that the Yes intervals are exactly those that contain 2. Thus, this is the Oracle of 2, as it should be. 

One can also ponder comparing $(\sqrt{2})^2$ to $(\sqrt{1.9\ldots93})^2$ where the number of 9s is, say, $10^{23}$. And let us say that this is in the context of solving $f(x) = 0$ for some function $f$ where one can use Newton's method to find the root, but not be able to explicitly produce it. It would be impossible to distinguish these computationally. This is an issue for all versions of real numbers though an interval approach has the advantage that it is explicitly giving the range of compatible numbers. 

It is not necessary to check all Yes intervals though it does still require proving something about infinitely many intervals.

\begin{proposition}
Let $R$ and $S$ be two oracles. Let $\mathcal{I}_R$ be a fonsi of Yes intervals of $R$ and $\mathcal{I}_S$ be a fonsi of Yes intervals of $S$. If all the elements of $\mathcal{I}_R$ intersect pairwise with the elements of $\mathcal{I}_S$, then $R=S$
\end{proposition}

Note that the fonsi requirement is actually just that there are arbitrarily small intervals in the sets as Yes intervals of an oracle always intersect one another. 

\begin{proof}\label{os:equal}
    The task is to show that an interval $a:b$ is either a Yes interval of $R$ and  $S$ or it is a No interval of both of them. Since all intervals are, nonconstructively, either Yes or No, it is sufficient to show that it is the case for just one type. The direction taken here is to show that all No intervals are the same. By relabelling, it is sufficient to show that an arbitrary No interval of $R$ is a No interval of $S$. 

    Let $a:b$ be a No interval of $R$. This implies there is a Yes interval $c:d$ of $R$ such that $a:b$ and $c:d$ are disjoint. Let $L$ be the distance between $a:b$ and $c:d$. Let $e:f$ be an interval in $\mathcal{I}_R$ whose length is less than $L/2$. As it is a Yes interval of $R$, it must intersect $c:d$. Let $m:n$ be an interval in $\mathcal{I}_S$ whose length is less than $L/2$. This intersects $e:f$ by assumption. Thus, all elements of $m:n$ strictly contained in $(c:d)_L$. This means that $m:n$ is disjoint from $a:b$ and, hence, $a:b$ is a No interval of $S$ as was to be shown. 
\end{proof} 

The choice to pursue the No intervals in the proof is reflective of the fact that there is a finite amount of work in computing out that an interval is a No interval while there can be a potentially infinite amount of work to establish that an interval is a Yes interval. 

\section{Inequality}

Two oracles $R_x$ and $R_y$ can be distinguished if there are disjoint intervals $a:b$ and $c:d$ such that $a \xora b$ and $c \xora[y] d$. It is clear that the oracles are not equal as they do not have the same Yes/No intervals. Having separated intervals allows for a comparison of oracles. 

On the interval level, $a:b < c:d$ means that for every $p \in a:b$ and every $q \in c:d$, it is the case that $p < q$. Necessarily, the intervals must be disjoint to have this be possible and, any two disjoint intervals, will be related by an inequality. 

We define the operators as:
\begin{enumerate}
    \item $R_x < R_y$, or more briefly, $x < y$, if there exists $a \xora b$ and $c \xora[y] d$ such that $a:b<c:d$.
    \item $R_x > R_y$, or more briefly, $x > y$, if there exists $a \xora b$ and $c \xora[y] d$ such that $a:b > c:d$.
\end{enumerate}

If the intervals $a:b$ and $c:d$ were not disjoint, then equality is possible. If it can be shown that all Yes intervals of $x$ are not greater than any of the Yes intervals of $y$, then that is noted as $x \leq y$. This would translate to the interval level as for any $x$-Yes interval $a \lte b$ and any $y$-Yes interval $c \lte d$ that it must be the case that $a \leq d$. 

\begin{proposition}[Transitivity]
    If $x <y$ and $y<z$, then $x < z$.
\end{proposition}

\begin{proof}
    Let $a \xora b$, $c \xora[y] d$, $e \xora[y] f$, and $g \xora[z] h$ such that $a:b < c:d$ and $e:f < g:h$. The goal is to show that $a:b < g:h$. Let $m:n$ be the intersection of $c:d$ and $e:f$. This exists as $c:d$ and $e:f$ are Yes intervals of the same oracle. Let $p \in a:b$ and $q \in g:h$ and $r \in m:n$. By assumption, $p < r$ and $r < q$. By transitivity of inequality of rational numbers, $p < q$. As this holds for all $p \in a:b$ and $q \in g:h$, it is the case the $a:b <g:h$.
\end{proof}

It does need to be shown that there can be no contradiction. 

\begin{proposition}
    If $x < y$, then it is not true that $x > y$ and it is also not true that $x = y$.
\end{proposition}

\begin{proof}
    The issue is that the comparison is based on two particular Yes intervals. It needs to be shown that two other Yes intervals would not contradict this statement. 

    Let $a:b < c:d$ be given that exemplifies $x<y$. Then $a:b$ is disjoint from $c:d$. This implies that $a:b$ is a No interval for $y$ while $c:d$ is a No interval for $x$. Thus, the two oracles are not equal as their rational betweenness relations differ.

    The other task is to show there are no Yes intervals that yield the opposite inequality. Let $p:q$ be a Yes interval for $x$ and $r:s$ be a Yes interval for $y$. If $p:q$ and $r:s$ overlap, then there is no contradictory information. 

    Assume, therefore, that they are disjoint. Since $p:q$ must intersect $a:b$ and $r:s$ must intersect $c:d$, it must be the case that $p:q < r:s$. This follows as given any two disjoint intervals, one must be wholly less than the other and the intersections demonstrates which inequality holds. 

\end{proof}

The oracle inequality operation satisfies transitivity, relying on the transitivity of inequality for rational intervals which in turn relies on the transitivity of rational numbers. 

The classical story is that the real numbers satisfy the Trichotomy property: Given $x$ and $y$, exactly one of the following holds: $x<y$, $x>y$, or $x=y$. This holds non-constructively for the oracles. If there exists two disjoint Yes intervals, one for $x$ and one for $y$, then one of the inequality holds as was just explored. The other case is that every Yes interval of $x$ intersects every Yes interval of $y$. Proposition \ref{os:equal} establishes that $x=y$ in that situation. 
 

This was non-constructive as, generically, it requires potentially checking infinitely many intervals. 

\begin{corollary}
    Let $x$ and $y$ be two oracles. Then, nonconstructively, exactly one of the following holds true: $x<y$, $x>y$, or $x=y$.
\end{corollary}

The constructivists use a property called $\varepsilon$-Trichotomy. This allows a definite determination with a finite, predictable amount of work. 

\begin{proposition}[$\varepsilon$-Trichotomy]
    Given oracles $x$ and $y$ and a positive rational $\varepsilon$, exactly one of the following holds: $x<y$, $x>y$, or there exists an interval $a:b$ of length no more than $\varepsilon$ such that $a:b$ is a Yes interval for both $x$ and $y$.
\end{proposition}

\begin{proof}
    By the Bisection algorithm, there exists an $x$-Yes interval $c:d$ and a $y$-Yes interval $e:f$ such that both intervals have length less than $\varepsilon/2$. If $c:d$ and $e:f$ are disjoint, then the oracles are unequal with the inequality being that of the intervals. If $c:d$ and $e:f$ overlap, then their union is a Yes interval for both $x$ and $y$. That interval has length less than $\varepsilon$.
\end{proof}

It is also the case that given $x < y$, there exists a length such that all Yes intervals of $x$ and $y$ of that length are disjoint. 

\begin{proposition}
    If $ x< y$, there exists $\delta$ such that if $a \xora b$, $c \xora[y] d$, $|a:b| < \delta$ and $|c:d| < \delta$, then $a:b < c:d$
\end{proposition}

\begin{proof}
    Assume $e \xora f$ and $g \xora[y] h$ are such that $e\lte f < g \lte h$. Let $L = g-f$. Then $\delta = L/2$ satisfies the requirements of the statement. 

    Let $a \xora b$ with $|a:b| < \delta$ and $c \xora d$ with $|c:d| \delta$. Since $a:b$ must intersect $e:f$, it is the case that $a:b$ is strictly contained in $(e:f)_\delta$. Similarly, $c:d$ is strictly contained in $(g:h)_\delta$. Thus, $a:b$ and $c:d$ are disjoint. This implies that $a:b < c:d$. 
\end{proof}


\section{Completeness}

Completeness is a defining feature of real numbers. It can come in a variety of guises as wonderfully detailed by James Propp in \cite{propp}. While any of the equivalent versions could be used, this paper will go with the choice Propp suggests as a good foundation: the Cut property. It is a simplified and symmetrized version of the least upper bound property. 

\begin{theorem}\label{th:cut}
    Let $A$ and $B$ be two disjoint, nonempty sets of oracles such that $A \cup B$ is the entire set of oracles.  Additionally, assume all oracles in $A$ are strictly less than all oracles in $B$. Then there exists an oracle $\kappa$ such that for all $x < \kappa$, it is the case that $x \in A$ and, for all $x > \kappa$, it is the case that $x \in B$.
\end{theorem}

Note that it should be clear that if $x < y$ and $y \in A$, then $x  \in A$. Also, if $x \in B$, then $y \in B$. To be otherwise would contradict the assumption that $A$ is strictly less than $B$. 

To say that a rational number $q$ is in a set of oracles will mean an oracle representative of that rational is in the set of oracles. The notation $\widehat{q}$ may be used to denote the oracle version.

\begin{proof}
    We need to define the oracle $\kappa$. The rule $R$ will be such that $R(a\lte b, \delta)$ returns $c\lte d$, a subinterval of $(a:b)_\delta$ if $\widehat{c} \in A$ and $\widehat{d} \in B$. If $\widehat{a-\delta} \in B$ or $\widehat{b+\delta} \in A$, then the empty set is returned. 

    We need to show that $\kappa$ is an oracle and satisfies the desired property. Let us show that the desired property holds first. 

    %The main implication of the relation of $A \cup B$ is that if $x < y$ and $y \in A$, then $x \in A$. Similarly, if $x \in B$, then $y \in B$. The proof of this is simply that because of the union being the whole set of oracles, each of $x$ and $y$ has to be in $A$ or $B$ and cannot be in both. Furthermore, if $y \in A$ and $x \in B$, but $x < y$, then we do not have the property holding that all oracles in $A$ are strictly less than all oracles in $B$. Thus, if we know that $y \in A$, then $x \in A$. And if we know that $x \in B$, then $y \in B$.

    For any $x \in A$ and $y \in B$ with $x$-Yes interval $a \lte b$  and $y$-Yes interval $c \lte d$, it is the case that $\widehat{a} \in A$ and $\widehat{d} \in B$. Hence, $a:d \in R$ and $a \xora[\kappa] d$. 

    Let $ x < \kappa$. Then there exists an $x$-Yes interval $u\lte v$ and a $\kappa$-Yes interval $e\lte f$ such that $v < e$. That is what that inequality means. By definition, $e$ is a lower endpoint of an interval that contains the interval $m\lte n$ where $m$ is itself the lower endpoint of a Yes interval for some element $\alpha \in A$; let's say that is the interval $m\lte g$. We clearly have $ u \lte v < e \lte g$ and thus $\alpha > x$. As $\alpha \in A$, we therefore have $x \in A$. 

    If $x > \kappa$, we can do the same argument except reversing the inequalities leading to $\beta \in B$ which is less than $x$ implying $x \in B$.

    As for $\kappa$ being an oracle, this is the usual checking of the properties: 
    \begin{enumerate}
        \item Range. This is true by definition. 
        
        \item Existence. By the non-emptiness, there exists oracles $\alpha \in A$ and $\beta \in B$ with Yes intervals $a:b$ for $\alpha$ and $c:d$ for $\beta$. Thus, $R(a:d, 1) \neq \emptyset$ as $\widehat{a-1} \in A$ and $\widehat{d+1} \in B$. 
        
        \item Separation. Let $a\lte b \in R$, $a:m:b$, and a subwidth $\delta $ be given.  By assumption of the disjointness and totality of the sets $A$ and $B$, it is the case that $\widehat{m}$ is in exactly one of $A$ or $B$.
        
        If $\widehat{m} \in A$, then $\widehat{m-\delta}$ is in $A$. If $\widehat{m+\delta}$ is in $A$, then $m+\delta:b \in R$. If $\widehat{m+\delta}$ is in $B$, then $m_\delta \in R$.

        Similarly, if $m$ is in $B$, then $\widehat{m+\delta} \in B$. If $\widehat{m-\delta} \in B$, then $a:m-\delta \in R$. If $\widehat{m-\delta} \in A$, then $m_\delta \in R$.

        \item Disjointness. Let $a \lte b$ be in the range of $R$ and $c : d$ disjoint from $a:b$. If $(c:d) < a$, then both $c$ and $d$ are in $A$. With a $\delta$ less than the distance to $a$, it would be the case that $c_\delta, d_\delta, c:d$ do not intersect $B$. Thus, $R(c:d, \delta) = \emptyset$. Similarly for $b < (c:d)$. 
         
        \item Consistency. Let $a \lte b \in \mathbb{I}_R$ and $\delta$ be given. Let $c \lte d \in R$ and contained in $a:b$. Note that $\widehat{c} \in A$ and $\widehat{d} \in B$ by definition of the range of $R$. Then $a- \delta < c$ implies $\widehat{a - \delta} \in A$ and $b + \delta > d$ implies $\widehat{b+\delta} \in B$. Thus, $R(a:b, \delta) \neq \emptyset$ and, in fact, $R(a:b, \delta) = c:d$.
        
        \item Closed. Let $a_\delta \in \mathbb{I}_R$ for all $\delta$. Let $a:b$ and $\delta$ be given. Let $c \lte d$ be such that $c:d$ is contained in $a_\delta$ and $c:d \in R$. Then $c \in A$ and $d \in B$. By the containment in $a_\delta$, $a- \delta < c$ implies $\widehat{a-\delta} \in A$ while $a + \delta > d$ implies $\widehat{a+\delta} \in B$. If $b \leq a$, then $\widehat{b-\delta} \in A$. If $b \geq a$, then $\widehat{b+\delta} \in B$. In either case, $R(a:b, \delta) \neq \emptyset$ as the condition to have the empty set is not present. In addition, $R(a:b, \delta) = c:d$. 
    \end{enumerate}
    
\end{proof}


\section{Arithmetic}

For arithmetic, the operators of addition and multiplication need to be defined. Then it needs to be shown that the usual properties hold including the existence of additive and multiplicative identities and inverses, as appropriate. 

The idea of arithmetic with operators is to apply the operations to the intervals. The ideal approach would be that the Yes intervals of $x+y$ would be the intervals that result from adding Yes intervals of $x$ to those of $y$. This almost works, but it fails with, say, $\sqrt{2} - \sqrt{2}$. This ought to be 0. All the Yes intervals of $\sqrt{2}$ added to those of $-\sqrt{2}$ do lead to intervals that contain 0. But there is no interval a result of those operations of the form $0:b$. This is where having the oracle rules becomes very useful. The range of $R_{x+y}$ will consist of the result of combining the range of $R_x$ and $R_y$ with the arithmetic operator, but with that scenario, the omission of $0:b$ intervals is not an issue as they can be deduced in the step going from the range of $R$ to the Yes intervals. The Yes step is theoretically doable, but may not be always actionable. For example, if solving $f(x)=0$ with Newton's method to something that looks like $\sqrt{2}$ but cannot be proven as such, then $x - \sqrt{2}$ would be seen as compatible with 0, but could not be proven to be so via any finite computation. 

Interval arithmetic is largely that of doing the operation on the endpoints. For addition, $a \lte b \oplus c \lte d = (a+c) \lte (b+d)$ and its length $|a:b|+|c:b|$. For multiplication, $a:b \otimes c:d = \min(ac, ad, bc, bd)\lte \max(ac, ad, bc, bd)$. For $0 \lte a \lte b$ and $0 \lte c \lte d$, the interval multiplication becomes $ac \lte bd$. For multiplication, the length is a bit more complicated. Let $M$ be an absolute bound for $a:b$ and $c:d$; an absolute bound on an interval is a number such that for $p$ in the interval, $|p| < M$. Then $|a:b \otimes c:d| < |M|(|a:b| + |c:d|)$. 

Most of the arithmetic properties hold for interval arithmetic, but the distributive property does not nor are there any additive and multiplicative inverses. The additive identity is $0:0$ while the multiplicative identity is $1:1$. Operating on subintervals of $a:b$ and $c:d$ yields a subinterval of the resulting interval operation on $a:b$ and $c:d$. 

The negation operator is $\ominus(a:b) = -a:-b$; this is not the additive inverse as $a:b \oplus (\ominus(a :b )) = (a-b):(b-a)$ which does contain 0, but not exclusively so. The reciprocity operator is $1 \oslash (a:b) = 1/a : 1/b$ though this only applies to intervals excluding 0. If 0 was included, with $a < 0$, the resulting set would be $-\infty:1/a \cup 1/b : \infty $ which is not an interval as used here.

The distributive property is replaced with $( a:b \otimes ( c:d \oplus e:f) \subset (a:b \otimes c:d) \oplus (a:b \otimes e:f)$. To see this, note that the left-side has boundaries chosen from $\{a(c+e), a(d+f), b(c+e), b(d+f)\}$ while the second has a boundary of the form $(\min(ac, ad, bc, bd) + \min(ae, af, be, bf) ) \lte (\max(ac, ad, bc, bd) + \max(ae, af, be, bf) )$. To demonstrate that they are indeed not equal for some examples, consider $2:3 \otimes ( 4:7 \oplus -6:-3) = 2:3 \otimes -2:4 = -6:12$ compared to $(2:3 \otimes 4:7) \oplus (2:3 \otimes -6:-3) = 8:21 \oplus -18:-6 = -10:15$. 

For more on interval analysis, see, for example, \cite{moore} or, in this context, \cite{taylor23main}.


Let oracles $x$ and $y$ be given with rules $R_x$ and $R_y$, respectively. The oracle $\xora[x+y]$ is defined by the rule that $R_{x+y}(a:b, \delta)$ can be $c:d$ if there exists $e':f'$ in the range of $R_x$ and $e'': f''$ in the range of $R_y$ such that $(e':f') \oplus (e'':f'') = c:d$ and $c:d$ is contained in $(a:b)_\delta$; it is the empty set otherwise. For multiplication, the oracle $\xora[xy]$ is defined by the rule that $R_{xy}(a:b, \delta)$ is $c:d$ if there exists $e': f'$  in the range of $R_x$ and $e'':f''$ such that $(e':f') \otimes (e'':f'') = c:d$ and $c:d$ is contained in $(a:b)_\delta$; it is the empty set otherwise. A variant of this is to enlarge this to use the Yes intervals of $x$ and $y$; the choice to use the ranges of $R_x$ and $R_y$ is to keep the arithmetic constructive. It should be clear that they generate the same set of Yes intervals for the operated on oracles as Yes intervals are associated with intervals in the range of the rule. This is then the link to saying that using different rules for $x$ and $y$ lead to the same $x+y$, $xy$-relations on the rationals. 


The first step is to establish that these are oracles. Addition and multiplication will be handled mostly simultaneously. The symbol $\odot$ will be used to represent a generic symbol for an operator operating on intervals and $\cdot$ will then be used for that same operator operating on individual numbers.

\begin{enumerate}
    \item Range. This holds by definition. 
    \item Existence. Let $a:b \in R_x$ and $c:d \in R_y$. Then $R_{x\cdot y}(a:b \odot c:d, \delta) = a:b \odot c:d$ establishes existence for the operator $\odot$; cycling over $\oplus$ and $\otimes$ yields it for both of them. 
    \item Separation. Let $a:b$ be the result of $a':b' \odot a'':b''$ for $a':b' \in R_x$ and $a'': b'' \in R_y$. Let $m$ be strictly contained in $a:b$ and let $\delta$ be given. 
    
    %Because $m$ is strictly contained, there exists $m'$ and $m''$ such that $m = m' \cdot m''$ with $m'$ strictly contained in $a':b'$ and $m''$ strictly contained in $a'':b''$. 
    
    For $\odot = \oplus$, choose $\delta'$ and $\delta''$ such that $\delta' + \delta'' < \delta$. 
    
    For $\odot = \otimes$, let $M$ be an absolute upper bound on the intervals $a':b'$ and $a'':b''$. Then choose $\delta'$ and $\delta''$ such that $M (\delta' + \delta'') + \delta' \delta'' < \delta$.
    
    Choose intervals $c':d' \in R_x$ and $c'':d'' \in R_y$  such that their lengths are less than $\delta'$ and $\delta''$, respectively, as allowed by the Bisection Algorithm. Define $c:d = c':d' \odot c'':d''$ Then $|c:d| < \delta$ per the respective operators bounds. 
    
    Since $a':b'$ and $c':d'$ must intersect as both are in the range of $R_x$, it must be the case that $c':d'$ is contained in $(a':b')_{\delta'}$. Similarly, $c'':d'' \subset (a'':b'')_{\delta''}$. The claim is that $c:d \subset (a':b')_{\delta'} \odot (a'':b'')_{\delta''} \subset (a:b)_\delta$. Let $p' \in a':b'$ and $p'' \in a'':b''$. 

    For $\odot = \oplus$, $p' \pm \delta' + p'' \pm \delta''$ is the generic form of an element in $(a':b')_{\delta'} \oplus (a'':b'')_{\delta''}$. By combining, $(p' + p'') \pm (\delta' + \delta')$, which will be called $q$. Then $|p' + p'' - q| = |\delta'| + |\delta''| < \delta$. Thus, $q \in (a:b)_\delta$.

    For $\odot = \otimes$, $(p' \pm \delta')(p'' \pm \delta'') = p'p'' + (\pm p' \delta'' \pm p'' \delta' \pm \delta'\delta'')$ which will be represented by $q$. Note that $|p'p'' - q| \leq |p' \delta''| + | p'' \delta'| + | \delta'\delta''| < M(\delta' + \delta'') + \delta' \delta'' < \delta$. Therefore, $q \in (a:b)_\delta$.

    The claim of containment has been established. 

    If $m$ is contained in $c:d$, then since its length is less than $\delta$, it is contained in $m_\delta$ and it serves in the role of $e:f$ in the property. If $m$ is not in $c:d$, let $a:c:d:m$ by relabelling. Then take $e$ to be the average of $d$ and $m$. Let $f$ be on the other side of $m$ and within $m_\delta$. The interval $c:d$ is then wholly contained in $|a_\delta:e$, satisfying the property. 

    Note that the Separation property on the input oracles was used via the Bisection Algorithm to generate the intervals of sufficiently small length. 
    
    \item Disjointness. Let $a:b \in R_{x\dot y}$ and $c:d$ disjoint from $a:b$. Let $e':f' \in R_x$ and $e'':f'' \in R_y$ be such that $e':f' \odot e'':f''$ is in $a:b$. Given any $m':n' \in R_x$ and $m'':n'' \in R_y$, there will be $p'$ and $p''$ such that $e':p':f'$, $m':p':n'$, $e'':p'':f''$, and $m'':p'':n''$. Thus, $p' \cdot p''$ will be in both $e':f' \odot e'':f''$ and $m':n' \odot m'':n''$.  Let $L$ be the distance from $c:d$ to $a:b$; this is positive as they are disjoint. Then $p' \cdot p''$, which is in $a:b$ is not in $(c:d)_\delta$ whenever $\delta < L$. This means that $m':n' \odot m'':n''$ is not contained in $(c:d)_\delta$. As $m':n'$ and $m'':n''$ were arbitrary intervals in the range of their oracles, the definition of the rule yields $R_{x \cdot y} (c:d, \delta) = \emptyset$. 

    \item Consistency. Let $a:b \in \mathbb{I}_{R_{x\cdot y}}$. This means there exists $c:d$ contained in $a:b$ such that $c:d = e':f' \cdot e'':f''$ for $e':f' \in R_x$ and $e'':f'' \in R_y$. By definition, $R_{x \cdot y}(a:b,\delta) = c:d$ in that case and it is not the empty set. 
        
    \item Closed.  Let $a_\delta \in \mathbb{I}_{R_{x\cdot y}}$ for all $\delta$. Let $a:b$ and $\delta$ be given. The task is to show $R(a:b, \delta) \neq \emptyset$. Let $c:d \in R_{x \cdot y}$ such that $c:d$ is contained in $a_\delta$. This exists by assumption. This implies that $c:d$ is contained in $(a:b)_\delta$. By definition, $R_{x \cdot y}(a:b,\delta) = c:d$ in that case and it is not the empty set.  
\end{enumerate}

Having established that these are oracles, it is necessary to check the field properties. 
\begin{enumerate}
    \item Commutativity. This follows from $a:b \odot c:d = c:d \odot a:b$ which in turn follows from the commutativity of $\cdot$ on the rationals. 
    \item Associativity. This follows from $(a:b \odot c:d) \odot e:f = a:b \odot (c:d \odot e:f)$ which in turn follows from the associativity of $\cdot$ on the rationals. 
    \item Identities. Since $a:b \oplus 0:0 = a:b$, the rule $R_0 (c:d, \delta) = 0:0$ for $c:0:d$ and $\emptyset$ otherwise leads to $R_x \oplus R_0 = R_x$. Thus, $x + 0 = x$ for all rational-betweenness relations $x$. Similarly, $a:b \otimes 1:1 = a:b$ leads to $x \times 1 = x$. 
    \item Distributive Property. As mentioned above, $( a:b \otimes ( c:d \oplus e:f) \subset (a:b \otimes c:d) \oplus (a:b \otimes e:f)$. This means that the oracle rule for both will not return the empty set unless the other one does as well. Thus, they generate the same betweenness relation and are therefore considered equal. 
    \item Additive Inverse.  The additive inverse for $x$ is given by the rule $R_{-x}(a:b, \delta) = \ominus (c:d)$ where $R_x (\ominus (a:b), \delta) = c:d$  and is the empty set if $R_x$ is the empty set for those inputs. It is an oracle basically because the rule returns the empty set in the same conditions as the original rule. For example, if $a:b \in \mathbb{R}_{-x}$, then $\ominus (a:b) \in \mathbb{R}_x$and thus $R_x(\ominus (a:b), \delta) \neq \emptyset$ implying $R_{-x} (a:b, \delta) \neq \emptyset$.

    To verify that this is the additive inverse, compute $x + (-x)$ by looking at $a:b \oplus c:d$ where $a:b \in R_x$ and $c:d \in R_{-x}$. For $c:d \in R_{-x}$, it must be the case that $-c:-d \in R_x$. The intervals $-c:-d$ and $a:b$ must intersect as they are both in the range of $R_x$. Let $p$ be a point of that intersection. Then $a:b \oplus c:d$ has $p + (-p) =0$ in its interval. Since these were random intervals in the ranges, it must be true that all intervals in the range of the sum rule must contain 0. This leads to the set of Yes intervals being those that contain 0, the same set as for the rooted oracle rule of 0. 
    
    \item Multiplicative Inverse. For the multiplicative inverse of $x$, it is slightly more complicated. This only applies if $x \neq 0$. Let $u:v$ be an $x$-Yes interval such that $0 \notin u:v$. Let $M < \min(|u|, |v|)$. 
    The easiest path is to consider the fonsi defined by $\{ a:b | 1 \oslash (a:b) \in R_x \wedge a:b \subset 1 \oslash (u:v) \}$. This defines a fonsi as the narrowing and intersection of the intervals in $R_x$ translates to that of their reciprocal nature. Indeed, $|1 \oslash (a:b)| = |a:b|/|a*b| < |a:b|/M^2$ as $(1/u : a : b: 1/v) < 1/M$ implying $1/(ab)< 1/M^2$. Thus, the interval clearly shrinks as $a$ and $b$ get closer to one another. Being a fonsi, there is an oracle associated with it. 

    Being the multiplicative inverse follows by multiplying the elements in the fonsi with the intervals of $R_x$. As with the additive inverse, given $a:b \in R_x$ and $c:d \in R_{1/x}$, there exists $p$ that is common to both $a:b$ and $1 \oslash (c:d)$. Thus, the multiplication of these intervals leads to $p *1/p = 1$ and, as this is true for all such intervals, this is equal to the oracle of 1. 

\end{enumerate}

This establishes that the oracles form a field. 


\section{The Real Numbers}

The oracles are ordered, complete, and a field. It needs to be shown that the arithmetic operations respects the ordering. To do this, it is sufficient to show that 1) $x + z < y +z$ implies $x<y$ and 2) $xy > 0 $ if $x >0 $ and $y >0$.

By definition of the inequality, there exists intervals $ a \xora[x+z] b$ and $c \xora[y + z] d$ such that $a:b < c:d$. Let $a' \xora b'$, $e' \xora[z] f'$, $c' \xora[y] d'$, and $e''  \xora[z] f''$ such that $a'\lte b'\oplus e' \lte f' \subset a \lte b$ and $c' \lte d' \oplus e'' \lte f'' \subset c \lte d$. Since $e':f'$ and $e'':f''$ are Yes intervals of the same oracle, their intersection is a Yes interval $e \lte f$ and can replace them both in the containment statements. Translating the containment statements, they become $a \leq a' + e \leq b' + f \leq b < c \leq c' + e \leq d' + f \leq d$. This implies $b' + f < c' + e$ and therefore $0 \leq f - e < c' -b'$. Therefore, $c' > b'$ and $a' \lte b' < c' \lte d'$.

The multiplication is a bit shorter. Let $x >0 $, $y >0$, $a \xora b$ such that $0<a \lte b$, $c \xora[y] d$ such that $0 < c \lte d$. Then $ac \xora[xy] bd$ and $0:0 < ac \lte bd$. Hence, the oracle $xy$ is greater than 0. 

The rationals are also dense in the oracles. Let $x < y$ be two given oracles. Let $a:b < c:d$ where $a \xora b$ and $c \xora[y] d$; this is allowed by the meaning of the inequality. Let $m$ be the average of $b$ and $c$. Then the oracle version of $m$, say $\widehat{m}$, satisfies $x < \widehat{m} < y$ as evidenced by $a:b < m:m < c:d$.

Thus, the oracles are a complete, ordered field with the rationals dense in it. These are the real numbers. 


\section{Concluding Thoughts}

The idea was quite simple: a real number is best represented by the intervals that contain it. To do this, there are two distinct levels. The first level is that of the oracle rules. These are computationally accessible. They are constructive in nature. Indeed, they are best thought of as a prescription as to what to compute and do rather than some given completed object. 

The second level is that of the betweenness relations. These are pure. They represent the intervals that contain the real number. In order to compute out all the Yes intervals for a given oracle, it may be necessary to check infinitely many intervals. This can mean that to state the relations is to delve into nonconstructivist territories. 

Both versions have their trade offs. In terms of practical uses, the rules are very implementable in, say, a computer program. The betweenness relations are very useful in theoretical uses. Some of the results in the arithmetic could have been smoother if the relations aspect was relied upon rather than the rules. The reason that was not chosen was to show how it works with the rules. 

A famous motivation for Dedekind was to prove that $\sqrt{2} \sqrt{3} = \sqrt{6}$. How might that go with the relations? The Yes intervals for $\sqrt{p}$ are those intervals $a \lte b$ that satisfy $\max(0,a)^2 : p : b^2$. Let $0:0 < a \lte b$ by a $\sqrt{2}$-Yes interval and $0:0 < c\lte d$ be a $\sqrt{3}$-Yes interval. Multiplication of them leads to $ac \lte bd$. Squaring the multiplication leads to $a^2 c^2 \lte b^2 d^2$. Since $a^2 \lte 2 \lte b^2$ and $c^2 \lte 3 \lte d^2$, multiplication leads to $a^2 c^2 \lte 6 \lte b^2 d^2$. Thus, $\sqrt{2} \sqrt{3} = \sqrt{6}$. This can obviously be generalized to any $n$-th roots. 

In general, using intervals is how arithmetic can be explored. The computation of $e + \pi$ can be explored with intervals. One would take Yes intervals of $e$ and Yes intervals of $\pi$ and then add them together. The intervals can be shrunk as small as one likes if one has the computational power to do it. There is nothing infinite in the computation up a certain desired level of precision. Using fractions allows for accurate precision. 

There are other works relevant to this idea being worked on by the author. The paper \cite{taylor24dedekind} explores the betweenness relations and how they relate to Dedekind cuts. The paper \cite{taylor23metric} extends this idea to give a new completion of metric spaces. Basically, closed balls replace the rational intervals. The separation property gets replaced by the property that given two points in a Yes ball, there exists a Yes ball that excludes at least one of them. There is also work, \cite{taylor23maudlin}, to extend this idea to the novel topological spaces of linear structures as described in \cite{maudlin}. 

Functions are also a topic of interest to extend oracles to. With the idea of a real number being based on rational intervals, it suggests that functions ought to respect that. Exploring the implications is the business of \cite{taylor23funora}. A comprehensive tome, \cite{taylor23main}, covers these various topics including comparing the other definitions of real numbers to this one and exploring its uses such as using mediants to compute continued fractions. 

\medskip

\normalem %restoring normal emphasis in bibliography 

\printbibliography

\end{document}
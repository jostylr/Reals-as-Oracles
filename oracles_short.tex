\documentclass[12pt]{article}
\usepackage{personal}
\usepackage{realoracles}


\newtheorem{theorem}{Theorem}[section]
\newtheorem{lemma}{Lemma}[section]
\newtheorem{corollary}{Corollary}[section]
\newtheorem{proposition}{Proposition}[section]



\title{Real Numbers As Rational Betweenness Relations}

\jtauthor
\date{\today}



%\sloppy%\openup-.1\jot
\begin{document}\maketitle
\begin{abstract}
Irrational numbers cannot be known in a precise way in the same fashion that rational numbers can be. But one can be precise about the uncertainty. The assertion is that if one knows whether or not a real number is in any given rational interval, then one can claim to know the real number. This insight leads to a two tiered definition of real number. The top tier is the idealized rational betweenness relation which details what properties a relation on rational numbers must satisfy to qualify as a real number. The other tier is the practical approach to real numbers. These are non-unique rules that give definite pathways to bringing forth the rational betweenness relations. Through these rules, it will be established that the rational betweenness relations satisfy the axioms of the real numbers. 
\end{abstract}

Real numbers are a fundamental part of mathematics. There are many different definitions, but they each have their own inadequacies. Some, such as decimals, Cauchy sequences, and series representation, emphasize computational aspects often at the costs of added complexity and non-uniqueness. Other approaches, such as Dedekind cuts and axiomatic definitions, offer little computational guidance. The definition in this paper is an attempt to offer a path towards computation while not requiring it for stating what the number is. As a guide in contemplating different notions, it is useful to consider how equality and arithmetic are handled, particularly how an irrational minus something that may be itself is computationally known to be 0 or not. 

The first step towards this idea is to think of a real number $x$ as providing a betweenness relation on rational numbers by $a$ and $b$ being $x$-related if $x$ is, inclusively, between $a$ and $b$. This almost works. But there are many situations in which one cannot definitively conclude that a given real number is between two particular rational numbers. The resolution of this is to have an underlying rule managing the information. This rule structure is then what the arithmetic is built on.

The rules under discussion have the property that given any interval of rational numbers, an answer can be provided as to whether there is a small interval containing the real number that intersects the interval in question. That small interval can then be used for further computations. The term oracle will be used for such rules to indicate a kind of prophesying which brings forth the prediction of the prophecy. The setup allows for the computation of intervals, but does not require the computations be done in order to define the real number. The main downside is that multiple rules can represent the same real number, but the ambiguity is nonconstructively removed at the higher level presentation of the $x$-relation viewpoint.  

This paper will first establish some basic notations for working with rational intervals before first defining the rational betweenness relations and then defining the properties that these rules must satisfy to represent a real number. A brief section on some examples will be given which will be light on details. See \cite{taylor23main} for a more in-depth discussion of examples, uses, and comparisons to other definitions of real numbers. Equality is discussed before several sections are explored in fully establishing the real number properties. 

This is certainly not the first approach to using interval containment for defining real numbers. There is a nested interval approach that dates back to the 1800s with Bachmann; see, for example, the overview of real number definitions by \cite{ittay-2015}. The constructivists, such as in \cite{bridger}, use an approach based on intervals being a measurement of the real number as if it was an experimental quantity. The approach of this paper is quite distinct from these other approaches, but is based on the same underlying idea that being contained in an interval is how a real number is fundamentally understood. All definitions of real numbers can be understood from this perspective though most definitions obscure this fundamental nature. 

\section{Interval Notations}

This section is in common to the other papers on this subject by this author, such as \cite{taylor24dedekind}.

The set of all rationals $q$ such that $q$ is between $a$ and $b$, including the possibility that $q=a$ or $q=b$, is a \textbf{rational interval} denoted by $a:b$. If $a=b$, then this is a \textbf{rational singleton} denoted by $a:a$. A singleton is a set of exactly one rational number, namely, $a$. To indicate $a \leq b$, the notation $a \lte b$ can be used. If $a < b$ and the interval is being presented, then the compacted notation $a \lt b$ may be used. 

The notation will also be used to indicate betweenness. If $a \leq b \leq c$ or $c \leq b \leq a$, then $a:b:c$ will be used to denote that. By definition, $c:b:a$ and $a:b:c$ represent the same betweenness assertion. This can be extended to any number of betweenness relations, such as $a:b:c:d$ implying either $a \leq b \leq c \leq d$ or $d \leq c \leq b \leq a$. There are also some trivial ways to extend given betweenness chains. For example, if $a:b:c$, then $a:a:b:c$ holds as well. Another example is that if $a:b:c$ and $b:c:d$, then $a:b:c:d$ holds true. This all follows from standard inequality rules for rational numbers. 

If $b$ and $c$ are between $a$ and $d$, but it is not clear whether $b$ is between $a$ and $c$ or between $c$ and $d$, then the notation $a:\{b,c\}:d$ can be used. This can also be extended to have, for example, $a:b:\{c,d\}$ which would be shorthand for saying that both $a:b:c$ and $a:b:d$ hold true. In addition, the notation \sout{$a:b:c$} will be used to indicate that $b$ is not between $a$ and $c$. One could also indicate this by $b:\{a,c\}$ which could be said in words that $a$ and $c$ are on the same side of $b$. 

Rational numbers satisfy the fact that, given three distinct rational numbers, $a, b, c$, exactly one of the following holds true: $a:b:c$, $a:c:b$, or $b:a:c$. That is, one of them is between the other two. It can be written in notation as $a:b:c$ holds true if and only if both \sout{$a:c:b$} and \sout{$b:a:c$} hold true. This follows from the pairwise ordering of each of them as provided by the Trichotomy property for rational numbers along with the transitive property. 

In this paper, often a potential relabeling will be invoked. This is to indicate that there are certain assumptions that are needed to be made, but they are notational assumptions and, in fact, some arrangement of the labels of that kind must hold. For example, if $\{a,d\}:b:c$ holds true, then either $a:d:b:c$ or $d:a:b:c$ holds true. If these are generic elements, then relabeling could be used to have $a:d:b:c$  be  true for definiteness, avoiding breaking the argument into separate, identical cases. If $a$ and $d$ were distinguished in some other way, such as being produced by different processes, then relabeling would not be appropriate to use. 

If $a:b:c:d$, then the union of $a:c$ with $b:d$ is the interval $a:d$. If $b \neq c$, then the union of $a:b$ and $c:d$ as a set is not an interval. One can still consider the \textbf{intervalized union} of $a:d$ as the shortest interval that contains both intervals. 

An \textbf{$a$-rooted} interval is an interval who has an endpoint that is $a$, that is, they are of the form $a:b$ for some $b$. An  \textbf{$a$-neighborly} interval is an interval that strictly contains $a$. The set of all $a$-rooted intervals will be denoted $\mathbb{I}_a$ while the set of all $a$-neighborly intervals will be denoted by $\mathbb{I}_{(a)}$

For rational $\delta > 0$, the notation $a_\delta$ represents the \textbf{$\delta$-halo of $a$} which is defined as $a -\delta : a+ \delta$. An interval is $a_\delta$ compatible if it is strictly contained in $a_\delta$ and is an $a$-neighborly interval. That is, the interval $c\lte d$ is $a_\delta$ compatible if $a- \delta < c < a < d < a+ \delta$. The halo can also extend to other intervals. The notation $(a:b)_\delta$ will refer to $a_\delta \cup a:b \cup b_\delta$. If $a \leq b$, this is the same as the interval $a-\delta:b+\delta$. 

The notation $b : |a_\delta$ will indicate the interval that goes from $b$ to the closest endpoint of $a_\delta$ to $b$ while the notation $b:a_{\delta}|$ will indicate that the interval goes from $b$ to the farthest endpoint of $a_\delta$ from $b$. If, for example, $b < a-\delta < a$, then $b:|a_\delta$ is the same as $b:a-\delta$ while $b:a_{\delta}|$ is the same as $b:a+\delta$. Also, $|a_\delta : b$ is the same as $b:a_\delta |$. This notation makes the most sense for $b$ outside of $a_\delta$. 

Given $m$ in $a:b$, a \textbf{subwidth} $\delta$ shall mean a positive rational number such that $m_\delta$ is strictly contained in $a:b$. For a subwidth to exist, it must be case that $m$ is strictly contained in $a:b$. 

The term subinterval of $a:b$ will include $a:b$ as a subinterval but it does not include the empty set as a subinterval. 

The set $\mathbb{I}$ will represent the set of all rational intervals. 

The length of $a:b$ will be denoted by $|a:b|$ and is equal to $|b-a|$.

Throughout the paper, unless noted otherwise, the letters $a$ through $w$ will represent rational numbers, $x, y, z, \alpha, \beta$ will represent real numbers, and $\delta, \varepsilon$ will represent positive rational numbers. Primes on symbols will be assumed to be of the same type. 

\section{Rational Betweenness Relations}

The definition of a \textbf{rational betweenness relation}, with relational symbol $\xrel$, is that it is a symmetric relation on rational numbers which satisfies the below properties. If $a \xrel b$, then the interval $a:b$ is said to be an $x$-interval. 
\begin{enumerate}
    \item Existence. There exists $a$ and $b$ such that $a\xrel b$.
    \item Interval Separation. If $a \xrel b$ and $c$ is strictly between $a$ and $b$, then exactly one of the following holds: 1) $a \xrel c$ and \sout{$c \xrel b$}, 2) $c \xrel b$ and \sout{$a \xrel c$}, or 3) $c \xrel c$. 
    \item Consistency. If $c : a : b : d$ and $a \xrel b$, then $c \xrel d$. 
    \item Singular. If $c \xrel c$ and $d \xrel d$, then $c=d$. 
    \item Closed. If $c$ is a rational number such that $c$ is included in every $x$-interval $a:b$, then  $c \xrel c$. 
\end{enumerate}

The set of all such relations when coupled with appropriate ordered field operations satisfies the axioms of the complete field of real numbers. This definition was shown to work in \cite{taylor24dedekind} by demonstrating that they are equivalent to Dedekind cuts. The focus of this paper is to establish these as a model of the real numbers using the practical approach of the oracle rules described in the next section. 

For a given a relation $\xrel$, it can be convenient to use the term Yes interval for intervals $a:b$ that satisfy $a \xrel b$ and the term No interval if \sout{$ a \xrel b$}. These terms will be reused in other sections to refer to intervals using oracle rules though it will eventually be established that the Yes intervals of an oracle representing $x$ coincide with the Yes intervals of $\xrel$.

It can be useful to recast the definitions above in the Yes/No interval language. Also of use is the term a \textbf{root of the relation} which is a rational number $c$ such that $c \xrel c$.

\begin{enumerate}
    \item Existence. There exists a Yes interval. Without this, one could have the null relation as a rational betweenness relation. This gives a starting point for further computations using Interval Separation. 
    \item Interval Separation. Any rational number $c$ strictly contained in a Yes interval will divide that Yes interval into two new intervals one of which will be Yes and the other will be No unless $c$ is a root of the relation in which case the two intervals and the singleton $c:c$ are Yes intervals. Interval Separation is inspired by the Intermediate Value Theorem and is the computational property allowing for a narrowing down of intervals. 
    \item Consistency. If an interval contains a Yes interval, then it is a Yes interval. This also implies that an interval contained in a No interval is a No interval. Consistency allows Interval Separation to rule out two disjoint separated Yes intervals. 
    \item Singular. There is at most one root of the relation. This prevents an interval of roots which is not ruled out by the interval separation property. Interval separation with consistency would rule out a finite number of roots, but not an interval of roots.  
    \item Closed. If a rational number $c$ is in every Yes interval, then $c$ is a root of the relation. This establishes a unique representative relation for rational numbers. 
\end{enumerate}

These properties ensure that two different rational betweenness relations cannot represent the same real number. 

Arithmetic can be defined using interval arithmetic, as will be discussed in the context of the rules, but there is a closure step that may be needed. For example, if $x$ is irrational, then $x - x = 0$ is still a statement that is desired, but the arithmetic of intervals will never produce any Yes intervals with an endpoint of $0$ let alone $0:0$. It will be the case that $0$ will always be present in every Yes interval and thus it is clear to close it by adding $0:b$ as Yes intervals. Since the mechanism to do so is also a convenient mechanism for how real numbers are actually used, it seems reasonable to explore that and establish arithmetic through that mechanism which is the mechanism of the oracles.  

\section{Oracles}

An \textbf{oracle} is a rule $R$, satisfying the properties listed below, which should be able to handle any input of the form of a rational interval along with a positive rational number. The output, which is not necessarily exclusively defined by the input, should be either a rational interval or the empty set. To make it into a function, one could view it as $R \mathbin{\col} \mathbb{I} \times \mathbb{Q}^+ \to \mathcal{P}(\mathbb{I} \cup \{\emptyset\})$. This would suggest, however, computing the full output of $R$ for a given input which is not something needed or desired. The expression $R(a:b, \delta) = c:d$ means that one of the outputs for that pair of inputs is $c:d$; $c:d$ can then be said to be in the range of $R$, denoted as $c:d \in R$. The whimsical language of calling the nonempty interval $c:d \in R$ a \textbf{prophecy} of the oracle may also be used. The expression $R(a:b, \delta) \neq \emptyset$ means that none of the outputs for that input pair is $\emptyset$. The notation $\mathbb{I}_R$ will be all of the intervals that have a subinterval in the range of $R$. 

To be an oracle, the rule must satisfy six properties.  The properties are:
\begin{enumerate}
    \item Range. 
    $R(a:b, \delta)$ should either be $\emptyset$ or a subinterval of $(a:b)_\delta$ which intersects $a:b$. 
    \item Existence. 
    There exists $a:b$ and $\delta$ such that $R(a:b, \delta) \neq \emptyset$.
    \item Separation. 
    If $a:b \in R$, then for a given $m$ contained in $a:b$ and given a $\delta$, there exists an $m_\delta$ compatible interval $e:f$ such that one of the following holds true:  $|a_\delta:e \in \mathbb{I}_R$, $e:f \in \mathbb{I}_R$,  or $f:b_{\delta}| \in \mathbb{I}_R$.
   \item Disjointness. 
   If $a:b \in R$ and $c:d$ is disjoint from $a:b$, then there exists a $\delta$ such that $R(c:d, \delta) = \emptyset$.
    \item Consistency. 
    If $a:b  \in \mathbb{I}_R$, then $R(a:b, \delta) \neq \emptyset$ for all $\delta$.
    \item Closed. 
    If $a_\delta \in \mathbb{I}_R$ for all $\delta $, then $R(a:b, \delta) \neq \emptyset$ for all $\delta$ and $b$. Such an $a$ is called a root of the oracle. 
\end{enumerate}

If multiple real numbers are being discussed, such as $x$ and $y$, then $R_x$ and $R_y$ will represent their respective rules. 

The rule $R$ may be thought of as a multi-valued function. The expression $R(a:b, \delta) \neq \emptyset$ implies that the empty set is not in the range of $R$ for that input. The expression $R(a:b, \delta) = \emptyset$ is implying that the range for that input into $R$ contains the empty set; it need not be exclusively the empty set.

The range of $R$ can be thought of as intervals that are known to include the real number based directly on its definition. The set $\mathbb{I}_R$ contains all the intervals that contain an interval in the range of $R$. Consistency ensures that the empty set is not assigned to any such interval and this is possible since all intervals in $\mathbb{I}_R$ contain an interval that can be returned. The Closed property covers an edge case dealing with describing rational numbers.

 For example, if approximating the square root of 2 and the interval is $1.3:1.4$ with a $\delta$ of $0.1$, then the rule computation might generate $1.39:1.42$, but another computation with the rule might generate $1.41:1.42$. Both $R(1.3:1.4, 0.1) = 1.39:1.42$ and $R(1.3:1.4, 0.1) = \emptyset$ would be valid outputs given these computations. Only the latter would be definitive in excluding $1.3:1.4$ from containing the square root. The interval $1.3:1.4$ would not be in $\mathbb{I}_R$ while $1.39:1.43$ would be as it contains $1.39:1.42$ which was a returned output of $R$. Note that $1.41:1.42$ would not work just yet as it was not returned by an output although presumably asking about a different interval that does include it would lead to the rule returning it. 

A rational interval $a:b$ is a \textbf{Yes interval} of the rule $R$ if $R(a:b, \delta) \neq \emptyset$ for all $\delta >0$. This includes all intervals that are in  $\mathbb{I}_R$, but it also includes the \textbf{$a$-rooted} intervals where $a$ is a root of the oracle.   A rational interval $a:b$ is a \textbf{No interval} if $R(a:b, \delta) = \emptyset$ for some $\delta > 0$. In a nonconstructive sense, each interval $a:b$ is either a Yes interval or a No interval. It is non-constructive since it requires a potentially infinite number of $\delta$s to check. 

If an interval $a:b$ is a Yes interval for $R$ and $R$ is to represent the real number $x$, then this can be expressed with the notation $a \xora b$. For No intervals, the notation is \sout{$a \xora b$}. The notation $a:|c_\delta$ extends to the Yes intervals as $a \xora |c_\delta$. These notations reflect that the Yes / No interval designation has created an $x$-betweenness relation on the rational numbers as will be established later.

It may be helpful to expand a little on what the properties mean. The basic idea is that a Yes interval ought to contain the real number; since this is defining the real number, this becomes more of a guiding idea, than a deduction. The concept of the rule is that a possible Yes interval is given along with a little error tolerance. The rule ought to respond with a Yes interval which helps move the process along in ascertaining what the real number is. If it cannot respond with a Yes interval, then the given interval is a No interval. 

Here is a bit of explanation for the properties:
\begin{enumerate}
    \item Range. Returning a subinterval of $a:b$ means $x$ is definitively in $a:b$ which is ensured by Consistency. Returning $\emptyset$ means that $x$ is not in $a:b$. The return of subintervals of $(a:b)_\delta$ that are not contained in $a:b$ are ambiguous on the question of $a:b$ containing $x$ though it is in the returned subinterval. The requirement to intersect $a:b$ is based on the fact that if the subinterval was disjoint from $a:b$ then Disjointness would say that $a:b$ does not contain $x$. This just helps bring that to the forefront. 
    
    \item Existence. Without this, the rule could always return the empty set. Such a rule would not represent any $x$. On a practical level, having a prophecy returned is the starting point of where to start narrowing in on the real number. It can be quite a large interval which makes this doable if some very rough knowledge of the number is known. 
    
    \item Separation. This property is inspired by the Intermediate Value Theorem. The idea is that a Yes interval should be able to be continually narrowed down by selecting a rational number $m$ inside of it and then testing which of the two created intervals is a Yes interval and the other one would then be a No interval. Because of the possibility of being able to not decide the issue at $m$, there is a $\delta$-halo extended in which to examine it. 
    
    It can also be the case that while $a:b$ is in the range of $R$, none of its subintervals are. For example, in representing the real number $2$, the range of $R$ could be $1:2$ along with intervals of the form $2:b$ with $b >2$. Intervals of the form $(a:2)_{\delta}$ would then contain, for example, the interval $2:(\delta/2)$ which is in the range of $R$. Picking $m=1.5$ and $\delta = 0.2$, an interval of $1.6:2.1$ would contain an element of the range. If it had to be strictly in the interval $1:2$, this rule would not satisfy the Separation property. 

    Having a bit of fuzziness outside of $a:b$ also allows $m$ to be one of the endpoints, say, $a$. Then $a_\delta = m_\delta$ and the subinterval of $|a_\delta : e$ would be disjoint from $a:b$ which would automatically imply by Disjointness that it is a No interval. But that is okay. The interval $e:f$ could still work to contain a prophecy as could $f:b_\delta|$.
 
    \item Disjointness. Disjointness ensures that a single real number is being discussed. Without an assertion of negativity, one could have multiple disjoint regions.  As an example, imagine a rule which returns small intervals around 2 and small intervals around 5. Done correctly, this will be able to satisfy the other properties, including the Separation property because that property only applies to prophecies. Consistency does not help as that is about intervals containing prophecies, but does not demand that there is a prophecy containing other prophecies.  
    
    \item Consistency. Consistency asserts that the oracle never contradicts itself. Since it is supposed to represent a real number being in the interval, if another interval contains it, then that containing interval would also contain the real number and ought to be a Yes interval. 
    
    \item Closed. This ensures that if there is a narrowing in to a single rational number, then the intervals with that rational number as endpoint are Yes intervals. It does not require that the interval in question be a prophecy nor contain one. The assumption that $a_\delta$ is in $\mathbb{I}_R$ for all $\delta >0$ implies that $R$ can always return a subinterval of $a_\delta$ for $R(a:b, \delta)$. Being Closed implies that $R$ does this or returns another suitable interval. 

\end{enumerate}

The term rule is being used instead of function to suggest a more constructive approach. While it is perfectly fine for an $R$ to be given as an explicit function, the more typical case is that $R$ is computed out as needed and may not return the same result for the same inputs. For example, if a rule is based on Newton's method restricted to a region of unique convergence but no specific starting point is given, then different choices of starting points could lead to different returned intervals. The design here is to allow for that variation. 

\subsection{Mountain Climbing a Real Number}

There are many different oracle rules for a given real number. The collection of all the Yes intervals is potentially unknowable from finite means. The practical exploration of a real number is the use of the rules to get increasingly narrower Yes intervals.  

A metaphor would be that of a mountain. A given Yes interval represents a slice of the mountain. The peak will be above the slice. An oracle rule is the scaffolding on the mountain to climb it. It does not represent a single path to go up the mountain, but it consists of the resources that allows one to climb it. 

A peak of a mountain is a singleton interval. This only occurs for rationals. For the sake of visualization, if the peak is $m/n$ in lowest terms, then the height of the mountain will be $n$. As one narrows the slicing of the mountains, generally the peaks that remain will get increasingly tall. 

Irrational real numbers will have infinitely high peaks. These are not reachable. But, with an ever-increasing amount of effort, one can go as high as one wants, resources permitting. For rationals, either the peak is achieved or one is scaling ever higher, but with a much lower peak in the slice. It is possible to have one end of a slice be that lower peak, but unable to confirm that it is the peak being sought. 

With actual, physical mountains, distinct mountains can be defined though there can be edge cases between multiple summits for a mountain versus multiple mountains. In the real number case, the mountains here have infinitely many other mountains surrounding them with infinitely high peaks being densely present. This is to give a sense of the difficulty of speaking of a real number. It is the oracle rules that allow one to stay on the mountain for the given real number. 

This is also to speak to what the spirit of seeking a definition of a real number ought to be. The climb is what is important. The mountain itself is the point, but without the climb, it is not distinct from the infinitely many other mountains clustered around it. It is the process of revealing itself which is at the heart of a real number. 

There are many rules for a given real number, just as there are many ways to ascend a mountain. A given rule is not the real number, but the real number is often not discernible without a rule. 

Ultimately, the real number ought to be taken as the completed set of Yes intervals for those formally inclined. But the attitude should not be to try to complete the set of Yes intervals. Instead, one should embrace the oracle rules and use them to produce as small an interval as desired. 

Two rules are equivalent if they produce the same Yes intervals. This is sufficient to allow the focus in developing arithmetic to be on developing useful rules. 

\subsection{Examples}

The first set of examples is that of rational numbers. Given a rational number $q$, one oracle rule, the \textbf{Singular Oracle at $q$}, is to have  $R(a:b, \delta) = q:q$ if $a:q:b$ and is the empty set otherwise. This is the nice version of a rational number. The \textbf{Reflexive Oracle at $q$} is the rule $R(a:b, \delta) = a:b$ if $q \in a:b$ and the empty set otherwise. A less nice version, but one more in line with what is produced from arithmetic operations on irrational numbers is the \textbf{Fuzzy Oracle at $q$} whose rule is that $R(a:b, \delta)$ returns $q_\delta$ if $q_\delta$ intersects $a:b$ and returns the empty set otherwise. A multi-valued version would be that given $a:b$ and $\delta$, consider the set of intervals of size $\delta$ or less that contain $q$ and select one. If it intersects $a:b$, then the interval is returned and, if not, the empty set is returned. All these versions have that the Yes intervals are the intervals that contain $q$ while the No intervals are those that do not contain $q$. The oracles are considered equal, as discussed in a later section. Oracles equal to them will be collectively referred to as the Oracle of $q$ or the Rooted Oracle of $q$.  

The next set of examples are the $n$-th roots. Let $x$ represent the positive real number such that $x^n$ ought to be the rational $q$. The comprehensive rule for these is that $R(a\lt b, \delta) = a:b$ whenever $\max(a, 0)^n:q:b^n$ and is the empty set otherwise.

A common situation for real number estimates is that the real number is computed by a sequence of intervals. For example, the $n$-th root can be computed using Newton's method. Let $a_0 >0$ be some positive rational. Given $a_m$, define $b_m = q/a_m^{n-1}$ and then $a_m^n : q : b_m^n$ will hold true. The iteration is defined as $a_{m+1} = a_m + (b_m - a_m)/n$. This is what results from applying Newton's method to $x^n - q$. The oracle rule $R$ would then be to compute $m, a_m, b_m$ such that $|a_m - b_m| < \delta$.  Then $R(a:b, \delta) = a_m:b_m$ if $a_m:b_m$ intersects $a:b$ and is the $\emptyset$ otherwise. Due to the length, $a_m:b_m$ will be in $(a:b)_\delta$ if it intersects $a:b$. 

This particular way of using Newton's method leads to a family of rules. Given a different $a_0$, the intervals computed will differ and will lead to some computed intervals overlapping the endpoints of some intervals that other starting points would not overlap with. The Yes / No determination, however, would not  change. If the root is the rational $p$, that is, $p^n = q$,  then intervals of the form $p:b$ will never be returned by this procedure unless $a_0 = p = b_0$. Nevertheless, the rule will never return the empty set for an interval that contains $p$. All Yes intervals in this case will contain $p$. 

More general uses of Newton's method can be discussed, but it is more efficient to discuss a general rule construct based on interval families. Such a rule is discussed in Section \ref{os:fonsis} about fonsis. 

Cauchy sequences can be viewed as pairs of rationals $a_n, \varepsilon_n$ where $a_n$ is the $n$-th element of the sequence and $\varepsilon_n$ is a bound for all future elements to be within $\varepsilon_n$ of $a_n$. This leads to the sequence of intervals $a_n-\varepsilon_n:a_n+\varepsilon_n = (a_n)_{\varepsilon_n}$. The Cauchy criterion ensures that the $\varepsilon_n$ exists and that they can be taken to approach 0. An oracle rule would then be that, for $n$ such that $\varepsilon_n < \delta$,  $R(b:c, \delta) = (a_n)_{\varepsilon_n}$ if $(a_n)_{\varepsilon_n}$ intersects $b:c$ and is the empty set otherwise. If $R(b:c, \delta) \neq \emptyset$, this implies that the entire tail of the sequence of intervals intersects $b:c$ and thus the sequence converges to a real number that is in $b:c$.  

For a set $E$ of rationals bounded above, the least upper bound can be defined as $R(a:b, \delta)$ to be $a:b$ when $a$ is bounded above by an element of $E$ and $b$ is an upper bound of $E$; it would be the empty set otherwise. For practical reasons, it may not be easy to determine this. Therefore, one could extend it such that if $a_\delta$ or $b_\delta$ contains both elements of $E$ and upper bounds, then $a_\delta$, respectively $b_\delta$, is returned. 

The last example of this section are the real numbers produced by the Intermediate Value Theorem. Let $f(x)$ be a function which is continuous and monotonic on $a:b$ with $f(a)*f(b) < 0$. For rational intervals $c:d$ contained in $a:b$, define the oracle $R$ by $R(c:d, \delta) = c:d$ if $f(c)*f(d) \leq 0$ and the empty set otherwise. For intervals that are not contained in $a:b$, take the intersection with $a:b$ and return its result. Monotonicity is required for the Disjointness property. Continuity is required to show that the oracle produced here is a zero of $f$. Without continuity, the most that can be said is that the oracle is the location of a sign change for the function. The initial $a:b$ yields the existence property. 

The above is assuming that a sign can be determined for $f(c)$. It is possible that if $f(c)$ is close to zero, something which is desired, then $f(c)$'s fuzziness will prevent a sign determination. In that case, the continuity of the function allows for the existence of a $\delta$ such that $f(c_{\delta})$ is contained in the interval $-\varepsilon:\varepsilon$ for a specified $\varepsilon$. Thus, one could specify an $\varepsilon$ tolerance and if $f$ is in that region, then $c_\delta$ is returned by the rule. Practically speaking, this is when a process will terminate, having hit the resolution limits of the computational system. 

Without monotonicity, one can still do the usual process of dividing up the interval and testing the signs to create a sequence of nested intervals. The sequence can be associated with an oracle. The oracle that is found would generally be path dependent on the choice of division points. 

\subsection{Small Yes Intervals}

To approximate a real number well, one needs a very small interval in which it is known to be in. Given an oracle, the existence and separation properties are the ingredients necessary to refine the intervals. This, and its implications, is what this section is about. 


\begin{proposition}[Bisection Algorithm]
    Given a rational $\varepsilon >0$ and an oracle $R$, there exists a prophecy whose length is less than $\varepsilon$.
\end{proposition}

\begin{proof}
    By Existence, there exists a prophecy $a_0:b_0$, that is, an interval in the range of $R$. If $|a_0:b_0| < \varepsilon$, then that interval is sufficient. Otherwise, the desired interval is found via iteration. The iteration step for determining $a_{i+1}$ and $b_{i+1}$ given $a_i:b_i$ begins with letting $m_i$ be the average of $a_i$ and $b_i$. Apply Separation to $a_i:b_i$ with $m_i$ the dividing point and choose a subwidth $\delta < \min(|a_i:b_i|/6, \varepsilon /2)$. By Separation and definition of $\mathbb{I}_R$, there exists an interval $e_i:f_i$ in $(m_i)_\delta$ and a prophecy $a_{i+1}:b_{i+1}$ such that one of the following holds: 
    \begin{enumerate}
        \item $a_{i+1}:b_{i+1}$ is in $e_i:f_i$. Since the length of $(m_i)_\delta$ is less than $\varepsilon$, this interval works as the desired interval.  
        \item  $a_{i+1}:b_{i+1}$ is in $|(a_i)_\delta:e_i$. The length will be no more than $2|a_i:b_i|/3$.
        \item  $a_{i+1}:b_{i+1}$ is in $f_i:(b_i)_\delta|$. The length will be no more than $2|a_i:b_i|/3$.
    \end{enumerate}
    The length of $a_n:b_n$ will be at most $(2/3)^n |a_0:b_0|$. To find the $n$ in which this procedure has definitively achieved the goal of $|a_n:b_n|<\varepsilon$,  take $\frac{p}{q} = \varepsilon/|a_0:b_0|$ and let $m$ be the number of digits in $q$ expressed in base 10. Then $n \geq  6m$ leads to  $(2/3)^n \leq (2/3)^{6m} < 10^{-m} < 1/q < p/q$. Thus, $n \geq 6m$ would have $|a_n:b_n| \leq (2/3)^n |a_0:b_0| < \varepsilon$ as desired. 
\end{proof}

Since every prophecy is a Yes interval, this immediately implies:
\begin{corollary}
    Given an oracle $R$, there are arbitrarily small Yes intervals. 
\end{corollary}



\begin{proposition}\label{os:yescat}
    Let an oracle $R$ be given. If $a:b$ is a Yes interval, then one of the following holds: $a:b \in \mathbb{I}_R$, $a$ is a root of $R$, or $b$ is a root of $R$.
\end{proposition}

\begin{proof}
    By definition of $a \xora b$, $R(a:b, \delta) \neq \emptyset$ for all $\delta > 0$. This implies that, by Disjointness, all prophecies must intersect $a:b$. If any of them are contained in $a:b$, then $a:b \in \mathbb{I}_R$ which satisfies what was to be shown. Thus, for the rest of this proof, assume all prophecies are not contained in $a:b$.

    Let $L = |a:b|$. By the Bisection Algorithm, there exists $c:d \in R$ such that $|c:d| < L/2$. Due to its length, $c:d$ cannot contain both $a$ and $b$. As $c:d$ is a prophecy, it must intersect $a:b$. For it not to be contained in it, it must overlap one of the endpoints. Let $q$ be that endpoint so that $q \in c:d$.
    
    Let $\delta < L/2$ be given. The task is to show that $q_{\delta} \in \mathbb{I}_R$. Let $e:f \in R$ be given such that $|e:f| < \delta$. By the same logic as before, the prophecy $e:f$ must contain one of the endpoints of $a:b$. Since it must intersect $c:d$ and the length of both intervals combined is less than $L$, the endpoint contained in $e:f$ must be $q$ as well. Since the length of $e:f$ is less than $\delta$, this implies that $e:f \subset q_{\delta'}$.

    Thus, $q$, one of the endpoints of $a:b$, is a root of $R$ as was to be shown in the case that $a:b \notin \mathbb{I}_R$.
\end{proof}

\begin{proposition}\label{os:singular}
    If an oracle $R$ has a root $q$, then that root is unique and an interval is a Yes interval exactly when $q$ is in the interval.
\end{proposition}

\begin{proof}
    There are three possible cases for how an interval $a:b$ can be related to $q$:

    \begin{enumerate}
        \item $q$ is an endpoint of $a:b$. Since $q$ is a root, any interval of the form $q:r$ is a Yes interval by the Closed property.
        \item $q$ is strictly contained in $a:b$. If $q \in a:b$ but is not an endpoint, then let $L$ be the distance from $q$ to the closest endpoint of $a:b$. Any interval $q_\delta$ with $\delta < L $ will then be contained in $a:b$ and hence $a:b \in \mathbb{I}_R$ since $q_\delta$ is. 
        \item $q$ is outside of $a:b$.  Let $L$ be the distance from $q$ to the closest endpoint. Then let $c:d$ be a prophecy contained in $q_{\delta}$ with $\delta < L$. This exists as $q$ is a root. Due to the length, $c:d$ is disjoint from $a:b$. Thus, Disjointness yields a $\delta'$ such that $R(a:b, \delta')= \emptyset$. Hence, $a:b$ is a No interval. 
    \end{enumerate}
    
    As for the uniqueness of the root, let $b$ be any rational not equal to $q$ and $\delta = \frac{|b-q|}{2}$. Then $b_\delta$ does not contain $q$ and hence is a No interval. This means it does not contain any prophecy and $b$ is not a root of the oracle. 
\end{proof}


\begin{proposition}\label{os:rootsmallpro}
    If $q$ is contained in arbitrarily small prophecies of an oracle, then $q$ is the root of the oracle. 
\end{proposition}

\begin{proof}
    Given $\delta$, let $a:b$ be a prophecy whose length is less than $\delta$ and contains $q$; this exists by assumption. Since $q$ is contained in $a:b$, $a:b$ is contained in $q_\delta$. Thus, $q_\delta \in \mathbb{I}_R$. As this was an arbitrary $\delta$, the Closed property applies to conclude  $q$ is a root of the oracle. 
\end{proof}

\begin{corollary}\label{os:root}
    If $q$ is contained in every prophecy of an oracle, then $q$ is the root of that oracle. 
\end{corollary}

\begin{proof}
    By the Bisection Algorithm, there exists arbitrarily small prophecies. $q$ is contained in them implying the proposition applies. 
\end{proof}



\begin{corollary}
    Given an oracle, $q$ is a root of the oracle if and only if $q$ is an element of every Yes interval. 
\end{corollary}

\begin{proposition}
    Given an oracle, $q$ is a root of the oracle if and only if $q$ is in arbitrarily small Yes intervals. 
\end{proposition}

\begin{proof}
   The one direction immediately follows the previous statements. For the other direction, assume that $q$ is in arbitrarily small Yes intervals. Let $\delta$ be given and let $a:b$ be a Yes interval whose length is $\delta$ and contains $q$. Then $a:b$ is contained in $q_\delta$. From above, either $a:b \in \mathbb{I}_R$ in which case $q_\delta$ is as well, or one of the endpoints is a root. Let $r$ be the endpoint that is the root. Then the distance from $r$ to $q$ is less than $\delta$; call that $L$. Let $L' = \delta - L > 0$. As $r$ is a root, $r_{L'} \in \mathbb{I}_R$ and it is contained in $q_\delta$ by the lengths. Thus, $q_\delta$ is in $\mathbb{I}_R$ in all cases and the result is established. 
\end{proof}


\subsection{Intersections}

The intervals, whether they are prophecies or the more general Yes intervals, ought to have the real number in question in those intervals. Thus, intersecting two such intervals should lead to a further narrowing towards the real number. This is indeed what happens. This section deals with various statements involving intersections and oracles. 


\begin{proposition}\label{os:prointer}
    Given an oracle $R$, any pair of prophecies $a:b$ and $c:d$ will intersect. 
\end{proposition}

\begin{proof}
    If they were disjoint, then the Disjointness property would say, for example, that $R(c:d, \delta)= \emptyset$ for some $\delta$. But $c:d \in \mathbb{I}_R$ as it contains itself. Thus, Consistency says that $R(c:d, \delta) \neq \emptyset$. This is a contradiction and the two intervals must intersect. 
\end{proof}

\begin{proposition}\label{os:yesinter}
    Given an oracle $R$, all Yes intervals intersect. 
\end{proposition}

\begin{proof}
    If $R$ is rooted at $q$, then $q$ is common to all Yes intervals. Hence, they intersect. 

    If $R$ is not rooted, then all Yes intervals contain at least one prophecy. Those prophecies intersect implying the intervals that contain them will intersect as well. 
\end{proof}


\begin{proposition}\label{os:inter}
    If $a:b$ is an interval that intersects every prophecy of an oracle, then, nonconstructively, $a:b$ is a Yes interval. 
\end{proposition}

\begin{proof}
    Let $\delta$ be less than half the length of $a:b$. By the Bisection Algorithm, there exists a prophecy $c:d$ whose length is less than $\delta$. By assumption, $c:d$ intersects $a:b$. This implies that either $c:d$ is contained in $a:b$, contains $a$ but not $b$, or contains $b$ but not $a$. If it is the first, then $a:b \in \mathbb{I}_R$ and is therefore a Yes interval. If not, let $q$ be the endpoint of $a:b$ that is contained in $c:d$. All prophecies whose lengths are less than $\delta$ will contain either $q$ or be contained in $a:b$; it cannot contain the other endpoint due to it needing to intersect $c:d$ and the two combined lengths are less than the length of $a:b$. If there is no prophecy that is a subinterval of $a:b$, then, nonconstructively, $q$ is contained in arbitrarily small prophecies. This in turn implies $q$ is a root of the oracle and, thus, $a:b$ is a Yes interval. 
\end{proof}

A useful idea is that every pair of disjoint intervals has at least one No interval in it. 

\begin{proposition}
    Given an oracle and two disjoint intervals $a:b$ and $c:d$, at least one of them is constructively known to be a No interval of that oracle. 
\end{proposition}

\begin{proof}
    As the intervals are disjoint, by potentially relabeling, it can be assumed $a:b:c:d$ with $b < c$. Then define $L = c-b > 0$. By the Bisection Algorithm, which is a finite constructive procedure, there is a Yes interval $e:f$ whose length is less than $L$. The interval $e:f$ must be disjoint from at least one of the intervals because of the length. The interval it is disjoint from is then a No interval by Disjointness. 
\end{proof}



   
\begin{proposition}
    If $ a\xora b$ and $c \xora d$, then the intersection of $a:b$ and $c:d$ is a Yes interval. 
\end{proposition}

\begin{proof}
    If there are Yes intervals that do not contain a prophecy, then by Proposition \ref{os:yescat}, it is a rooted oracle. If the oracle is rooted with root $q$, then the Yes intervals all contain $q$. Thus, they intersect and the intersection is an interval that contains $q$. Thus, it is a Yes interval as well. 

    The other situation is that all Yes intervals contain a prophecy. This will be assumed for what follows below. Note that there is no assumption that the oracle is not rooted; it can be the case that all Yes intervals contain prophecies and  the oracle is rooted. 

    Up to relabeling, the various cases are: 
    \begin{enumerate}
        \item $a:c:d:b$. In this case, the intersection is $c:d$ which is a Yes interval. 
        \item $a:b:c:d$ with $b \neq c$. This is the case of two disjoint Yes intervals. That was established to not be possible. 


        \item $a:(b=c):d$. In this case, $b=c$ is the intersection.  If $a=b$ or $c=d$, then the intersection is a Yes interval as it is one of the given intervals. Assume, therefore, that $a \neq b$ and $c \neq d$. 
        
        By assumption, $a:b$ and $c:d$ both contain prophecies. The prophecies must intersect and their intersection can only be the rational $b=c$. Since all prophecies must intersect both of these prophecies, they must all contain $b=c$. Any rational that is  contained in arbitrarily small prophecies is a root of the oracle by Proposition \ref{os:rootsmallpro}.
        
        \item $a:c:b:d$ with $b \neq c$. By assumption there is a prophecy $a':b'$ in $a:b$ and a prophecy $c':d'$ in $c:d$. The goal is to establish that there is a prophecy in $c:b$. This would then imply $c:b$ is a Yes interval. 
        
        If $a':b'$ is contained in $c:b$ or $c':d'$ is contained in $c:b$, then $c:b$ is a Yes interval. So assume not. Note that $a':b'$ and $c':d'$ must intersect as they are both prophecies. By potentially relabeling, it can be assumed that $a:a':c:c':b':b:d':d$ with $a'\neq c$ and $d' \neq b$. If $c'=b'$, then the case before can be adapted to claim that $c'=b'$ is the root of the oracle implying $c:b$ is a Yes interval. 

        Assuming $b' \neq c'$, let $L$ be less than half the distance between $b'$ and $c'$. Note that $b'_L$ and $c'_L$ are disjoint. Thus, at least one of them is a No interval. By potentially relabeling, let $c'_L$ be a No interval. 

        Use Separation on the prophecy $a':b'$ with separation point $c'$ and $\delta < L$ taken to be a subwidth of $a':b'$. Separation gives us a $c'_\delta$ compatible interval $e:f$ such that $a':\{e,c\}:c':f:b':b$  with one of $a':e$, $e:f$, or $f:b'$ being a Yes interval. As $a':e$ is disjoint from $c':d'$, it is a No interval. The interval $e:c':f$ is contained in $c'_\delta$ which is contained in the No interval $c'_L$. Thus, it is a No interval as well due to Consistency. This implies that $f:b'$ must be a Yes interval and, as it is contained in $c:b$, $c:b$ is a Yes interval as well. 
     \end{enumerate}
\end{proof}

\begin{corollary}
    If $a \xora b$, $b \xora c$ and $a:b:c$, then $b:b$ is a Yes interval.
\end{corollary}

\begin{proof}
    $b:b$ is the intersection of $a:b$ and $b:c$.
\end{proof}

A common trick is to look at small enough Yes intervals, or prophecies, such that they are all containd within a known interval. This is an immediate consequence of them all intersecting. 

\begin{proposition}\label{os:yescontain}
    Let $a:b$ be a Yes interval of a given oracle. Then any Yes interval of that oracle whose length is less than a given $\delta$ will be contained in $(a:b)_\delta$.
\end{proposition}

\begin{proof}
    Given any Yes interval $c:d$ whose length is $\delta$, it is the case that $c:d$ intersects $a:b$ as they are both Yes intervals of the same oracle. Since the maximum distance an element of $c:d$ can be from another element in $c:d$ is less than $\delta$, the maximum distance an element can be from $a:b$ is less than $\delta$. Thus, $c:d$ is contained in $(a:b)_\delta$.
\end{proof}

\section{Fonsis}\label{os:fonsis}

 A \textbf{Family of Overlapping, Notionally Shrinking Intervals} (fonsi, pronounced faan-zee,) is a set of rational intervals such that any pair of rational intervals in the set intersect and, given a rational $\varepsilon >0$, there exists at least one interval in the fonsi such that its length is less than $\varepsilon$. Sequences of nested intervals whose lengths approach 0 is an example of a fonsi. In constructivist works, such as \cite{bridger}, these objects are often taken as the definition of a real number and are likened to a set of measurements. 

Fonsis cover a range of examples. They can be thought of as the set of prophecies of an oracle without the machinery of rule assignment. 

\begin{proposition}
    Given a fonsi, there is an oracle associated with it such that all of the elements of the fonsi are Yes intervals. 
\end{proposition}

\begin{proof}
    Let $a:b$ and rational $\delta>0$ be given. Choose an interval $c:d$ from the fonsi such that $|c:d| < \delta$. 

    The rule $R$ will have that if $c:d$ intersects $a:b$, then $R(a:b, \delta) = c:d$; otherwise it is equal to the empty set. Given the length is less than $\delta$, the intersection with $a:b$ forces $c:d$ to be contained in $(a:b)_{\delta}$



    To show that $R(a:b, \delta) \neq \emptyset$, it is necessary to demonstrate that every element of the fonsi whose length is less than $\delta$ intersects $a:b$. 
    
    The intervals in the range of $R$ are precisely the elements of the fonsi. That the range is contained in the fonsi is clear from the definition. That every element of the fonsi is in the range follows by considering $R(c:d, |c:d|+1)$ for a given interval $c:d$ of the fonsi. Since $|c:d| < |c:d|+1$, $c:d$ itself matches the rule constraint and is therefore in the range of $R$. 

    The properties are verified as follows: 
    \begin{enumerate}
        \item Range. Satisfied by definition. 
        \item Existence. The fonsi must be non-empty since given a length, it is required to produce an interval. Let $a:b$ be any element of the fonsi and $\delta =1$. Let $c:d$ be any element of the fonsi whose length is less than 1. By definition of a fonsi, $c:d$ and $a:b$ intersect. Thus, $R(a:b, 1) 
        \neq \emptyset$.
        \item Separation. 
        Let $a:b \in R$, that is, it is an element of the fonsi. Let $m$ and a subwidth $\delta$ be given such that $a:m:b$. Let $c:d$ be an element of the fonsi whose length is less than $\delta$. Since $a:b$ and $c:d$ are in the fonsi, they intersect. This implies $c:d$ is contained in $(a:b)_\delta$. If $c:m:d$, then, because of the length being less than $\delta$, $c:d$ is wholly contained in $m_\delta$. Thus, take $m_\delta$ as the $e:f$ with $a:e:f:b$ and $e:f \in \mathbb{I}_{R}$ as required. 
        
        If $m$ is not in $c:d$, then, possibly via relabeling, $c:d$ will be contained within $|a_\delta:m$ with $c:d:m$ as a relabeling choice. Let $e$ be, say, the midpoint between $m$ and $d$ and $f$ be some number that is strictly contained in $m_\delta$ and $m:b$. By the containment of $c:d$, $|a_\delta:e$ is in $\mathbb{I}_R$.
        
        \item Disjointness. Let $a:b \in R$ and $c:d$ be disjoint from $a:b$. Let $\delta$ be less than the distance from $a:b$ to $c:d$. Let $e:f$ be any element of the fonsi whose length is less than $\delta$. Since $e:f$ must intersect $a:b$, it cannot intersect $c:d$ given the lengths. Thus, $R(c:d, \delta) = \emptyset$. 
    
        \item Consistency. Let $a:b \in \mathbb{I}_R$ which implies there exists an element $c:d$ of the fonsi which is a subinterval of $a:b$. Let  $ \delta$ be given. The task is to show $R(a:b, \delta) \neq \emptyset$. Let $e:f$ be any element of the fonsi whose length is less than $\delta$. Since $e:f$ and $c:d$ intersect as they are both in the fonsi, it is the case that $R(a:b, \delta) = e:f$. As $e:f$ was arbitrary given the length, $R(a:b, \delta) \neq \emptyset$. 
        
        \item Closed. Assume that $a$ is given such that $a_\delta \in \mathbb{I}_R$ for all $\delta$. Given $a:b$ and $\delta$, the task is to show $R(a:b, \delta) \neq \emptyset$. Let $c:d$ be an element of the fonsi whose length is less than $\delta$. The question is whether $c:d$ intersects $a:b$ or not. Suppose $a:b$ does not intersect $c:d$, then let $\delta'$  be less than the length from $c:d$ to $a:b$. This implies that $a_{\delta'}$ is disjoint from $c:d$. Since $a_{\delta'} \in \mathbb{I}_R$, there exists an interval $e:f$ in the fonsi which is contained in $a_{\delta'}$. But being an element of the fonsi, it must intersect $c:d$. As this is not possible if $a_{\delta'}$ is disjoint from $c:d$, it must be the case that $a_{\delta'}$ and $c:d$ intersect. As this is a contradiction of the choice of $\delta'$ whose existence is based on $c:d$ being disjoint from $a:b$, it must be the case that $c:d$ intersects $a:b$. Thus, $R(a:b, \delta) \neq \emptyset$. As a side note, since this is true for all $b$ and elements $c:d$ of the fonsi, it must be the case that $a$ is contained in every element of the fonsi. 
    \end{enumerate}

\end{proof}

A fonsi gives rise to oracles and the set of Yes intervals is a fonsi, in fact, a maximal fonsi. 


\begin{corollary}
    The prophecies of an oracle form a fonsi.
\end{corollary}

\begin{proof}
    The bisection algorithm establishes the notionally shrinking portion of the fonsi. That the intersection of the prophecies is nonempty was established in Proposition \ref{os:prointer}.
\end{proof}


\begin{corollary}
    The Yes intervals form a fonsi. 
\end{corollary}

\begin{proof}
    The containment of the prophecies handles the notionally shrinking. The intersection of the Yes intervals is the claim of Proposition \ref{os:yesinter}.
\end{proof}


\begin{corollary}
    The Yes intervals form a maximal fonsi.
\end{corollary}

\begin{proof}
A maximal fonsi is one in which no new interval can be added without it failing to be a fonsi. By Proposition \ref{os:inter}, all intervals that intersect every prophecy, which is contained in the Yes intervals, is a Yes interval. 
\end{proof}




\section{Oracles and Rational Betweenness Relations}

Each oracle is associated with a single rational betweenness relation, but a rational betweenness relation is associated with multiple oracles. Two oracles represent the same relation if all the elements of their respective ranges intersect. This section will establish these claims. 

\subsection{Oracles Give Rise to Rational Betweenness Relations}

The complete set of Yes intervals for an oracle is the rational betweenness relation associated with that oracle. Recall that $a \xora b$ exactly when $a:b$ is a Yes interval for the oracle; the notation $a \xrel b$ is used for the Yes intervals of the rational betweenness relation. Complete means in this context that the complement of the set of Yes intervals is the set of No intervals; there is no interval that is considered unknown. Practically speaking, this cannot necessarily be achieved.


\begin{lemma}[Consistency]\label{os:con}
    Let an oracle $R$ be given representing $x$. Assume $c:a:b:d$. If $a \xora b$, then $c \xora d$. If \sout{$c \xora d$}, then \sout{$a \xora b$}. 
\end{lemma}

\begin{proof}
    For $a \xora b$, either $a:b \in \mathbb{I}_R$ or, potentially by relabeling, $R$ is rooted at $a$. In the first instance, there is a prophecy contained in $a:b$. This implies $c:d$ contains $e:f$ since $c:d$ contains every interval contained in $a:b$. Thus, $c:d \in \mathbb{I}_R$ and hence $c \xora d$. 
    
    If $R$ is rooted at $a$, there are two cases. In the first case, $a$ is strictly contained in $c:d$. Then there is a positive distance $L$ from $a$ to the closest endpoint of $c:d$. Let $\delta < L$. Then $a_\delta$ is contained in $c:d$ and hence $c:d \in \mathbb{I}_R$. The other case is that $a$ is an endpoint of $c:d$, say, $ a= c$. Then $R(a:d, \delta) \neq \emptyset$ for all $\delta$ by the Closed property. Thus, $c \xora d$ in both cases.  

    With the nonconstructive assumption that $a:b$ is either a Yes interval or No interval and since being a Yes interval leads to $c:d$ being a Yes interval, if $c:d$ is known to be a No interval, then $a:b$ must be a No interval as well. 

    In terms of $R$, this is implying that $R(c:d, \delta) = \emptyset$ implies the existence of $\delta'$ such that $R(a:b, \delta') = \emptyset$. 
    
    
\end{proof}

The Separation property as postulated does not apply to all Yes intervals; this makes being an oracle easier to verify, but harder to use. Separation, however, can be extended to all Yes intervals.

\begin{lemma}[Separation]
    If $a: b$ is a Yes interval for an oracle $R$, $m$ is a rational number strictly between $a$ and $b$, and a subwidth $\delta > 0$ is given, then there exists an $m_\delta$ compatible interval $e:f$ such that one of the following holds true:
    \begin{enumerate}
        \item $a \xora e$, \sout{$e\xora f$}, \sout{$f\xora b$};
        \item $e \xora f$, \sout{$a\xora e$}, \sout{$f\xora b$};
        \item $f \xora b$, \sout{$a\xora e$}, \sout{$e\xora f$}.
    \end{enumerate}
\end{lemma}

\begin{proof}
    Yes intervals are either in $\mathbb{I}_R$ or they are rooted. 
    
    If it is a rooted interval, then relabeling allows for the root to be $a$ which implies $a_{\delta'} \in \mathbb{I}_R$ for all rational $\delta' >0$. With $a:e:b$, it is immediately the case that $a:e$ is an $a$-Rooted interval and hence is a Yes interval by the Closed property. Any $m_\delta$ compatible interval $e:f$ will then have the property that $a \xora e$ while \sout{$e\xora f$} and \sout{$f\xora b$} because $e:f$ and $f:b$ are disjoint from the Yes intervals $a_\delta$ for $\delta < |a:e|$.
 
    The other case is that of $a:b \in \mathbb{I}_R$. This consists of two cases. Let $c:d$ be a prophecy contained in $a:b$. 
    
    If $m$ is not in $c:d$, then, by possibly relabeling, $a:m:c:d:b$ with $m$ strictly contained in $a:c$.  Let $e:f$ be an $m_\delta$ compatible interval such that $a:e:m:f:c:d:b$ with $f \neq c$. Then $a:e$ and $e:f$ are disjoint from $c:d$ while $f:b$ contains $c:d$. Thus, $f \xora b$, \sout{$a\xora e$} and \sout{$e\xora f$}.

    If $m$ is in $c:d$, then apply the Separation property using  $\delta' < \delta$. If the produced $e':f'$ is in $\mathbb{I}_R$, then, as it is in $m_{\delta'}$ which is strictly contained in $m_\delta$, there exists an $m_\delta$ compatible interval $e:f$ that strictly contains $e':f'$. Such an interval $e:f$ would then lead to both $a:e$ and $f:b$ being disjoint from $e':f'$. Thus, $e \xora f$, \sout{$a\xora e$}, \sout{$f\xora b$}.

    If the produced $e':f'$ is not in $\mathbb{I}_R$, then, by relabeling, it can be assumed that $|c_{\delta'}:e' \in \mathbb{I}_R$. Then $e$ can be chosen to be strictly contained in $e':m$ so that \sout{$e\xora f$} and \sout{$f\xora b$} by Disjointness. If $a:c_{\delta'}:e$, then $a \xora e$. If not, then a non-constructive approach is needed.

    If there is any Yes interval contained in $a:e$, then that is sufficient to conclude $a:e$ is a Yes interval. So assume there is not; this is a nonconstructive step. This means that while every Yes interval must intersect both $a:b$ and $a:e'$, none of them can be contained in $a:e$. Intervals whose lengths are less than the distance between $e'$ and $b$ imply that $a$ is the common point of intersection for all of these intervals. As established in Corollary \ref{os:root}, this implies $a$ is a root of the oracle. Thus, $a \xora e$ is a Yes interval. 

\end{proof}

\begin{lemma}[Interval Separation]\label{os:intsep}
  If $a \xora b$, then one exactly one of the following holds: 
  \begin{enumerate}
        \item $a\xora m$, \sout{$m \xora b$};
        \item $m \xora m$; 
        \item $b \xora m$, \sout{$m \xora a$}.
    \end{enumerate}
\end{lemma}

\begin{proof}
Given any $\delta$, if $e:f$, as in the previous proof, is not the Yes interval, then Consistency and Disjointness leads to either the first or third outcome, depending on which one contains the Yes interval. 

If $e:f$ is the Yes interval for each $\delta$, then $m_\delta \in \mathbb{I}_R$ for all $\delta$. This is a nonconstructive step. The Closed property yields that $m$ is the root of the oracle and thus, $m \xora m$.
\end{proof}


\begin{theorem}
    Given an oracle $R_x$ and assuming that for every interval $a:b$, either $a \xora b$ or \sout{$a \xora b$} can be established, then $\xora$ is a rational betweenness relation. 
\end{theorem}


In what follows, properties from the oracle will be prepended with the term Oracle while those of the rational betweenness relations will have RBR prepended. 

\begin{proof}
    By assumption, $\xora$ is a relation on all pairs of rational numbers. It is symmetrical as $R(a:b, \delta) = R(b:a, \delta)$ for all intervals $a:b$ and positive rationals $\delta$.

    The properties are established as follows: 
    \begin{enumerate}
        \item RBR Existence. By Oracle Existence, there exists $a:b$ such that $R(a:b, \delta) \neq \emptyset$. This implies there exists $c:d$ such that $R(a:b, \delta) = c:d$. The interval $c:d \in \mathbb{I}_R$ which, by Oracle Consistency, implies $R(c:d, \delta) \neq \emptyset$ for all $\delta$.  
        \item RBR Interval Separation. This is Lemma \ref{os:intsep}.
        \item RBR Consistency. This is Lemma \ref{os:con}.
        \item RBR Singular. Since $c \xora c$ is equivalent to saying that $c$ is a root of the oracle, Proposition \ref{os:singular}  states that $c$ is unique. Thus, if $d \xora d$, then $d$ must be $c$. 
        \item RBR Closed. Assume $c$ is such that $a \xora b$ implies $c \in a:b$. This is equivalent to saying that $c$ is in every Yes interval. By Corollary \ref{os:root}, $c$ is a root of the oracle. By the Oracle Closed property, $R(c:c, \delta) \neq \emptyset$ for all $\delta$. Thus, $c \xora c$.
    \end{enumerate}
\end{proof}

\subsection{Equivalent Oracles}

Having established that a given oracle does have an associated rational betweenness relation, at least nonconstructively, this allows an equivalence between oracles based on having the same rational betweenness relation. Specifically, two oracles are \textbf{equivalent}, denoted $R_x \equiv R_y$, exactly when $\xora$ and $\xora[y]$ are equal as completed relations. Since the rational betweenness relation is unique for a given rule, a relation between oracles based on equality of the rational betweenness relation will automatically be reflexive, symmetric, and transitive. 

 \begin{proposition}
Let $R_x$ and $R_y$ be two oracles. All of the prophecies of $R_x$ will intersect all of the prophecies of $R_y$ if and only $R_x \equiv R_y$.
\end{proposition}

\begin{proof}\label{os:equal}
    If $R_x \equiv R_y$, then they have the same Yes intervals, which will include both sets of prophecies. Since all Yes intervals of a given oracle intersect each other, the two sets of prophecies intersect as well. 
    
    Assume that all of the prophecies of $R_x$ intersect all of the prophecies of $R_y$. The task is to show that a given interval $a:b$ is either a Yes interval of both $R_x$ and  $R_y$ or it is a No interval of both of them. Since all intervals are, nonconstructively, either Yes or No, it is sufficient to show that it is the case for just one type. The direction taken here is to show that all No intervals are the same. By relabeling, it is sufficient to show that an arbitrary No interval of $R_x$ is a No interval of $R_y$. 

    Let $a:b$ be a No interval of $R_x$. By the nonconstructive Proposition \ref{os:inter}, this implies there is a prophecy $c:d$ of $R_x$ such that $a:b$ and $c:d$ are disjoint. Let $L$ be the distance between $a:b$ and $c:d$. By the Bisection Algorithm, let $e:f$ be a prophecy of $R_x$  whose length is less than $L/2$. As it is a Yes interval of $R_x$, it must intersect $c:d$. Let $m:n$ be a prophecy of $R_y$ whose length is less than $L/2$. This intersects $e:f$ by assumption. Thus, all elements of $m:n$ are strictly contained in $(c:d)_L$. This means that $m:n$ is disjoint from $a:b$ and, hence, $a:b$ is a No interval of $R_y$ as was to be shown. 
\end{proof} 

The choice to pursue the No intervals in the proof is reflective of the fact that there is a finite amount of work in computing out that an interval is a No interval while there can be a potentially infinite amount of work to establish that an interval is a Yes interval. 

Because of the nature of having to affirm equality of infinitely many intervals, in practice, it may be hard to establish that two oracles are equivalent. Instead, one can say that they are \textbf{$a:b$ compatible} for a given interval $a:b$ if $a:b$ is a Yes interval for both oracles. All oracles share Yes intervals as can be seen by intervalizing the union of two of their separate Yes intervals. For example, if $a:b$ is a Yes interval of the one, $c:d$ is a Yes interval of the other and, by potentially relabeling, $a:\{b,c\}:d$, then the two oracles are $a:d$ compatible.  


 Given a rational number $q$, the Singular Oracle at $q$, the Fuzzy Oracle at $q$, and the Reflexive Oracle at $q$ are all equivalent as oracles. This is easy to see by considering their prophecies. 
 
 The prophecies of the Fuzzy Oracle at $q$ are all of the form $q:q$. The prophecies of the Fuzzy Oracle at $q$ are of the form $q_\delta$. The prophecies of the Reflexive Oracle at $q$ are of the form $a:b$ where $q \in a:b$. In all three oracles, every prophecy contains $q$. Thus, all the prophecies intersect and these are equivalent oracles. 

The associated rational betweenness relation is $a \xora[q] b$ if and only if $q \in a:b$. The above would constitute a proof of these claims if the rules in question were, in fact, established to be oracles, which, while not difficult, was not done here. See \cite{taylor23main} for a more full treatment of these oracles. 

The Oracle of $q$ will refer to any oracle which is equivalent to the Singular Oracle at $q$ which will be the canonical example. 

A particular example to explore is $(\sqrt{2})^2$. Let $R$ be the oracle of this number. It should be equal to the Oracle of 2. One rule can be the following. Given $a:b$ and $\delta$, considering $c:d$ such that $c^2:2:d^2$ and $|c^2:d^2|< \delta$. Then $R(a:b, \delta) = c^2:d^2$ if $c^2:d^2$ intersects $a:b$ and is the empty set otherwise. After showing that this is an oracle, it is, by definition, trivial to observe that all of the prophecies contain 2. Thus, this is the Oracle of 2. 

One can also ponder comparing $(\sqrt{2})^2$ to $(\sqrt{1.9\ldots93})^2$ where the number of 9s is, say, $10^{23}$. And let us say that this is in the context of solving $f(x) = 0$ for some function $f$ where one can use Newton's method to find the root, but not be able to explicitly produce it. It would be impossible to distinguish these computationally. This is an issue for all versions of real numbers though an interval approach has the advantage that it is explicitly giving the range of compatible numbers. Indeed, one can at least say that they are $1.9999999999:2.000000001$ compatible. 

There is also a slightly different approach that can be used to establish equivalence of rules. It is essentially that they agree on the No intervals. 

\begin{proposition}
    Let $R_x$ and $R_y$ be oracles such that whenever there is a prophecy of one of them that is disjoint from an interval $a:b$, then there exists a prophecy in the other oracle that is also disjoint from $a:b$. Then $R_x$ and $R_y$ are equivalent as oracles. 
\end{proposition}

\begin{proof}
    It is necessary to show that all the prophecies intersect. Let us suppose not and assume that $a:b$ is a prophecy of $R_x$ which is disjoint  from a prophecy $c:d$ of $R_y$. By the hypothesis, there is a prophecy of $R_x$ disjoint from $a:b$. But all prophecies of an oracle must intersect. Thus, $a:b$ must intersect $c:d$.
\end{proof}



\subsection{Rational Betweenness Relations Give Rise to the Reflexive Oracle}

The above has established that given an oracle, there is a rational betweenness relation associated with it. The other direction needs to be established as well. While there is not a unique oracle given a rational betweenness relation, there is a natural one. 

Given a rational betweenness relation $\xrel$, the \textbf{Reflexive Oracle of $x$} is defined by the rule $R(a:b, \delta)$ is $a:b$ if $a \xrel b$ and is the empty set otherwise. This rule is single-valued and the result is independent of $\delta$. As all the prophecies of $R$ are the Yes intervals of the relation, it should be clear that the oracle does generate this rational betweenness relation. 

It does have to be shown that the rule defined here is an oracle. One quick fact needs to be established first before proving the rule is an oracle. 

\begin{proposition}
    Given a rational betweenness relation $\xrel$, it is the case that if $a \xrel b$ and $c \xrel d$, then $a:b$ and $c:d$ intersect. 
\end{proposition}

\begin{proof}
    Assume they are disjoint. By potentially relabeling, it can be assumed that $a:b:m:c:d$ where $m$ is strictly contained in $b:c$. By RBR Consistency, $a \xrel d$. By RBR Separation using $m$, one of the following holds true: $a \xrel m$ with \sout{$m \xrel d$} or $d \xrel m$ with \sout{$m \xrel a$} or $m \xrel m$. By RBR Consistency, since $c \xrel d$, it is the case that $m \xrel d$. Also since $a \xrel b$, it is also the case that $a \xrel m$. Thus, $m \xrel m$. Next use RBR Separation on $a \xrel m$ with $b$ the Separation point. Since $a \xrel b$ by assumption and $b \xrel m$ by Consistency using $m \xrel m$, it must be the case that $b \xrel b$. But the RBR Singular property would then assert that $m = b$. This is a contradiction and these intervals must intersect. 
\end{proof}

\begin{proposition}
    Given the rational betweenness relation $\xrel$, the associated Reflexive rule is an oracle. 
\end{proposition}

\begin{proof}

    The first step is to describe the set $\mathbb{I}_R$. If $a:b \in \mathbb{I_R}$, then there must be an $x$-related interval $c:d$ contained in $a:b$. But by RBR Consistency, this implies $a \xrel b$. Thus, $a:b \in \mathbb{I}_R$ if and only if $a \xrel b$ if and only if $R(a:b, \delta) \neq \emptyset$ for all $\delta$. 

    \begin{enumerate}
        \item Oracle Range. Given $a:b$ and $\delta$, $R(a:b, \delta)$ is either $\emptyset$ or $a:b$ which certainly intersects $a:b$ and is a subinterval  of $(a:b)_\delta$.
        \item Oracle Existence. RBR existence yields an interval $a:b$ such that $a \xrel b$. Thus, $R(a:b, 1) = a:b$. 
        \item Oracle Separation. Let $a:b \in R$, i.e., $a \xrel b$, $m$ contained in $a:b$, and $\delta >0$ be given. If $m$ is an endpoint, then choose any $m_\delta$ compatible interval $e:f$. By potentially relabeling, assume $m =a$ and $|a_\delta:e:a:\{f,b\}:b_\delta|$. If $a:b:f$, then $e:f$ contains $a:b$ and is the required Yes interval. If $a:f:b$, apply RBR Separation to determine whether $a \xrel f$, $f \xrel f$, or $f \xrel b$. If is the first, then $e:f$ is a Yes interval. If is the latter two, then $f:b_\delta|$ can be taken as the Yes interval. 

        Assume $m$ is not an endpoint. Then $m$ is strictly contained in $a:b$ and RBR Separation applies to give three outcomes. If $m:m$, then choose any $e:f$ $m_\delta$ compatible interval contained in $a:b$ and that is the required Yes interval. If not, then either $a \xrel m$ or $m \xrel b$. By relabeling, assume $a \xrel m$ with \sout{$m \xrel b$}. Pick a $m_\delta$ compatible interval $e:f$ such that $a:e:m:f:b$. As $a \xrel m$, RBR Separation applies using $e$ as the separation point. If $a \xrel e$, then $|a_\delta:e$ is the required Yes interval. Otherwise, $e:f$ can be taken as the Yes interval based on the other two possible outcomes of the Separation property. 

        \item Oracle Disjointness. Assume $a:b \in \mathbb{I}_R$, i.e, $a \xrel b$, and that $c:d$ is disjoint from it. Since all Yes intervals of the relation intersect as shown above, $c:d$ is a No interval. By definition of the rule, $R(c:d, \delta) = \emptyset$ for all $\delta$. 
        \item Oracle Consistency. If $a:b \in \mathbb{I}_R$, then by definition of the rule, $R(a:b, \delta) = a:b$ for all $\delta >0$ and it is never equal to the empty set as was to be shown. 
        \item Oracle Closed. Assume $c_\delta \in \mathbb{I}_R$ for all $\delta$ which implies $c_\delta$ is an $x$-related interval of the relation. Let $a \xrel b$. If $c$ is an endpoint, then $c \in a:b$. Otherwise, let $L$ be the distance from $c$ to the closest endpoint. Consider $c_\delta$ for $\delta < L$. It cannot contain either of the endpoints. Thus, it is either contained in $a:b$ or disjoint from it. But since $c_\delta$ is an $x$-related interval, it must intersect the $x$-related interval $a:b$. Therefore, it is contained in it. The conclusion is that $c$ is in every $x$-related interval implying by RBR Closed that $c \xrel c$. This in turn implies by RBR Consistency that every interval that contains $c$ is a an RBR Yes interval and hence a Yes interval for the oracle. In particular, $R(c:b, \delta) \neq \emptyset$ for all $b$ and $\delta$. 
    \end{enumerate}
\end{proof}


It has now been established that every rational betweenness relation arises from at least one oracle and every oracle gives rise to one rational betweenness relation. It is possible to modify the rule above so that many different rules can give rise to the same rational betweenness relation. For example, instead of always returning $a:b$, it could return a smaller interval if $a:b$ is over a certain length. There are many other variations as well. 




\section{Establishing the Real Number Field}

In this section, the task is to establish that the rational betweenness relations with the appropriate inequality relations and field operations are a model for the complete real number field. The approach is to define these relations and operations on oracles and establish that they respect the equivalence relation on the oracles. While in a theoretical sense it seems as if this filters the operations from the RBR to the oracles and then back to the RBRs, in practice, it is oracles that are generally at hand to use. 

As a note, equality of RBR is as relations and equality of oracles is via the equivalence relation as already established. 

\subsection{Inequality}

Two oracles $R_x$ and $R_y$ can be distinguished if there are disjoint intervals $a:b$ and $c:d$ such that $a \xora b$ and $c \xora[y] d$. It is clear that the oracles are not equal as they do not have the same Yes/No intervals. Having separated intervals allows for a comparison of oracles. 

On the interval level, $a:b < c:d$ means that for every $p \in a:b$ and every $q \in c:d$, it is the case that $p < q$. Necessarily, the intervals must be disjoint to have this be possible and, any two disjoint intervals, will be related by an inequality. 

We define the operators as:
\begin{enumerate}
    \item $R_x < R_y$, or more briefly, $x < y$, if there exists $a \xora b$ and $c \xora[y] d$ such that $a:b<c:d$.
    \item $R_x > R_y$, or more briefly, $x > y$, if there exists $a \xora b$ and $c \xora[y] d$ such that $a:b > c:d$.
\end{enumerate}

If the intervals $a:b$ and $c:d$ were not disjoint, then equality is possible. If it can be shown that all Yes intervals of $x$ are not greater than any of the Yes intervals of $y$, then that is noted as $x \leq y$. This would translate to the interval level as for any $x$-Yes interval $a \lte b$ and any $y$-Yes interval $c \lte d$ that it must be the case that $a \leq d$. 

\begin{proposition}[Transitivity]
    If $x <y$ and $y<z$, then $x < z$.
\end{proposition}

\begin{proof}
    Let $a \xora b$, $c \xora[y] d$, $e \xora[y] f$, and $g \xora[z] h$ such that $a:b < c:d$ and $e:f < g:h$. The goal is to show that $a:b < g:h$. Let $m:n$ be the intersection of $c:d$ and $e:f$. This exists as $c:d$ and $e:f$ are Yes intervals of the same oracle. Let $p \in a:b$ and $q \in g:h$ and $r \in m:n$. By assumption, $p < r$ and $r < q$. By transitivity of inequality of rational numbers, $p < q$. As this holds for all $p \in a:b$ and $q \in g:h$, it is the case the $a:b <g:h$.
\end{proof}

It does need to be shown that there can be no contradiction. 

\begin{proposition}
    If $x < y$, then it is not true that $x > y$ and it is also not true that $x = y$.
\end{proposition}

\begin{proof}
    The issue is that the comparison is based on two particular Yes intervals. It needs to be shown that two other Yes intervals would not contradict this statement. 

    Let $a:b < c:d$ be given that exemplifies $x<y$. Then $a:b$ is disjoint from $c:d$. This implies that $a:b$ is a No interval for $y$ while $c:d$ is a No interval for $x$. Thus, the two oracles are not equal as their rational betweenness relations differ.

    The other task is to show there are no Yes intervals that yield the opposite inequality. Let $p:q$ be a Yes interval for $x$ and $r:s$ be a Yes interval for $y$. If $p:q$ and $r:s$ overlap, then there is no contradictory information. 

    Assume, therefore, that they are disjoint. Since $p:q$ must intersect $a:b$ and $r:s$ must intersect $c:d$, it must be the case that $p:q < r:s$. This follows as given any two disjoint intervals, one must be wholly less than the other and the intersections demonstrates which inequality holds. 

\end{proof}

The oracle inequality operation satisfies transitivity, relying on the transitivity of inequality for rational intervals which in turn relies on the transitivity of rational numbers. 

The classical story is that the real numbers satisfy the Trichotomy property: Given $x$ and $y$, exactly one of the following holds: $x<y$, $x>y$, or $x=y$. This holds non-constructively for the oracles. If there exists two disjoint Yes intervals, one for $x$ and one for $y$, then one of the inequality holds as was just explored. The other case is that every Yes interval of $x$ intersects every Yes interval of $y$. Proposition \ref{os:equal} establishes that $x=y$ in that situation. 
 

This was non-constructive as, generically, it requires potentially checking infinitely many intervals. 

\begin{corollary}
    Let $x$ and $y$ be two oracles. Then, nonconstructively, exactly one of the following holds true: $x<y$, $x>y$, or $x=y$.
\end{corollary}

The constructivists use a property called $\varepsilon$-Trichotomy. This allows a definite determination with a finite, predictable amount of work. 

\begin{proposition}[$\varepsilon$-Trichotomy]
    Given oracles $x$ and $y$ and a positive rational $\varepsilon$, exactly one of the following holds: $x<y$, $x>y$, or there exists an interval $a:b$ of length no more than $\varepsilon$ such that $a:b$ is a Yes interval for both $x$ and $y$.
\end{proposition}

\begin{proof}
    By the Bisection algorithm, there exists an $x$-Yes interval $c:d$ and a $y$-Yes interval $e:f$ such that both intervals have length less than $\varepsilon/2$. If $c:d$ and $e:f$ are disjoint, then the oracles are unequal with the inequality being that of the intervals. If $c:d$ and $e:f$ overlap, then their union is a Yes interval for both $x$ and $y$. That interval has length less than $\varepsilon$.
\end{proof}

It is also the case that given $x < y$, there exists a length such that all Yes intervals of $x$ and $y$ of that length are disjoint. 

\begin{proposition}
    If $ x< y$, there exists $\delta$ such that if $a \xora b$, $c \xora[y] d$, $|a:b| < \delta$ and $|c:d| < \delta$, then $a:b < c:d$
\end{proposition}

\begin{proof}
    Assume $e \xora f$ and $g \xora[y] h$ are such that $e\lte f < g \lte h$. Let $L = g-f$. Then $\delta = L/2$ will satisfy the requirements of the statement. 

    Let $a \xora b$ with $|a:b| < \delta$ and $c \xora d$ with $|c:d| < \delta$. By Proposition \ref{os:yescontain}, $a:b$ is contained in $(e:f)_\delta$ and $c:d$ is contained in $(g:h)_\delta$. By choice of $\delta$, $(e:f)_\delta$ and $(g:h)_\delta$ are disjoint and thus obey the same inequality as $e:f$ and $g:h$. Their subintervals are therefore disjoint and related in the same fashion. Thus, $a:b < c:d$. 
\end{proof}

In this section, the focus has been on the Yes intervals. It is also the case that if two oracles have Yes intervals that are disjoint, then there are prophecies of each that are disjoint from each other.

\begin{proposition}
    Given oracles $R_x$ and $R_y$ such that $a \xora b$ and $c \xora[y] d$ with $a:b$ disjoint from $c:d$, then there exists prophecies of $R_x$ and $R_y$ that are disjoint. 
\end{proposition}

\begin{proof}
    Choose $\delta$ to be less than half the distance from $a:b$ to $c:d$. Then $(a:b)_\delta$ and $(c:d)_\delta$ are disjoint. By Proposition \ref{os:yescontain}, any prophecy of $R_x$ whose length is less than $\delta$ will be contained in $(a:b)_\delta$; the Bisection Algorithm guarantees the existence of such prophecy and let that be $a':b'$. Similarly, any prophecy of $R_y$ whose length is less than $\delta$ is contained in $(c:d)_\delta$ and, by the Bisection Algorithm, let $c':d'$ be such a prophecy. Thus, the $R_x$-prophecy $a':b'$ and the $R_y$-prophecy $c':d'$ are disjoint. 
\end{proof}

Thus, all the computations about inequalities can be done solely with the prophecies which may be a more concrete set of intervals to inspect. For example, this immediately establishes that if $q < r$, then the Oracle of $q$ is less than the Oracle of $r$ using the fact that the Singular Oracle version of these oracles have the sole prophecies of $q:q$ and $r:r$, respectively. 

It is also important to establish that equivalent oracles have the same relation with non-equivalent oracles. 

\begin{proposition}
    Let $R_x \equiv R'_x$, $R_y \equiv R'_y$, and $R_x < R_y$. Then $R'_x < R'_y$.
\end{proposition}

\begin{proof}
    Let $a:b$ be a prophecy of $R_x$ and $c:d$ be a prophecy of $R_y$ such that $a:b < c:d$. This can be done by the previous proposition. Let $\delta$ be less than half the distance from $a:b$ to $c:d$. Let $a':b'$ be a prophecy of $R'_x$ such that $|a':b'|< \delta$. Let $c':d'$ be a prophecy of $R'_y$ such that $|c':d'| < \delta$. As equivalent oracles have all of their prophecies intersecting, it is the case that $(a:b)_\delta$ contains $a':b'$ and $(c:d)_\delta$ contains $c':d'$. By choice of $\delta$, these are disjoint and the inequality relation transfers. In particular, $a':b' < c':d'$ which implies $R'_x < R'_y$. 
\end{proof}

With the oracle inequalities explored, the rational betweenness relations can be ordered in the following way. The relation $\xrel$ is less than $\xrel[y]$ if an oracle representative $R_x$ of $\xrel$ is less than $R_y$ of $\xrel[y]$. Due to the equivalence proposition, different representatives will lead to the same conclusion. By considering the Reflexive oracle of the relation, it is clear that this inequality is the same one as obtained by saying that $\xrel < \xrel[y]$ if and only if there exists $a \xrel b$ and $c \xrel[y] d$ such that $a:b < c:d$. 

Letting $x$, $y$, and $z$ represent relations, the statement $x:y:z$ will mean that either $x \leq y \leq z$ or $z \leq y \leq x$. That is, $y$ is between $x$ and $z$. This notation can be extended as was done with rational numbers. 

\begin{proposition}
    If $a \xrel b$, then $\xrel[a] : \xrel : \xrel[b]$.
\end{proposition}

That is, if $a:b$ is a $x$-Yes interval, then $x$ is between $a$ and $b$.

\begin{proof}
    By the Bisection Algorithm, given a length $\delta$, there exists $c \xrel d$ such that $|c:d|<\delta$. Choosing $\delta$ to be less than the length of $a:b$, at least one of the endpoints is excluded; by relabeling, this can be chosen so that $b$ is excluded. This establishes that $\{a, x\} : b$. If $c:d$ excludes $a$ as well, then $a : c:d:b$ and then either $a:a < c:d < b:b$ or $b:b < c:d < a:a$. This establishes that as relations, $a : x : b$. 

    The other possibility is that there is no $\delta$ such that there is an $x$-interval excluding $a$. This implies that $a_\delta \in \mathbb{I}_R$ for all $\delta$. This implies that $a$ is the root of the oracle of $x$ and thus $x = a$ as relations which implies $a:x:b$. 
\end{proof}


\subsection{Completeness}

Completeness is a defining feature of real numbers. It can come in a variety of guises as wonderfully detailed by James Propp in \cite{propp}. While any of the equivalent versions could be used, this paper will go with the choice Propp suggests as a good foundation: the Cut property. It is a simplified and symmetrized version of the least upper bound property. 

For this section, the letters $x$ and $y$ will represent relations. That is, $x$ is used instead of $\xrel$. For rational numbers, the symbols such as $q$ will also represent the relation version $\xrel[q]$. Thus, rational $q \in A$ means the relation $\xrel[q]$ is an element of the set of relations $A$. 

\begin{theorem}[The Cut Property] 
Let $A$ and $B$ be two disjoint, nonempty sets of rational betweenness relations such that $A \cup B$ is the whole set of such relations. In addition, for all $x \in A$ and for all $y \in B$, it is the case that $x < y$. Then there exists a rational betweenness relation $\kappa$ such that whenever $x < \kappa < y$, it will be the case that $ x \in A$ and $y \in B$.
\end{theorem}

Note that if $a \leq b$ and $a-\delta \in B$, then $(a:b)_\delta$ is entirely contained in $B$. Similarly, if $b + \delta \in A$, then $(a:b)_\delta$ is entirely contained in $A$. If neither of these hold, then $a-\delta \in A$, $b+\delta \in B$, and $(a:b)_\delta$ has elements of both $A$ and $B$ in it. It is also useful to note that if $a \in A$, then $a-\delta \in A$ and if $b \in B$, then $b+\delta \in B$ for all lengths $\delta$.

\begin{proof}
This property can be established for the rational betweenness relations by using a rule which returns intervals that intersect both $A$ and $B$. In particular, given $a \lte b$ and a length $\delta$, if $a-\delta \in A$ and $b+\delta \in B$, then $R(a:b, \delta) = (a:b)_\delta$. Otherwise, the empty set is returned. The expansion beyond $a:b$ facilitates dealing with rational cut points as detailed by the Closed property. 

To establish this is an oracle, the properties follow by: 
\begin{enumerate}
    \item Range. By definition, a prophecy will be the interval $(a:b)_\delta$ which contains $a:b$. 
    \item Existence. Let $a \in A$ and $b \in B$. These exist by assumption. Thus, $R(a:b, \delta)= (a:b)_\delta$  and it is not the empty set.
    \item Separation. Let a prophecy $a \lte b$ be given. This implies $a \in A$ and $b \in B$. Let $m$ be given in $a:b$ and a length $\delta$ be given. If $m$ is strictly contained in $a:b$, consider $\delta' < \delta/2$ such that $m_\delta' \subset a:b$. Let $e = m-\delta'$ and $f=m+\delta'$. If $e \in A$ and $f \in B$, then $R(e:f, \delta') = e-\delta':f+\delta' \subset m_\delta$ which satisfies the requirement of Separation. If $ e\in B$, then $R(a:m-\delta, \delta') = a-\delta' : e$ and this satisfies the requirement. If $f \in A$, then $R(m+\delta:b, \delta') = f:b +\delta'$ and this satisfies the requirement.

    If $m$ is an endpoint with $n$ being the other endpoint, then consider $m:c:n$ with $c$ being $\delta/2$ away from $m$. Let $s$ be the sign required for $sc$ to get closer to $m$. If $c$ is in the same set as $m$, then $R(c:n, \delta/3) = c+s\delta/3 : n-s\delta/3$ will satisfy the requirement with $e=m+s\delta/3$ and $f=c+s\delta/3$. Then $f:n_\delta$ will contain the prophecy $c+s\delta/3 : n-s\delta/3$. If $c$ is in the same set as $n$, then $R(m:c, \delta/3) =  m+s\delta/3:c-s/\delta/3$ and that prophecy can be $e:f$ which will contain itself and is containd in $m_\delta$. 

    Note that the prophecies of this rule can never be singletons thus that case need not be handled. 
    \item Disjointness. If $a\lte b$ is a prophecy, then $a \in A$ and $b \in B$. If $c:d$ is disjoint from $a:b$, then let $\delta$ be less than the distance from $a:b$ to $c:d$. If $a:b < c:d$, then $(c:d)_\delta$ is entirely contained in $B$; if $a:b > c:d$, then $(c:d)_\delta$ is entirely contained in $A$. In either case, the rule returns the empty set. 
    \item Consistency. If $a \lte b \in \mathbb{I}_R$, then let $c\lte d$ be a prophecy in $a:b$. This means $c \in A$ and $d \in B$. Thus, $a-\delta < a \leq c < d \leq b < b + \delta$ implying $a-\delta \in A$ and $b+\delta \in B$. Thus, the rule returns $(a:b)_\delta$ and not the empty set. 
    \item Closed. Assume $a_\delta \in \mathbb{I}_R$ for all $\delta$. Let $b$ and $\delta$ be given. The task is to show that $R(a:b, \delta) \neq \emptyset$. By assumption, $a-\delta \in A$ and $a+\delta \in B$. If $b \leq a$, then $b-\delta \leq a- \delta$ and so $b-\delta \in A$. Thus, the rule returns $(a:b)_\delta$. If $b \geq a$, then $b+\delta \geq  a+\delta$ and so $b+\delta \in B$. Thus, the rule returns $(a:b)_\delta$. In all cases, the empty set is not returned and $a$ is a root of the oracle. 
\end{enumerate}

Having established that $\kappa$ is an oracle, it now has to be established that it satisfies being a cut point. Let $x < \kappa < y$. That statement implies the existence of the prophecies $a \xrel b$, $c \xrel[\kappa] d$, and $ e \xrel[y] f$ such that $a:b < c:d < e:f$. Being a prophecy of $\kappa$, $c \lte d$ has the property that $c \in A$ and $d \in B$. Since $c > b$, $b \in A$ and since $x < b$, $x \in A$. Similarly, $d < e$ implies $e \in B$ and $y > e$ implies $y \in B$.

\end{proof}

Having shown the Cut property, the rational betweenness relations satisfies all the equivalent completeness properties. 

The above assumes that membership in the sets $A$ and $B$ can always be determined. It is possible to handle a situation in which the boundary between $A$ and $B$ is fuzzier. The oracle rules are particularly helpful here as they come with a bit of fuzziness built in. The assumption on the sets is that given a fuzziness tolerance, any relation can be determined to be in one of the sets or in an interval whose endpoints are in the opposing sets from each other. 


\subsection{Arithmetic}

For arithmetic, the operators of addition and multiplication need to be defined. Then it needs to be shown that the usual properties hold including the existence of additive and multiplicative identities and inverses, as appropriate. 

The idea of arithmetic with oracles is to apply the operations to the intervals. The ideal approach would be that the Yes intervals of $x+y$ would be the intervals that result from adding Yes intervals of $x$ to those of $y$. This almost works, but it fails with, say, $\sqrt{2} - \sqrt{2}$. This ought to be 0. All the Yes intervals of $\sqrt{2}$ added to those of $-\sqrt{2}$ do lead to intervals that contain 0. But there is no interval of the form $0:b$ that is a result of subtracting two Yes intervals of $\sqrt{2}$. This is where having the oracle rules becomes very useful. The range of $R_{x+y}$ will consist of the result of combining the range of $R_x$ and $R_y$ with the arithmetic operator, but with that scenario, the omission of $0:b$ intervals is not an issue as they can be deduced in the step going from the range of $R$ to the Yes intervals. The Yes step is theoretically doable, but may not be always actionable. For example, if solving $f(x)=0$ with Newton's method to something that looks like $\sqrt{2}$ but cannot be proven as such, then $x - \sqrt{2}$ would be seen as compatible with 0, but could not be proven to be so via any finite computation. 

Interval arithmetic is largely that of doing the operation on the endpoints. For addition, $a \lte b \oplus c \lte d = (a+c) \lte (b+d)$. The length of the new interval is $|a:b \oplus c:d| = b+d - a- c = b-a+ c-d = |a:b|+|c:b|$. 

For multiplication, $a:b \otimes c:d = \min(ac, ad, bc, bd)\lte \max(ac, ad, bc, bd)$. For $0 \lte a \lte b$ and $0 \lte c \lte d$, the interval multiplication becomes $ac \lte bd$. For multiplication, the length is a bit more complicated. Let $M$ be an absolute bound for $a:b$ and $c:d$; an absolute bound on an interval is a number such that for any $p$ in the interval, $|p| \leq M$. Then $|a:b \otimes c:d| \leq |M|(|a:b| + |c:d|)$. This follows by considering cases. The first case is that of all four endpoints being involved. By relabeling in this case,  $ac \lte bd$ can be taken to be the multiplication interval leading to $bd - ac = bd - cb + cb - ac = b(d-c) + c (b-a) \leq M (|a:b| + |c:d|)$. The second case is that of three of the endpoints being repeated which implies one of them is repeated. By relabeling, $ac \lte ad$ can be taken to be the multiplication interval leading to $ad - ac= a(d-c) < M (|c:d|) \leq M (|c:d| + |a:b|)$. The third case is that of only two endpoints being used. By relabeling, $ac:ac$ can be taken to be the interval with a length of 0 which is less than or equal to any non-negative quantity.  That last case is just the scenario of multiplying two singletons. 

Most of the arithmetic properties hold for interval arithmetic, but the distributive property does not nor are there any additive and multiplicative inverses. The additive identity is $0:0$ while the multiplicative identity is $1:1$. Operating on subintervals of $a:b$ and $c:d$ yields a subinterval of the resulting interval operation on $a:b$ and $c:d$. 

The negation operator is $\ominus(a:b) = -a:-b$; this is not the additive inverse as $a:b \oplus (\ominus(a :b )) = (a-b):(b-a)$ which does contain 0, but not exclusively so. The reciprocity operator is $1 \oslash (a:b) = 1/a : 1/b$ though this only applies to intervals excluding 0. If 0 was included, with $a < 0$, the resulting set would be $-\infty:1/a \cup 1/b : \infty $ which is not an interval as used here.

The distributive property is replaced with $ a:b \otimes ( c:d \oplus e:f) \subset (a:b \otimes c:d) \oplus (a:b \otimes e:f)$. To see this, note that the left-side has boundaries chosen from $\{a(c+e), a(d+f), b(c+e), b(d+f)\}$ while the second has a boundary of the form $(\min(ac, ad, bc, bd) + \min(ae, af, be, bf) ) \lte (\max(ac, ad, bc, bd) + \max(ae, af, be, bf) )$. To demonstrate that they are indeed not equal for some examples, consider $2:3 \otimes ( 4:7 \oplus -6:-3) = 2:3 \otimes -2:4 = -6:12$ compared to $(2:3 \otimes 4:7) \oplus (2:3 \otimes -6:-3) = 8:21 \oplus -18:-6 = -10:15$. 

For more on interval analysis, see, for example, \cite{moore} or, in this context, \cite{taylor23main}.

Having discussed interval arithmetic, oracle arithmetic can now be defined based on that. The essential additional ability is that of being able to use narrower intervals. Much of the discussion will be the same for both arithmetic and multiplication. To facilitate that, the symbol $\odot$ will be used to represent a generic symbol for an operator operating on intervals and $\cdot$ will then be used for that same operator operating on individual numbers.

Let oracles $R_x$ and $R_y$ be given. The oracle $R_{x \cdot y}$ is defined by the following. Given $a:b$ and $\delta$, take an interval $c:d = (e:f) \odot (g:h)$ such that $e:f$ is a prophecy of $R_x$, $g:h$ is a prophecy of $R_y$, and $|c:d| < \delta$. Then $R_{x \cdot y}(a:b, \delta) = c:d$ if $c:d$ intersects $a:b$ and is the empty set otherwise. Note that if $c:d$ intersects $a:b$, it will be contained in $(a:b)_\delta$. 

The existence of an interval of the form $c:d$ follows from the computational bounds on the arithmetic operators as well as the Bisection Algorithm applied to $R_x$ and $R_y$. For addition, that choice is ensured by choosing the lengths of each to be less than $\frac{\delta}{2}$. For multiplication, let $M$ be a bound on given intervals of $R_x$ and $R_y$. Then choose intervals of $R_x$ and $R_y$ such that their lengths are less than $\frac{\delta}{2M}$. 

 A variant of this is to enlarge this to use the Yes intervals of $x$ and $y$; the choice to use the prophecies is to keep the arithmetic constructive. It will be the case that using equivalent rules leads to an equivalent rule for the result; this will be established below. 

The first step is to establish that these are oracles. 

\begin{enumerate}
    \item Range. This holds by definition. 
    \item Existence. Let $a:b \in R_x$ and $c:d \in R_y$. Then $R_{x\cdot y}(a:b \odot c:d, \delta) = a:b \odot c:d$ establishes Existence for the operator $\odot$. 
    \item Separation. Let $a:b$ be the result of $a':b' \odot a'':b''$ for $a':b' \in R_x$ and $a'': b'' \in R_y$. Let $m$ be strictly contained in $a:b$ and let $\delta$ be given. 
    
    %Because $m$ is strictly contained, there exists $m'$ and $m''$ such that $m = m' \cdot m''$ with $m'$ strictly contained in $a':b'$ and $m''$ strictly contained in $a'':b''$. 
    
    For $\odot = \oplus$, choose $\delta'$ and $\delta''$ such that $\delta' + \delta'' < \delta$, such as $\frac{\delta}{3}$ for both of them. 
    
    For $\odot = \otimes$, let $M$ be an absolute upper bound on the intervals $a':b'$ and $a'':b''$. Then choose $\delta'$ and $\delta''$ such that $M (\delta' + \delta'') + \delta' \delta'' < \delta$. Taking $M> 1$ and letting $\varepsilon = \min(1, \delta)$, then using $\frac{\varepsilon}{3M}$ for both $\delta'$ and $\delta''$ works as the $M$ terms combine to become $\frac{2 \varepsilon}{3}$ and the last term becomes $\frac{\varepsilon^2}{9M^2} \leq \frac{\varepsilon}{9}$. Thus the sum will be less than $\delta$.
    
    Choose intervals $c':d' \in R_x$ and $c'':d'' \in R_y$  such that their lengths are less than $\delta'$ and $\delta''$, respectively, as allowed by the Bisection Algorithm. Define $c:d = c':d' \odot c'':d''$ Then $|c:d| < \delta$ per the respective operators bounds. 
    
    Since $a':b'$ and $c':d'$ must intersect as both are in the range of $R_x$, it must be the case, given the length of $c':d'$, that $c':d'$ is contained in $(a':b')_{\delta'}$. Similarly, $c'':d'' \subset (a'':b'')_{\delta''}$. The goal is to show that $c:d \subset (a:b)_\delta$. Since the interval arithmetic operators preserve containment, it is the case that $c:d \subset (a':b')_{\delta'} \odot (a'':b'')_{\delta''}$. Thus, we are done if we can show that  $(a':b')_{\delta'} \odot (a'':b'')_{\delta''} \subset (a:b)_\delta$. Let $p' \in a':b'$, $|r'| < \delta'$, $p'' \in a'':b''$, and $|r''| < \delta''$.  The task is to show $(p'+r') \cdot (p''+ r'') \in (a:b)_\delta$. Note that $p' \cdot p'' \in a:b$ by definition of of these intervals. 

    For $\odot = \oplus$, the computation is $s = (p'+r') + (p''+ r'') = (p' + p'') + (r' + r'')$ and as $|r'+r''| < \delta$, it is definitional that $s \in (a:b)_\delta$. 

    For $\odot = \otimes$, $t = (p'+r')(p''+ r'') = p'p'' + p'r'' + r' p'' + r' r''$ . The first term is in $a:b$ and the need is to show that $|p'r'' + r'p'' + r' r''|< \delta$. For this, the triangle inequality applies along with the bound $M > |p'|, |p''|$ to yield $|p'r'' + r'p'' + r' r''| < M \delta'' + M \delta' + \delta' \delta'' < \delta$ where the last inequality follows from the definition of $\delta'$ and $\delta''$.  Thus, $t \in (a:b)_\delta $

    The claim of containment has been established. 

    If $m$ is contained in $c:d$, then since its length is less than $\delta$, it is contained in $m_\delta$ and it serves in the role of $e:f$ in the property. If $m$ is not in $c:d$, let $a:c:d:m$ by relabeling. Then take $e$ to be the average of $d$ and $m$. Let $f$ be on the other side of $m$ and within $m_\delta$. The interval $c:d$ is then wholly contained in $|a_\delta:e$, satisfying the property. 

    Note that the Separation property on the input oracles was used via the Bisection Algorithm to generate the intervals of sufficiently small length. 
    
    \item Disjointness. Let $a:b \in R_{x\cdot y}$ and $c:d$ disjoint from $a:b$. Let $L$ be the distance from $c:d$ to $a:b$; this is positive as they are disjoint. Take $\delta < L$. The claim is that $R_{x \cdot y} (c:d, \delta) = \emptyset$. 
    
    Let $a':b' \in R_x$ and $a'':b'' \in R_y$ be such that $a':b' \odot a'':b'' = a:b$; this is what being in the range of $R$ means for $a:b$. Choose $e':f' \in R_x$ and $e'':f'' \in R_y$ such that $e:f = e':f' \odot e'' : f'' $ and that $|e:f| < \delta$; this can be done using the operator bounds as done before.  There will be $p'$ and $p''$ such that $e':p':f'$, $a':p':b'$, $e'':p'':f''$, and $a'':p'':b''$. Thus, $p' \cdot p''$ will be in both $e:f$ and $a:b$. Since $|e:f|< L$, there exists no point $p$ which is in $c:d$. As $e:f$ is disjoint from $c:d$ and is not wholly contained in $(c:d)_\delta$, the rule for the operator yields the empty set. 

    The above also establishes that all intervals in $R_{x \cdot y}$ intersect. 

    \item Consistency. Let $a:b \in \mathbb{I}_{R_{x\cdot y}}$. This means there exists $c:d$ contained in $a:b$ such that $c:d = e':f' \odot e'':f''$ for $e':f' \in R_x$ and $e'':f'' \in R_y$. By definition, $R_{x \cdot y}(a:b,\delta)$ is allowed to be $ c:d$ for any $\delta$. Furthermore, if $g':h' \in R_x$ and $g'':h'' \in R_y$, then they intersect $e':f'$ and $e'':f''$. This implies $g':h' \odot g'' : h''$ intersects $c:d$. Thus, it can never be disjoint from $a:b$ and the result of $R(a:b, \delta)$ can never be the empty set. 
        
    \item Closed.  Let $a_\delta \in \mathbb{I}_{R_{x\cdot y}}$ for all $\delta$. Let $a:b$ and $\delta$ be given. The task is to show $R(a:b, \delta) \neq \emptyset$. The empty set can be a result only happen if there exists $c:d = c':d' \odot c'':d''$ such that $c:d$ is disjoint from $a:b$ where $c':d' \in R_x$, $c'':d'' \in R_y$. 
    
    Let $c:d$ be given and disjoint from $a:b$. Then there exists a $\delta'$ such that the distance from $c:d$ to $a:b$ is more than $\delta'$. This would imply that $a_{\delta'}$ is disjoint from $c:d$. By assumption, there exists an interval $e:f \in R_{x \cdot y}$ contained in $a_{\delta'}$. As $e:f$ is disjoint from $c:d$ and all intervals in $R_{x \cdot y}$ intersect as established above, $c:d$ is not in $R_{x \cdot y}$. Thus, the rule never yields the empty set for any input interval of the from $a:b$. 

    Practically speaking, this ensures that if one computes out $c':d' \in R_x$ and $c'':d'' \in R_y$ such that $c:d = c':d' \odot c'':d''$ has length less than $\delta$, then $c:d$ will be contained in $(a:b)_\delta$ which is all that is required. It will, in fact, also intersect $a:b$ as it should.
    
\end{enumerate}

One also has to establish that different rules for a real number lead to the same oracle under these operations. 

\begin{proposition}
    Let $R_x \equiv R'_x$ and $R_y \equiv R'_y$, then $R_{x \cdot y} \equiv R'_{x \cdot y}$.
\end{proposition}

\begin{proof}
The task is to show that the prophecies of $R_{x \cdot y}$ intersect the prophecies of $R'_{x \cdot y}$. By the equivalence, all the prophecies of $R_x$ intersect all the prophecies of $R'_x$ and all the prophecies of $R_y$ intersect the prophecies of $R'_y$. Let $a:b \in R_{x \cdot y}$ which means there exists $c:d \in R_x$ and $e:f \in R_y$ such that $a:b = c:d \odot e:f$. Similarly, let $a':b'\in R'_{x \cdot y}$ with $c':d' \in R'_x$, $e':f' \in R'_y$ such that $a':b' = c':d' \odot e':f'$. Then by assumption of the equivalences, there exists $p$ that is in both $c:d$ and $c':d'$ along with a $q$ in both $e:f$ and $e':f'$. Then $p \cdot q$ will be in both $a:b$ and $a':b'$. Thus, they intersect. 
\end{proof}

Having shown them equivalent, this means they represent the same completed rational betweenness relation.

Having established that these are oracles, it is necessary to check the field properties. 
\begin{enumerate}
    \item Commutativity. This follows from $a:b \odot c:d = c:d \odot a:b$ which in turn follows from the commutativity of $\cdot$ on the rationals. 
    \item Associativity. This follows from $(a:b \odot c:d) \odot e:f = a:b \odot (c:d \odot e:f)$ which in turn follows from the associativity of $\cdot$ on the rationals. 
    \item Identities. Since $a:b \oplus 0:0 = a:b$, the rule $R_0 (c:d, \delta) = 0:0$ for $c:0:d$ and $\emptyset$ otherwise leads to $R_x \oplus R_0 = R_x$. Let $0$ represent $\hat{R}_0$, that is, $0$ is the rational betweenness relation associated with $R_0$. Thus, $x + 0 = x$ for all rational-betweenness relations $x$. Similarly, $a:b \otimes 1:1 = a:b$ leads to $R_1$ and $x \times 1 = x$. This relies on the equivalence of using different representatives. 
    \item Distributive Property. As mentioned above, $( a:b \otimes ( c:d \oplus e:f) \subset (a:b \otimes c:d) \oplus (a:b \otimes e:f)$. This means that the oracle rule for both will not return the empty set unless the other one does as well. Thus, they generate the same betweenness relation and are therefore considered equal. 
    \item Additive Inverse.  The additive inverse for $x$ is given by the rule $R_{-x}(a:b, \delta) = \ominus (c:d)$ where $R_x (\ominus (a:b), \delta) = c:d$  and is the empty set if $R_x$ is the empty set for those inputs. It is an oracle basically because the rule returns the empty set in the same conditions as the original rule. For example, if $a:b \in R_{-x}$, then $\ominus (a:b) \in R_x$and thus $R_x(\ominus (a:b), \delta) \neq \emptyset$ implying $R_{-x} (a:b, \delta) \neq \emptyset$.

    To verify that this is the additive inverse, compute $x + (-x)$ by looking at $a:b \oplus c:d$ where $a:b \in R_x$ and $c:d \in R_{-x}$. For $c:d \in R_{-x}$, it must be the case that $-c:-d \in R_x$. The intervals $-c : -d$ and $a:b$ must intersect as they are both in the range of $R_x$. Let $p$ be a point of that intersection. Then $a:b \oplus c:d$ has $p + (-p) =0$ in its interval. Since these were random intervals in the ranges, it must be true that all intervals in the range of the sum rule must contain 0. This leads to the set of Yes intervals being those that contain 0, the same set as for the rooted oracle rule of 0. 
    
    \item Multiplicative Inverse. For the multiplicative inverse of $x$, it is slightly more complicated. This only applies if $x \neq 0$. Let $u:v$ be an $x$-Yes interval such that $0 \notin u:v$. Let $M < \min(|u|, |v|)$. 
    The easiest path is to consider the fonsi defined by $\{ a:b | 1 \oslash (a:b) \in R_x \wedge a:b \subset 1 \oslash (u:v) \}$. This defines a fonsi as the narrowing and intersection of the intervals in $R_x$ translates to that of their reciprocal nature. Indeed, $|1 \oslash (a:b)| = |a:b|/|a*b| < |a:b|/M^2$ as $(1/u : a : b: 1/v) < 1/M$ implying $1/(ab)< 1/M^2$. Thus, the interval clearly shrinks as $a$ and $b$ get closer to one another. Being a fonsi, there is an oracle associated with it. 

    Being the multiplicative inverse follows by multiplying the elements in the fonsi with the intervals of $R_x$. As with the additive inverse, given $a:b \in R_x$ and $c:d \in R_{1/x}$, there exists $p$ that is common to both $a:b$ and $1 \oslash (c:d)$. Thus, the multiplication of these intervals leads to $p *1/p = 1$ and, as this is true for all such intervals, this is equal to the oracle of 1. 

\end{enumerate}

Having established the field properties hold for oracles, this translates to rational betweenness relations establishing that they are a field. 


\subsection{The Real Numbers}

The rational betweenness relations are ordered, complete, and a field. It needs to be shown that it is an ordered field, that is, that the arithmetic operations respects the ordering. To do this, it is sufficient to show that 1) for all $x, y, z$ in the field, if $x<y$, then $x + z < y +z$; and 2) for all $x, y$ in the field, if $x >0$ and $y>0$, then $xy > 0 $.

By definition of the inequality, there exists intervals $ a \xora[x] b$ and $c \xora[y] d$ such that $a:b < c:d$. Thus, $c-b > 0$. By Bisection, there exists an interval $e \xora[z] f$ such that $0 \leq f-e < c-b$. Then $a+e \xora[x+z] b+f$ and $c+e \xora[y+z] d+f$. Since $a+e \leq  b+f$ and $c+e \leq d+f$, it suffices to show that $b+f < c+e$. This is equivalent to showing $f-e < c-b$, but that is true by choice of $e:f$. Thus, $x+z < y+z$. As there was nothing special about $x$, $y$, or $z$, this holds for all oracles. 

The multiplication is a bit shorter. Let $x >0 $, $y >0$, $a \xora b$ such that $0<a \lte b$, $c \xora[y] d$ such that $0 < c \lte d$. Then $ac \xora[xy] bd$ and $0:0 < ac \lte bd$. Hence, the oracle $xy$ is greater than 0. 

The rationals are also dense in the oracles. Let $x < y$ be two given oracles. Let $a:b < c:d$ where $a \xora b$ and $c \xora[y] d$; this is allowed by the meaning of the inequality. Let $m$ be the average of $b$ and $c$. Then the oracle version of $m$, say $\widehat{m}$, satisfies $x < \widehat{m} < y$ as evidenced by $a:b < m:m < c:d$.

As this holds for all oracles, it holds for the rational betweenness relations. Thus, they are a complete, ordered field with the rationals dense in it. These are the real numbers. 


\section{Concluding Thoughts}

The idea was quite simple: a real number is best represented by the intervals that contain it. To do this, there are two distinct levels. 

The first level is that of the oracle rules. These are computationally accessible. They are constructive in nature. Indeed, they are best thought of as a prescription as to what to compute and do rather than some given completed object. 

The second level is that of the betweenness relations. These are pure. They represent the intervals that contain the real number. In order to compute out all the Yes intervals for a given oracle, it may be necessary to check infinitely many intervals. This can mean that to state the relations is to delve into nonconstructivist territories. 

Both versions have their trade offs. In terms of practical uses, the rules are very implementable in, say, a computer program. The betweenness relations are very useful in theoretical uses. Some of the results in the arithmetic could have been smoother if the relations aspect was relied upon rather than the rules. The reason that was not chosen was to show how it works with the rules. 

A famous motivation for Dedekind was to prove that $\sqrt{2} \sqrt{3} = \sqrt{6}$. How might that go with the relations? The Yes intervals for $\sqrt{p}$ are those intervals $a \lte b$ that satisfy $\max(0,a)^2 : p : b^2$. Let $0:0 < a \lte b$ be a $\sqrt{2}$-Yes interval and $0:0 < c\lte d$ be a $\sqrt{3}$-Yes interval. Multiplication of them leads to $ac \lte bd$. Squaring the multiplication leads to $a^2 c^2 \lte b^2 d^2$. Since $a^2 \lte 2 \lte b^2$ and $c^2 \lte 3 \lte d^2$, multiplication leads to $a^2 c^2 \lte 6 \lte b^2 d^2$. Thus, $\sqrt{2} \sqrt{3} = \sqrt{6}$. This can obviously be generalized to any $n$-th roots. 

In general, using intervals is how arithmetic can be explored. The computation of $e + \pi$ can be explored with intervals. One would take Yes intervals of $e$ and Yes intervals of $\pi$ and then add them together. The intervals can be shrunk as small as one likes if one has the computational power to do it. There is nothing infinite in the computation up a certain desired level of precision. Using fractions allows for accurate precision. 

There are other works relevant to this idea being worked on by the author. The paper \cite{taylor24dedekind} explores the betweenness relations and how they relate to Dedekind cuts. The paper \cite{taylor23metric} extends this idea to give a new completion of metric spaces. Basically, inclusive balls replace the inclusive rational intervals. The Separation property gets replaced by the property that given two points in a Yes ball, there exists a Yes ball that excludes at least one of them. There is also work, \cite{taylor23maudlin}, to extend this idea to the novel topological spaces of linear structures as described in \cite{maudlin}. 

Functions are also a topic of interest to extend oracles to. With the idea of a real number being based on rational intervals, it suggests that functions ought to respect that. Exploring the implications is the business of \cite{taylor23funora}. A comprehensive tome, \cite{taylor23main}, covers these various topics including comparing the other definitions of real numbers to this one and exploring its uses such as using mediants to compute continued fractions. 

\medskip

\normalem %restoring normal emphasis in bibliography 

\printbibliography

\end{document}
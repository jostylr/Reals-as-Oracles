\documentclass[12pt]{article}
\usepackage{amsfonts,amsmath,amssymb,mathrsfs,amsthm}
\usepackage{realoracles}
\usepackage[hidelinks]{hyperref}
% \usepackage[pagewise]{lineno}\linenumbers   %AIMS Mathematics Recommendation
\usepackage[backend=biber,style=alphabetic,sorting=ynt]{biblatex}
\addbibresource{bib.bib}

%Dangerous, possibly remove if weird
\interfootnotelinepenalty=10000 

\title{Defining Real Numbers as Oracles, an Overview}
\author{
  James Taylor\footnote{Arts \& Ideas Sudbury School, 4915 Holder Ave, Baltumore, MD 21214, james@aisudbury.org \\
  \hspace*{15px} Johns Hopkins University, Lecturer, Advanced Academic Programs, james.taylor@jhu.edu}
}
\date{January 31, 2023}

\addtolength{\textwidth}{2.0cm}
\addtolength{\hoffset}{-1.0cm}
\addtolength{\textheight}{3.0cm}
\addtolength{\voffset}{-1.5cm}


\newtheorem{theorem}{Theorem}
\newtheorem{lemma}{Lemma}
\newtheorem{corollary}{Corollary}
\newtheorem{proposition}{Proposition}

\theoremstyle{remark}
\newtheorem{remark}{Remark}



%\sloppy%\openup-.1\jot
\begin{document}\maketitle
\begin{abstract}
A new definition of a real number is that it is a rule which says Yes or No based on whether the real number ought to be in a given rational interval. This is a summary paper for formalizing, exploring, and generalizing this definition. The full exploration is given in the paper ``Defining Real Numbers as Oracles''. 
\end{abstract}



\section{What is a real number?}

The question of how to define a real number has a long history in mathematics. There is a notion that it was settled 150 years ago with the introduction of Dedekind cuts and Cauchy sequences as real numbers. The former has the issue of not being very helpful in computations while the latter is rather too involved in the computations. Neither definition seems to have influenced the thinking of anyone with regards to what a real number is. Indeed, it is quite common to take real numbers as axiomatically given rather than delve into these definitions.

The first, and perhaps most lasting, notion of what a real number is may very well be its decimal expansion. It is problematic as the full decimal expansion cannot be written down and yet the definition seems to suggest that is what it is. The expansion is generally infinite and it is generally not easy to specify a pattern for the digits. As a compromise, a sufficient number of digits is written down to do the computations given one's constraints though the arithmetic operations tend to expand the errors while the resulting computation produces more digits in the decimal expansion, suggesting the opposite to the uninitiated.

Upon reflection, one way of viewing what is actually needed, given finite limitations, is an interval which is sufficiently small that the calculated resulting interval is sufficiently small for what the actual need is. The core task then seems to be able to produce intervals as small as desired. Arithmetic operations will expand the length of the resulting intervals, but the loss of precision is embedded in the interval presentation. 

One approach might be to specifically view a real number as something that produces these intervals. The problem is that the production of an interval representative requires a choice which is external to the real number. The approach here is different. The approach involves no arbitrary choice in defining a real number. It does, however, make central a mechanism for producing a useful interval of any length once an initial interval is given.\footnote{Even if one were to want to define the real numbers as being a function that produced approximations given a starting interval, there would still be the requirement to say that the starting interval contains the real number.}

The idea is that a real number is an oracle that, when given a rational interval, says Yes or No. If it says Yes, then the real number is to be considered to be in that interval. If it says No, then the real number is to be considered to not be in that interval. The definition will formalize what ought to be true about the oracle's answers given that it is describing a single number. The main paper establishes that this definition does lead to the real number field. This paper gives the definition and then highlights the results with some examples of how to use it. This definition balances the need to be computationally friendly while defining an existence independent of computation. 

This is not the first time that a definition of real numbers as rational intervals has been entertained. For example, both \cite{bridger} and \cite{bridges} use the notion of a family of intersecting and arbitrarily fine intervals to define a real number; this is what this paper refers to as a fonsi. They use equivalence relations to handle different families representing the same real number while the approach here, for a fonsi, would be to take the maximal fonsi as the unique definition of a real number. The fonsi definition of real number fits well with constructivist ideas. It was apparently motivated by the idea of measuring physical quantities and noting that an interval is always returned from that procedure. 

This is an overview of the main paper, ``Defining Real Numbers as Oracles'' \cite{taylor23main}. It contains the full details of what is glossed over in this paper as well as several numerical examples of oracles in practice. 

\section{Defining the oracles}\label{sec:ora}

An oracle is a rule, satisfying a few properties below, that takes in a inclusive rational interval and returns a 1 or a 0. The 1 should be returned when an interval ought to contain the real number.\footnote{Set theoretically, an oracle is a function $R \col \ \mathbb{Q}^2 \to \{0,1\}$ where each pair $(a,b)$ is identified with the  the inclusive rational interval $a:b$. This viewpoint is suppressed here as the goal is to provide a framework for just-in-time determination. Another way of viewing the rule is that it is the characteristic function of the set of intervals that contain the real number, a viewpoint which is also not adopted here.} This includes the inclusive rational intervals consisting of exactly one point which may be called a singleton. The main paper uses the unconventional, but useful, notation $a:b$ to represent the inclusive rational interval since the notation $[a,b]$ suggests a real-filled interval. There is also no implication that $a < b$.  Indeed, the singleton can be written as $a:a$. The notation $a\lte b$ is used to indicate $a\leq b$ when knowing which is the lower bound is important with equality being allowed. This paper will primarily follow the shortened notation. 

In what follows, Yes represents a 1 and No represents a 0 result for the interval. The following are the properties that a rule $R$ must satisfy to be an oracle. 

\begin{enumerate}
    \item Consistency. If an interval contains a Yes interval, then it needs to be a Yes interval. This maintains the illusion of the real number being in the Yes intervals and also leads to this being a unique representative of a real number. 
    \item Existence. There should be a Yes interval. All other properties are conditional so this is the only one that says there is something there. 
    \item Closed. If a rational number is in every Yes interval, then its singleton should also be a Yes interval. This is the other part in making sure that there is only one representative of the real number. It is a primary reason for using inclusive intervals. This also makes arithmetic with rationals have an easy version.
    \item Rooted. There is at most one Yes singleton.  This helps ensure that the oracle is narrowed in on a single number. 
    \item Interval Separation. This is the key property. For a given Yes interval, a point $c$ inside that interval creates two subintervals in addition to its own singleton. The requirement is that exactly one of those intervals is Yes unless $c:c$ is a Yes singleton, in which case all three are Yes since $c$ is in all three.  
\end{enumerate}

If $q$ is contained in every Yes interval, then $q$ is the \textbf{root of the oracle}.  An oracle with a root may be called a rooted oracle or a singleton oracle. Rooted oracles are the rational real numbers. If an oracle is not rooted, then it may be called a neighborly oracle. Neighborly oracles are the irrational real numbers. 

While the first three properties above seem to be basic requirements, there is flexibility on the last two. They can be replaced with equivalent properties.

The first equivalent replacement would be the two properties: 1) Two Point Separation Property, which states that given any two rational numbers in a Yes interval, there exists a Yes interval that does not contain at least one of the two numbers, and 2) Disjointness Property, which states that if two intervals are disjoint, then at most one of them can be a Yes interval. These two properties are more useful in generalizing to metric spaces, where inclusive balls are considered instead of inclusive intervals. The interval separation does not generalize to that context, but two point separation does. The paper \cite{taylor23metric} gives more details about metric completion using oracles. 

The second equivalent replacement would be the two properties: 1) Narrowing Property, which states that given a rational positive length, there is a Yes interval that is shorter than that length, and 2) Intersection Property, which is that all Yes intervals intersect. If a set of intervals satisfy these two properties, then they are called a Family of Overlapping, Notionally Shrinking Intervals, or a \textbf{fonsi}. Given a fonsi, there is a unique oracle whose set of Yes intervals contain the fonsi. The oracle is defined via the rule that an interval is considered to be a Yes interval if it contains an intersection of elements of the fonsi. This intersection can be finite or infinite. The infinite intersection is mainly of interest for rooted oracles as the singleton may not be in the fonsi nor included in finite intersections. This is a very practical tool in establishing oracles in a variety of contexts, such as a converging sum. A maximal fonsi is a fonsi in which all intersections of sets in the fonsi are in the fonsi and all intervals that contain an element of the fonsi is in the fonsi. A maximal fonsi is exactly the set of Yes intervals for the unique related oracle. 


\subsection{Basic properties of an oracle}

There are a variety of properties that can be deduced from these definitions. Proofs can be found in the main paper though in general these are fairly easy to check. In what follows below, there is a fixed oracle which the Yes intervals are referring to.

\begin{enumerate}
    \item The union of Yes intervals is a Yes interval. 
    \item Any interval contained in a No interval is a No interval. 
    \item The intersection of two Yes intervals is non-empty. The intersection is a Yes interval. 
    \item The union of two intersecting No intervals is a No interval. 
    \item If two intervals are disjoint, then at most one can be a Yes interval. 
    \item If an interval is disjoint from a Yes interval, then it is a No interval. 
    \item The separation property extends to a finite partition of a Yes interval. There will be exactly one subinterval that is a Yes interval unless one of the partition elements is a root of the oracle.   
    \item Given a positive length, there is a Yes interval that is shorter than the length. This follows by using the Separation property applied to successive bisection points to a given Yes interval. This includes the possible case of a singleton being a Yes interval whose length is 0. 
    \item The Two Point Separation property holds by applying the Separation property potentially twice. 
\end{enumerate}

\subsection{Ordering of Oracles}

The relation of two oracles can be seen by their relation on sufficiently narrow Yes-intervals. An interval $a\lte b$ is less than the interval $c\lte d$ exactly when $b < c$. If $r$ and $s$ are two oracles, then they are:
\begin{enumerate}
\item $r<s$ if there exists a $r$-Yes interval which is strictly less than a $s$-Yes interval.
\item $r>s$ if there exists a $r$-Yes interval which is strictly greater than a $s$-Yes interval.
\item $r=s$ if $r$ and $s$ have the same answer on every interval.
\item $r?s$ if $r$ and $s$ have the same answer on every interval that they have returned answers on. The intersection of all such intervals is the current Resolution of Compatibility. 
\end{enumerate}

One notation that is helpful is to use a bracket around an oracle in a rational interval to indicate that there is a Yes interval that is between the two. Namely, $a: [r] : b$ could be used to indicate that there exists an $r$-Yes interval whose endpoints are between $a$ and $b$, potentially inclusive. This could be extended to have betweenness of oracles as well. Something like $[r] : [s] : [t]$ can be viewed as $ r \leq s \leq t$ or $r \geq s \geq t$.

The first three are what one would expect. The last one is a capitulation to the reality that it is often not possible to fully differentiate an oracle from a nearby one. The main paper gives an example of an oracle that is based on the Collatz conjecture. The resolution of compatibility with the oracle of 0 is $0:2^{-68}$. This is the best known result about that conjecture and that number at this time. This example is, of course, not of practical significance, but it does illustrate potential limitations. 

 It is a merit of the oracle approach that this can be said clearly and naturally. 

The main paper does establish that this is an ordering with the usual properties. 

One key tool in determining that two oracles are equal is that they are equal if their Yes intervals can be shown to always overlap. This can be further refined to them being equal if the Yes interval of one of them can always be shown to contain an interval of the other. This is used in establishing the distributive property. 

The distance between two oracles can be computed from computing the distances between the Yes-intervals and taking the greatest lower bound of the distances. This exists since the distances are bounded below by 0. The distance of two intervals is the maximal distance of any two numbers in the interval which is the difference of the two farthest endpoints of the intervals. The distance of an interval from itself is its length. 

With the ordering established, closed and open intervals can be defined either as usual or one can define an oracle as being in the interval if certain intervals are Yes intervals for the oracle. Indeed, let $\gamma$ be an oracle and the closed interval $[\alpha, \beta]$ be given. For $\gamma$ to be in the interval, every rational interval $a:b$ must be a $\gamma$-Yes interval whenever $a$ is a lower endpoint of an $\alpha$-Yes interval and $b$ is an upper endpoint of a $\beta$-Yes interval. In other words, if an interval is both $\alpha$-Yes and $\beta$-Yes, then it must also be a $\gamma$-Yes interval. For an open interval, $\gamma$ is in that interval if there exists a rational interval $a:b$ that is $\gamma$-Yes with $a$ being an upper endpoint of an $\alpha$-Yes interval, $b$ a lower endpoint of a $\beta$-Yes interval, and $a < b$. Another way of phrasing that is to say that $\gamma$ is in the open interval if there exists a $\gamma$-Yes interval that is between, and disjoint from, two intervals, one of which is $\alpha$-Yes and the other is $\beta$-Yes. 


\subsection{Rationals, roots, and zeros}

The first examples to explore as oracles are the rationals and $n$-th roots. 

The oracle of a rational number $q$ is defined as the rule that $a:b$ is a Yes interval exactly when $a \leq q \leq b$. It is easy to check that the properties hold. It has a canonical representative interval in the form of the singleton $q:q$. This works because the rational numbers independently exist outside of this definition. 

The $n$-th root of a positive rational number $q$ is defined as the rule that $a:b$ is a Yes interval exactly when $\max(a, 0)^n \leq q \leq \max(b,0)^n$. If $b > a> 0$, this is more simply stated as $q$ being between the $n$-th powers of $a$ and $b$. That this defines an oracle can be established with the help of the monotonicity of $x^n$. The Closed Property takes a little work to show that if $p^n \neq q$, then there is a Yes interval that excludes $p$. 

The oracle of the $n$-th root exists. The next step is to start finding some Yes intervals. A great algorithm is the one from Newton's Method. Start with a given positive number $x$ for the $n$-th root of $q$, ideally somewhat close to the root though one can always use $1$. This guess has the associated Yes interval $x:\frac{q}{x^{n-1}}$ which satisfies $x^n : q : (\frac{q}{x^{n-1}})^n$. The next interval is obtained by taking the weighted average, specifically $\frac{1}{n}( (n-1) x + \frac{q}{x^{n-1}} )$ becomes the next $x$. Continue to iterate. At each stage, there is a Yes interval for $\sqrt[n]{q}$. This sequence of intervals is nested and the lengths are eventually shrinking quadratically to zero.  

There are, of course, many techniques for obtaining the roots. One technique given in the main paper is to experiment with using right triangles to get an interval trapping of $\sqrt{2}$. This helps illustrate the natural perspective of interval containment.  

Perfectly motivating examples for oracles are those produced by the foundation of the Intermediate Value Theorem. In particular, given a function $f$ and a value $y$, the oracle $\alpha$ of interest is the solution to $f(\alpha) = y$. The assumption is that there is a limited interval $a:b$ on which the function $f$ is continuous, strictly monotonic, and $y$ is between $f(a)$ and $f(b)$, that is, $[f(a)] : [y] : [f(b)]$. Note that these outputs could be oracles. Then for $c:d$ in $a:b$, the interval is a Yes interval if $[f(c)]: [y] : [f(d)]$. This defines the answer and the oracle properties, other than Consistency, can be established; the proof relies on the assumptions about $f$ in that interval. The proto-oracle's definition is augmented to have the Yes intervals also include those intervals that contain a Yes interval $c:d$ in $a:b$. This fully defined oracle then satisfies the Consistency property. If $f$ is monotonic and continuous on $(-\infty, \infty)$, then the extension would be unnecessary. Note that a singleton is included automatically if $f(a) =y $ for some rational $a$.

Some examples to explore include: $\pi$ as the zero of $\sin(x)$ on the interval $3:4$, $e$ as the solution to  $\ln(x)=1$ on the interval $2:3$, and $\sqrt{2}$ as the solution of $x^2 = 2$ on the interval $1:2$.\footnote{This assumes that the transcendental functions can be defined and sufficiently evaluated to be able to precisely determine the sign for a given input. For some discussion on these matters, see the main paper, particularly the section on function oracles.} The requirement for these real-valued functions to be used is that the function evaluated on rationals can indicate whether the ultimate value is above, below, or the same as the target value. 

This method provides more than just the existence of a solution, which is often insufficient for practical use. Oracles in general come with the ability to construct a sequence of narrowing intervals. To do so, take the bisection point of $a:b$, namely $m =\frac{a+b}{2}$, and test it out. Compute $f(m)$ and decide how it relates to $y$, $f(a)$, and $f(b)$. Replace one of the input endpoints with $m$ and then $f(m)$ replaces the corresponding output endpoint. The replacement should be chosen to maintain $y$ being in the resulting interval, something which can always be done. Iterate this until sufficient precision is obtained.  The midpoint strategy will halve the interval at each step which is convenient for estimating how many steps to take to achieve a desired precision. 

Another useful choice is to use mediants. For the fractions $\frac{a}{b}$ and $\frac{c}{d}$, the mediant is $\frac{a+c}{b+d}$ and it is a number in between them. The main paper devotes a section to their use, the relation to continued fractions, the Farey process, and the Stern-Brocot tree. By using mediants and the above function evaluation method, the narrowing process generates the best rational approximations to the solution of the equation as well as generating the continued fraction representation of that solution. 

If the solution of the equation is rational, then the mediant process will produce it exactly while the midpoint process will typically generate a sequence of approximations that never achieves the full solution. For example, solving $3x = 1$ starting in $0:1$ will produce the singleton solution of $\frac{1}{3}:\frac{1}{3}$ in two steps using mediants ($\frac{0}{1}:\frac{1}{1}$, $\frac{0}{1}:\frac{1}{2}$, $\frac{1}{3}:\frac{1}{3}$), while the midpoint process produces $0:\frac{1}{2}, \frac{1}{4}: \frac{1}{2}, \frac{1}{4}: \frac{3}{8}, \frac{5}{16}: \frac{3}{8}, \ldots$

The key point is that oracles provide the consistent framework for exploring these process and approaching how the infinite is practically done. 

\subsection{Being complete}

There are a few tasks that real numbers have to be able to handle that the rationals are not capable of doing. These are the completion properties. Details are in the main paper.

The first is the least upper bound property. Given a non-empty set $E$ of rationals with an upper bound, the goal is  to establish that there is an oracle which will serve as a least upper bound. Define the set of upper bounds to be $U$. Then define the oracle by the rule that $a:b$ is a Yes interval exactly when $a\leq y$ for all $y \in U$ and $b\geq x$ for all $x\in E$. Most of the proof is simply writing down the statements and working out the inequalities. 

The main paper also augments this to any bounded set of oracles having a least upper bound. The details mainly differ by having to use the interval representation of the oracles in getting a lower and upper bound for the intervals. Namely, the lower endpoints of the oracles in $E$ and the upper endpoints of the oracles in $U$. In both cases, if the rational is the least upper bound, then it does appear in the set. 

The second completion property is that Cauchy sequences should have limits. The basic definition of the limit oracle is that an interval is a Yes interval if it contains the tail of the sequence. The property of the narrowing of the Cauchy sequence establishes the No's of the Separation and Rooted property. For a Cauchy sequence converging to a rational number, it is required that the singleton be explicitly added since no singleton can contain a non-constant tail.

Related to this is the general notion of a fonsi as mentioned above. A primary example is that of nested, converging sequences of intervals which Cauchy sequences can be converted into. Absolutely converging series can be directly established to be oracles using a fonsi. 

But a fonsi need not be sequential. As an example, consider a nesting function which takes in lengths and produces intervals shorter than that length that are nested within larger intervals. A fonsi could also be a set that consists entirely of a single singleton. The main criterion is that every interval in the family overlaps and that it is possible to find intervals shorter than any given length. Converting them into oracles is largely filling in any missing intervals, particularly making sure consistency and the closed property hold.  

Two fonsis are equivalent to the same oracle if one can show that each of their elements intersect. This also follows if, given any small length, there exist intervals in both fonsis which are smaller than that length and overlap each other. Two Cauchy sequences whose tails are arbitrarily close can thus be seen to have the same limit. 

Fonsis are the objects that the constructivists, cited earlier, use to define real numbers. The concern for them is being explicit which a fonsi very much allows. The downside is that a real number is not defined uniquely. The approach here embraces having a unique definition of a real number which has the downside of having very long intervals that are not of any practical concern. 

\section{Arithmetic}

Arithmetic is done with the intervals. Interval arithmetic is based on, but different than, the usual number arithmetic. Conceptually, combining intervals with arithmetic operators is to look at the smallest interval that contains the results of performing the indicated operation on every pairing of the elements from the interval. In practice, it means looking at the endpoint combinations. Division only works on same signed intervals. Addition is easy, multiplication with signed intervals is a little more involved. Computing the power of intervals is rather involved. 


Many of the properties of ordinary arithmetic apply to intervals, but not all. There are no inverses. One side of the distributive property is included in the other. Generically, combining intervals leads to longer intervals. All of this is gone into detail in the main paper, but it is straightforward to explore interval arithmetic. 

Oracle arithmetic, while based on intervals, does create a single oracle answer instead of an interval. This is the result of oracles having notionally shrinking intervals and thus the combination of all of those intervals does produce an oracle with intervals as small as needed. The properties of usual arithmetic can then be established, including the distributive property as well as the existence of additive and multiplicative inverses.   

The main paper goes through all the different arithmetic operators, produces bounds for the interval arithmetic operators, and establishes the necessary results from that. This paper will simply illustrate it with an example. 

\subsection{An example of arithmetic}

The task is to compute $x = \frac{e-\sqrt{2}}{\pi}$. This should be an oracle and so it should be able to answer questions about any interval given to it. For example, is $x$ in the interval $Q = \frac{41}{100}: \frac{42}{100}$? 

The first step is to start computing the interval arithmetic with some narrow inputs. If it is sufficiently narrow, either the output interval will be contained in $Q$, in which case the oracle says Yes, or it will be disjoint from $Q$ in which case the oracle says No. After testing this interval, the mediant process will be demonstrated also using this one computed interval. 

To compute out a Yes interval, the inputs needed are interval representations for $e$, $\sqrt{2}$, and $\pi$. Using mediant approximations for these numbers is recommended: 
\begin{enumerate}
\item $e$-Yes interval $A  = \frac{106}{39}: \frac{87}{32}$,
\item $\sqrt{2}$-Yes interval $B = \frac{41}{29}: \frac{17}{12}$, and
\item $\pi$-Yes interval of $C= \frac{333}{106}: \frac{22}{7}$.
\end{enumerate}
The interval operations are then as follows:
\begin{enumerate}
\item Subtraction is $A\ominus B = \frac{106}{39} - \frac{17}{12}: \frac{87}{32} - \frac{41}{29} = \frac{203}{156}: \frac{1211}{928}$ 
\item Reciprocating, to convert division into multiplication, leads to  $1\oslash C = \frac{7}{22}: \frac{106}{333}$
\item Multiply the reciprocal by the subtraction result to get $D = (A\ominus B)\oslash C = \frac{203}{156} *\frac{7}{22}: \frac{1211}{928} * \frac{106}{333} = \frac{1421}{3432}: \frac{64183}{154512}$
\end{enumerate}
If that precision is insufficient, then the steps can be repeated with more precise input intervals. The main paper does have bounds that would allow for the computation of the precision ahead of time. 

To test whether $Q$ is in that interval, the task is to compute whether the endpoints are greater than or less than one another. A useful fact to know is that $\frac{a}{b} < \frac{c}{d}$ exactly when $ad - bc < 0$. Equality happens if that quantity is 0. Computing $1421*100 - 41*3432 = 1388$, the conclusion is that $\frac{41}{100} < \frac{1421}{3432}$. For the other endpoint, the computation $64183*100 - 154512*42 = -71204$ yields $ \frac{64183}{154512} < \frac{42}{100}$. The Yes interval $D$ is therefore contained in the interval $Q$ which, by consistency, forces the oracle to reply Yes for $Q$. 

The rational intervals obtained from the arithmetic operators can have larger denominated fractions as endpoints. The mediant process can reduce that complexity. The process starts with the unique Yes interval which is of length 1 with integral endpoints. For this oracle, that interval is $\frac{0}{1}:\frac{1}{1}$ as $D$ is clearly contained in it. 

The approach is to compute the mediant of the interval and then see which of the two subintervals it defines still contains the interval $D$. This continues until the mediant is inside $D$ and it ceases to be possible to determine from $D$ which interval is a Yes interval. 
\begin{enumerate}
\item Mediant is $\frac{1}{2}$. Computation with the the upper endpoint:  $2*64183 - 1*154512 = -26146 < 0$. The mediant is therefore greater than the upper endpoint leading to $D$ being in the interval $\frac{0}{1}: \frac{1}{2}$. (L)
\item Mediant is $\frac{1}{3}$. This turns out to be below the lower endpoint:  $3*1421 - 1*3432 = 831>0$. Thus, $D$ is in the interval $\frac{1}{3}: \frac{1}{2}$. (R)
\item Mediant is $\frac{2}{5}$. It is below the lower endpoint: $5*1421 - 2*3432 = 241 > 0$. Thus, $D$ is in the interval $\frac{2}{5}: \frac{1}{2}$. (R)
\item Mediant is $\frac{3}{7}$. It is above the upper endpoint: $7*64183- 3*154512 = -14255 < 0$. Thus $D$ is in the interval $\frac{2}{5}: \frac{3}{7}$. (L)
\item Mediant is $\frac{5}{12}$. It is above the upper endpoint: $5*64183 - 12*154512 = -2364 < 0$. Thus $D$ is in the interval $\frac{2}{5}: \frac{5}{12}$. (L)
\item Mediant is $\frac{7}{17}$. It is below the lower endpoint: $17*1421 - 7*3432 = 133 > 0$. Thus $D$ is in the interval $\frac{7}{17}: \frac{5}{12}$. (R)
\item Mediant is $\frac{12}{29}$. It is below the lower endpoint: $29*1421 - 12*3432 = 25 > 0$. Thus $D$ is in the interval $\frac{12}{29}: \frac{5}{12}$. (R)
\item Mediant is $\frac{17}{41}$. It is not below the lower endpoint: $41*1421-17*3432 = -83$. It is not above the upper endpoint: $41*64183-17*154512 = 2368836 >0$. $D$ is not contained in either $\frac{12}{29}: \frac{17}{41}$ or $\frac{12}{29}:\frac{17}{41}$. To proceed further, a more narrow Yes interval must be computed to replace $D$. 
\end{enumerate}
The conclusion so far is that $\frac{12}{29}: \frac{5}{12}$ is a $\frac{e-\sqrt{2}}{\pi}$-Yes interval. If an error of $.0029$ is sufficient, then the computation can stop here. In comparison, the error in interval $D$ is $.0013$. Notice that the mediant representative interval has much simpler fractions to work with which is the advantage of using mediants when possible. To check on the initial interval of concern, the computations are $12*100 - 29*41 = 11$ and $5*100-12*42 = -4$. This leads to the same conclusion, but with simpler arithmetic involved. 

It is useful to note that the mediant process above is computing out the continued fraction for the oracle. As mentioned in the main paper, the continued fraction is an accounting of whether the left or right subinterval is selected. The pattern was L,L,R,R,L,L,R,R.\footnote{The first L comes from selecting $\frac{0}{1}: \frac{1}{0}$ from the actual starting ``interval'' of $\frac{0}{1}: \frac{1}{0}$. The leading 0 in the continued fraction represents the number of initial right selections which is equal to the integer portion of the number.} Thus, doing a count, the continued fraction at this point is $[0;2,2,2,2]$ and the next one, representing the choice to be made with $\frac{17}{41}$, is either $[0;2,2,2,2,1]$ or $[0;2,2,2,3]$. As a continued fraction, those two choices both represent $\frac{17}{41}$, but for the mediant process they represent either a left selection or a right selection, respectively. In fact, the next two selections will be a right selection before switching, leading to $[0;2,2,2,4,1]$. The next three intervals are therefore, $\frac{17}{41}: \frac{5}{12}$, $\frac{22}{53}: \frac{5}{12}$, and then with the switch, $\frac{22}{53}: \frac{27}{65}$. That final interval has a length of about $0.00029$. 

This has been an example of oracle arithmetic. It is messy and it will rarely end with an exact production of a number. But it can produce intervals to arbitrary precision given sufficient computational resources and the ability to be arbitrarily precise on the input intervals. The mediant process allows for the reduction of complexity. 

\subsection{The decimal version}

It can also be useful to review how this computation would be viewed in a decimal computation. In what follows, the symbol $=$ will stand in for approximately equal. The first difficulty is what exactly that means. There are four likely interpretations of this:

\begin{itemize}
\item Short Rounding. $x$ is in the interval $a.d_1d_2d_3\ldots d_n \pm 0.5*10^{-n}$. $\pi = 3.14$ implies the interval $3.135:3.145$.
\item Truncation. $x$ is in the interval $a.d_1d_2d_3\ldots d_n: a.d_1d_2d_3\ldots (d_n+1)$. $\pi=3.14$ implies the interval $3.14:3.15$.
\item Long Rounding. $x$ is in the interval $a.d_1d_2d_3\ldots d_n \pm .1*10^{-n}$. $\pi = 3.14$ implies the interval $3.13:3.15$. 
\item Big Uncertainty in Last Digit. $x$ is in the interval $a.d_1d_2d_3\ldots d_n \pm 5*10^{-n}$. $\pi = 3.14$ implies the interval $3.09:3.19$. 
\end{itemize}

Redoing the computation of $x = \frac{e-\sqrt{2}}{\pi}$  using $\pi = 3.14$, $e=2.72$, $\sqrt{2} = 1.41$, the single value computation yields roughly $u = 0.417195$. To about that precision, $x$ is $0.4150978$. This does suggest that the rough computation needs to be rounded or truncated to avoid giving a false precision. In what follows, two intervals per each method will be produced. The first interval is the interval level that the default interpretation would contain both the short computation result as well as the longer one. The second interval will be the interval that gets produced by doing interval arithmetic and then the appropriate representative to produce something that contains that interval will also be given. 

\begin{itemize}
    \item Short Rounding. $0.42$ represents the interval $0.415:0.425$ which contains both $x$ and $u$.  Computing the implied interval, the result is $\frac{2.715-1.415}{3.145}: \frac{2.725-1.405}{3.135} = 0.4133:0.4211$. This does include $x$, but to translate it into a single number with the implied spread covering this entire interval would yield $0.4$ because $0.4133$ is not contained in $0.42$'s interval under short rounding. The implied interval is $0.35:0.45$.
    \item Truncation. $0.41$ would work as the interval of $0.41:0.42$ contains both $x$ and $u$. The interval computation is $\frac{2.72 - 1.42}{3.15}:\frac{2.73 - 1.41}{3.14} = 0.4126: 0.4204$. Again, while $x$ is in this interval, to cover the edge, here $0.4204$, one cannot write $0.41$. Thus, as before, $0.4$ is the single number representative; it implies $0.4:0.5$ as the interval. 
    \item Long Rounding. $0.42$ implies $0.41:0.43$ and $0.41$ implies $0.40:0.42$. So either version will contain both $x$ and $u$. As for the interval computation, $\frac{2.71 - 1.42}{3.15}:\frac{2.73 - 1.40}{3.13} = 0.4095: 0.4250$. Neither $0.42$ or $0.41$ will have that whole interval contained in it. So $0.4$ is to be chosen implying $0.3:0.5$. 
    \item Big Uncertainty in the Last Digit. There are several representatives that can cover both $x$ and $u$. $0.41$ leads to the interval $0.36:0.46$. The interval computation is $\frac{2.67 - 1.46}{3.19}:\frac{2.77 - 1.36}{3.09} = 0.3793: 0.4564$. The representative $0.42$ leads to $0.37:0.47$ which just barely covers the computed interval.  
\end{itemize}

The problem with decimal arithmetic is that the imprecision grows making it difficult to use a single representative without sacrificing claims of precision. The last two options for interpretation also imply multiple reasonable representatives for a given decimal and level of precision. 

How about calculators? The calculator used here will be a TI-84. The calculator reports  $e = 2.718281828$, $\pi = 3.141592654$, and $\sqrt{2} = 1.414213562$. The calculator seems to do short rounding to produce decimals, for example, if expanding $\pi $ just a little further, the result is  $3.14159265358$ which the calculator rounds up. Computing out the decimals, the calculator reports $0.4150978214$ with the implication that the result should be in the interval $0.41509782135:0.41509782145$.  WolframAlpha reports it as $0.415097821354$ so it is indeed in that interval, but just barely. 

This success is because the calculator is computing more digits than is shown. If the directly displayed digits were used for the numbers, then the result is $0.4150978213$ which implies the interval $0.41509782125:0.41509782135$ and this is not an interval that contains the quantity being computed. 

Computing an interval from the display version and using WolframAlpha's abilities to handle longer strings of digits, the result is roughly $0.41509782088:0.41509782166$ which does contain $x$. The closest representative that would include this interval, however, would be $0.41509782$ as $0.415097821$ would not imply including the numbers above $0.4150978216$. 

On a practical level, our tools have been engineered to produce sufficiently accurate numbers for any practical concerns, but it can be potentially misleading, particularly for those just learning how to use decimals. Intervals lead to precise statements that convey the limitations inherently and unambiguously at the cost of length and time.  

\subsection{Arithmetic with rationals}

For rational oracles, arithmetic can be performed with their singletons. Doing the multiplication $2 * \pi$ using the $\pi$-Yes interval $\frac{333}{106}: \frac{22}{7}$ leads to $\frac{333}{53}: \frac{44}{7}$ being a $2 \pi$-Yes interval. If one then tried to recover 2 by dividing  by $\pi$, using that same interval for $\pi$, the result is $\frac{333}{53} * \frac{7}{22}: \frac{44}{7} * \frac{106}{333} = \frac{2331}{1166}: \frac{4664}{2321} \approx 1.9991: 2.0009$. That is, while 2 is in that interval and will be in all such intervals, the interval arithmetic does not lead to an immediate conclusion of the singleton. This is a practical problem with the arithmetic of real numbers which seems to apply to all definitions of real numbers. This approach at least gives immediate bounds as part of the process.

The above mediant process hits an immediate problem when attempting to apply it to simplify the interval endpoints. The first step of the mediant process is to get the integer part. One needs to choose between the intervals $\frac{1}{1}:\frac{2}{1}$ and $\frac{2}{1}:\frac{1}{0}$. Since $2$ is the solution but cannot be seen as such, this process will not be able to decide which interval. It cannot proceed as above. But it is possible to modify the other endpoints using mediants. That is, the process can consider choosing both subintervals and proceeding until the known Yes interval cannot distinguish any further. On the left, the $k$-th mediant subinterval will be $\frac{1 + 2k}{1+k}:\frac{2}{1}$, that is, the process is repeatedly adding $\frac{2}{1}$ to the left endpoint. Each time, the comparison is $\frac{1+2k}{1+k}$ to $\frac{2331}{1166}$. Solving for $k$, the result is that $\frac{2331}{1166}$ is one of these mediants. The other side is adding $\frac{2}{1}$ to the right endpoint repeatedly. This is $\frac{2k + 1}{k+0}$ and comparing it to $\frac{4664}{2331}$. Setting the $k$ fraction to be less than the fixed endpoint and solving, the inequality is $4664k > 2*2331 k + 2331$ leading to $k > 1165.5$. Testing $1165$ and $1166$ leads to finding $\frac{2333}{1166} < \frac{4664}{2331} < \frac{2331}{1165}$. If it could be established that this pattern continues with finer initial input intervals, then the 2 could be established to be the oracle. 

Oracle ordering is compatible with arithmetic. Namely, adding the same oracle to both sides of an inequality preserves the inequality and two positive oracles multiplied together are positive. These are easy to show with sufficiently narrow Yes-interval arithmetic. 

Note that with interval arithmetic, some care has to be taken with signs and multiplication. 

\subsection{The reals}

The oracles with the above relations and arithmetic operators do form an ordered field. The rationals are embedded as a subfield. The rationals are dense as given any two distinct oracles, there are disjoint Yes intervals for them and any rational between the two closest endpoints will root a rational oracle in between them. 

The Archimedean property follows by  writing down disjoint intervals of the two reals and then scaling the lesser one past the upper one using the property that rationals can be scaled past each other. 

Another property of the real numbers is that they are uncountable. The proof of that with oracles is fairly simple. Given a list of oracles, one can use the mediant process, possibly applying it twice in one step, to choose a mediant interval which does not contain the next item on the list. Since the oracle answers questions about intervals and since the mediant process will go past the point where a fraction with a given denominator can appear in that interval, this oracle will always give an answer that satisfies the required properties.  Hence, this is an oracle and it is not on the list by construction. 

\section{Function oracles}

The next idea to tackle is that of functions. The definition of functions should respect this oracle idea.  In particular, it is clear that the rationals can be precisely specified, but irrationals cannot be. This translates into being able to assign rationals particular values while irrationals can only get a neighborly agreement on their values. 

The idea is to have Yes rectangles whose wall should contain the image of the base. The wall is allowed to be larger than the image. The core requirement is that there should exist smaller and smaller rectangles that shrink down to what would classically be understood as the value of the function at a given real number. 

Singletons are allowed to be the base or wall of a rectangle. This is what allows rational numbers to be independent of the their neighbors.

More explicitly, a function oracle $f$ is a rule that answers Yes or No when provided with a rational rectangle and satisfies the following properties: 
\begin{enumerate}
    \item Elongating Consistency. If the wall of rectangle $M$ contains the wall of the Yes rectangle $N$ and they share the same base, then $M$ is a Yes rectangle. 
    \item Narrowing Consistency. If the base of rectangle $M$ is contained in the base of the Yes rectangle $N$ and they have the same wall, then $M$ is a Yes rectangle.  
    \item Intersection. If two Yes rectangles intersect, then the intersection is also a Yes rectangle. 
    \item Single-valued. Given two disjoint rectangles $M$ and $N$ sharing the same base, at most one of them can be a Yes-rectangle. 
    \item Separating. Given a Yes rectangle $M$, an oracle $\alpha$ contained in the base of $M$, and two $y$-values $r$ and $s$ contained in the wall of $M$, then there exists a Yes rectangle contained in $M$, not containing at least one of those values in its wall, and $\alpha$ is contained in its base.
\end{enumerate} 

The first two properties allow for elongating and narrowing rectangles. In particular, rectangles can be narrowed as much as needed. Rectangles with overlapping bases must intersect on the wall as well. That is, there are no vertically disjoint rectangles. 

The oracle is answering the question "Does the wall of the rectangle contain the image of the base under this function?" Just like real numbers, as this is defining the function, the answer to the question is on the aspirational side. 

By working with the properties, it is shown in the  main paper that for every oracle $\alpha$ in the base of a Yes rectangle, there is a unique oracle $\beta$ whose Yes intervals are the walls of the Yes rectangles whose base contains $\alpha$. The common notation of $f(\alpha) = \beta$ for this association is therefore justified. 

It can be proven that oracle functions are equivalent to the set of usual real functions that are continuous everywhere except possibly at the rationals. Thomae's function is the classic example. Thomae's function qualifies as a function oracle because singletons can be the entire base of a rectangle. Any given rational can be assigned to any oracle. The family of rectangles for that rational would include rectangles whose base is the singleton and whose walls are the Yes intervals of the output oracle. The irrationals, however, cannot be picked out in such a way. The bases of their rectangles must include some neighbors and this is how continuity arises for them.

The rationals are not entirely independent of their neighbors in the aggregate. Any particular rational can be whatever value, but eventually the rationals must do some kind of alignment with the irrationals as in Thomae's function. As an example, the characteristic function of the rationals cannot arise from a function oracle. Indeed, any non-singleton rectangle will include both rational and irrational oracles which means the wall must include at least the interval $0:1$. This is why it fails to satisfy the Separating property. 

Given a function oracle $f$ and a classically continuous function $g$, another function oracle can be defined by composition, specifically, $g$-mapping the wall of a Yes rectangle. Specifically, if $a:b \times c:d$ is a $f$-Yes rectangle, then $a:b \times g(c:d)$ is the new rectangle. As a special case, every classically continuous function can be represented by a function oracle composed with the function oracle $x$ whose foundational rectangles are $a:b \times a:b$.

With the usual error estimates, Taylor polynomials can also be used to construct function oracles. 

Given a $f$-Yes rectangle, $f$ is Riemann integrable over the base. Indeed, the function oracle setup is that of the bounding rectangles one could use to compute the areas. 

A version of the intermediate value theorem can be proven. Given a function oracle $f$, an interval $a:b$ in the domain of the function oracle,\footnote{An interval is in the domain if every oracle in the interval is in the base of a Yes rectangle. An oracle being in an interval means the interval is a Yes-interval for that oracle.} and an oracle $y$ satisfying $[f(a)]:[y]:[f(b)]$ , then there exists an oracle $c$ such that all non-singleton rectangles containing $c$ in the base will have $y$ in the wall. If the oracle $c$ is not a singleton, then $f(c) = y$. If the oracle $c$ is a singleton, then it could be defined differently from $y$ and the function oracle would be necessarily discontinuous at $c$. 


\section{Relation to other definitions}

The main paper has a lengthy discussion on how this definition relates to other ones. This will be brief version of that. For a reference of an overview of different real number definitions, please see \cite{ittay-2015}. For those curious about the deficiencies of other definitions in detail, please check out the videos by Norm Wildberger's videos, such as the provocative ``Real numbers as Cauchy sequences don't work!''\footnote{\url{https://www.youtube.com/watch?v=3cI7sFr707s}} 

Many definitions of real numbers consist of a particular numerical representation of a real number, such as summations or the wonderful continued fraction representations. Decimals are another example which includes the non-essential choice of base 10. 

Cauchy sequences are different in that there is no particular specification of a single Cauchy sequence for a real number. In some sense, Cauchy sequences encompass all those particular representations in addition to many others. This arbitrariness in any single sequence leads to using equivalence classes of Cauchy sequences. While that is a unique representation, it has the unfortunate side effect of making every real number look the same. Indeed, if one looks at the first $n$ elements of the sequences, for whatever natural number $n$ one chooses to use, then there are representatives from each Cacuchy sequence class that all agree up to the first $n$ terms. For comparing Cauchy sequences in an equivalence class, the control defined in the Cauchy sequence is lost across the equivalence classes. For example, given any $10$ trillion numbers, there are infinitely many Cauchy sequences for any given real number with those numbers as the first $10$ trillion terms. For a particular sequence, it is possible to talk about a particular control, but that is completely lost in the equivalence classes. A similar claim can be made of equivalence classes of sequences of nested, shrinking intervals. 

Cauchy sequences are good tools to describe a particular real number in a particular way, but there is arbitrariness in the choices defining a sequence. They also essentially demand the computation be done in order to define the number. Oracles have no such arbitrariness in them nor does there need to be computations done just to define the number. Large, universe spanning intervals are present as Yes intervals in an oracle, but the burden of what is relevant is on the questioner, not the oracle. 

Continued fraction representations are not that arbitrary. Finite continued fractions do have a non-uniqueness representing which subinterval to select, but infinite ones are unique. Arithmetic with continued fractions is unpleasant though ordering is very pleasant to work with. Many aspects of them do feel natural, but most of that naturality is brought out even more from the perspective of the mediant approximation method which flows naturally from oracles. 

There are other approaches involving a fonsi in one way or another. Bachmann's approach of 1892 was to use nested sequences of shrinking intervals. The constructivist texts of \cite{bridger} and \cite{bridges} take a fonsi as a real number. The constructivists also prefer to think of equality of real numbers as being demonstrated that two fonsi have the same overlaps rather than thinking of an equivalence class defining a single version of a real number. An even larger approach, as discussed in \cite{ittay-2015}, is that of minimal Cauchy filters. This is roughly equivalent to defining a real number as the set of all sets that contain a given fonsi without including the singleton. 

The approach taken here sits in between the fonsi and the Cauchy filter. There is a single version of the number, just like the minimal Cauchy filter and unlike the other two approaches. In terms of singletons, the only version which is guaranteed to have a rational in the real number representative is this approach. The other fonsi approaches may or may not include the rational. The filter approach specifically excludes it. A more substantial difference, and what is at the core of this idea, is this is the only approach in which the interval separation property is guaranteed to work. The fonsi approach guarantees a smaller interval can be obtained, but not how and not conveniently. The Yes intervals are a part of the filter though it suggests much more general sets are relevant. 

The final definition under consideration is that of Dedekind cuts. They are not arbitrary as many of the other representations are. Indeed, they are the lower bounds of the Yes-intervals, excluding singletons. They work, but they seem unhelpful. For example, in the intermediate value definition for a monotonic function, the solution by Dedekind cuts is defined as $\{x | f(x) < y\}$. This does not actually suggest a way of computation nor does this seem clearer or easier to handle than $f(x) = y$. The oracle perspective provides an easy definition and suggests how to compute a good representative. It is not even clear with Dedekind cuts what one ought to compute while for oracles, computing smaller intervals is strongly suggested by the formulation. 

It can be useful to also consider how these definitions handle the completion properties. Cauchy sequences are their own limits? Dedekind cuts yield a kind of completed version of the set $E$ itself as the least upper bound of the set $E$? For the least upper bound, Cauchy sequences lead to having to do the algorithm to define it. For Dedekind cuts, the Cauchy sequence limit definition of an $x$ being less than a tail segment feels like just a partial specification. 

Ultimately, all of these definitions of real numbers have their place as viewpoints and useful representations, but as a definition they fall short. With oracles, there is no arbitrariness in the representative. Oracles suggest strongly how to compute and approach them without demanding the computation be done to define the number. In many applications of trying to find a real number, the definition fits quite naturally in both assuring the existence of the solution in addition to yielding a computation of it with built in precision information. The arithmetic and ordering seem fairly natural in their definition and use. 

The final criteria is how well does the definition generalize. The specific representations, such as continued fractions, are very particular to the real numbers. Dedekind cuts rely on the ordering and do not generalize to unordered spaces. They are intimately connected to the 1-dimensionality of real numbers as is the Interval Separation property used here. But the oracle idea is easy to generalize using the Two Point Separation property. Cauchy sequences and the ultra-filters do generalize, but their problems carry over as well. 

\section{Completing a metric space}

To complete a metric space, the rational inclusive intervals are replaced with inclusive balls, specified by a center $c$ and a radius $r$. An oracle in that space is then a rule that says Yes or No when presented with an inclusive ball. Singletons are balls of radius $0$. The properties the oracle needs to satisfy are
\begin{enumerate}
    \item Consistency. If a ball contains a Yes ball, then it is a Yes ball. 
    \item Existence. There exists a Yes ball.
    \item Two Point Separating. Given a Yes ball and two points in it, then there is a different Yes ball inside the first one which does not contain at least one of the given points. 
    \item Intersecting. Two Yes balls must intersect and their intersection must contain a Yes ball.
    \item Closed. If a point $q$ in the original space is contained in all Yes balls, then the singleton of $q$ is a Yes ball.
\end{enumerate}

As discussed in the main paper, such oracles form a new metric space where the distance is inherited through ball representatives, the triangle inequality, and the greatest lower bound. The original space's elements are represented by the balls that contain them, including their singleton ball; these are the rooted ones. Given a Cauchy sequence of points in these spaces, the balls that contain the tails are the Yes-balls of the limit oracle. The properties of the limit as an oracle can be easily established. For the closed property, it is necessary to add in the singleton as a Yes ball if it converges to a point in the original space; it is easy to show that there are no difficulties with doing so. 

Function oracles can be extended to this space where the sides of the rectangles are balls in the metric space instead of intervals. 

Of the definitions of real numbers, most are particular to real numbers. Cauchy sequences are a notable exception, but using them requires the equivalence class approach which is a bit like looking for information in the Library of Babel.\footnote{\url{https://libraryofbabel.info/}} 

\section{Conclusion}

Oracles give a new perspective on what a real number is. It is about determining which intervals contain it. It is about trying to locate the number by asking about where it is. It rejects the arbitrariness in other definitions of a real number, but embraces external arbitrariness in practical representation.

The oracle approach motivates how to approximate a real number. The mediant process gives the best rational representation and this framework naturally suggests using it. It elevates rational numbers as the key elements to understanding real numbers. It also highlights the differences between rational and irrational numbers in a clear fashion. 

Oracles give a language for handling the imprecision that are inherent with real numbers. A Yes interval can express how much is known about a real number and that expression is reflective of the purpose of the definition. There is no  trail of dots with oracles. 

The hope is that oracles become something that changes the way to teach and understand real numbers. There should be clear rules of the game for real numbers and those rules should be given early on. Oracles naturally lead to better manipulation of fractions, mediants, intervals, geometric bounds, and infinite processes. The computations with them can be fairly easy. 

There is no need to pretend that the completed infinity exists. All that is needed is to have a mechanism that answers human questions, particularly those of practical concerns. That is what an oracle does. 

\medskip

\printbibliography

\end{document}


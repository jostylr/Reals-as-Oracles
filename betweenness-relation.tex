\documentclass[12pt]{article}
\usepackage{personal}
\usepackage{realoracles}

\newtheorem{theorem}{Theorem}[section]
\newtheorem{lemma}{Lemma}[section]
\newtheorem{corollary}{Corollary}[section]
\newtheorem{proposition}{Proposition}[section]

\newcommand{\qcut}[2][x]{\ensuremath{\mathbb{Q}_{#2 #1}}}
\newcommand{\qlt}[1][x]{\qcut[#1]{<}}
\newcommand{\qeq}[1][x]{\qcut[#1]{=}}
\newcommand{\qgt}[1][x]{\qcut[#1]{>}}
\newcommand{\qgeq}[1][x]{\qcut[#1]{\geq}}
\newcommand{\qleq}[1][x]{\qcut[#1]{\leq}}
\newcommand{\cut}[1][x]{{\vert}_{#1} }

\newcommand{\yrel}{\xrel[y]}

\title{Real Numbers as Rational Betweenness Relations}
\jtauthor
\date{\today}

%\sloppy%\openup-.1\jot
\begin{document}\maketitle
\begin{abstract}
A new definition of a real number is that a real number is a rational betweenness relation in which two rationals are related if the real number ought to be between the two rationals, inclusively. There are five properties that a relation must satisfy for it to represent a real number with the key property being the Separation property which is roughly that any number in between either is related to exactly one of the two endpoints or is related to itself. The set of betweenness relations with the natural order and arithmetic of intervals is canonically isomorphic as a complete ordered field to Dedekind cuts. 
\end{abstract}

\section{Introduction}

The purpose of this paper is to give a new definition of real numbers and establish that this definition works. This paper will not delve deeply into the use of this definition nor argue for it over other definitions. For that, please refer to \cite{taylor23main} in which this definition is slightly altered and presented differently with the intent of establishing a firm foundation in line with how real numbers are practically used.

This paper will first define the term rational betweenness relation. This will involve introducing some convenient notion. After establishing the definition and some immediate properties, the next task is to recall the definition of Dedekind cut, which will be slightly modified in presentation from the standard definition as given, for example, in \cite{rudin}. The first main result is to establish the explicit canonical bijection from the space of Dedekind cuts to the space of rational betweenness relations. 

An important part of defining real numbers is defining the order relation and arithmetic operators. For the betweenness relation, interval ordering and arithmetic naturally suggest definitions for the relations. The bijection to the Dedekind cuts will be shown to respect the order relation and arithmetic operators which can then be used to establish their properties. The bijection is an isomorphism of the order and arithmetic structures which also establishes the completion property.  This establishes that the space of rational betweenness relations is a valid model of the real numbers. 

Throughout this paper and without further explicit comment, the letters $a, b, c, d$ will always be rational numbers and often take the role of endpoints of intervals. The letters $m, n, p, q, r, s$ will also always be rational numbers and often take the role of numbers in an interval. The letters $x, y, z, \alpha, \beta$ will represent real numbers.

\section{Definition of a rational betweenness relation}

To claim to know a real number is to at least be able to answer definitively whether that real number is between any two given rational numbers. With that capability, one can do arbitrarily precise arithmetic with that number. The capability can be codified as a relation between rational numbers. 

A little notation will be helpful. The set of all rationals $q$ such that $q$ is between $a$ and $b$, including the possibility that $q=a$ or $q=b$, is a \textbf{rational interval} denoted by $a:b$. If $a=b$, then this is a \textbf{rational singleton} denoted by $a:a$. It is a set of exactly one rational number, namely, $a$. 

The notation will also be used to indicate betweenness. If $a \leq b \leq c$ or $c \leq b \leq a$, then $a:b:c$ will be used to denote that. By definition, $c:b:a$ and $a:b:c$ represents the same betweenness assertion. This can be extended to any number of betweenness relations, such as $a:b:c:d$ implying either $a \leq b \leq c \leq d$ or $d \leq c \leq b \leq a$. If $b$ and $c$ are between $a$ and $d$, but it is not clear whether $b$ is between $a$ and $c$ or between $c$ and $d$, then the notation $a:\{b,c\}:d$ can be used. This can also be extended to have, for example, $a:b:\{c,d\}$ which would be shorthand for saying that both $a:b:c$ and $a:b:d$ hold true. 

Rational number satisfy the fact that, given three rational numbers, $a, b, c$, we have either $a:b:c$, $a:c:b$, or $b:a:c$. That is, one of them is between the other two. This follows from the pairwise ordering of each of them as provided by the Trichotomy property. 

In this paper, often a relabelling will be mentioned. This is to indicate that there are certain assumptions that are needed to be made, but they are notational assumptions and, in fact, some arrangement of the labels of that kind must hold. For example, if $\{a,d\}:b:c$ holds true, then either $a:d:b:c$ or $d:a:b:c$ holds true. If these are generic elements, then relabelling could be used to have $a:d:b:c$  be  true for definiteness, avoiding breaking the argument into separate, identical cases. If $a$ and $d$ were distinguished in some other way, such as being produced by different processes, then relabelling would not be appropriate to use. 

The $x$ rational betweenness relation is a relation on pairs of rational numbers that is supposed to express that there is a real number $x$ between the two rational numbers. That is, if $a : b$ is a rational interval, then $a \xrel b$ denotes that $x$ ought to satisfy $a \leq x \leq b$ or $b \leq x \leq a$;  both $(a,b)$ and $(b,a)$ are in the betweenness relation. We say that $a:b$ is an $x$-interval and that $a$ is $x$-related to $b$. If $(a,b)$ is not in the relation, then that can be denoted by \sout{$a \xrel b$} and $a:b$ is said to not be an $x$-interval. 

Since this is defining the real number $x$, the above is only a notional idea behind the definition. The definition is that a \textbf{rational betweenness relation} is a symmetric relation on rational numbers 
which satisfies the following properties:
\begin{enumerate}
    \item Existence. There exists $a$ and $b$ such that $a\xrel b$.
    \item Interval Separation. If $a \xrel b$ and $a : c : b$, then exactly one of the following holds: 1) $a \xrel c$ and \sout{$c \xrel b$}, 2) $c \xrel b$ and \sout{$a \xrel c$}, or 3) $c \xrel c$. 
    \item Consistency. If $c : a : b : d$ and $a \xrel b$, then $c \xrel d$. 
    \item Rooted. If $c \xrel c$ and $d \xrel d$, then $c=d$. 
    \item Closed. Given a $c$ such that $a:b$ being an $x$-interval implies $a:c:b$, then $c \xrel c$. 
\end{enumerate}

If $q \xrel q$, then $x$ represents the rational number $q$ and is the \textbf{root of the relation}; $q:q$ may be called an $x$-singleton. If such a $q$ exists, then $x$ will be called a rational. Also note that the rooted property implies that $q$ would be unique. If no such $q$ exists, then $x$ is irrational. 

The Interval Separation property is at the heart of this procedure being computationally useful. It is modelled on how the Intermediate Value Theorem is operationalized. This version of the property is idealized in the sense that there is an assumption that $c:c$ can be evaluated as to whether it is in the relation or not. For an approach that loosens that restriction, see \cite{taylor23main}. The idea there is to view the relation more as a a mechanism for providing an answer to a question asked by a user. The questioner can provide a little bit of fuzziness. For those interested in practical usage of this idea, that approach is highly recommended. 

There are also other versions of the Separation property that lead to generalizations beyond real numbers. This version was selected for its simplicity both conceptually and in its ease of use. Some of the variants are discussed in the previously mentioned paper. 

\subsection{Propositions for Rational Betweenness Relations}

This is where we prove some useful general statements about rational betweenness relations. 

\begin{proposition}\label{br:rooted}
    If the $x$-between relation has a rational root $q$, then $a \xrel b$ if and only if $a : q : b$.
\end{proposition}

Note that $a \xrel b$ is stating that the interval $a:b$ is in the relation while $a:q:b$ is stating that the rational number $q$ is between the rational numbers $a$ and $b$.

\begin{proof}
    Let $a : b$ be given such that $a : q : b$ which is the same as saying $a:q:q:b$. Then since $q \xrel q$, Consistency says that $a \xrel b$. For the other direction, assume that $\sout{a:q:b}$. By trichotomy and potentially relabeling, assume that $a$ is between $q$ and $b$. By Consistency, $q \xrel b$ and $q\xrel a$.  Since $a \neq q$ and $q \xrel q$, Rooted states that \sout{$a \xrel a$}. Since $q \xrel a$ and $q:a:b$, Separation states that \sout{$ a \xrel b$}. 
\end{proof}


\begin{proposition}\label{br:existence}
    Given $\xrel$ and rational $a$, there exists $b$ such that $a \xrel b$. 
\end{proposition}

\begin{proof}
    Let $u \xrel v$ be an $x$-interval; that such an interval exists follows from the Existence property. If $a$ is not between $u$ and $v$, then by relabelling, it can be assumed that $a:u:v:v$. Then Consistency tells us that $a \xrel v$. In this case, $v$ is the required $b$. 

    The other case is $u:a:v$. If $a = u = v$, then $a \xrel a$. This is sufficient but also $a \xrel a+1$ would work as well. If $a = u \neq v$, then $a \xrel v$ applies. If $a  = v \neq u$, then $a \xrel u$. The last case is that $a$ is strictly between $u$ and $v$. Again, if $a \xrel a$, then that satisfies the requirements. Otherwise, Separation states that either $a \xrel u$ or $a \xrel v$. In either case, the desired $b$ exists. 
\end{proof}


\begin{proposition}\label{br:endpointed}
    If $a \xrel p$ and $p \xrel b$ with $a:p:b$, then $p \xrel p$.
\end{proposition}

\begin{proof}
    By Consistency, $a \xrel b$ as $a:a:p:b$.  By Separation, exactly one of the following holds true: $p \xrel p$,  \sout{$a \xrel p$}, or \sout{$p \xrel b$}. Since both $a:p$ and $p:b$ are in the relation, it must be the case that $p \xrel p$, as was to be shown. 
\end{proof}


\begin{proposition}\label{br:intersect}
    If $a \xrel b$ and $c \xrel d$, then $a:b$ and $c:d$ intersect in an $x$-interval. 
\end{proposition}

\begin{proof}
    The betweenness of the four rational numbers $a$, $b$, $c$, $d$ is well handled by cases. To minimize the cases, a relabelling assumption is that $a$ is one of the outer endpoints and $c$ is closer in the betweenness string to $a$ than $d$ is. We have the following cases: 

    \begin{enumerate}
        \item $a:b:c:d$, $b \neq c$. Let $m = \frac{b+c}{2}$. This is a rational number strictly between $b$ and $c$ with the strictness following from them not being equal to each other. Thus, $a:b:m:c:d$. By Consistency, $a\xrel m$ and $d \xrel m$. From Proposition \ref{br:endpointed}, this implies $m \xrel m$. But then Proposition \ref{br:rooted} states that since $m$ is not in $a:b$ nor $c:d$, they cannot be $x$-intervals. Having arrived at a contradiction, this case is incompatible with the assumptions of the statement. 
        \item $a:(b=c):d$. By Proposition \ref{br:endpointed}, $b:b$ is an $x$-singleton. This is the intersection of the two intervals and it is an $x$-interval. 
        \item $a:c:d:b$. The interval $c:d$ is the intersection and it was given as an $x$-interval. 
        \item $a:c:b:d$, $b \neq c$. The interval $c:b$ is the intersection. By Separation using $c$ and the interval $a:b$, one of the following holds true: 
        \begin{enumerate}
            \item $c \xrel b$. Since $c:b$ is the intersection, this case is exactly what is desired.
            \item $c \xrel c$. This implies $c \xrel b$ which is the intersection. 
            \item $a \xrel c$. By Proposition \ref{br:endpointed} and $a : c: d:$, this leads to $c \xrel c$. While this, by itself, would be fine and establishes $c \xrel b$, this is actually a contradiction as the Separation property also asserts in this case that $\sout{c \xrel b}$.
        \end{enumerate}
    \end{enumerate}
\end{proof}


\begin{proposition}\label{br:something-inside}
    If $a \xrel b$, then either $a \xrel a$ or there exists $c \neq a$ such that $a : c : b$ with $c \xrel b$. 
\end{proposition}

\begin{proof}
    Assume $\sout{a \xrel a}$. By the Closed property, there exists an $x$-interval $u \xrel v$ that does not contain $a$.

    By Proposition \ref{br:intersect}, let $c:d$ be the $x$-interval that is the intersection of $c:d$ and $a:b$. The intersection interval excludes $a$ as $a$ is not in $c:d$. Up to relabelling, the cases are $a:c:d:b$ or $a:c:b:d$. In the second case, $c:b$ is the intersection and is the $x$-interval. For the first case, $c:d$ is the intersection. By Consistency, $c \xrel b$ as $c:c:d:b$. 
\end{proof}

\begin{corollary}\label{br:doublesomething-inside}
    If $a \xrel b$ and neither $a$ nor $b$ are roots of $x$, then there exists $c \neq a$ and $d \neq b$ such that $a : c : d : b$ with $c \xrel d$. 
\end{corollary}

\begin{proof}
Apply Proposition \ref{br:something-inside} twice, with the intermediate step of $c \xrel b$ and then using it again with $b$ in the role of $a$. 
\end{proof}


\begin{proposition}\label{br:something-outside}
    If \sout{$a \xrel b$}, then there exists an $x$-interval disjoint from $a:b$.
\end{proposition}

\begin{proof}
    Since \sout{$a \xrel b$}, all subintervals, including the singletons $a:a$ and $b:b$, are not $x$-intervals by Consistency. By being Closed and using Existence, there exist $x$-intervals $A:A'$ and $B:B'$ such that $a$ and $b$ are not in those intervals, respectively. Since $x$-intervals intersect, there exists an $x$-interval $c:d$ which excludes $a$ and $b$. Thus, $c:d$ is either disjoint from $a:b$ or contained in $a:b$. It cannot be contained in $a:b$ as Consistency would then imply $a:b$ is an $x$-interval. Thus, $c:d$ is both disjoint from $a:b$ and $c:d$ is an $x$-interval.
\end{proof}


Two rational betweenness relations, $\xrel$ and $\xrel[y]$ are different if there exists an interval $a:b$ such that $a \xrel b$ and \sout{$a \xrel[y] b$} or vice versa. 

\begin{proposition}\label{br:different}
    If $\xrel$ and $\xrel[y]$ are different, then there exists two disjoint intervals $a:b$ and $c:d$ such that $a \xrel b$, \sout{$a \xrel[y] b$}, $c \xrel[y] d$, and \sout{$c \xrel d$}.
\end{proposition}


\begin{proof}
    By the definition of difference, there exists an interval $a:b$ which is an $x$-interval but not a $y$-interval, where a relabelling may be necessary as only one interval for one of the relations is guaranteed by the definition. By Proposition \ref{br:something-outside}, there exists a $y$-interval $c:d$ disjoint from $a:b$. Since it is disjoint from $a:b$, Proposition \ref{br:intersect} implies $c:d$ is not an $x$-interval. This is what was to be shown. 
\end{proof}


The final proposition demonstrates that there are arbitrarily small intervals in the relation. 

\begin{proposition}
    Given rational $\varepsilon >0$ and $x$ relation $\xrel$, there exists $a:b$ such that $|a-b| < \varepsilon$ and $a \xrel b$. 
\end{proposition}

This is the Intermediate Value Theorem stepping algorithm. 

\begin{proof}
    Let $a_0:b_0$ be an $x$-interval which exists by the Existence property. If $a_0 = b_0$, then the length is 0 and $a_0:b_0$ satisfies the result. 
    
    Assume $a_i:b_i$ is defined and $a_i \neq b_i$. Then define $L_i = |a_i - b_i|$. Let $m_i = \frac{a_i + b_i}{2}$; this is distinct from $a_i$ and $b_i$ due to them not being equal. By Separation, either 1) $a_i \xrel m_i$, 2) $b_i \xrel m_i$, or 3) $m_i \xrel m_i$. If it is 3), then $m_i:m_i$ satisfies the requirements as its length is 0. Otherwise, for 1), let $a_{i+1}= a_i$, $b_{i+1} = m_i$ and for 2) let $a_{i+1} = m_i$, $b_{i+1} = b_i$. Then $L_{i+1} = |a_{i+1} - b_{i+1}| = \frac{L_i}{2} = \frac{1}{2^{i+1}} L_0 $. 

    Since there exists an $N$ such that $2^N < \frac{\varepsilon}{L_0}$, this process will finish in at most $N$ steps. 

    
\end{proof}




\section{Dedekind cuts}

A first attempt at defining a Dedekind cut is that a cut is a pair of sets of rationals, $(A, B)$, such that: 1) $A \cup B = \mathbb{Q}$, 2) if $a \in A$ and $b \in B$, then $a < b$, and 3) both $A$ and $B$ are non-empty. This definition works well for representing irrational numbers uniquely, but fails to be unique for rational numbers. For example, let $q$ be the rational to represent. Then $A = \{a | a < q\}$ and $B= \{b | b \geq q\}$ is a cut representing $q$, but so is  $A' = \{a | a \leq q\}$ and $B'= \{b | b > q\}$. One possibility is to choose to always take one kind, such as requiring that $A$ has no greatest element. This is a common choice such as in \cite{rudin}, page 17. Another option, which will be used here, is to modify the definition slightly to have three potential sets in a cut.

A Dedekind cut is defined as a triplet of sets of rationals, $(A, B, C)$, such that: 
\begin{enumerate}
    \item Nonempty. Both $A$ and $C$ are non-empty.
    \item Ordered. If $a \in A$ and $c \in C$ then $a < c$. If $b \in B$, then also $a < b < c$. 
    \item Comprehensive. $A \cup B \cup C = \mathbb{Q}$.
    \item Singular. There is at most one element in $B$.
    \item Open. There is no greatest element in $A$ and no least element in $C$.
\end{enumerate}
If the cut is called $x$, then the notation that will be used will be $\qlt$ for $A$, $\qeq$ for $B$ which may be an empty set, and $\qgt$ for $C$. The notation $\qleq$ will be used for the set $\qlt \cup \qeq$ while $\qgeq$ denotes $\qgt \cup \qeq$. To refer to the cut itself, the notation $\cut$ will be used. 

If $a \in \qlt$ and $b < a$, then $b \in \qlt$. If $a \in \qgt$ and $ b > a$, then $b \in \qgt$. These follow by using the Comprehensive property to say that $b$ is in one of those sets and then ruling out the others by the Ordered property. 

If $\qeq = \{q\}$, then $\qlt$ is the set of rational numbers less than $q$ and $\qgt$ is the set of rational numbers greater than $q$. The notation $\qeq[q]$ where $q$ is a rational number will imply that this is the case. Cuts of this form represent the rational numbers and $q$ is the root of the cut. 

If $\qeq$ is empty, then the cut represents an irrational number. In that case, $\qgt$ and $\qlt$ are complements in the set of rational numbers. 


\subsection{Propositions for Dedekind Cuts}

This is a collection of some useful statements regarding Dedekind cuts. 

\begin{proposition}
    For a cut $x$, the sets $\qlt$, $\qeq$, $\qgt$ are pairwise disjoint. 
\end{proposition}

\begin{proof}
    This follows from the Ordered property. If an element was in two of the sets, then it would be less than itself which cannot be. 
\end{proof}

\begin{proposition}\label{br:cut-unbounded}
    Let $x$ be a cut, then if $a \in \qlt$, $b \in \qgt$, $p < a$ and $b < q$, then $p \in \qlt$ and $q \in \qgt$.
\end{proposition}

\begin{proof}
    Let $a,b, q, p$ be as in the statement. By the Comprehensive property, $p$ is an element of one of the sets. If it is in $\qgeq$, then $p > a$ by the Ordered property, but it is assumed that $p < a$. Thus, $p \in \qlt$. Similarly, if $q \in \qleq$, then $q < b$. Since it is not, $q \in \qgt$. 
\end{proof}

Define a set $A$ to be a \textbf{lower cut} if it is a nonempty set of rationals $A$, whose complement is also nonempty, and it satisfies the following two properties: 1) [Unbounded Below] if $p \in A$ and $q < p$, then $q \in A$; 2)[Open Above] if $p \in A$, then there exists $q \in A$ such that $p < q$.

The notation $A^C$ will denote the set complement of $A$. That is, $A$ and $A^C$ are disjoint with $A \cup A^C = \mathbb{Q}$.

\begin{lemma}
    For a lower cut $A$, the set $A^C$ is the set of upper bounds of $A$.
\end{lemma}

\begin{proof}
    Let $q \in A$ and $r \in A^C$. Since the sets are disjoint, $r \neq q$. If $r < q$, then $r \in A$ by the Unbounded Below property. This condition leads to $r \nless q$. By the Trichotomy property for rationals, $q > r$. As $q$ and $r$ were arbitrary elements in their sets, all of the elements in $A^C$ are strictly greater than the elements in $A$. To establish that all upper bounds are in $A^C$, let $s$ be such that no element of $A$ is less than $s$. If $s \in A$, then, by the Open Above property, there would be a $q \in A$ such that $q > s$. By assumption, that was not possible and $s \in A^C$. 
\end{proof}


\begin{proposition}
    If a set $A$ is a lower cut, then there is a cut $x$ such that $A = \qlt$.
\end{proposition}

\begin{proof}
    Let $A^C$ be the complement of $A$. If there exists $q \in A^C$ such that $q \leq r$ for all $r \in A^C$, then define $B = \{q\}$ and $C = A^C - \{q\}$. If there is no such least number $q$, then define $B = \{\}$ and $C = A^C$. 

    The claim is that $(A, B, C)$ forms a cut. Let us check the properties. 

    \begin{enumerate}
        \item Nonempty. $A$ is nonempty by definition. If there was no least element $q$ in $C$, then $C = A^C$ and, by assumption, $A^C$ was nonempty. If there is a least element, then $q+1 > q$ is an upper bound of $A$ and hence in $A^C$ by the lemma. Since $C$ and $A^C$ differ only by $q$, $q+1 \in C$. 
        \item Ordered. Let $a \in A$ and $c \in C$. Then the lemma says that $ a< c$. If $b \in B$, then the lemma also says that $a < b$. As $b$ is the least element of $A^C$ and $b \notin C$, $ b < c$. 
        \item Comprehensive. $A \cup B \cup C = A \cup A^C = \mathbb{Q}$ by definition. 
        \item Singular. By definition, $B$ has at most one element. 
        \item Open. The Open Above property establishes that $A$ has no greatest element in $A$. The extraction of the least element from $A^C$ ensures $C$ has no least element as given any number $r \in C$, the number $\frac{q+r}{2} > q$ is an upper bound of $A$ and thus in $C$. 
    \end{enumerate}
\end{proof}

The construction of a cut from a lower cut does require accepting the existence of a complement set as well as the ability to determine the existence of a least element of a set in addition to finding it if it exists. 

\section{From Relation to Cut and Back Again}

This section will establish a canonical bijection between the rational betweenness relations and Dedekind cuts. The term canonical refers to the fact that this bijection does not require any arbitrary choices in addition to the sense that this is the natural bijection. It maps the representative of rationals to each other, e.g., the zero cut is mapped to the zero relation. 

Given the betweenness relation $\xrel$, the associated Dedekind cut is defined by \begin{enumerate}
    \item $\qlt = \{ a | \exists b, c (a < c < b) \wedge (c \xrel b) \} $
    \item $\qgt = \{ b | \exists a, c (a < c < b) \wedge (a \xrel c) \} $
    \item $\qeq = \{ q | q \xrel q \} $
\end{enumerate}
Essentially, $\qlt$ is the set of lower endpoints of $x$-intervals, $\mathbb{Q}_{>x}$ is the set of upper endpoints of $x$-intervals, and $\qeq$ is the set of singletons in the $x$-between relation. The use of $c$ is present to ensure that if a singleton is present in the relation, then it does not appear in either of the inequality sets. 

Is this a cut? Yes: 
\begin{enumerate}
    \item Nonempty. Existence yields an $x$-interval $a:b$. By relabelling, $a \leq b$. Then $a-1< a$, $b+1 > b$. By definition of the cut, $a-1 \in \qlt$ and $b+1 \in \qgt$.
    \item Ordered. Let $a \in \qlt$ and $b \in \qgt$.  Then there exists $c, m$ such that $a < c < m$ and $c \xrel m$. There is also a $d, n$ such that $d < n< b$  and $d \xrel n$. By Proposition \ref{br:intersect}, $c \xrel m$ and $d \xrel n$ intersect. Let $p$ be an element of that intersection. Then $a < c \leq p \leq n < b$ implying $a < b$.  If $\qeq$ is not empty, then let $u$ be an element of $\qeq$. By Proposition \ref{br:rooted}, $c \xrel m$ and $d \xrel n$ implies that $\{c,d\}:u:\{m,n\}$ implying $a < u$ and $u < b$. 
    \item Comprehensive. Given a rational $a$, Proposition \ref{br:existence} yields the existence of $b$ such that $a \xrel b$. Proposition \ref{br:something-inside} establishes the existence of a $c \neq a$ such that $a:c:b$ with $c \xrel b$ unless $a \xrel a$ in which case $a \in \qeq$. In the other case, $a \in \qlt$ if $a < c$ while $a  \in \qgt$ if $a > c$. 
    \item Singular. The Rooted property of the relation prevents $\qeq$ from having more than one element. 
    \item Open. The existence of $c$ in the definition of the cuts establishes no greatest element in $\qlt$ and no least element in $\qgt$. 
\end{enumerate}

In the other direction, given the Dedekind $x$-cut, the associated $x$-betweenness relation is defined by, for $a \leq b$, $a \xrel b$ exactly when $a \in \qleq$ and $b \in \qgeq$. This automatically takes care of singletons and representing rationals. 

Is this a betweenness relation? Yes: 
\begin{enumerate}
    \item Existence. By the Nonempty property, we have the existence of $a \in \qlt$ and $b \in \qgt$ which implies $a \xrel b$. The Ordered property does imply $a < b$, but that is not needed here. 
    \item Separation. Let $a \xrel b$ with $a \leq b$.  Then $a \in \qleq$ and $b \in \qgeq$. Let $c$ be given such that $a:c:b$. If $c \in \qlt$, then  $c \xrel b$. It is also the case that \sout{$a \xrel c$} as both $a$ and $c$ are in $\qlt$ implying neither of them are in $\qgeq$. Similarly, if $c \in \qgt$, then \sout{$c \xrel b$} and $a \xrel c$. If $c \in \qeq$, then $c \in \qleq$ and $c \in \qgeq$ which implies $c \xrel c$. By the Comprehensive property, those are the only three possible cases for $c$ and they correspond to having the three different possible outcomes of the Separation property. 
    \item Consistency. Let $c:a:b:d$ with $ a \xrel b$. The claim is that $c \xrel d$. By relabelling, take $a \leq b$ which implies $a \in \qleq$ and $b \in \qgeq$.  The betweenness leads to $c \leq a$ and $ b \leq d$. This implies that $c \in \qleq$ and $ d \in \qgeq$. Thus, $c \xrel d$.
    \item Rooted. The three pieces of the cut are disjoint. The only way to get an $x$-singleton $q:q$ is if $q \in \qeq$. That set has at most one element. 
    \item Closed. Assume that $c$ is such that if $a \xrel b$, then $a:c:b$. Since the three cut sets disjointly span the rationals by the Comprehensive property, $c$ is in exactly one of those sets. If $c \in \qlt$, then, by the Open property, there exists $a \in \qlt$ such that $c < a$. Take any $b \in \qgt$ which can be done by the Nonempty property. Then $a \xrel b$ by definition, but $c: a:b$ and is therefore \sout{$a:c:b$}. Similarly, if $c \in \qgt$, then there would exist $b \in \qgt$ with $b < c$ and any $a \in \qlt$ would lead to $a:b:c$ with $a \xrel b$. With neither of these options available, it must be the case that $c \in \qeq$ implying $c \xrel c$.    
\end{enumerate}

The final task is to establish that these two are bijections and, in particular, they are the inverses of one another. 

To show that mapping from relations to cuts is one-to-one, let $\xrel$ and $\yrel$ represent two distinct rational betweenness relations. By Proposition \ref{br:different}, there exists $x$-interval $a:b$ and $y$-interval $c:d$ such that they are disjoint. By relabelling, we can assume that $a \leq b < c \leq d$. Then $c \in \qleq[y]$, $d \in \qgeq[y]$ while $c, d \in \qgt[x]$. Thus, the relations are mapped to different cuts. 

To show that mapping from the relations to the cuts is onto, the path will be to show that the cut-to-relation mapping is undone by the relation-to-cut mapping. Let a cut $\cut$ be given. Let $\yrel$ represent the relation that the cut is mapped to. Let $\cut[z]$ represent the cut that $\yrel$ is mapped to. The goal is to show that $\cut = \cut[z]$.  

Let $a \in \qlt[z]$ and $b \in \qgt[z]$. Part of the task is to show $a \in \qlt$ and $b \in \qgt$. By the definition of the relation-to-cut mapping and Proposition \ref{br:intersect}, there exists $c$ and $d$ such that $a < c \leq d < b$ such that $c \yrel d$. By the definition of the cut-to-relation mapping, $c \in \qleq$ and $d \in \qgeq$. Because $a < c$ and $ d < b$, $a \in \qlt$ and $b \in \qgt$. 

The only set left to explore is $\qeq[z]$. This is two parts. If $\qeq[z]$ is empty, then that implies that $c:c$ is not a $y$-interval for all rational $c$. Since singletons $c:c$ are a $y$-interval exactly when $c$ is in both $\qleq$ and $\qgeq$, that intersection is empty. Since the elements of $\qeq$ are common to both unions, this implies $\qeq$ is empty. The other part is if $c \in \qeq[z]$. This means $c \yrel c$. But that means that $c$ was an element of the intersection of $\qleq$ and $\qgeq$. Since $\qlt$, $\qeq$, and $\qgt$ are disjoint, this implies $c \in \qeq$ as was needed to be shown. 


We have established that
\begin{theorem}
    The set of rational Dedekind cuts is in canonical bijective correspondence with the set of rational betweenness relations. The representative of the rational $q$ in the relations does correspond to the cut associated with $q$.
\end{theorem}

From this point on, using $x$ in the relation notation $\xrel$ and the cut notation $\cut$, or in the sets of the cut $(\qlt, \qeq, \qgt)$, implies that they are related by this canonical bijection. 

\section{Ordering the Relations}

Dedekind cuts have been established to be a complete ordered field. This paper will establish that for the betweenness relations by showing that the natural operations on the betweenness relations is mapped to those operations on their associated cuts. This section will be on the order relation and least upper bound property. The following section will detail the arithmetic. 

Given two intervals $a:b$ and $c:d$, the notion that $a:b < c:d$ is that given any $p$ and $q$ such that $a:p:b$ and $c:q:d$, it can be concluded that $p < q$. This includes the endpoints and can be determined by comparing endpoints. Indeed, if $a\leq b$ and $c \leq d$, then $a:b < c:d$ exactly when $b < c$. 

From the relation perspective, $x < y$ occurs, by definition, exactly when there exists an $x$-interval $a:b$ and a $y$-interval $c:d$ such that $a:b < c:d$. By Proposition \ref{br:different}, given two distinct relations $\xrel$ and $\yrel$, there exists disjoint intervals $a:b$ and $c:d$ such that $a \xrel b$ and $c \yrel d$. Disjointness means that one of the intervals is wholly greater than the other. By Proposition \ref{br:intersect}, that order relation cannot be reversed.  This may be denoted as $\xrel < \yrel$. 

If both relations happen to be  rooted with roots $p$ and $q$, respectively to the relations $\xrel$ and $\yrel$, then $x<y$ if and only if $p < q$ if and only if $p:p < q:q$. 

For Dedekind cuts, $x < y$ occurs, by definition, exactly when $\qlt \subset \qlt[y]$. This would imply that $\qgt[y] \subset \qgt$. For cuts representing rationals, the solitary elements of $\qeq$ and $\qeq[y]$ will have that ordering though there is no set relation that expresses it. To emphasize it as a cut, the inequality may be denoted as $\cut < \cut[y]$.

The ordering definition of Dedekind cuts is well-established to obey all the desired properties. The task here is to show that the relation ordering and the Dedekind ordering correspond under the bijection which implies the relation ordering also has the same desired properties. 

Assume that $\xrel < \yrel$ which implies there exists $x$-interval $a:b$, $a \leq b$,  and $y$-interval $c:d$, $c \leq d$, such that $a:b < c:d$. Then $a \in \qleq$, $b \in \qgeq$, $c \in \qleq[y]$, $d \in \qgeq[y]$. Since $ b< c$, $b \in \qlt[y]$. Since all elements of $q \in \qlt$ are less than $b$, $\qlt \subset \qlt[y]$. Because $b\notin \qlt$, these are distinct sets and $\cut < \cut[y]$. 

For the other direction, assume $\cut < \cut[y]$. Let $b \in \qlt[y]$ which is not in $\qlt$; this implies $b \in \qgeq$. Let $a \in \qlt$ and $d \in \qgt[y]$.  Because there is no greatest element of $\qlt[y]$, let $c\in \qlt[y]$ such that $b < c$. By the definition of the bijection,  $a \xrel b$ and $c \yrel d$. Because $a < b < c < d$, it is that case that $a:b < c:d$ so that $\xrel < \yrel$, as was to be shown. 

\begin{theorem}
For relations $\xrel$ and $\yrel$, $\xrel < \yrel$ if and only if $\cut < \cut[y]$.
\end{theorem}

\subsection{Least Upper Bound}

With an ordering, the least upper bound of a set can be queried as to whether it exists or not. Given a non-empty set $E$ of cuts that is bounded above by a cut $z$, then the least upper bound $\alpha$ of $E$ is defined by $\qlt[\alpha] = \bigcup_{x \in E} \qlt$. This is a lower cut: 
\begin{enumerate}
    \item Nonempty. Let $\cut$ be an element of $E$ as $E$ is nonempty. Let $q \in \qlt$. Then $q \in \qlt[\alpha]$.
    \item Complement is Nonempty. Let $\cut[z]$ be an upper bound of $E$. This implies $\qlt \subseteq \qlt[z]$ for all $x \in E$. Let $r \in \qgt[z]$. Then $r > p$ for any $p \in \qlt[z]$ implying $r \notin \qlt[x]$ for any cut $x$ in $E$. Thus, it is not in the union of the lower cuts and is in the complement. 
    \item Unbounded Below. Let $q \in \qlt[\alpha]$ and $p < q$. Then there exists a cut $x$ in $E$ such that $q \in \qlt$. By Proposition \ref{br:cut-unbounded}, $p \in \qlt$ and thus $p \in \qlt[\alpha]$.
    \item Open Above. Let $q \in \qlt[\alpha]$. Then there exists a cut $x \in E$ such that $q \in \qlt$. By the Open property, there exists an element $r \in \qlt$ such that $r > q$. Thus, $r \in \qlt[\alpha]$ satisfying the Open Above property. 
\end{enumerate}
Being a lower cut, $\qlt[\alpha]$ generates the full cut $\cut[\alpha]$.

That $\alpha$ is an upper bound is clear from $\qlt[\alpha]$ containing every $\qlt$ for $x \in E$. Being the least upper bound requires showing that if $\beta$ is an upper bound of $E$, then $\alpha \leq \beta$. What this amounts to showing is that $\qlt[\alpha] \subset \qlt[\beta]$. Let $p \in \qlt[\alpha]$. Then $p \in \qlt$ for some $x \in E$. Since $\beta > x$ as it is an upper bound, $\qlt \subset \qlt[\beta]$ and thus $p \in \qlt[\beta]$. As $p$ was an arbitrary element of $\qlt[\alpha]$, this  establishes that $\qlt[\alpha] \subset \qlt[\beta]$.

The above has shown that Dedekind cuts are complete. For the relations, the bijection  maps the least upper bound $\cut[\alpha]$ to the relation $\xrel[\alpha]$. Because the bijection respects the ordering, the image will be the least upper bound. 

By definition of the bijection, $a \xrel[\alpha] b$, $a \leq b$, exactly when $a \in \qleq [\alpha]$ and $b \in \qgeq[\alpha]$. In words, $b$ is a rational upper bound of of $E$ while $a$ is a rational lower bound of the set of upper bounds of $E$. This is what one would expect of the least upper bound of the relations. 

\begin{theorem}
    The set of rational betweenness relations satisfies the least upper bound property. 
\end{theorem}


\section{The Arithmetic of Relations}

The arithmetic of the relations is most simply described as doing the arithmetic on the endpoints of intervals. This creates a core of intervals for which Consistency and the Closed property then expand it to the full relation. This section first presents the core of the arithmetic of relations with a discussion of the trickier aspects of the arithmetic. The well-known arithmetic of Dedekind cuts is then explored. The section concludes by showing that the canonical bijection between the relations and cuts commutes with the arithmetic definitions which allows for the use of Dedekind cuts to establish the full arithmetic operation definition and their properties. 

To make the correspondence easier, the intervals in the core of the arithmetic will be taken to be Spaced intervals. An $x$-\textbf{Spaced interval} in a relation $\xrel$ is an interval $a:b$, $a < b$, such that $a \in \qlt$ and $b \in \qgt$. This is essentially stating that neither $a$ nor $b$ are roots of the relation. 

In what follows, the intervals discussed are the core intervals of defining the arithmetic operations. To fully define the relation, Consistency and the Closed property would require adding more intervals. This is done in this paper by defining the operations using the lower endpoints of the core intervals to define the lower cut associated with the operation; this aligns with the customary Dedekind cut definition. Using the bijection, the cuts being defined then complete the definition of the relation.  

\begin{enumerate}
    \item Addition. $u \xrel[x+y] v$ if it is an interval of the form $(a+c):(b+d)$ where $a:b$ is an $x$-Spaced interval, $c:d$ is a $y$-Spaced interval, $a < b$, and $c < d$. The definition of cut addition is based on $\qlt[x+y] = \{a + c | a \in \qlt, c \in \qlt[y]\} $. This is exactly the lower endpoints of the core addition intervals. 
    \item Negation. $u \xrel[-x] v$ if it is an interval of the form $u:v$ contains an interval of the form $-b:-a$ where $a:b$ is an $x$-Spaced interval. 
    \item Multiplication. $u \xrel[xy] v$  if it is an interval of the form $m:M$ where $m = \min(ac, bc, ad, bd)$, $M = \max(ac, bc, ad, bd)$, where $a:b$ is an $x$-Spaced interval, $c:d$ is a $y$-Spaced interval. For $0 \leq a < b$ and $0 \leq c < d$, this translates to $ac:bd$. 
    \item Reciprocation. $u \xrel[\frac{1}{x}] v$ if it is an interval of the form $\frac{1}{b}:\frac{1}{a}$ where $a:b$ is an $x$-Spaced interval which does not contain 0. 
    \item Powering. This could be repeated multiplication or it can be directly defined as $u \xrel[x^n] v$ is an interval of the form $a^n: b^n$ where $a:b$ is an $x$-Spaced interval that does not contain 0. For $x = 0$, $0^n$ is defined as $0$. 
\end{enumerate}

A tricky aspect of real number arithmetic is that of two irrationals combining together to form a rational number. For example, $\sqrt{2} - \sqrt{2} = 0$. From the relations point of view, every interval in the relation will contain $0$, indicating that the root is $0$, but neither $0:0$ nor $0:1$ would be intervals in the relation via direct combination. This point is addressed in the context of the relations in \cite{taylor23main} without relying on the Dedekind cuts. That paper also establishes the ordered field properties without relying on the cuts. 

For the Dedekind cuts, the definition of these operators focuses on generating the set $\qlt$ which is then used to generate the other two sets. This is useful as the generation of a rational from irrationals will not produce the $\qeq$ one element set directly. Using the previous example, $ x = \sqrt{2} - \sqrt{2}$ will correctly have $\qlt = \{ q | q < 0\}$, but it can not directly generate $\qeq = \{0\}$. By using the complement of $\qlt$, 0 automatically is part of $\qgeq$ and can be extracted by being the least element of that set.

In what follows, the arithmetic of Dedekind cuts will be defined, roughly following \cite{rudin}, pages 17-21.  At times, it is necessary to focus on $x > 0$ at which point the notation $\qlt^+$ is useful and it denotes the positive elements of the set of $\qlt$. It is also important to note that defining $\qlt^+$ defines $\qlt$ by adding in all of the non-positive rational numbers.
 
\begin{enumerate}
    \item Addition. $\qlt[x+y] = \{q + r | q \in \qlt, r \in \qlt[y]\} $. Its complement are all the numbers that are not of that form. Part of that will be $\qlt[x+y] = \{q + r | q \in \qgt, r \in \qgt[y]\}$. It can be established that the least upper bound of $\qlt[x+y]$ is the greatest lower bound of $\qgt[x+y]$; if this happens to be a cut representing a rational number, then that is the sole element of $\qeq[x+y]$; otherwise that set is empty. 
    \item Negation. $\qlt[-x] = \{-q | q \in \qgt\}$ and this leads to $\qgt[-x] = \{-q | q \in \qlt\}$. If $x$ is rational, then $\qeq[-x] = \{-q\}$ where $\qeq = \{q\}$; if $x$ is not rational, then $\qeq[-x] = \{\} = \qeq$. It is easy to establish that $x-x = 0$ for any cut $x$. Indeed, if $q \in \qlt[-x]$, then $-q \in \qgt$. Thus, if $p \in \qlt$, it is the case that $p+q < 0$ as this is just $p < -q$ which follows from how the sets are related. 
    \item Multiplication. This is tricky from a cut perspective since multiplication of negative numbers reverses inequalities. For that reason, multiplication via the cut definition is generally defined for multiplying positive rationals and then imposing the usual negation and zero relations in defining multiplication of cuts involving negative cuts. Following this, let $x , y > 0$. Then $\qlt^+[xy] = \{qr | q \in \qlt^+, r \in \qlt[y]^+ \}$ and $\qgt^+[xy] = \{qr | q \in \qgt^+,  r \in \qgt[y]^+ \}$. Again, $\qeq[xy]$ is populated with the least upper bound of $\qlt[xy]^+$ which is the same as the greatest lower bound of $\qgt[xy]^+$. 
    \item Reciprocation. This is restricted to non-zero numbers. Since one can double negate, with reciprocating in between, it is sufficient to discuss positive cuts.  The set $\qlt[1/x]^+ = \{\frac{1}{q} | q \in \qgt\}$ and $\qgt[1/x] = \{ \frac{1}{q} | q \in \qlt^+ \}$. If $x$ is rational (and non-zero), then $\qeq[1/x] = \{\frac{1}{q}\}$ where $\qeq = \{q\}$; it is empty if $x$ is not rational. 
    \item Powering. One can tell a similar story to multiplication, restricting to positive numbers and expanding by manual figuring of signs. Alternatively, one can work with the definition via repeated multiplication. 
\end{enumerate}

With the operations defined for both cuts and relations, the task is to show the bijection commutes with the operations. Once that is established, then all of the properties for an ordered field will flow by using the bijection and the fact that those properties have been established for Dedekind cuts. 

\begin{enumerate}
    \item Addition. Given $x$ and $y$, it needs to be shown that the core of $\xrel[x+y]$ is mapped to the cut $\cut[x+y]$ under the canonical bijection. This is done by showing that for any given $u \xrel[x+y] v$, with $u < v$, that $u \in \qleq[x+y]$ and $v \in \qgeq[x+y]$. By definition of relational addition, there exists $a \leq b$, $c \leq d$ such that $a \xrel b$, $c\xrel d$, $u = a+c$, and $v = b+d$. But, by definition of the cuts, $a \in \qleq$, $b $
\end{enumerate}
The mapping of this cut to the relations is pretty clear as the endpoints are of the form of the elements in these sets. For example, if $a \xrel[x+y] b$ with $ a \in \qlt[x+y], b \in \qgt[x+y]$, then there exists $q <r, s < t$ such that, $a = q+s$, $b = r+t$, $q \xrel r$,  and $s \yrel t$ where the relations imply that  $q \in \qlt$, $s \in \qlt[y]$, $r \in \qgt$, and $t \in \qgt[y]$.  The one notable exception to this scheme is if there is a singleton, but that follows from the agreement on least upper bounds under the bijective correspondence. 

The relational mapping conforms to picking the bounds from these sets. 
The same argument for addition applies here for the cut $\cut[xy]$ mapping to the relation $\xrel[xy]$ being pulled from the bounds of the sets. The singleton inclusion matches the least upper bound bijective correspondence. 

The mapping to the relations yields the core relation to be $a \xrel b$ when $\frac{1}{a} \xrel[\frac{1}{x}] \frac{1}{b}$, assuming $0:a:b$. The intervals containing zero are dealt with by Consistency. 


\begin{theorem}
The canonical bijection between cuts and rational betweenness relations is an isomorphism which establishes that the rational betweenness relations form a complete, ordered field with the rational numbers as a dense subset. 
\end{theorem}

\section{Commonly Used Real Number Setups}

It can be useful to see how well-known real numbers are given by the rational betweenness relations. Rationals, as previously mentioned, are the relations containing a root with that root being the rational in question; an interval is in the relation if it contains that rational number. The least upper bound has also been mentioned as essentially the intervals whose lower endpoint is a lower bound of the upper bounds and the upper endpoint is one of the upper bounds. 

The next step after rationals is that of the $n$-th root. Given positive $q$, the relation defining the $n$-th root of $q$ is $a \xrel[q^{1/n}] b$ exactly when $a^n: q: b^n$. Extending this to the $n$-th root of $x$, one would have $a \xrel[x^{1/n}] b$ when there exists $u \xrel v$ such that $a^n:u:v:b^n$.

A Cauchy sequence $x_n$ converges to $x$ where $a \xrel b$ exactly when $a:b$ contains the tail of the sequence. In particular, there exists an $N$ such that for any $x_n$ with $n \geq N$, there exists $a_n \xrel[x_n] b_n$ such that $a:{a_n, b_n}:b$. That such an $N$ happens follows from the Cauchy criterion.  

With Cauchy sequences, one can then have convergent sums fairly easily. By bounding the tail of a sum, one can construct an interval and for convergent sums, that bound can be made as small as one pleases. Thus, there is a rational betweenness relation that it represents. This easily extends to absolutely convergent sums, i.e., if $ \sum_{i=0}^{\infty} |a_i|$ converges, then $\sum_{i=0}^n a_i - \sum_{n+1}^{\infty} |a_i| \xrel[\sum_{i=0}^{\infty} a_i] \sum_{i=0}^n a_i + \sum_{n+1}^{\infty} |a_i|$ 

The inspiration of the Separation property is that of the Intermediate Value Theorem. If a function is continuous and strictly monotonic on an interval $a:b$ with $f(a)*f(b) \leq 0$, then there exists a unique $x$ in $a:b$ such that $f(x) = 0$. The interval $c:d$ is an $x$-interval exactly when $c:d$ intersects $a:b$ and letting $m:n$ be the intersection interval, it is that case that $f(m)*f(n) \leq 0$. The conditions allows the Separation property to hold. 

Many algorithms, such as Newton's method in the context of Kantorovich's theorem, can be readily seen as establishing the existence of a rational betweenness relation. The requirement is that there is a family of intervals that are overlapping and there are intervals that exist for any given small $\varepsilon>0$. Such families can be readily extended to a full betweenness relation. 

The relations also suggest useful presentations. For example, $\lim_{n \to \infty} (1+\frac{1}{n})^n = e$ is a classic example from introductory calculus. But most presentations lack a notion of precision over the imprecision of each $n$ approximation. From the relations perspective, having an upper bound is essentially demanded. In this example, it is simple: $(1 + \frac{1}{n})^{n+1}$ is an upper bound for all terms past $n$; see \cite{mend} for example. For example, $n=1$ yields $2:4$, $n=2$ yields $\frac{9}{4}:\frac{27}{8}$, and $n=3$ yields $\frac{64}{27}:\frac{256}{81}$. Not only is that a bit of a pleasing pattern, but it yields a concrete sense of converging on the limiting  value with clarity as to the uncertainty. 

The bisection method comes freely out of the Separation property. But the Separation property facilitates choosing any rational number in the interval. This allows for using mediants instead of midpoints. While generally slower convergence, it has the advantage of being more efficient in the size of the numerator and denominator. In fact, if started on an interval with consecutive integer endpoints, the mediant method will generate the continued fraction expansion of the real number $x$ which provides the best fitting approximations for a given size of the denominator. For rational numbers, the mediant process will produce the rational in a finite number of steps while the midpoint process is not guaranteed to do so. 

All of these results, and more, are explored and established in \cite{taylor23main}.

\section{Relations versus Cuts}

There are many definitions of real numbers; see \cite{ittay-2015} for an excellent overview of many of them. Most have a flavor in common with the decimal presentation and Cauchy sequences, that is, there is an element of computation in the very definition of the numbers. Often the representation is not unique; Cauchy sequences are extremely non-unique while decimals only falter on the trailing 9s vs trailing 0s though this comes from having chosen a base to represent the number which seems unappealing for a fundamental definition.   

There are also definitions related to intervals. A nested sequence of intervals being a real number was proposed around the same time as Cauchy sequences and Dedekind cuts. The constructivists, such as in \cite{bridger}, use families of intersecting, arbitrarily fine intervals to represent real numbers. Both of those approaches also lack uniqueness of representatives. 

Dedekind cuts and the rational betweeenness relations have the benefit of being unique in their representation of a real number. They both focus on irrationals being holes between rational numbers. The question then arises, which one is better as a definition for a real number? 

One can look at generalizations, as mentioned in the next section, and argue that the relations generalize better than the cuts as the cuts require a splitting while the relations require containment, a much more generalizable concept. 

Setting that aside and focusing on real numbers, the relations are superior from a computational point of view. Think about computing a real number for some other computation. The relation approach naturally leads to using intervals. If one has a specific length requirement, the tools and structures required to get to that level come with the relation definition itself. For cuts, it has to be built up. The cut structure itself seems less relevant. This may happen with definitions, but it is preferable to have definitions that align with the use of the object, if possible. 

In particular, imagine trying to get close to $\pi$ and let's say it is known that $\pi$ is between 3 and 4. For the relations, this is asserting that $3 \xrel[\pi] 4$. It is just a restatement of that fact. For the cuts, it would be the assertion that $3 \in \qlt[\pi]$ and $4 \in \qgt[\pi]$. It feels as if the message of the cuts is that we are filling up the sets $\qlt[\pi]$ and $\qgt[\pi]$. It has the feeling of looking away from the real number rather than the relations viewpoint which is to enclose the real number. 

Focusing on intervals containing real numbers seems a more natural viewpoint and one which the rational betweenness relations provides a unique representative for a given real number. 


\section{Idealizations and Generalizations}

The rational betweenness relations as presented here conform perfectly to the typical Dedekind cut presentation and usual real number story. But real numbers are messier than this story can actually handle. 

Take, for example, asking whether $\sqrt{2}^2 = 2$. By the explicit definitions, one can argue that this is true. But given just the relations or just the contents of the set, could one argue this? One can easily produce intervals as close to 2 as one likes, but to establish that 2 itself is in there requires a proof rather than a computation. 

This is a problem with both the relations and the Dedekind cuts. For the relations, it is not possible to answer for every singleton whether it is in the relation or not. This make the Separation property falter. For the cuts, it is not possible to always answer which of the three sets a given rational number is part of. 

The solution to this is to give a bit of fuzziness. It requires recasting this into more of a procedure that yields an answer when asked about a number, with an error tolerance also provided. The relation definition is modified in part by adding to the Separation property a little region around the dividing number $c$ to focus on instead of $c:c$. This is explored and developed in \cite{taylor23main}.

There is an alternate version of the Separation property in which two rational numbers are given in the interval and the property requires there be an interval in the relation which excludes at least one of those numbers. This property is not as crisp as the one presented here, but it does have the advantage of handling the fuzziness automatically as well as generalizing to metric spaces, as explored in \cite{taylor23metric}. Intervals are replaced with closed balls. 

This relational idea can also be extended to functions. Instead of intervals, the relation would contain rectangles whose wall would notionally contain the image of its base under the application of the function. With the assumption of what is used here, namely, that all singletons can be evaluated, the natural conclusion of this setup are functions that are continuous on the irrationals but potentially discontinuous on the rationals. For the fuzzier version of this setup, the natural conclusion is that these functions need to be continuous. Mark Bridger explores  these ideas in a similar spirit and setup to here in \cite{bridger}.

Rational betweenness relations are a new way of perceiving the real numbers. They give a clear and accessible definition while also being able to help with computing out the number. They encourage precision thinking about the imprecision associated with a real number. The concept nicely generalizes not only to other topological spaces, but also to the very notion of what functions on these spaces might mean. They represent how finite beings such as ourselves can meaningfully claim knowledge of an inherently infinite real number. 


\medskip

\normalem %restoring normal emphasis in bibliography 
\printbibliography

\end{document}


\documentclass{beamer}
\usetheme{CambridgeUS}
\usecolortheme{seahorse}
\usefonttheme{serif}

\usepackage{lipsum}
\usepackage{graphicx,xcolor}
\usepackage{amsmath,amssymb,amsfonts}
\usepackage{realoracles}

\AtBeginSection[]{ 
    \begin{frame}{Outline} 
        \tableofcontents[currentsection] 
    \end{frame} }

\title[Oracles]{Defining Real Numbers as Oracles}
\subtitle{\texorpdfstring{Is $x$ in $a:b$?}{Is x in a:b}}
\author{James Taylor}
\institute{ratmath.com}
\date{August 2023}

\setlength{\parskip}{\baselineskip} 

\begin{document}

\begin{frame}
\titlepage
\end{frame}


\begin{frame}{What is a real number?}
    A real number is an object that tells us whether or not it is between any given two rational numbers. 

    It is a rule which, when given $a \leq b$, says Yes (1) if the real number $x$ satisfies $a \leq x \leq b$ and No (0) otherwise. 

    Since we are defining $x$, this is aspirational. What we need are properties that will tell us whether we are talking about such an object. 

    These rules are what we call oracles. 
    
\end{frame}

\begin{frame}{Interval Notation}

    We use $a \lte b$ to indicate the rule that, when given rational $q$, it says Yes if  $a \leq q \leq b$. We use dots to indicate that we are only talking about rationals. 

    Most of the time, the ordering of $a$ and $b$ does not matter. We use $a : b$ in that case. $q \in a:b$ iff $a \leq q \leq b$ or $b \leq q \leq a$. This is the suborder property of betweenness. 

    $a:a$ is the singleton, namely $q \in a:a$ iff $q = a$. 

    $a:b$ is a neighborly interval if $a \neq b$.

    $c: a:b :d$ means if $q \in a:b$ then $q \in c:d$.
    
\end{frame}


\begin{frame}{Oracle Properties}
    An oracle is a rule $R$ that, when given the rational $a:b$, answers either $0$ or $1$ and satisfies:
    \begin{enumerate}
    \item Consistency. If  $c: a:b : d$ and $R(a:b) = 1$, then $R(c:d) = 1$.
    \pause
    \item Existence. There exists $a:b$ such that $R(a:b) = 1$.
    \pause
    \item Closed. If $q \in a:b$ for every Yes interval $a:b$, then $R(q:q) = 1$. 
    \pause
    \item Rooted. If $q \neq p$ and $R(q:q) =1$, then $R(p:p) = 0$.
    \pause
    \item Interval Separation. If $c \in a:b$ and $R(a:b) = 1$, then either $R(a:c) \neq R(c:b)$ or $R(c:c) =1$.
    \end{enumerate}

    Note an interval $a:b$ is an $R$-Yes interval iff $R(a:b) = 1$. It is an $R$-No interval otherwise. We often abbreviate to just Yes or No interval if speaking of  just one oracle. 

\end{frame}


\begin{frame}{Rational Oracles}
    Given a rational $q$, the corresponding oracle is given by the rule $R(a:b) = 1$ iff $q \in a:b$. We say it is rooted at $q$.

    \begin{enumerate}
    \item Consistency. If $R(a:b) = 1$, then $q \in a:b$. If $c:d$ contains $a:b$, then it contains $q$. So $R(c:d) = 1$.
    \pause
    \item Existence. $R(q:q) = 1$. 
    \pause
    \item Closed. The only rational in all Yes intervals is $q$ which satisfies $R(q:q) = 1$. 
    \pause
    \item Rooted. If $q \neq p$, then $p:p$ does not contain $q$ and so $R(p:p) = 0$.
    \pause
    \item Interval Separation. If $c \in a \lte b$ and $R(a:b) = 1$, then we could have: $q = c$ leading to $R(c:c) = 1$, or $ q< c$ in which case $R(a:c) = 1$, $R(c:b) = 0$, or $q > c$ in which case $R(c:b) = 1$ and $R(a:c)=0$. 
    \end{enumerate}
    
\end{frame}


\begin{frame}{\texorpdfstring{$x^3 = 2$}{xcubeeq2}}
    The oracle of the cube root of 2 is the rule such that $a:b$ is a Yes interval exactly when $a^3 : 2 : b^3$.

    \begin{enumerate}
    \item Consistency. Follows from monotonicity since if $c:a:b:d$ and $a:b$ is Yes, then  $c^3 : a^3 : 2 : b^3 : d^3$
    \pause
    \item Existence. $1:2$ satisfies $1^3 : 2 : 2^3 = 1:2:8$. 
    \pause
    \item Closed. Given any rational $q$, we can find rational $a:b$ such that $q^3 : a^3 : 2 :b^3$. This requires a bit of a technical argument.
    \pause
    \item Rooted. There are no rationals such that $q^3 : 2 : q^3$.
    \pause
    \item Interval Separation. If $a^3 : 2 : b^3$ and $a\lte c \lte b$, then if $c^3 > 2$, we have $a^3 : 2 : c^3 :b^3$ and so $a:c$ is Yes while $b:c$ is No. It is reversed if $c^3 < 2$. 
    \end{enumerate}

    As explained elsewhere, the cube is more or less cubing the endpoints of the interval. So this is the positive cube root of 2. 
    
\end{frame}



\begin{frame}{Consistency}
    Consistency says that once we say the real number is in an interval then everything consistent with that statement is determined. 

    This means that Yes propagates upwards, No propagates downwards. 

    This is to help ensure maximality and uniqueness. The need for it can be seen by saying that $a:b$ is Yes exactly when $-1 : a: 0 : b: 1$. 

    This breaks the upwards Yes consistency but otherwise picks out 0 as the root of the oracle. There is nothing wrong with this from a practical consideration, but breaks uniqueness. 
    
\end{frame}


\begin{frame}{Existence}
    This ensures non-triviality. It also highlights that we ought to have a starting place. 

    The Yes interval can be a very large encompassing interval. It basically says we know something about the number. 

    The rule which says No to all intervals satisfies all of the other (conditional) properties. Existence is what stops it from being considered an oracle. 
\end{frame}



\begin{frame}{Closed}
    We use this for uniqueness and to ensure that we can use the rational singleton in oracle arithmetic. 

    Example for need is the rule $R(a<b)= 1$ exactly when $a < 0 \leq b$ and No otherwise. 

    This satisfies Consistency and Existence easily.  

    It satisfies Rooted since there are no Yes singletons. 

    It satisfies Separation since if we choose $0$, then $a \leq 0$ is Yes while $0 < b$ is No and $0:0$ is No. For other partition points, the subinterval containing $0$ is Yes. 

    Note $S(a \leq 0 < b )= 1$ would be a different rule which would work the same way. Thus, two different pseudo oracles for representing 0. 
    
\end{frame}

\begin{frame}{Rooted}
     Rootedness partners with separation to ensure narrowing down. 

     Consider $R(a:b) = 1$ if $0:a:b:1$ or contains such an interval. 

     Then we have Consistency and Existence. 

     We have $R(q:q) = 1$ iff $0:q:1$. There are no rationals contained in all Yes intervals so Closed is satisfied. 

     Separation is satisfied as well since if we take $0:a:c:b:1$ then $R(c:c)=1$. If $c$ is outside $0:1$, then we can take the partitioned subinterval that intersects $0:1$.
\end{frame}



\begin{frame}{Separation}
    This forces the narrowing of the intervals. 

    The rest do not force this. For example, $a<b$ is Yes if either $1:a:b:2$ or $3:a:b:4$ or contains such an interval. All singletons are No.

    Consistency is ensured by that last bit. 
    
    Existence is given by $1:2$. 

    Rooted is satisfied since there are no Yes singletons. 
    
    It is closed since there are disjoint Yes intervals. 
    
\end{frame}



%
%\section*{References}
%\begin{frame}
%    \frametitle{References}
%    \bibliographystyle{amsalpha}
%    \bibliography{bib.bib}
%\end{frame}

\end{document}
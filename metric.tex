\documentclass[12pt]{article}
\usepackage{personal}
\usepackage{realoracles}

\title{Topological Completions with Oracles}
\jtauthor
\date{March 1, 2023}




%\sloppy%\openup-.1\jot
\begin{document}\maketitle
\begin{abstract}
A real number can be defined as an oracle that responds Yes or No depending on if that real number ought to be considered to be in the given rational interval or not. This idea can be extended to more general settings. This paper investigates oracles in the context of metric spaces, general topological spaces, and a relatively new approach to topology, the theory of linear structures. This paper also explores generalizing the notion of function oracles, namely, oracles that take a generalized rectangle and decide whether or not the image of the base is contained in the wall of the rectangle. This is a proposed replacement for classical functions in order to respect a space being a completed space. 
\end{abstract}

\section{Introduction}

Our task is to explore how to complete various spaces using the notion of oracles. Oracles are defined on a collection of containers of points in a space. For completing the rationals to real numbers, these containers were inclusive rational intervals which includes an interval of exactly one rational. In a general topological space, our containers will be the closed sets. For it to work out well, the topological space should be completely regular which, among other things, implies singletons are closed sets. For metric spaces, the containers will be closed balls with the data being the center point and the radius. Balls of radius 0 are allowed, producing the singletons. For Linear Structures, the containers will be closed sets, but the notion of being closed is distinctly different from that of topological structures. 

For our purposes, singletons should be examples of containers. Non-empty pairwise intersections should either again be a container or contain containers. Containment of containers by other containers should make sense. 

For all of these different kinds of containers, the definition of oracles is largely the same. The intuition is that an oracle says Yes to a container exactly when the desired point ought to be in the container. Since we are defining the desired point, we cannot actually do this. 

We define an oracle to be a rule R which, given a container, says Yes or No (1 or 0) and satisfies the following conditions ($A$ and $B$ are containers, $q$ is a point in the space): 
\begin{enumerate}
    \item Consistency. If $A$ contains $B$ and $B$ is a Yes-container, then $A$ is a Yes-container.
    \item Existence. There exists $A$ such that $A$ is a Yes-container.
    \item Two Point Separating. Given a Yes-container $A$ and two points in it, then there is a Yes-container inside $A$ which does not contain at least one of the given points. 
    \item Intersecting. If $A$ and $B$ are Yes-containers, then they have non-zero intersection and that intersection is either a Yes-container or contains a Yes-container if it is not a container. 
    \item Closed. If $q$ is contained in all Yes-containers, then the singleton $q$ is a Yes-container. 
\end{enumerate}


\section{Real Oracles}

We briefly review what was done with oracles generating the real numbers from the rationals. See \cite{taylor23main}.

...

\section{Metric Spaces}

Given a dense metric space, we can complete it by replacing intervals with balls. The interval separation property is unpleasant in this context as it involves splitting a space and one point will not be sufficient. Instead, we will use a generalization of the two point separation property. 

We will use $E$ and primes on that to denote different metric spaces. A metric space is a set $E$ equipped with a distance function, $d$, which takes in pairs of points from $E$ and returns a non-negative number. We will use $d_E$ if we need to distinguish different distance functions for different metric spaces. A distance function must be symmetric and is zero exactly when computing the distance from a point to itself. In addition, it must satisfy the triangle inequality $d(x,y) \leq d(x,y) + d(y,z)$. 

A ball is specified by a point $c$ (the center) and a radius $r$ and is a rule $B[c,r]$ such that $B[c,r](q) = 1$ if and only if the distance $d(c,q) \leq r$; we say that $q$ is contained in $B[c,r]$. The radius could be either rationals or, now that we have them, real numbers. The function $d$ is the distance function, or metric, whose existence is given by this being a metric space; the distance must satisfy some properties such as positivity, symmetry, and the triangle inequality. Containment of balls is that any point in the contained ball is also in the containing ball. We do allow singletons which is the ball of radius $0$ with center $p$. It has the property that only $p$ is in it. Two balls $B$ and $C$ are disjoint if there does not exist any point $q$ that is contained in both $B$ and $C$.

Note that our balls would be closed balls if we had a complete metric space. We are including the boundary sphere. 

Our metric spaces will be dense metric spaces which we take to mean that in any given ball $B[c,r]$, there exists $q \in E$ such that $q \neq c$ and $q$ is in $B[c,r]$.


\subsection{Oracles in a Metric Space}

We can now give our definition of an oracle for a metric space $E$. The Oracle of $\alpha$ is a rule defined on balls that decides on whether they are a Yes ball or No ball and satisfies ($B$ and $C$ are balls in what follows, $p$ and $q$ are points in the original metric space ): 
\begin{enumerate}
    \item Consistency. If $B$ contains $C$ and $C$ is a Yes-ball, then $B$ is a Yes ball.
    \item Existence. There exists $B$ such that $B$ is a Yes-ball.
    \item Two Point Separating. Given a Yes-ball $B$ and two points in it, then there is a Yes-ball inside $B$ which does not contain at least one of the given points. 
    \item Intersecting. If $B$ and $C$ are Yes balls, then they must intersect and their intersection must contain an Yes ball.
    \item Closed. If $q$ is contained in all Yes-balls, then the ball containing $q$ of radius 0 is a Yes-ball. 
\end{enumerate}

Similar to before, this is equivalent to looking at a maximal family of overlapping, notionally shrinking balls. And, as before, the oracle is preferred to emphasize the algorithmic nature of it all. 

One would need to establish that the singletons represent themselves in this new space, that the metric extends to these new creations, and that it is complete as a metric space.



\subsection{Family of Overlapping Notionally Shrinking Balls}

An equivalent replacement for the Two Point Separating Property is the Narrowing Property: Given any Yes-ball $B$ for the oracle $\alpha$ and real number $\varepsilon >0$, we can find a Yes-ball whose diameter is less than $\varepsilon$ inside $B$.

\begin{proposition}
    The Two Point Separating Property and the Narrowing Property are equivalent.
\end{proposition}

\begin{proof}
    
\end{proof}


Two oracles are equal if they agree on all intervals. If they are not equal, then there exists a Yes-ball for one of them which is a No-ball for the other one. Because of consistency, there can be no Yes-balls within the No-ball. 


\subsection{Completion of a Metric Space}



The distance can be defined as follows. Let $x$ and $y$ be two oracles. Then $\overline{d}(x,y)$ is defined as the real number oracle which is the infimum of the set of distances $d(B, C)$ where $B$ is a $x$-Yes ball and $C$ is a $y$-Yes ball. The distance between two balls in the original space say with centers $q$ and $s$ and radii $r$ and $t$, respectively, is defined as $r + d(q,s) + t$. This should encompass the distance defined as the supremum over all the distances of the points within the ball. Indeed, let $q' \in B$ and $s'\in C$. Then $d(q', s') \leq d(q',q) + d(q,s') \leq d(q',q) + d(q,s) + d(s,s') = r + d(q,s) + t$ by two applications of the triangle inequality.  

We need to show that this does define a metric, that the distances between pairs on the original space is the same, and that this new space is closed. We will also establish the equivalence of the fonshb. 

\begin{proposition}
    The function $\overline{d}$ defined above is a distance function for the space of oracles of $E$. 
\end{proposition}

\begin{proof}
We need to check the properties. 
\begin{enumerate}
\item Non-negativity. The function $\overline{d}$ is non-negative as this is the greatest lower bound of a set of numbers bounded below by 0. 
\item Symmetry. The function $\overline{d}$ is symmetric. This follows as $d(B,C) = d(C,B)$ for distances between balls and so the sets for the greatest lower bound of $\overline{d}(x,y)$ is the same as the set for $\overline{d}(y,x)$.
\item Equality. We have $\overline{d}(x,y) = 0$ exactly when $x=y$. If $x=y$, then we can consider the set of distances $d(B,B)= 2r$ of balls that contain $x$ to itself. Since we can take the balls to be arbitrarily small, we have the greatest lower bound is 0. If $x \neq y$, then there exists disjoint balls $B$ and $C$ with $x \in B$ and $y \in C$. The greatest lower bound is then bounded below by the $\inf\{d(q,p) | q \in B, p \in C\}$ since contained balls in them will have distances of their centers at least as large as that infimum. 
\item Triangle Inequality. We have $\overline{d}(x,z) \leq \overline{d}(x,y) + \overline{d}(y,y)$. Let $A$, $A'$, and $A''$ be Yes-balls for $x$, $y$, and $z$, respectively. Let $(q,c, r)$, $(q',c', r')$, and $(q'', c'', r'')$ be points in, centers and radii for, respectively, the similarly primed balls. Then $d(q,c) + d(c, c'') + d(q'',c'')$ is the quantity that represents an element of the distance set whose infimum is $\overline{d}(x,z)$. We need to compare it to the representatives of the other two distances summed. Well, $d(c,c'') \leq d(c,c') + d(c', c'')$. Substituting that in, we get $(q,c) + d(c, c'') + d(q'',c'') \leq (q,c) + d(c,c') + d(c', c'') + d(q'',c'')$ which is less than the representatives of the sum $\overline{d}(x,y) + \overline{d}(y,z)$. When we take the infimum, the inequality is preserved. 
\end{enumerate}
\end{proof}


One would need to check that this does define a distance. The triangle inequality should follow largely from the original triangle inequality and some inequality work involving infimums. 

To establish the original space is still in there, we identify the singletons as their own representatives and note that the distance as defined above for oracles immediately gives us that the distance between the singletons is unchanged from the original space. 

The final step is to show that the new space is complete. This could follow largely on how we did Cauchy sequences. Define a Cauchy sequence as a sequence of points such that we have a sequence of nested balls that contain the tail of the sequence and the size can be taken as small as we like. A Yes-ball is then any ball which contains one of these nesting balls. Consistency and existence are immediate from the definitions. Intersection is easy to see since the nesting balls contain one another and thus there must be a common nested ball inside any ball which contains a nested ball. The Closed property, as we did before, is simply postulated as part of the definition of the balls. As for the two point separation property, given two points, there exists a small enough ball that cannot contain them both due to the non-zero distance between them. At that point, we should have a nested Yes ball that does not contain at least one of them. 

We call the space of oracles with this induced metric $\overline{E}$, the \textbf{closure} of $E$.


\subsection{Fixed Point Theorem}

We will establish the fixed point theorem. First, we need to define the natural extension of a map $f: E \to E$ to a map $\bar{f} : \bar{E} \to \bar{E}$. Here we will content ourselves with the classical continuous functions being extended; see the function oracle section for a slightly more generalized notion. 

\begin{definition}
    A function $f: E \to F$ satisfies $\varepsilon-\delta$ continuity if, for given $x \in E$ and $\varepsilon > 0$, there exists a $\delta > 0$ such that $f(B_{x} (\delta)) \subseteq B_{f(x)} (\varepsilon)$.
\end{definition}

\begin{definition}
If a function $f:E \to F$ satisfies $\varepsilon-\delta$ continuity, then the \textbf{natural extension} of $f$ to $\bar{E}$ is the function $\bar{f}:\bar{E} \to \bar{F}$ defined by, for oracle $\alpha \in \bar{E}$, $\bar{f}(\alpha)$ is the unique oracle in $\bar{F}$ defined to be the collection of all containers in $F$ that contain images under $f$ of Yes containers of $\alpha$. 
\end{definition}

We need to establish that for a given $\alpha$, the images of its Yes containers does form an oracle in $\bar{F}$. 

To do so, let $f$ and $\alpha$ be given. Then: 

\begin{itemize}
    \item Consistency. If a container $A$ contains a container $C$ that contains the image of a $\alpha$-Yes container, then it is contained in $C$ as well.
    \item Existence. By existence applied to $\alpha$, there is a closed ball $B$ that is $\alpha$-Yes. Need to use continuity as a scrambling $f$ could disperse image of a ball across the space and not be contained in a ball. 
    \item Two Point Separating.
    \item Intersecting. The image of an image is contained in the image? 
    \item Closed. 
\end{itemize}

\begin{proposition}
    Given a function $f:E \to E$ that satisfies $d(f(x), f(y)) \leq q d(x, y)$ for all $x, y \in E$ and for a $0 \leq q < 1$, then there exists an oracle $\alpha$ such that the natural extension of $f$ to $\bar{E}$, called $\bar{f}$, has the property that $\bar{f}(\alpha) = \alpha$.
\end{proposition}



\section{Standard Topology}

We now proceed onto general topological spaces. Given our Two Point Separation Property, we expect the Separation axioms to be important for developing useful, robust oracle structures. The containers in a standard topology setting will be closed sets. These are dramatically weaker in content than the closed balls and will require more assumptions. 

One aspect of applying oracles in this context is the explicit requirement that the space itself should be contained after the application of the oracles. This is what drives our separation example. A completely regular space is one in which singletons are closed sets and that for every singleton and closed set not containing that singleton, there exists a continuous function on the interval $[0,1]$ such that the function is 1 on the singleton and 0 on the given closed set. This is almost what a rule demands (can we prove that it demands this? ) 

The condition of complete regularity is what is needed for the embedding of the space into its Stone-Cech Compactification to be a homeomorphism. This can be proven by considering the product space of all continuous bounded functions from the space into the real numbers. We will produce the various compactifications directly using oracles. 

In particular, given a continuous function from the space into a compact space, we define the oracles to be closed sets whose image contains a given point, narrowing down, need some more separation. 


\subsection{Different Compactifications of the Open Unit Interval}

circle, closed interval, topologist sine curve

the toplogist sine curve. Fix a number u in [-1, 1]. The containers will be 

for (0,1) maybe also sin(1/x) <= 1/2,  sin(1/1-x) >= 1/2. is continuous. Also (0,1) with one point on the end, sine curve on the other. 


\subsection{One Point Compactification}

We will assume the space is locally compact, Hausdorff, and completely regular, but not compact. We will call these simply \textbf{noncompact spaces}. 

If we define the space of oracles on such a space, using closed sets as the containers, then we end up with the space itself again, under homeomorphism. To get something different, we consider the non-compact closed sets as another oracle. Then we define the oracle of $\infty$ to be the oracle whose containers are the complements to  compact sets. When included with the oracles of closed sets, we end up with the one point compactification using the topology of open sets being the open sets of the original space along with the open sets forming this oracle. 

We need to verify that this is an oracle.  
\begin{itemize}
    \item Consistency. Any container of this type is a Yes implying that if contains another Yes container, then it says yes. 
    \item Existence. The space itself is the complement of the compact empty set so the whole space is there. 
    \item Two Point Separation. Given a point $q$ in the complement of a compact set, we can add its local compact neighborhood to the compact set and that complement will be contained in the original complement but not contain $q$. 
    \item Disjoint. If we had two complements that were disjoint, then we would have $X-C$ and $X-B$ being disjoint. This means that $B \cup C = X$ (if $q$ is not in $B$ or $C$, then $q$ is in both of their complements). Since $C$ and $B$ are compact, their union is and therefore the space would be compact. We are assuming it is not.
    \item Rooted. There is no point $q$ in every complement. 
\end{itemize}

So this is indeed an oracle. Now we need to show the new space is compact. 

More or less, use the finite intersection property of collections. Let the collection be closed sets whose finite intersections are non-empty. Then we need to show that the full intersection is non-empty. Essentially, no new oracles, I guess. So either it contains infinity or it does not. If the union of the sets is compact, then we have the result. If it is not, is it the complement of a compact set? If so, then that means the oracle of infinity is a member? 

\subsection{General Compactifications}

Let $f$ be a continuous function from a non-compact space $X$ to a compact space $K$. 

want to intersect the inverse images of compact sets with the complements above. Is it still compact? Does this work? Maybe use open pre-images? 

\section{Theory of Linear Structures}

so topological space allowed completion (metric space too of course since it is topological and the oracle is equivalent to complement of closed balls),  but is there a submetrical space which allows both to be done? 

Also, would like to add finite / discrete space. Essentially, minimal resolution is the minimal neighborhood for the oracle. Need to modify the two point separation to exclude pairs of points that are in a designated minimal neighborhood. 

\subsection{Brief Review of Linear Structures as Basis of Topoology}

linear structures, directed linear structures, neighborhoods, open, closed, compact, continuous, 

chain-continuous



\section{Function oracles}



The Function oracles we defined can be extended to the realm of the completed metric spaces. There is very little we need to change here. Instead of the sides of the ``rectangle'' being an interval, we use a ball. When we are considering the intersection of these function rectangles, we take the intersection of the sides to be the largest ball contained in them. 

We expect to find again that such functions are continuous at all the new points and potentially discontinuous at the old points. 


A \textbf{function oracle} $f$ over the metric space $E$ and into the metric space $E'$ is a rule that, given any rectangle $R= A \times B$ with $A$ a ball in $E$ and $B$ a ball in  $E'$, either says Yes or No and satisfies: 
\begin{enumerate}
    \item Elongating Consistency. If the wall of rectangle $M$ contains the wall of rectangle $N$ with $N$ and $M$ having the same base,  then $M$ is a $f$-Yes rectangle if $N$  is an $f$-Yes Rectangle. 
    \item Narrowing Consistency. If the base of rectangle $M$ contains rectangle $N$ with $N$ and $M$ having the same wall, then $N$ is a $f$-Yes rectangle if $M$ is. 
    \item Intersection. If two $f$-Yes rectangles intersect, then there is an $f$-Yes rectangle contained in the intersection. 
    \item Single-valued. Given two disjoint rectangles $M$ and $N$ with the same base, at most one of them can be a Yes-rectangle for $f$. 
    \item Separating. Given an $f$-Yes rectangle $M$, an oracle $\alpha$ contained in the base of $M$, and two points $r$ and $s$ contained in the wall of $M$, then there exists an $f$-Yes rectangle whose wall does not contain at least one of those points and whose base contains $\alpha$.
\end{enumerate} 

The development of this is almost identical to what was done before. We start with establishing the Narrowing Property: 

\begin{proposition}\label{pr:metricfshrink}
Given a function oracle $f$ over the metric space $E$ to the metric space $E'$, an oracle $\alpha$ in the $f$-Yes rectangle $M$, and an $\varepsilon > 0$, then we can find an $f$-Yes rectangle contained in $M$ whose base contains $\alpha$ such that its wall has diameter less than $\varepsilon$.
\end{proposition}

The distance function that appears below is that of $E'$ as the Separating property handles the $E$ part. 

\begin{proof}
    Let $f$, $\alpha$, and $\varepsilon$ be given. Define $B[c_0, r_0]$ to be the wall of $M_0 =M $. Assume $M_i$ is defined with $B[c_i, r_i]$ being the wall of $M_i$. If $2r_i$ is less than $\varepsilon$, then we are done. If not, by the density of the metric space, there exists $p$ in $B[c_i, \frac{r_i}{3}]$. By the separating property, there exists a rectangle $M_{i+1}$ whose wall, $B[c_{i+1}, r_{i+1}]$, excludes either $ p$ or $c_i$. We wish to argue that $r_{i+1} < \frac{2r_i}{3}$. Let $q$ be in $B[c_i, r_i]$. Then $d(q, c_i) \leq r_i$ by definition of the ball.  By the triangle inequality, $d(q, p) \leq d(q,c_i) + d(p, c_i) \leq r_i + \frac{r_i}{3}= \frac{4 r_i}{3}$. Thus, the diameter must be less than $\frac{4r_i}{3}$ if $p$ is excluded while it must be less than $r_i$ if $c$ is excluded. Therefore, the radius $r_{i+1} < \frac{2r_i}{3}$. 

    We repeat this at most until $i$ satisfies $(\frac{2}{3})^i r_0 < \varepsilon$. 
\end{proof}

The Narrowing Property also will imply the separating property by taking $\varepsilon$ to be less than the distance between the two points. We favor the separating property as it usually does not require more than two steps to establish, but the narrowing property is crucial in establishing a variety of important properties. The separating property can be viewed as the foundation for the powerful machinery of the narrowing property. 

We can now establish the classical existence of the function value.

\begin{proposition}
Let $E$ and $E'$ be metric spaces with $f$ being a function oracle from $E$ to $E'$. If an oracle $\alpha$ in $\overline{E}$ has a Yes-interval which is the base of a $f$-Yes rectangle, then $f(\alpha)$ is defined as the unique oracle $\beta$ in $\overline{E'}$ associated with the family of overlapping notionally shrinking balls given by the walls of the $f$-Yes rectangles that contain $\alpha$. 
\end{proposition}

\begin{proof}
We need to show that the set of walls is a fonsb. Given two such walls, their bases intersect and we can restrict them to the same base by the Narrowing property. By the Single-valued property, the walls must intersect. By the Intersection property, that intersection must contain an $f$-Yes rectangle whose wall will be contained in the wall of the original two rectangles. Thus, that wall is in the fonshb. As for notionally shrinking, that is directly the narrowing property. 

By the fonshb proposition, we have a unique oracle associated with it. 
\end{proof}

The main addition that the fonshb does is to add in the singleton if it is missing. 

We say that an oracle $\alpha$ is in the domain of the function oracle $f$ if there is at least one $f$-Yes rectangle $M$ whose base is a $\alpha$-Yes interval. By the proposition above, being in the domain does imply that the function will have a value at $\alpha$ in the classical sense. 


\subsection{Composition of Function Oracles}

To compose function oracles, say $h = f \circ g$, we define it as follows. Let $M$ be a $g$-Yes rectangle. Then we define the family of $h$-Yes rectangles $\mathcal{F}_M$ to be any $f$-Yes rectangle whose base is contained in the wall of $M$. The collection of all such families defines a new function oracle which is what we define $f \circ g$ to mean. 

If $f$ is a function oracle from the metric space $E'$ to the metric space $E''$ and $g$ is a function oracle from metric space $E$ to the metric space $E'$, then $f \circ g$ is a map from $E$ to $E''$.

We need to establish that this is a function oracle. 

\begin{enumerate}
    \item Elongating Consistency. Let rectangle $A$ $y$-contain $B$ with $A$ and $B$ being $x$-same  and assume $B$ is a $f \circ g$-Yes rectangle. We need to show that $A$ is also a $f \circ g$-Yes rectangle. Let $M$ be the $g$-Yes rectangle such that $B$ is in the family $\mathcal{F}_M$. Specifically, the wall of $M$ contains the base of $B$. $B$ is a $f$-Yes rectangle and so $A$ is $f$-Yes as well by $f$'s Elongating Consistency. Since $A$ and $B$ have the same base, the base of $A$ is contained in the wall of $M$ and is included in $\mathcal{F}_M$. Thus, it is a $f \circ g$-Yes Rectangle.
    \item Narrowing Consistency. Let the base of $A$ contain the base of $B$ with $B$ and $A$ having the same wall. Further assume $A$ is a $f \circ g$-Yes rectangle. Then we need to show that $B$ is a $f \circ g$-Yes rectangle. Let $M$ be the $g$-Yes rectangle whose wall contains the base of $A$. Then $M$ contains the base of $B$. Since $A$ is $f$-Yes, by $f$'s Narrowing Consistency, we have $B$ is $f$-Yes. We conclude $B$ is $f \circ g$ Yes. 
    \item Intersection. Let $A$ and $B$ be two $f\circ g$-Yes rectangles that intersect. We need to show that the intersection contains a $f \circ g$-Yes rectangle. 
    \item Single-valued. Let $A$ and $B$ be two disjoint rectangles with the same base. We need to show that at most one is a $f\circ g $-Yes rectangle. 
    \item Separating. Let $A$ be a $f\circ g$-Yes rectangle, $\alpha$ an oracle in the base of $A$, and $r, s$ be two points in the wall of $A$. We need to show that there exists a $f\circ g$-Yes rectangle whose base includes $\alpha$ but whose wall excludes at least one of $r$ or $s$. 
    
\end{enumerate}

\subsection{Product Metric Spaces}

A product metric space is ... 

Need to show closure of the product is the product of the closures. 



\subsection{Arithmetic of Function Oracles}

We can now apply the composition of function oracles to establish the arithmetic of function oracles. $\mathcal{Q}^n$ is a metric space, which we will call the rational $n$-space. We can view arithmetic operators as a composition of maps between rational $n$-spaces. For example, addition is 


\subsection{Multivariable Function Oracles}

We could generalize this to multivariable functions. But we will soon be discussing metric spaces in general which we can then apply to the special case of $\mathbb{Q}^n$. 


\subsection{Continuity}

\subsection{Integral of Function Oracles}

Need a volume for metric spaces which is actually a way to compute volumes for balls. 

\subsection{Derivative of Function Oracles}

Need a linear structure over the metric space. 



\medskip

\normalem %restoring normal emphasis in bibliography 
\printbibliography

\end{document}

\documentclass[12pt]{article}
\usepackage{amsfonts,amsmath,amssymb,mathrsfs,amsthm}
\usepackage{hyperref}

\title{Defining Real Numbers as Oracles}
\author{
  James Taylor
}
%\date{September 1, 2003}

\addtolength{\textwidth}{2.0cm}
\addtolength{\hoffset}{-1.0cm}
\addtolength{\textheight}{3.0cm}
\addtolength{\voffset}{-1.5cm}


\newtheorem{theorem}{Theorem}
\newtheorem{lemma}{Lemma}
\newtheorem{corollary}{Corollary}
\newtheorem{proposition}{Proposition}

\theoremstyle{remark}
\newtheorem{remark}{Remark}


%https://tex.stackexchange.com/questions/432778/reduce-spacing-around-colons-in-math-mode
\DeclareMathSymbol{:}{\mathord}{operators}{"3A} 
%\mathrel{:} if need colon spaced. 
%want small less than for colon replacement
\newcommand{\lt}{\mathord{<}}

%\sloppy%\openup-.1\jot
\begin{document}\maketitle
\begin{abstract}
We explore a new definition of real numbers, namely, a real number is an oracle that gives an affirmation if the real number is in a given rational interval and gives a negative if not. We establish that this is a proper definition of real numbers, compare and contrast with other common definitions, and mention some immediate algorithmic uses of them. This definition moves the real number definition to be in line with how they are pragmatically used. 
\end{abstract}



\section{The Oracle of $r$}\label{sec:ora}

The idea is that a real number is an oracle that reveals itself by answering, under repeated questioning, whether it is in a given rational interval. 

The heuristic is that the Oracle of $r$ is a rule which, given two rational numbers, will return 1 (Yes) if $r$ is in the closed interval defined by the two rational numbers and returns 0 (No) otherwise. If the answer was a 1, then we say the interval is an $r$-Yes interval and similarly for 0 and No. If we are just speaking of one oracle, we may simplify say a Yes-interval or No-interval. 

\subsection{Interval notation}

To facilitate slightly easier notation, we will define a rational interval $a:b$ as giving a rational number $q$ the property of being in the interval if $q$ is between $a$ and $b$, inclusive. We identify $b:a$ as the same property. The interval $a:b$ contains the interval $c:d$ if being in the interval $c:d$ implies being in the interval $a:b$. We will write $a\lt b$ if we want to indicate the ordering relation between the rationals; it is sometimes convenient to slip that into notation, but it is often not needed in which case we just use $a:b$. We will also write $a:b < c:d$ to indicate the situation in which $a$ and $b$ are less than $c$ and $d$ which implies that all rationals $q$ in $a:b$ are less than all rationals $r$ in $c:d$.

We also have the singleton intervals of $a:a$ consisting of just $a$. 

A rational $q$ is strictly contained in $a\lt b$ if $a < q < b$. 

\subsection{Defining the Oracle}

The Oracle of $r$ is a rule $R$ defined on rational intervals that returns values of 1 and 0, and satisfies: 
\begin{enumerate}
    \item Consistency. If $c:d$ contains $a:b$ and $R(a:b) = 1$, then $R(c:d) = 1$.
    \item Finite. $R(a:b) = 1$ for some rationals $a$, $b$.
    \item Separating. If $R(a:b)=1$, then for a given $c \neq a, b$ in $a:b$, either $R(c:c) = 1$ or $R(a:c) \neq R(c:b)$. 
    \item Rooted. There is at most one $c$ such that $R(c:c) =1$.
    \item Closed. If $c$ is contained in all $R$-Yes intervals, then $R(c:c) = 1$.
\end{enumerate}

Note that in the above $a$, $b$, $c$, and $d$ are all rationals; we do not apply this to irrationals since we are defining them here. We generally obey the convention that $r$ represents naming the oracle as a number and $R$ is the rule. There is not a distinction in fact of $r$ and $R$, but the notation helps make the connection to how we customarily think of real numbers versus the oracle version. 

The binary nature of the output of $R$ tells us that if $c:d$ contains $a:b$ and $R(c:d) = 0$, then $R(a:b) = 0$. For if $R(a:b)=1$, then consistency implies $R(c:d)=1$ which contradicts our starting assumption. We will refer to this as part of the Consistency property. 

For further elaboration on what each of these properties means:

\begin{enumerate}

    \item Consistency says that Yes propagates upwards to larger intervals while No propagates downwards to smaller intervals. That is, the Oracle never contradicts itself. 
    
    In our examples, this is usually true by definition of the rule. 
    
    If $R(a:b)= 0$, then $R(c:d)=0$ if $c:d$ is contained in $a:b$.

    \item The finite requirement says that there is definitely an interval on which the Oracle exists. It is required to avoid the trivial $R(a:b) = 0$ for all $a:b$ which would satisfy all the others, including the separating property due to it only applying with Yes-intervals. 
    
    The existing interval also gives us a place to start in our approximation schemes that we discuss later. 
    
    It is usually easy to verify in any practical example as any Yes-interval will do and we can usually find a very bad approximate interval that is easy to establish.  

    \item Separation is needed to ensure that the Yes propagates downwards and we can, therefore, make progress on deciding which interval has $r$. This is crucial to the use of the Oracle in approximating what we take to be a single number $r$. The possibility of $R(c:c) = 1$ occurs exactly when the real number $r$ is the rational number $c$. 
    
    Proving that a given oracle is, in fact, a rational can be impossible, mirroring difficulties with other real number approaches. 

    Separation could alternatively be phrased as insisting that if $R(a:b) = 1$, then at least one of $R(a:c)$ or $R(c:b)$ is 1 and if they are both 1, then $R(c:c) = 1$. We chose our formulation to emphasize that being able to separate the intervals is a key feature we desire to implement and that it fails only if the overlapping endpoint is the number of interest. 
    
    \item Being Rooted ensures that there is just one single number under discussion. Without this, we could have the rule, for example, $R(a:b) = 1$ for all $a:b$ intersecting a given interval $c:d$. 
    
    In the arithmetic operations, this is proven by assuming the existence of two such rational singletons being Yes-intervals and then exhibiting an interval of sufficiently small length that the two can be distinguished. 

    \item The Closed property is required to ensure that we have only one rule per rational number. Without this, we would face a similar problem to $0.\bar{9} = 1.\bar{0} = 1$ in the presentation of infinite decimals. Since our oracles are modeling having a given number present, it seems reasonable to include the rational number $c$ if it is present in all Yes intervals. 
    
    In our constructions, we often need to add this in explicitly. It is a bit awkward.

\end{enumerate}


This definition does not avoid the common downsides of the other definitions of real numbers, but it does reduce them to a context that reflects how real numbers get used in practice and illustrates those downsides as fundamental to the nature of real numbers.  


\section{Basic Properties of an Oracle}

We establish some basic facts of oracles which we will find useful throughout. 

We start with the extremely helpful properties involving intersections and unions. We then discuss the uniqueness of oracles, with the related notions of determining ordering and compatibility. In particular, we define what we mean by $r = s$, $r<s$, $r>s$, and $r?s$ for oracles $r$ and $s$. 


The relation $r?s$ means that they have not been established to be equal, but they are compatible with that hypothesis. Typically, we will want to specify and interval of compatibility, which is the smallest known interval tested that they both reported Yes on. To denote that, we use $a:r?s:b$ 

\subsection{ Intersections and Unions}

Let $a \leq b \leq c \leq d$ with all of them being rationals. The intersection of $a:c$ and $b:d$ is $c:d$ while the union is $a:d$. The two intervals $a:b$ and $c:d$ are disjoint if $b < c$. 

One can either take those as statements from the usual set definitions or take these as the definitions of those terms. It is immediate from the Consistency property that Yes-intervals are closed under unions and that No-intervals are closed under intersection. We will prove that they are closed under intersections and unions, respectively. 

We start with establishing that the intersection of two $R$-Yes intervals is an $R$-Yes interval.

\begin{proposition}
Let $R$ be an oracle and $a \leq b \leq c \leq d$. If $R(a:c) = 1 = R(b:d)$, then $R(b:c) = 1$.
\end{proposition}

\begin{proof}
  
  This follows from the Separation and Consistency properties. From Consistency, we have $R(a:d) = 1$ since it contains an $R$-Yes interval (two in fact, but we just need one). Then apply separation to $a:c$ and $c:d$. If $R(c:c) = 1$, then $R(b:c) = 1$ since it contains $c:c$ and we are done. Otherwise, Separation tells us $R(a:c) \neq R(c:d)$. By assumption, we know $R(a:c) = 1$ so that implies $R(c:d) = 0$. 
  
  We can then consider what follows from $R(b:d) = 1$.  From Separation based on $c$, we have $R(b:c) \neq R(c:d)$ since we are specifically in the case that $R(c:c) \neq 1$. As we established $R(c:d) = 0$, we must have $R(b:c) = 1$.
  
  We have established that Yes-intervals are closed under intersection. 
\end{proof}

Let us now establish that the No-intervals are closed under union. 

\begin{proposition}
Let $R$ be an oracle and $a \leq b \leq c \leq d$.  If $R(a:c) = 0$ and $R(b:d) = 0$, then $R(a:d) = 0$. 
\end{proposition}

\begin{proof}
    Let's assume that $R(a:d) = 1$ and find a contradiction. With that assumption, $R(a:b) = 1$ from Separation applied to $a:b$ and $b:d$ with the fact that $b:d$ was a No-interval. But this contradicts $R(a:c)= 0$ since it contains $a:b$ and therefore Consistency would demand $a:c$ must be a Yes-interval. We can conclude that $R(a:b) =0$.
\end{proof}

Also, two disjoint intervals cannot be both yes. Notice the distinction in the next proposition that $b$ and $c$ are separated by a strict inequality. 

\begin{proposition} \label{pr:disjoint}
Let $R$ be an oracle and $a:b < c:d$. Then we cannot have both $R(a:b) = 1$ and $R(c:d) = 1$. 
\end{proposition}

\begin{proof}
By being Rooted, we must have at least one of $R(b:b) = 0$ or $R(c:c) = 0$ holding true, probably both. Let's assume $R(b:b) = 0$ without loss of generality.
 
Either $R(a:b) = 0$ or $R(a:b)=1$. If it is the first case, then we have established what we want to claim. So let us assume we are in the second case. 
 
Since $R(a:b) = 1$,  by Consistency, we have $R(a:d) = 1$. Then Separation applies for the intervals $a:b$ and $b:d$. Since $1 = R(a:b) \neq R(b:d)$, we must have $R(b:d) = 0$. As $c:d$ is contained in $b:d$ and No intervals propagate downwards by Consistency, we must have $R(c:d)=0$  as we were to establish in this case. 
\end{proof}

We can thus see that one method of demonstrating that $R(a:b)=0$ for some given interval $a:b$ is to produce an interval $c:d$ disjoint from $a:b$ such that $R(c:d)=1$. On the other hand, if all non-singleton $R$-Yes intervals intersect $a:b$, but are not contained in $a:b$, then either $a$ or $b$ is contained in all such intervals. Let's say $a$ is. Then $R(a:a)=1$ by the Closed property and thus $R(a:b)=1$ as well. This situation is where the usual computational difficulties arise and we often have to resort to the notion of compatibility as described in the next section. 

\subsection{Equality and Ordering}

Two oracles are equal if they agree on all (rational) intervals. To prove inequality, it is therefore sufficient to find a disagreement. The basic tool is to find two disjoint intervals that we can guarantee different results. 

If we have two different ``real numbers'', do we get different oracles for them? Since we are defining real numbers, this gets a little tricky, but we can imagine that what we mean by knowing the two numbers are different is that they are separated by a rational number. Let's assume $a < r < p < s < b$ where $r$ and $s$ are our two distinct real numbers and $p$ is a known rational that separates them, and $a$ and $b$ are two rationals we know sandwich the two reals. Then $a:p$ is a Yes-interval for the Oracle of $r$ while $p:s$ is a Yes-interval for the Oracle of $s$, and each of those intervals are No-intervals for the other oracle. 

For example, if we we want to distinguish the square roots of 2 and 3 from one another (we will discuss square roots later, just assume the usual ordering properties for now), then $a = 1$, $p = \tfrac{3}{2}$, and $b = 2$ would suffice as $1 < \sqrt{2} < \tfrac{3}{2} < \sqrt{3} < 2$. Let $R$ be the Oracle of $\sqrt{2}$ and $S$ be the Oracle of $\sqrt{3}$, then we should have $R(1:\tfrac{3}{2}) = 1$, $R(\tfrac{3}{2}:2) = 0$,  $S(\tfrac{3}{2}:2) = 1$, $S(1, \tfrac{3}{2}) = 0$. Thus, these oracles are different. In particular, from this presentation, it would be reasonable to say $R < S$ and this i is in line with our definition below.

If the oracles $R$ and $S$ cannot find an interval on which they disagree, but we also cannot prove that they agree on all intervals, then we can talk about compatibility and the resolution of that compatibility.

Let $R$ and $S$ be two oracles. We adopt the following definitions:

\begin{enumerate}
    \item $R=S$. $R$ and $S$ are equal if and only if they agree on all intervals. 
    \item $R < S$. $R$ is less than $S$ if and only if there exists $a:b < c:d$ such that $R(a:b) =1 = S(c:d)$ and $R(c:d) = 0 = S(a:b)$. 
    \item $R < S$. $R$ is greater than $S$ if and only if $S < R$.
    \item $a:R?S:b$. $R$ and $S$ are $a:b$ compatible if $R(a:b)=S(a:b)$.
    \item $R ? S$. $R$ and $S$ are compatible if and only if $R(a:b) = S(a:b)$ for all known intervals $a:b$. 
\end{enumerate}

The interval $a:b$ is the \textbf{resolution of the compatibility of $R$ and $S$} if it the shortest $R$-Yes and $S$-Yes interval; this is necessarily the intersection of all mutually Yes intervals for $R$ and $S$. If we cannot compute the actual resolution, we will say the \textbf{known resolution of their compatibility is $a:b$} to indicate our tentative knowledge of $a:b$ being the shortest interval that they agree on. All oracles are at least compatible on the interval $-\infty:\infty$.

Consistency implies that the inequalities are definitive and not contradicted. We can prove the transitive law quite easily. Let the following oracles satisfy $R < S < T$. Then we have intervals $a:b < c:d < e:f$ such that $a:b$ is $R$-Yes, $c:d$ is $S$-Yes, $e:f$ is $T$-Yes, $a:b$ and $e:f$ are $S$-No, and $c:d$ is $R$-No and $T$-No. Because the intervals are disjoint, we can deduce that $a:b$ is $T$-No and $e:f$ is $R$-No. The main fact here is that inequality of intervals is transitive. REWRITE THIS. 

If an $a:R?S:b$, then they also agree (Yes) on all intervals that contain $a:b$ as well as all intervals disjoint from $a:b$ (No). 

Thus, two oracles could be equal, unequal, or compatible. The latter is for rules that we cannot find an interval of disagreement, but we have not been able to establish equality. 

We want to formalize that two oracles whose Yes intervals always overlap are, in fact, the same oracle.

\begin{proposition}\label{pr:overlap}
Let $R$ and $S$ be two oracles such that whenever $R(a:b)=1$ and $S(c:d)=1$, we have that $a:b$ and $c:d$ must have non-zero intersection. Then $R =S$
\end{proposition}

\begin{proof}
We want to show that given an interval $a:b$ such that $R(a:b)=1$, then $S(a:b)=1$. Let $c:d$ be such that $S(c:d)=1$. By the assumption, the two intervals must overlap. We have some cases. 

First, if $c:d$ is contained in $a:b$, then $S(a:b)=1$ by consistency. 

So next, let's assume that $c<d$ contains $a<b$. By separation, we must have $S(c:a) =1$ or $S(a:d) = 1$ or $S(a:a) =1$. If $S(c:a) = 1$, then we can argue that $S(a:a) =1$. Indeed, let $c < e < a$. Then separation tells us that $S(c:e) = 1$ or $S(e:a)  = 1$ or $S(e:e)=1$. Both the first and the third are not possible because that would require $c:e$ or $e:e$ to overlap $a:b$ which neither does. So $S(e:a) = 1$ for all $e$ between $c$ and $a$. This means that $a$ is contained in all $S$-Yes intervals and therefore $S(a:a)=1$. And so $S(a:b)=1$ by consistency. We also could then argue in this case that $S$ and $R$ are the rational Oracles of $a$. 

So we can look at the case that $S(a:d) = 1$. By the same argument above, but applying it to a separation involving $b$, we either have $S(b:d)=1$ descending down to $S(b:b)=1$, or $S(b:b)=1$ directly from separation, or $S(a:b)=1$ which is what we were to show. 

This argument can also apply to the Yes intervals of $S$ just as equally well. 
\end{proof}

\subsection{Bisection Approximation}

Let $R$ be the Oracle of $r$. We will use $R$ to compute an approximation of $r$ to any desired level of accuracy. 

$R$ being Finite gives us a starting interval, say $a:b$ with length $L$. Then take $c = \frac{a+b}{2}$. Since $R$ is separating, we can use it to determine whether $R(a:c) = 1$ or $R(c:b) = 1$ or $R(c:c) = 1$. If it is the latter, we are done and $r = c$. If it is one of the former, then we use that new interval to repeat the same process, but the length of the interval is now $\frac{L}{2}$. 

If we do this $n$ times, then the length will be $\frac{L}{2^n}$ which allows us to compute $r$ to be within any level of given accuracy. 

This is the bisection method, of course, and oracles provide exactly what we need to use it. 

We have proven that: 

\begin{proposition}\label{pr:short}
For a given oracle and length, we can produce an interval shorter than that length. 
\end{proposition}

This is helpful in establishing the arithmetic properties. Later we will discuss the mediant approximation which is generally a pretty pleasant computational method to employ with a nice relationship to continued fractions. 


\section{Examples}

It is always good to start with examples. In particular, how do we obtain various oracles in common situations? 

We shall start with how the rational numbers appear. We then define oracles for $n$-th roots, for numbers with approximation schemes, least upper bounds of sets, and then we investigate a couple of examples of indeterminate rules. For each of them, we will define the rule and then establish the properties by the definition. 

\subsection{Rational Oracles}

Given a rational $q$, we define the Oracle of $q$ as the rule $R(a:b) = 1$ if and only if $q$ is contained in $a:b$. This includes the interval $q:q$.  

We can verify the properties of the rational Oracle of $q$ as follows: 

\begin{enumerate}
    \item Consistency. If $R(a:b)=1$, then $q$ is contained in $a:b$. If $c:d$ contains $a:b$, then $q$ is contained in $c:d$. Thus, $R(c:d)=1$.
    \item Finite. $q$ is contained in $q:q$ so $R(q:q)=1$.
    \item Separating. If $R(a:b) =1$, then $q$ is contained in $a:b$. Let $c$ be strictly in $a:b$. We have two possibilities. Either $c = q$, in which case $R(c:c)=R(q:q)=1$ or $q$ is strictly contained in $a:c$ or $c:b$. The interval that contains it will be a Yes interval while the one that does not is a No interval. Therefore $R(a:c) \neq R(c:b)$ as required in this case. 
    \item Rooted. For $c \neq q$, $c:c$ does not contain $q$ and therefore $R(c:c)=0$.
    \item Closed. Assume $c$ is contained in all $R$-Yes intervals. Then, in particular, $c$ is in $q:q$ and thus $c=q$ and $R(c:c)=1$. That feels a little to reliant on the inclusion of the singleton. To make this a little more robust, assume $c \neq q$, say, $c < q$. Then there exists $a$, $b$, and $d$ such that $a < c < d < q < b$. Since $q$ is in $b:d$ but not in $a:d$, we have $R(a:d)=0$ and $R(b:d)=1$. 
\end{enumerate}

We will see with the operations that these oracles are the natural representatives of the rational numbers, obeying the arithmetic that we would want them to obey.  

We also claim that if we have an oracle with rule $R$ such that there is a rational $q$ with $R(q:q)=1$, then it is the Oracle of $q$, whose rule we shall call $Q$. If the oracle is different, then there is an interval $a:b$ on which they disagree. Since $R(q:q) =1$, all the  $Q$-Yes intervals are also $R$-Yes intervals. Therefore, we need to prove that for a given  $Q$-No interval $a:b$, we must also have it be an $R$-No interval. Because it is $Q$-No, it does not contain $q$. It is therefore disjoint from $q:q$. But by the disjoint property, Proposition \ref{pr:disjoint}, we have $R(a:b)=0$. Thus, the two oracles must agree on all intervals and we have uniqueness. 

The property of being closed also prevents having an oracle which agrees with all the $Q$-Yes intervals except $q:q$. Closed forces $R(q:q)=1$ if all $R$-Yes intervals contain $q$.

\subsection{Roots}

For the positive $n$-th root of a positive rational number $q$, the Oracle rule would be for $R(a\lt b) = 1$, if and only if $q$ is either contained in $a^n:b^n$ for $a>0$ or contained in $0:b^n$ for $a \leq 0$ and $b>0$. If we have $a:a$, then $R(a:a) = 1$ if and only if $a^n = q$ and $a>0$.

We can verify the properties of the Oracle of $\sqrt[n]{q}$ as follows: 

\begin{enumerate}
    \item Consistency. Because of the monotonicity of $x^n$ for positive $x$, consistency holds. Namely, assume $a\lt b$ is contained in $c \lt d$ and $R(a:b)=1$. If $c>0$, then $c^n \leq a^n \leq q \leq b^n \leq d^n$ and we see that this holds. If $c<0$ then we need to show $0 \leq q \leq d^n$. Since $0 \leq q \leq  b^n \leq d^n$, this holds. 
    \item Finite. Let $M = \max(q, 1)$. Then $0 < q \leq M^n$ as either $q^n = M^n \geq 1$ or $0 < q < 1^n = 1$. Either way, $R(0:M) = 1$. 
    \item Separation. Let $a\lt b$ be given such that $R(a:b)=1$. Let $a < c <b$ be given. We need to show that either $R(c:c) = 1$ or $R(a:c) \neq R(b:c)$. We proceed by cases:
    \begin{enumerate}
        \item If $c \leq 0$, then $a<0$ and $q$ is contained in $0:b^n$. Thus, $R(c:b) = 1$ and $R(a:c) = 0$. 
        \item Assume $c>0$. If $c^n = q$, then $R(c:c)=1$ and we are done. If $c^n < q$, then $c^n < q< b^n$ and $R(c:b)=1$. Since $c^n$ would be the upper bound on the $a:c$ interval computations, we have $R(a:c)=0$. Alternatively, if $c^n > q$, then we have that $R(a:c)=1$ since $a$ or $0$ will be a sufficient lower bound in the cases. Meanwhile, $R(c:b) = 0$ since $q$ is not between $c^n$ and $b^n$. 
    \end{enumerate}
     \item Rooted. This relies on the equation $x^n = q$ having at most one positive solution. This follows from the fact that $0<a<b$ implies $0 < a^n < b^n$. See appendix if in doubt. 
    \item Closed. Consider a rational number $r$. We want to show that either $r>0$ with $r^n = q$ in which case $R(r:r)=1$ or that if $r^n \neq q$ then there exists an interval $a:b$ such that $R(a:b)=1$ but $r$ is not in $a:b$. Let $M = \max(q, 1)$; we need this since if $q<1$, then $q^n < q < 1$ and we want to make sure we have a number greater than $q$ after being raised to the $n$-th power. 
    
    If $r < 0$, then the interval $0:M$ contains $q$ and does not include $r$ so that is a $R$-Yes interval. Let us therefore assume $r \geq 0$ and that $r^n \neq q$. In the case that $r^n > q$, there is an $s$ such that $s<r$ and $s^n > q$ (well-known fact, but see Appendix). Therefore, $R(0:s) = 1$ and $r$ is not in $0:s$. For the case of $r^n < q$, we have the existence of an $s$ such that $s > r$ and $s^n < q$. We thus have $R(s:M)=1$ and $r < s$ is not in the interval $s:M$. 
    
\end{enumerate}

\subsection{Nested intervals}

An example which is almost an outline version of an oracle rule is that of nested intervals whose lengths are going to 0. This can be literally a sequence of nested intervals or it can be a number with an error bound, such as a sequence of decimals with $\pm$ bounds. It is very common to have error bounds in applications and this is how to incorporate that into this framework. 

We will formalize the nested interval concept as a rule $I$ that takes in a positive rational number and produces a rational interval whose length is less than that number. We will call this a \textbf{nesting function}. It should have the property that $I(p)$ is contained in $I(r)$ if $p<r$ in addition to the length of $I(p)$ being less than $p$. If it is possible to define $I(0)$ in such a way that this maintains the nesting property, then it should be defined. This happens if and only if there is a rational number $q$ contained in each of the intervals and that rational number is unique. 

We can now define the Oracle of $I$ as $R(a:b) = 1$ if and only if $a:b$ contains $I(p)$ for some non-negative rational $p$.

\begin{enumerate}
    \item Consistency. Let $c:d$ contain $a:b$ and $R(a:b)=1$. Then there exists a $p$ such that $I(p)$ is contained in $a:b$. Therefore $c:d$ contains $I(p)$ and $R(c:d) = 1$
    \item Finite. By assumption, $I(1)=a:b$ exists and is finite so $R(a:b)=1$.
    \item Separating. Assume $R(a:b)=1$ and let $c$ be in $a:b$. If $c$ is in $I(p)$ for all $p$, then $I(0) = c:c$ is defined and $R(c:c)=1$. So let us assume there is a length $p$ such that $I(p)$ does not contain $c$. Because of nesting, $I(p)$ is contained in $a:b$ and, since it does not contain $c$, it must be either in $a:c$ or $b:c$, but not both. Thus, $R(a:c) \neq R(b:c)$.
    \item Rooted. This is pretty relevant to the $I(0)$ definition, but we can argue it here. Given rational numbers $c < d$, we claim that there is at most one which is in all $I(p)$. Consider the distance $d-c$. Look at the interval $I( \tfrac{d-c}{2} )$. This interval has length strictly less than $d-c$ and therefore both $d$ and $c$ cannot be both in there. 
    \item Closed. This is the property of having $I(0)$. Either that is defined and there is a rational $c$ with $R(c:c)=1$ or $I(0)$ is not definable in which case there is no rational that is contained in all the intervals. 
\end{enumerate}

As we said, nested intervals are closely related to the notion of oracles. It is the version which our latter approximation schemes produce and it is therefore more immediate in its use for getting an approximation. A given nesting function, however, is not a unique representative of a real number and the implication is that to get different representatives from this, one has to essentially take the steps done here. So converting to an Oracle is a first step to exploring other approximations, such as finding the best rational approximation via the mediant method described below. 

\subsection{Least Upper Bound}

We are given a non-empty set $E$ with an upper bound $M$, meaning that if $x \in E$ then $x$ has the property $x < M$. We want to create the oracle for the least upper bound of $E$. Let $U$ be the collection of upper bounds of $E$, namely, $y$ in $U$ if $y > x$ for every $x$ in $E$. 

We define the Oracle of $\mathrm{sup} E$ to be the rule such that $R(a\leq b) = 1$ if and only if $a \leq y$ for all $y$ in $U$ and $b \geq x$ for all $x$ in $E$. That is, $a$ is a lower bound for $U$ and $b$ is an upper bound for $E$.

\begin{enumerate}
    \item Consistency. Assume $R(a:b)=1$ and $c:d$ contains $a:b$. Then $c < a < y$ for all $y$ in $U$ and $d > b > x$ for all $x$ in $E$. Thus, $R(c:d) = 1$.
    \item Finite. By assumption, there is an $x \in E$ and an upper bound $M$ in $U$. Thus, $R(x:M) = 1$. 
    \item Separating. Let $R(a\lt b)=1$ and $a < c< b$. We need to show that either $R(c:c)=1$ or $R(a:c) \neq R(c:b)$. 
    
    If $c \geq x$ for all $x$ in $E$, then $R(a:c)=1$ by definition and the fact that $a$ being a lower bound for $U$ still holds. If $c \leq y$ for all $y$ in $U$, then $R(c:c)=1$ and we are done. If $c > y$ for some $y$ in $U$, then $R(c:b) = 0$ since $b > c$ and therefore neither of them can be less than $y$ for all $y$ in $U$. 
    
    The other case is that $c < x$ for some $x$ in $E$. Since $a < c$, we know that neither $a$ nor $c$ is greater than all $x$ in $E$. Thus, $R(a:c) = 0$. Since $c < x$ for some $x$ in $E$, it must be the case that since any upper bound $y$ has the property that $x < y$, we have $c < y$ for all $y$ in $U$. We therefore have $R(c:b)=1$.
    
    \item Rooted. For $R(c:c)=1$, we would need to have that $c \leq y$ for all $y$ in $U$ and $c \geq x$ for all $x$ in $E$. Assume we had another rational such that $R(d:d) = 1$. Then we also have $d \geq x$ and $d \leq y$ for all $y$ in $E$. Since both $c$ and $d$ are upper bounds in $E$, we have that each are in $U$. So $c \leq d$ and $d \leq c$. This implies that $c = d$. 
    \item Closed. Assume $c$ is in all Yes intervals. Then given any upper bound $y$ in $U$ and element $x$ in $E$, we have $c$ is in $x:y$ since $R(x:y)=1$ by definition. We therefore have that $c \leq y$ and $c \geq x$. Since $x$ and $y$ were arbitrary elements of $E$ and $U$, $c$ satisfies the conditions such that $R(c:c)=1$.
\end{enumerate}


\subsection{The Collatz number}

At the present time, the \href{https://en.wikipedia.org/wiki/Collatz_conjecture}{Collatz conjecture} is not proven. We can define an Oracle of Collatz $R$ such that $R(-\tfrac{1}{n}:\tfrac{1}{n})$ is 1 if the Collatz sequence terminates at 1 for all $k \leq n$ and 0 otherwise. For all other intervals $a:b$, $R(a:b) = 1$ if $a:b$ contains such a Collatz Yes interval. If there is a $n$ which does not satisfy the Collatz conjecture, then, taking $n$ to be the first such $n$, we define $R(\tfrac{1}{n}:\tfrac{1}{n}) = 1$. If there is no such $n$, then $R(0:0) = 1$

This is a rule which, as of 2020, is known to be $-2^-68:2^-68$ compatible with the Oracle of 0. But unless it gets proven or falsified, we cannot establish equality or inequality with respect to 0. 

In the ultimate form of a known result for the conjecture, this will be an Oracle of 0 or an Oracle of $\tfrac{1}{N}$ where $N$ is the first number where the conjecture fails to hold. Until that happens, we cannot say that the Separating or Rooted properties hold. 

An alternative approach to using this conjecture is to define an oracle that implies the $n$-th digit in decimal approximations is $1$ if the Collatz conjecture fails on it and $0$ otherwise.

Let $C(n)$ be an indicator function which is 1 if $n$ does not satisfy the Collatz conjecture and 0 otherwise. Define the following function as follows: $f(1) = 0$. $f(n) = \sum_{i=0}^n C(i)*10^{-i} $.  We can then define a nesting function as follows: $I(m)=0:2$ for $m > 1$ and $I(p) = I(2 \times 10^{-n}) = I_n = f(n): f(n)+2 \times 10^{-n}$ for $10^{-(n+1} \leq p < 10^{-n}$. The length property should be clear. The nesting nature follows from the fact that, for $m > n$,  $0 \leq  f(m)-f(n) = \sum_{i=n+1}^m C(i)*10^{-i} \leq \sum_{i=n+1}^m 10^{-i} = 10^{-n+1}\sum_{i=0}^{m-n-1} 10^{-i} < \tfrac{1}{9} \times 10^{-n} < 10^{-n} $ where the last inequality comes from the finite geometric sum form of $\dfrac{1-x^{m-n}}{1-x}$ with $x=10^-1$ leading to estimate the numerator with 1 and the denominator being replaced with $\tfrac{10}{9}$. 

As noted above, we can define an oracle based on the nesting function. This oracle works as far as we can compute. Note that, unless the conjecture was proven in the affirmative which would make this the Oracle of 0, we would still not be able to write down more for the rule which would need to be applied to each $n$. 


\subsection{Coin tosses}

We want to model an oracle that is probabilistic in nature and model how they might work. 

The first version is to create a nesting function by a bisection method. We start with an initial interval, say $I(q) = I_1 = 0:1$ for all $p\geq 1$. Then, given $I_n = a:b$, we let $c = \tfrac{a+b}{2}$ and then use a random function that yields 1 with probability $p$ and 0 with probability $1-p$. If it is 1, then, for $2^{-(n+1)} \leq q  < 2^{-n}$, $I(q) = a:c$. Otherwise, $I(q) = c:b$. With this nesting function, we can have an oracle. This oracle, however, will only be defined fully after this infinite subdividing process completes. In particular, similar to the Collatz conjecture example, at any given level, we will have infinitely many rationals that are not yet distinguished.  

A process closer to defining an oracle directly is the following; we will use $R$ for the rule. Let $I$ be a given starting interval and we set $C=I$ where $C$ stands for current. Let $a:b$ be given. If $a:b$ contains $C$ (or is $C$), then $R(a:b) = 1$. If $a:b$ and $C$ are disjoint, then $R(a:b) = 0$. If neither, then $a:b$ and $C$ intersect. Let $c:d$ then be that intersection. If the interval $c:d$ divides $C$ in half (they have a common endpoint), then we use a random process to determine if $c:d$ is a Yes interval or a No interval; the complement interval in $C$ is then the opposite. Then $a:b$ is a Yes or No interval based on $c:d$. We also redefine $C$ to be $c:d$ or its complement, depending on which one is Yes. It is possible for $c:d$ to divide $C$ into three intervals. We check $c:d$ first; if it is Yes, then the other two are No. If $c:d$ is No, then we do the random process to decide which of the two intervals making up the complement are Yes. Whichever one is Yes, becomes the new $C$ for future Rule consultations.  

This oracle will satisfy consistency, being finite, and separating all by definition. Rooted will apply as well. As for closed, $R(c:c)=1$ can only happen if we ask about it and comes up with a 1 in which case we have finished our process. 

This is an example of a rule which is guaranteed to give an answer for any question we ask even if the answer cannot be known in advance. In fact, the very nature of what is possible is based on what we ask. For example, if we never ask about  $c:c$, then it will not happen. But if we do ask, then it could come up as Yes. 

In a certain sense, our rule here is not "complete". An infinite being looking at it would see the finiteness. But from our limited perspective, it works just as much as any "complete" rule. In contrast, the Collatz example is one which we might not get an answer to. It is "incomplete" on a practical level. 

\section{Interval Arithmetic}

This section is not novel, but rather a review of applying arithmetic to intervals. We will need this in defining the arithmetic of oracles. This material can be seen, for example, in the videos by NJ Wildberger... These are of interest as this helps us propagate error intervals in scientific computations and our oracle approach reflects the usefulness of this thinking. 

The guiding idea of interval arithmetic is to do the arithmetic operations on each pair of elements and then have the minimal interval that contains that interval. We could try to define it that way and then prove the following. We will, instead, define these operations below and then establish that given any two elements in the operated intervals will fall in that resultant interval. 

Let $a \lt b$ and $c \lt d$, all of them being rational numbers. Then we define:
\begin{enumerate}
    \item Addition. $a:b + c:d = (a+c):(b+d)$
    \item Negation. $-(a:b) = -b:-a$
    \item Subtraction. $a:b - c:d = a:b + (-d:-c) = a-d:b-c$
    \item Multiplication. $a:b * c:d = \min(ac, ad, bc, bd): \max(ac,ad,bc,bd)$. For $0<a<b$ and $0<c<d$, this is equivalent to $a:b*c:d = ac:bd$. 
    \item Reciprocity. $1/(a:b) = 1/b:1/a$ as long as $a:b$ does not contain 0. If 0 is contained in $a:b$, then the reciprocal is undefined as it actually generates the split interval of $-\infty:1/a$ and $1/b:\infty$. 
    \item Division. $(a:b) / (c:d) = a:b * 1/d:1/c$ where $c:d$ does not contain 0. Applying the multiplication rule, we find that we can view it as:
    
    $\min(a/c, a/d, b/c, b/d): \max(a/c,a/d,b/c,b/d)$. 
    \item Natural Powers. Let $n$ be a natural number. $(a:b)^n = a^n:b^n$ if $a$ and $b$ have the same sign, 0 inclusive. If exactly one of them is negative, then let $c = \max (|a|, |b|)$ and $d=-\min(|a|, |b|)$. Then $(a:b)^n = c^n: c^{n-1}d $. 
    \item Negative powers ($a:b$ not containing 0). $(a:b)^{-n}$ for natural $n$ is defined as $(1/b:1/a)^n$, and if they are the same sign has $a^{-n} : b^{-n}$ but if they have opposite signs, then using  $c = \max (|a|, |b|)$ and $d=-\min(|a|, |b|)$, we find that the interval becomes $d^{-n} : c d^{n-1}$. 
\end{enumerate}

Let $a \leq p \leq b$ and $c\leq  q \leq  d$. Then from normal inequality arithmetic, we have: 

NEED TO MAKE ALL $<$ into $\leq$

REDO these. a little unclear what would be good to show, seems kind of not helpful right now. 

\begin{enumerate}
    \item Addition.   $a +c \leq  p + q \leq  b +d$
    \item Negation.  $-b \leq -p \leq -a$
    \item Subtraction.  $a - d \leq p-q \leq b -c$
    \item Multiplication. We have multiple cases depending on the sign of the items. We can have wholly positive, wholly negative, or mixed intervals and that can then change the inequalities involved. 
    \begin{itemize}
        \item All the numbers are positive. Then we multiply the inequalities to get $ac < pq < bd$. We also have $ac < bc, cd  < bd$. So the interval is $ac:bd$ and that is what the max/min choice yields. 
        \item All the numbers are negative. This is the same except the inequalities are reversed. 
    \end{itemize}
    Let $0 < a < p < b$
    \item Reciprocity. Let $0 < a < p < b$, then $1/a > 1/p  > 1/b$. Similarly, $a < p < b< 0$ has $1/a > 1/p > 1/b$. What fails is if $a < p < 0 < b$, we would have $1/b > 0 1/a > 1/p $ and similarly if $p$ was positive, it would flip over $1/b$ but not $1/a$.
    \item Division. This follows from Multiplication and Reciprocity. 
    \item Natural Powers. We could go to basic principles and play around with inequalities of the elements of the products, breaking into cases and dealing with various sign flippings. Instead, we can consider what happens under iterative multiplication. For squaring, we have that we need to find the maximum and minimum of $a^2, ab, b^2$. If they are the same sign, then $a^2 < ab < b^2$. If they are different signs, then $ab < 0 < a^2, b^2$ and we need to compare the size of $-a$ and $b$.  In general, if they are of the same sign, $a^n \leq a^{n-1} b \leq \cdots \leq ab^{n-1} \leq b^n$. If they are of different signs, then let $c = \max( |a|, |b|)$ and $d=min(|a|, |b|)$. Then we have the same sign inequality setup above, but replacing $a$ and $b$ with $c$ and $d$. In particular,  $c^n \geq d c^{n-1} \geq d^{i}c^{n-i}$  for $i > 1$.  Since the signs are different, we have the products are negative when the power of $a$ is odd and positive for even powers of $a$. Thus, $c^n$ and $d c^{n-1}$ are the two endpoints. We have three cases: $c=b$ in which case $b^n$ will be the largest product and we have $b^n > 0 > ab^{n-1}$; $c=|a|$ and $n$ is even in which case $a^n > 0 > a^{n-1} b$ is the relevant endpoints; $c=|a|$ and $n$ is odd in which case $a^{n-1} b > 0 > a^n$ is the relevant setup.  

As an example of the powers, consider $(-2:3)^4$. All of the products of the endpoints are: $16, -24, 36, -54, 81$. We therefore have the interval being $-54:81$ which is the product of $ab^3$ and $b^4$. 

    
     
    \item Negative powers are defined by combining reciprocity with natural powers. 
\end{enumerate}


We also want to point out that if $a:b$ and $c:d$ are contained in $e:f$ and $g:h$, respectively, then $a:b + c:d $ is contained in $e:f + g:h$ and similarly for the various other operations, with the understanding of the appropriate restrictions on not containing 0 for the division and reciprocals. For multiplication, we would need to show that $\max(ac, ad, bc, bd) \leq \max(eg, eh, fg, fh)$ and similarly for the min. 

Also need to argue that if we add / multiply two intervals contained in two other intervals, then the subintervals combined are in the superintervals combined. 


\subsection{Verifying the rules}

The associative and commutative rules of arithmetic apply to intervals. The distributive rule somewhat applies. Each of these is a distinct computation, but very straightforward. We will do the distributive property separately. 

\begin{enumerate}
    \item Interval addition is closed, namely, the sum of two intervals is another interval by definition. 
    \item Addition is Commutative. $a:b + c:d= a+c : b+d = c+a:b+d = c:d + a:b$. We used the commutativity of rational addition in the middle step. 
    \item Addition is Associative. $(a:b + c:d) + e:f = (a+c:b+d)+e:f = ((a+c)+e:(b+d)+f) = (a+c+e:b+d+f$ where the last step is the associative property of rationals.  On the other hand, $a:b + (c:d+e:f) = a:b + (c+e:d+f) = a+(c+e):b+(d+f) = a+c+e:b+d+f$ again by associativity of addition of rationals. Since they are equal to the same quantity, we have the associative rule of addition applying to intervals and we can comfortably write $a:b + c:d + e:f$ without requiring parentheses. 
    \item $0:0$ is the additive identity as $a:b+0:0 = a+0:b+0 = a:b$. 
    \item There is no additive inverse since subtraction increases the length of the interval. The singleton intervals do have additive inverses. 
    \item Interval multiplication is closed, namely, the product of two intervals is another interval by definition. 
    \item Multiplication is Commutative. $a:b * c:d= ac : bd = ca:bd = c:d * a:b$. We used the commutativity of rational multiplication in the middle step. 
    \item Multiplication is Associative. $(a:b * c:d) * e:f = (ac:bd)*e:f = ((ac)e:(bd)f) = (ace:bdf$ where the last step is the associative property of rational multiplication.  On the other hand, $a:b * (c:d*e:f) = a:b * (ce:df) = a(ce):b(df) = ace:bdf$ again by associativity of multiplication of rationals. Since they are equal to the same quantity, we have the associative rule of modification applying to intervals and we can comfortably write $a:b * c:d * e:f$ without requiring parentheses.     
    \item $1:1$ is the multiplicative identity as $a:b*1:1 = a1:b1 = a:b$. 
    \item There is no multiplicative inverse as multiplication has a non-zero length for non-singleton intervals. The non-zero singleton intervals do have multiplicative inverses. 
\end{enumerate}

For the distributive property, we do not have equality of the intervals. But we do have that one contains the other, which will be sufficient for our purposes to have the distributive rule apply to oracles. 

\begin{proposition}
We have the subdistributive property: $I = a:b*(c:d + e:f)$ is contained in $J = a:b*c:d + a:b*e:f$. 
\end{proposition}

\begin{proof}
We can compute the interval limits. The interval $I$ has boundaries $\max(a(c+e), a(d+f), b(c+e), b(d+f))$ and $\min(a(c+e), a(d+f), b(c+e), b(d+f))$ while the interval $J$ has boundaries $\max(ac, ad, bc, bd) + \max(ae, af, be, bf)$ and $\min(ac, ad, bc, bd) + \min(ae, af, be, bf)$. Since the boundaries of $I$ are contained in the possibilities of $J$, we do have $J$ containing $I$.
\end{proof}

As we can see, the arithmetic of intervals has some common properties with normal number arithmetic, but it is not the same.

\subsection{Lengths}

It is of some interest to compute out the lengths of these intervals that are generated by the arithmetic. We often want to have the lengths be as small as possible. 

\begin{enumerate}
    \item $a:b + c:d$ has length $(b+d) - (a+c) = (b-a) + (d-c)$, that is, it is the length of the two intervals separately. 
    \item $a:b*c:d$ for positive endpoints is $bd-ac = bd -bc + bc -ac =  b(d-c) + c(b-a)$ though it is also equal to $bd - ad + ad - ac = d(b-a) + a(d-c)$. For mixed signs, we have the maximum of $|a(d-c)|$, $|b(d-c)|$, $|c(b-a)|$, $|d(b-a)|$, $|d(b-a)+a(d-c)| = |b(d-c) + c(b-a)|$. 
    
    For a simple bounding estimate on the multiplicative length, we can take the maximum $M$ of $|a|, |b|, |c|, |d|$ and multiply that by the maximum length $L$ of $b-a$ and $d-c$ and then double that. So $2*M*L$. 
    \item Negation does not change the interval length. 
    \item Subtraction has length $b-c - (a-d) = (d-c) + (b-a)$, the same as addition.
    \item Reciprocity. $1/a - 1/b = \tfrac{|b-a|}{ab}$. Note $a$ and $b$ must be the same sign to avoid having 0 in there, which was part of the definition.
    \item For division, we can apply the simple bound to $M = \max(|a|, |b|, |1/c|, |1/d|)$ and $L= \max(|b-a|, |1/c - 1/d|)$ to get $2*M*L$ as an upper bound. 
    \item For natural powers, we have that either the interval is $a^n:b^n$ (length $|a^n-b^n|$ or, if $c = \max(|a|, |b|)$ and $d=-\min(|a|,|b|)$, we would have $c^n : d c^{n-1}$ with length $c^n + d c^{n-1}= |a-b|c^{n-1}$. For the first one, the length can be summarized as $|a-b|n c^{n-1} $
\end{enumerate}



\section{Oracle Arithmetic}

We can now define oracle arithmetic. The basic idea is that if an interval contains the result of combining the Yes intervals of the oracles being combined, then it is a Yes interval itself. 

We will prove a general statement about creating an oracle out of other oracles based on the property that shrinking intervals of the inputs lead to shrinking interval of the output. We will then apply it to the various forms of arithmetic operators. 

This shrinking is what allows us to go from the arithmetic of intervals, which does not have a mechanism to use this property, to an arithmetic of oracles, which can use it since we can shrink the intervals of interest for oracles. 

\subsection{Narrowing of Intervals}

An \textbf{interval operator} $f$ is a mapping that takes in a finite number of intervals and outputs an interval. An oracle operator $F$ is a mapping that takes in a finite number of oracles and produces another oracle.

We will use $|I|$ to denote the length of the intervals. 

The operator has the \textbf{narrowing} property if and only if $f(I_1, I_2, \ldots, I_n)= J$ has the property that $|J| \leq M \max_{i=1}^{n} (|I_i|) $ where $M$ can be chosen such that it would apply equally for any subintervals of the $I_i$. That is, once we have chosen starting intervals, we can narrow them down and the length of $J$ will get narrowed as well. 

\begin{theorem}
If $f$ is an interval operator with the narrowing property, then there is an associated oracle operator $F$. It is defined for $\beta = F(\alpha_1, \alpha_2, \ldots, \alpha_n)$ as the unique oracle such that the interval $J$ is an $\beta$-Yes interval if it contains an interval $I = f(I_1, I_2, \ldots, I_n)$ where $I_i$ is a $\alpha_i$-Yes interval for each $i$. Additionally, if a rational $q$ is contained in all such intervals, then it's singleton interval is a $\beta$-Yes interval. 
\end{theorem}

Essentially, we use the interval operator to translate Yes intervals into a Yes interval. Unfortunately, we cannot rely on singletons appearing directly from that process so we must add that in. It still is not entirely satisfactory as we generally must rely on an infinite set of computations to determine whether the singleton should be included. We can say, however, that there will be at most one such singleton. 

\begin{proof}
First, the rule $\beta$ is well-defined, up to computational limits. If we can produce an interval that 

\begin{enumerate}
    \item Consistency. If $c:d$ contains $a:b$ and $R(a:b) = 1$, then $R(c:d) = 1$.
    \item Finite. $R(a:b) = 1$ for some rationals $a$, $b$.
    \item Separating. If $R(a:b)=1$, then for a given $c \neq a, b$ in $a:b$, either $R(c:c) = 1$ or $R(a:c) \neq R(c:b)$. 
    \item Rooted. There is at most one $c$ such that $R(c:c) =1$.
    \item Closed. If $c$ is contained in all $R$-Yes intervals, then $R(c:c) = 1$.
\end{enumerate}

\end{proof}


\subsection{Addition}

Given the Oracle of $r$ and the Oracle of $s$, the sum is the Oracle $r+s$ defined by the rule $T(a:b) = 1$ if and only if there exists intervals $c:d$ and $e:f$ such that $a:b$ is contained in $c:d + e:f$ and both $R(c:d)=1$ and $S(e:f) = 1$. We must add in that if a rational number $q$ is part of every such YES interval, then $R(q:q) = 1$.

Let us establish that this is an oracle. The key fact is that the bisection algorithm allows us to make the summand intervals as small as we like. 

\begin{enumerate}
    \item Consistency. By definition, the containment will travel up of the existing additive intervals.
    \item Finite. By finiteness of $r$ and $s$, there exist an $R$-Yes rational interval $a:b$ and an $S$-Yes rational interval $c:d$. Their sum will be a rational interval and a $T$-Yes interval by definition.
    \item Separating. Let $T(a:b)=1$ be given. We know there are summand intervals whose sums are contained in $a:b$. Taking $a < c< b$, we can repeatedly bisect the intervals of $R$ and $S$. We have three cases: 
    \begin{enumerate}
        \item Eventually, one sum interval is contained in either $a:c$ or $c:b$. Then we have our result. 
        \item We can prove that $c$ is contained in all such intervals. Then $T(c:c) =1$ and we have separation. 
        \item We cannot prove that $c$ is contained in all such intervals, but we also cannot produce an interval independent of $c$. This is similar to the Collatz conjecture example. All we can do is talk about being resolved to a certain interval. We should be able to exclude $a$ and $b$ as we descend as well as any other rational other than $c$. That is, we can separate from all points except possibly one. 
    \end{enumerate}
    then for a given $c \neq a, b$ in $a:b$, either $R(c:c) = 1$ or $R(a:c) \neq R(c:b)$. 
    \item Rooted. Given any two rational numbers $p$ and $q$, we define the distance $D = |p-q|$. If we can produce an $R$-Yes interval $a:b$ and an $S$-Yes interval $c:d$ such that $a:b + c:d$'s length is smaller than $D$, then both $p$ and $q$ cannot be in the interval and addition is rooted. But this is what the bisection method provides. From finiteness, we have starting intervals for $r$ and $s$ and by repeated bisecting, we can get both interval lengths to be below $D/2$ and so their sum is less than $D$.
    \item Closed. This is added by hand into the definition. It seems unavoidable given an example of $(e + 2) - e$ resulting in $2$ but not having any singleton in that interval. 
\end{enumerate}




\subsection{Multiplication}

\begin{enumerate}
    \item Consistency. If $c:d$ contains $a:b$ and $R(a:b) = 1$, then $R(c:d) = 1$.
    \item Finite. $R(a:b) = 1$ for some rationals $a$, $b$.
    \item Separating. If $R(a:b)=1$, then for a given $c \neq a, b$ in $a:b$, either $R(c:c) = 1$ or $R(a:c) \neq R(c:b)$. 
    \item Rooted. There is at most one $c$ such that $R(c:c) =1$.
    \item Closed. If $c$ is contained in all $R$-Yes intervals, then $R(c:c) = 1$.
\end{enumerate}


\subsection{Inverses}



\subsection{Derived Operators}

Checking $R+S=T$ is a rule. Specifically, any interval $a:b$ is a YES interval if and only if it contains, possibly is, a sum interval from $R$ and $S$' s YES interval. If a rational number $q$ is part of every such YES interval, then $R(q:q) = 1$. 

Consistency is by definition. 

Non-vacuous is clear as $\infty:\infty$ adds to itself. 

Rooted. Assume there is $p<q$, both rational, such that $T(p:p)=1$ and $T(q:q) =1$. Let $L$ be $q-p$. Take an $R$-interval and an $S$-interval of length $L/3$. Then the sum interval is less than $L$ and therefore cannot contain both $p$ and $q$. Since at least one of them is not part of a YES interval, by definition, that one is not YES.  .... 

Closed. This holds by definition. We are essentially cheating by closing it. 

Separating. This is the hard one and where the real work is. Let $T(a:b) = 1$, which means that it contains an interval which is additive, say $c:d$. Let $q$ be given in $a:b$ as the dividing point. If $T(q:q) = 1$, then we are done. So we can assume that is not the case. We therefore have to establish that $T(a:q)=1$ or $T(q:b) = 1$. If $c:d$ does not contain $q$, then it is wholly contained in either $a:q$ or $q:b$ and thus one of them is YES. The other is no because .... Assume a sum interval is in $q:b$. Look at those intervals, they must be disjoint from the other ones? So then that contradicts R, S being rules. 

If $q$ is contained in $c:d$, then there must exist a sum interval $e:f$ which $q$ is not a part of since we are assuming $T(q:q)=0$. Take the intersection of the underlying summand intervals and then add. That is an interval which will not contain $q$ but will be contained in one of them. So back to the previous one. 



doing these computations is important:

The Oracle of 0 is the additive identity

$a-a = 0$. Let $R$ be the Oracle of $a$. We want to show that $S= R-R$ is the Oracle of 0. Let $c:d$ be any $R$-Yes interval. Then $c:d-c:d = c:d+ -d:-c = c-d:d-c$ is the corresponding $S$-Yes interval. Clearly, this interval contains 0. As 0 is contained in each $S$-Yes interval, we have $S(0:0)=1$ and $S$ is the Oracle of 0. 

The Oracle of 1 is the multiplicative identity

$\tfrac{a}{a} = 1$, for $a \neq 0$. Let $R$ be the oracle of $a$. We want to show $S=R/R$ is the Oracle of 1. Let $c:d$ be any $R$-Yes interval that does not contain 0. By the closed property, this exists because $a \neq 0$. Then $\dfrac{c:d}{c:d} = c:d * \tfrac{1}{d}:\tfrac{1}{c}$. By the definition of the multiplication, we need to consider the four products of the endpoints which are: 1, $\tfrac{d}{c}$, $\tfrac{c}{d}$, 1. This translates into the $S$-Yes interval $\tfrac{d}{c}:\tfrac{c}{d}$ which clearly contains 1. As this applies to all $R$-Yes intervals that do not include 0, we can conclude that $S(1:1) =1$.



Distributive rule:  We have subsetted intervals, but consider the two things as separate oracles. each yes interval in the one has yes interval in the other that it intersects. Claim is that this makes the two the same oracle. 


\section{Complete Ordered Field}

To what extent this is true depends on one's attitude about infinite sets and so forth, but to the extent that the other definitions are valid, we are just as valid. ....

Let $\mathbb{R}$ be the set of all oracles. 

From above, given two 



\section{Mediant Approximation}

\subsubsection{Square Root 2}

\subsubsection{Best Approximations}

\subsubsection{Continued Fractions}



\section{Relation to other definitions}

It is useful to compare this approach with other common approaches and some nearby alternatives to this. 

What are some good properties of a definition of real numbers? This is subjective, of course, but I was guided by the following: 

\begin{itemize}
    \item Uniqueness. Given a target real number, there should be only one version of that in the real number system and a feeling that it is generally possible to distinguish two different real numbers. 
    \item Reactive. This is a key feature. Real numbers generally have an infinite flavor to them. It was important to me to not pretend we could present the infinite version of that, but rather to present a method of answering queries. 
    \item Rational-friendly. Ideally, rationals would be easily spotted, treated reasonably, and arithmetic with them would be easy to do. 
    \item Supportive. The definition should be in line with and, ideally, supportive of standard practice of numbers. In particular, how numbers are used in science, applied mathematics, and numerical analysis. 
    \item Arithmeticizable. It should feel like the arithmetic laws are approachable and computable. That is, one can take an action to a certain level of precision and be confident in the result.
\end{itemize}

The Oracle approach fits the first four of these quite easily. The last two are pretty subjective and perhaps the best way forward on those is to contrast them with the other definitions. 

Much of what follows was heavily inspired by NJ Wildberger's excellent videos....


\subsection{Infinite Decimals}

Let us run through our criteria. 

\begin{itemize}
    \item Uniqueness.
    \item Reactive. 
    \item Rational-friendly. 
    \item Supportive. 
    \item Arithmeticizable. 
\end{itemize}

\subsection{Nested Intervals}

One can think of expanding the concept of infinite decimals as being a sequence of nested intervals where the length goes down by a tenth at each level. We could generalize this to be a more arbitrary sequence of nested and shrinking intervals. From the Oracle point of view, we could use the mediant method to define such a sequence of intervals. A sequence of such intervals would also give rise to an Oracle. These are closely related concepts. 

Let us run through our criteria. 

\begin{itemize}
    \item Uniqueness. This clearly fails. We can have two entirely distinct nested interval sequences describing the same real number in addition to arbitrarily changing a given sequence (cut out half of them, double their lengths, ...)
    \item Reactive. Not at all. The sequence of intervals is given. We could recast this as a function that, given an $n$, we generate the $n$-th nested interval based on what came before. But notice that even with a fixed real number target, such a method requires arbitrary choosing to be done as we unravel it.
    \item Rational-friendly. The rationals are those whose nested intervals converge to a rational number. There does not seem a particularly clear property that establishes this. Depending on the definition, we could have a finite nested interval sequence that ends in the singleton $q:q$ if it is allowed. 
    \item Supportive. On a practical level, we do like shrinking intervals. But it is not generally predefined intervals. Mostly, it is intervals that are generated when working and we would want to know that the given interval has a non-zero intersection with all the nested intervals. 
    \item Arithmeticizable. Interval arithmetic works. If we tried to build in a constrained size, such as the $n$-th interval has to be no longer than $1/n$ in length, then the arithmetic would become difficult to manage.  
\end{itemize}

While we will discuss this more in the Cauchy sequence section, we could try to solve the uniqueness problem by considering the equivalence class of nested intervals, presumably, something like nested intervals all of whom have a common intersection. This would solve the uniqueness problem, but makes most of the rest of the properties more problematic. It also has the issue, similar to the Cauchy sequence, though less severe, that the nesting intervals can be quite huge for as much of a large finite sequence as one wants. 


\subsection{Cauchy Sequences}

Let us run through our criteria. 

\begin{itemize}
    \item Uniqueness.
    \item Reactive. 
    \item Rational-friendly. 
    \item Supportive. 
    \item Arithmeticizable. 
\end{itemize}

\subsection{Dedekind Cuts}

Let us run through our criteria. 

\begin{itemize}
    \item Uniqueness.
    \item Reactive. 
    \item Rational-friendly. 
    \item Supportive. 
    \item Arithmeticizable. 
\end{itemize}

Explicit correspondence between these as much as possible: 

Infinite decimals, Dedekind cuts, cauchy sequences, nested intervals. 
Options: digit presentation of the number, an algorithm for computing the number,  or data on whether it contains it. 

The current definitions of real numbers has some unsatisfactory aspects. This
has been explored by Norm Wildberger ....

\begin{itemize}
    \item Infinite Decimals. Arithmetic is very problematic. Infinite choice vs
  algorithm. Unlimited carries basically stop us at some point. Related to
  this are other representations of numbers, such as continued fractions. The
  main issue is that these are all based on a particular representation which
  sidesteps existence and what the thing is, but feels very comfortable from a
  numerical point of view. 
  \item Equivalence Classes of Cauchy Sequences. Arbitrarily long initial portions
  of the sequences implies that, in a finite universe such as we have, all
  Cauchy sequences classes will look the same up to any given n. Could modify
  it by requiring the difference between terms to be less than a prescribed
  term, say 1/n, for a given point in the sequence. But then one needs to
  ensure that the arithmetic works out to keep that in place. But it does deal
  with the initial idiocy. That's really bad. 
  \item Dedekind Cuts. Infinite sets, the construction is not really in line with
  how we produce or use real numbers. If one is okay with infinite sets and
  few explicit examples, then it seems like it is a foundation. 
  \item Nested Intervals. This works and is strongly related to Dedekind cuts, but
  to see it is a given sequence of nested intervals has uniqueness issues. So
  one needs a refinement, such as Cauchy sequences. Also, nothing to stop the
  same objection in Cauchy sequences in which an initial amount of the
  sequence is not specific at all. It is a little bit more controlled, but it
  could take a trillion intervals to get anywhere near the number of interest.
\end{itemize}


We propose a new definition, one which is in line with how we actually use it.
It comes with it a couple of immediate to use algorithms, but it is not an
attempt to produce a string of better approximations. It is, rather, an
attempt to give form to an approximation.

If we recast the Dedekind cut into a statement about a mapping of rational
numbers to the numbers -1, 0, 1, as we describe presently, then this
similarity of our approach becomes even more apparent: 
Indicator function of a partition of Q based on mythical r: 
R(q) = -1 if q is less than r, R(q) = 0 if q is r, and R(q) = 1 if q is
greater than r. 

We can then get away from thinking of having to produce a set for the Dedekind
cut and instead view it as something to compute when we want to know whether
we are less than or greater than a given real number. This gets us closer to a
useful description, but it still isn't quite what we are looking for. 

\subsection{Extended Reals}

It turns out to be almost trivial to extend the Oracles to include extended version of the real numbers that includes $\pm \infty$. We need to include infinite intervals in our definition and drop our finite clause.

For unbounded intervals, we could write that as $a:$ or $a:\infty$ for all rationals greater than or equal to $a$ and write $:a$ or $-\infty:a$ for all rationals less than or equal to $a$. The special interval $-\infty:\infty$ consists of all rationals. 

If we changed the Finite property to the assertion that $R(-\infty:\infty)=1$, we could then define the Oracle of $-\infty$ as the rule $R(:a) = 1$ for all rational $a$ and 0 otherwise. The Oracle of $\infty$ is similarly defined as $R(a:)=1$  for all rational $a$ and 0 otherwise. 


\section{Concluding Thoughts}

Our definition is designed to be a tool for using the number. Some of the other definitions are designed at giving the approximations as the number. This definition resists doing so because of the issue of uniqueness and not having to actually make all those choices. 

An alternative definition could have been to have a function that given a rational epsilon, we get an interval. Again, this would be problematic for us having to define the interval amongst many choices. A unique oracle per real number is very useful. It also helps with defining the arithmetic. 


\bigskip

\noindent \textbf{Acknowledgements. } We gratefully acknowledge NJ Wildberger for criticisms of 

\appendix

\section{Technical Lemmas}

This is a place to collect some technical facts that are commonly known, but that we would like to have collected proofs of. In particular, these do not rely on the results of any real number analysis. 

\begin{lemma}
$x^n$ is monotonic for $x>0$. That is, if $0 < a < b$, then $a^n<b^n$.
\end{lemma}

\begin{proof}
$b^n-a^n= (b-a)\sum_{k=1}^n b^k a^{n-k}$. Since both $a$ and $b$ are positive, the sum is positive. The sign is therefore determined by $b-a$. If $b>a$, then $b^n-a^n > 0$ as was to be shown. 
\end{proof}

\begin{lemma}
Let $r \geq 0 $ and $q > 0$ be rational numbers such that $r^n < q$. Then there exists a rational number $s$ such that $r < s$ and $s^n < q$.
\end{lemma}

The basic idea is to find $N$ $s = r + \tfrac{1}{N}$ such that $s^n < q$. We use the completely rational binomial theorem.  

\begin{proof}
Define $a = q - r^n$. Define $N = \tfrac{3}{a} \max(1,n r^{n-1}, (r+1)^n$.  Take $s = r + \tfrac{1}{N}$. Then $s^n = (r+ \tfrac{1}{N})^n = r^n + \tfrac{n r^{n-1}}{N} + \sum_{k=2}^{n} \binom{n}{k} \tfrac{r^k}{N^{n-k}}$. We can factor out a $\tfrac{1}{N}$ in the sum and, since $N > 1$, we have $\tfrac{b}{N^i} < b$ for all $b$ and natural number $i$. Thus, $\sum_{k=2}^{n} \binom{n}{k} \tfrac{r^k}{N^{n-k}} < \tfrac{1}{N} \sum_{k=2}^{n} \binom{n}{k} r^k$  But that sum is part of the expansion of $(r+1)^n$ and is therefore bounded by it since those are all positive terms thanks to $r$ being positive. Thus, we have $s^n < r^n + n \tfrac{r^{n-1}}{N} + \tfrac{ (r+1)^n }{N}$.  By definition, we have $N > \tfrac{3}{a} n r^{n-1}$ implying $\tfrac{a}{3} > \tfrac{ r^{n-1}}{N}$. We also have $N > \tfrac{3}{a} (r+1)^n$ implying $\tfrac{a}{3} > \tfrac{(r+1)^n}{N}$. Therefore $s^n < r^n + \tfrac{2 a}{3} < q$. Since $r<s$, we have shown our result. 
\end{proof}

\begin{lemma}
Let $r \geq 0 $ and $q > 0$ rational numbers such that $r^n > q$. Then there exists a rational number $s$ such that $r > s$ and $s^n > q$.
\end{lemma}

The trick here is to consider $(r-\tfrac{1}{N})^n$ instead. 

\begin{proof}
Define $a = r^n - q$. Define $N = \tfrac{3}{a} \max(1,n r^{n-1}, (r+1)^n$.  Take $s = r - \tfrac{1}{N}$. Then $s^n = (r- \tfrac{1}{N})^n = r^n - \tfrac{n r^{n-1}}{N} + \sum_{k=2}^{n} \binom{n}{k} \tfrac{ (-1)^(n-k) r^k}{N^{n-k}}$. We can factor out a $\tfrac{1}{N}$ in the sum and, since $N > 1$, we have $\tfrac{b}{N^i} < b$ for all $b$ and natural number $i$. Since we are looking to prove $s^n > q$, making the expression smaller is what we are set to do. If we replace any positive terms in the sum with negative terms, we will make it smaller. So $s^n > r^n - \tfrac{n r^{n-1}}{N} - \sum_{k=2}^{n} \binom{n}{k} \tfrac{r^k}{N^{n-k}}$. As before, $\sum_{k=2}^{n} \binom{n}{k} \tfrac{r^k}{N^{n-k}} < \tfrac{1}{N} \sum_{k=2}^{n} \binom{n}{k} r^k$  But that sum is part of the expansion of $(r+1)^n$ and is therefore bounded by it since those are all positive terms thanks to $r$ being positive. Thus, we have $s^n > r^n - n \tfrac{r^{n-1}}{N} - \tfrac{ (r+1)^n }{N}$.  By definition, we have $N > \tfrac{3}{a} n r^{n-1}$ implying $\tfrac{a}{3} > \tfrac{ r^{n-1}}{N}$. We also have $N > \tfrac{3}{a} (r+1)^n$ implying $\tfrac{a}{3} > \tfrac{(r+1)^n}{N}$. Therefore $s^n > r^n - \tfrac{2 a}{3} > q$. Since $r>s$, we have shown our result. 
\end{proof}



\begin{thebibliography}{28.}
d

\end{thebibliography}
%\tableofcontents
\end{document}


    
Our definition is designed to be a tool for using a real number. Some of the other definitions are designed at giving the approximations as the number. This definition resists doing so because of the issue of uniqueness and not having to actually compute the infinite amount required by other definitions. 

An alternative definition could have been to have a function that given a rational $\varepsilon$, we get an interval. Again, this would be problematic for us having to define the interval amongst many choices. A unique oracle per real number is very useful. It also helps with defining the arithmetic. 

One of the salient features that comes from this point of view is that rational numbers are the numbers we can see explicitly while irrational numbers are those that we can only get a sense of their presence from their rational neighbors. They are distinctly different kinds of numbers with very different properties. 

We have given a suggestion for a function definition that goes along with the oracle notions. It very much embraces this difference, emphasizing the separate roles of rational and irrational. The rational can take on any values independent of those around them while the irrational can only take on the values compatible with those around them. This is largely because we can directly use a rational number in a computation while the irrationals we cannot except in symbolic form. 

Whether these ideas have any practical import is not clear to me. To the extent that they do, it is presumably already being used in applied areas. This approach may simply provide a clean and useful language to describe what is being done. It may also inspire new perspectives and uses of important theorems, such as perhaps giving rise to a more prescriptive Fundamental Theorem of Algebra. 
